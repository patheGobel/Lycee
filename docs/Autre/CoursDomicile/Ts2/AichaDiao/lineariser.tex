\documentclass[12pt]{article}
\usepackage{amsmath,amssymb}
\usepackage{calligra} % Police manuscrite
\usepackage[T1]{fontenc}
\usepackage{eso-pic}  % Pour ajouter du contenu en arrière-plan
\usepackage{xcolor}    % Pour gérer la couleur
\title{Linéarisation}
\author{A.DIAO}
\date{Année scolaire 2025-2026}
% Commande pour la signature en arrière-plan
\newcommand{\BackgroundSignature}{
    \AddToShipoutPictureBG*{
        \AtPageCenter{%
            \makebox(0,0){\textcolor[gray]{0.8}{\fontsize{50}{60}\selectfont \textcalligra{PGB}}}
        }
    }
}
\begin{document}
\maketitle % Affiche le titre, l'auteur et la date
\BackgroundSignature

On linéarise $(\cos x)^4$ en utilisant les nombres complexes.

\medskip
\textbf{Rappel (formule d’Euler) :}
\[
\cos x=\frac{e^{ix}+e^{-ix}}{2}
\]

\medskip
\textbf{Écriture complexe :}
\[
(\cos x)^4=\left(\frac{e^{ix}+e^{-ix}}{2}\right)^4
=\frac{1}{16}(e^{ix}+e^{-ix})^4
\]

\medskip
\textbf{Développement :}
\[
(e^{ix}+e^{-ix})^4
= e^{4ix}+4e^{2ix}+6+4e^{-2ix}+e^{-4ix}
\]

\medskip
\textbf{Regroupement en cosinus :}
\[
e^{ikx}+e^{-ikx}=2\cos kx
\]

Ainsi,
\[
(e^{ix}+e^{-ix})^4
= 2\cos 4x + 8\cos 2x + 6
\]

\medskip
\textbf{Expression finale :}
\[
(\cos x)^4
=\frac{1}{16}\left(2\cos 4x + 8\cos 2x + 6\right)
\]

\[
\boxed{(\cos x)^4=\frac{3}{8}+\frac{1}{2}\cos 2x+\frac{1}{8}\cos 4x}
\]

\end{document}
