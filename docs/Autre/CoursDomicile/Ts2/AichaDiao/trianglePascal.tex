\documentclass[12pt]{article}
\usepackage[utf8]{inputenc}
\usepackage[T1]{fontenc}
\usepackage{amsmath,amssymb}
\usepackage{xcolor}
\usepackage{tcolorbox}
\usepackage{geometry}
\geometry{margin=2.5cm}

\definecolor{titre}{RGB}{180,0,0}

\begin{document}

\begin{center}
    {\LARGE \textcolor{titre}{\textbf{Mini-cours : Triangle de Pascal}}}\\[0.3cm]
    {\large Pour trouver les coefficients du binôme}
\end{center}

\bigskip

%------------------------------------------------
\section*{\underline{\textcolor{titre}{1. Objectif du triangle de Pascal}}}

Le \textbf{triangle de Pascal} permet de trouver rapidement les \textbf{coefficients}
dans le développement de :
\[
(a+b)^n
\]
sans calcul compliqué.

\medskip
Il sert uniquement à connaître les \textbf{nombres placés devant chaque terme}.

%------------------------------------------------
\section*{\underline{\textcolor{titre}{2. Construction du triangle de Pascal}}}

\begin{tcolorbox}
\begin{itemize}
    \item Le premier nombre est toujours \textbf{1}.
    \item Les nombres sur les bords sont toujours \textbf{1}.
    \item Chaque nombre est la \textbf{somme des deux nombres situés juste au-dessus}.
\end{itemize}
\end{tcolorbox}

%------------------------------------------------
\section*{\underline{\textcolor{titre}{3. Premières lignes du triangle}}}

\[
\begin{array}{l}
\text{Ligne 0 :}\quad 1 \\[0.2cm]
\text{Ligne 1 :}\quad 1 \quad 1 \\[0.2cm]
\text{Ligne 2 :}\quad 1 \quad 2 \quad 1 \\[0.2cm]
\text{Ligne 3 :}\quad 1 \quad 3 \quad 3 \quad 1 \\[0.2cm]
\text{Ligne 4 :}\quad 1 \quad 4 \quad 6 \quad 4 \quad 1 \\[0.2cm]
\text{Ligne 5 :}\quad 1 \quad 5 \quad 10 \quad 10 \quad 5 \quad 1
\end{array}
\]

\[
\boxed{
\begin{array}{cccccc}
1 \\
1 & 1 \\
1 & 2 & 1 \\
1 & 3 & 3 & 1 \\
1 & 4 & 6 & 4 & 1 \\
1 & 5 & 10 & 10 & 5 & 1
\end{array}
}
\]


\medskip
 La \textbf{ligne n} correspond au développement de \((a+b)^n\).

%------------------------------------------------
\section*{\underline{\textcolor{titre}{4. Méthode de développement}}}

Pour développer \((a+b)^n\) :

\begin{tcolorbox}
\begin{enumerate}
    \item Écrire la \textbf{ligne n} du triangle de Pascal
    \item La puissance de \(a\) \textbf{diminue}
    \item La puissance de \(b\) \textbf{augmente}
\end{enumerate}
\end{tcolorbox}

%------------------------------------------------
\section*{\underline{\textcolor{titre}{5. Exemples}}}

\textbf{Exemple 1 :} \((a+b)^3\)

Ligne 3 :
\[
1 \quad 3 \quad 3 \quad 1
\]

\[
(a+b)^3 = a^3 + 3a^2b + 3ab^2 + b^3
\]

\bigskip

\textbf{Exemple 2 :} \((x+2)^4\)

Ligne 4 :
\[
1 \quad 4 \quad 6 \quad 4 \quad 1
\]

\[
(x+2)^4 = x^4 + 4x^3(2) + 6x^2(2^2) + 4x(2^3) + 2^4
\]

\[
(x+2)^4 = x^4 + 8x^3 + 24x^2 + 32x + 16
\]

%------------------------------------------------
\section*{\underline{\textcolor{titre}{6. À retenir absolument}}}

\begin{tcolorbox}
\begin{itemize}
    \item Le triangle donne seulement les \textbf{coefficients}
    \item Les lignes sont \textbf{symétriques}
    \item Méthode rapide et fiable en examen
    \item Très efficace pour les petites puissances
\end{itemize}
\end{tcolorbox}

\bigskip

\begin{center}
\textit{Fin du mini-cours}
\end{center}

\end{document}
