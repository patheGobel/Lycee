\documentclass[12pt]{article}
\usepackage{amsmath,amssymb}
\usepackage{calligra} % Police manuscrite
\usepackage[T1]{fontenc}
\usepackage{eso-pic}  % Pour ajouter du contenu en arrière-plan
\usepackage{xcolor}    % Pour gérer la couleur
\title{Linéarisation}
\author{E.DIA}
\date{Année scolaire 2025-2026}
% Commande pour la signature en arrière-plan
\newcommand{\BackgroundSignature}{
    \AddToShipoutPictureBG*{
        \AtPageCenter{%
            \makebox(0,0){\textcolor[gray]{0.8}{\fontsize{50}{60}\selectfont \textcalligra{PGB}}}
        }
    }
}
\begin{document}
\maketitle % Affiche le titre, l'auteur et la date
\BackgroundSignature

On linéarise $(\cos x)^5$ en utilisant les nombres complexes.

\medskip
\textbf{Rappel (formule d’Euler) :}
\[
\cos x = \frac{e^{ix} + e^{-ix}}{2}
\]

\medskip
\textbf{Écriture complexe :}
\[
(\cos x)^5 = \left(\frac{e^{ix} + e^{-ix}}{2}\right)^5
= \frac{1}{32} (e^{ix} + e^{-ix})^5
\]

\medskip
\textbf{Développement :}
\[
(e^{ix} + e^{-ix})^5 = e^{5ix} + 5 e^{3ix} + 10 e^{ix} + 10 e^{-ix} + 5 e^{-3ix} + e^{-5ix}
\]

\medskip
\textbf{Regroupement en cosinus :}
\[
e^{ikx} + e^{-ikx} = 2 \cos kx
\]

Ainsi,
\[
(e^{ix} + e^{-ix})^5 = 2 \cos 5x + 10 \cos 3x + 20 \cos x
\]

\medskip
\textbf{Expression finale :}
\[
(\cos x)^5 = \frac{1}{32} \big( 2 \cos 5x + 10 \cos 3x + 20 \cos x \big)
= \frac{1}{16}\cos 5x + \frac{5}{16}\cos 3x + \frac{5}{8}\cos x
\]

\[
\boxed{(\cos x)^5 = \frac{1}{16}\cos 5x + \frac{5}{16}\cos 3x + \frac{5}{8}\cos x}
\]

+++++++++++++++++++++++++++++++++++++++++++++++++++++++

On linéarise $(\sin x)^3$ en utilisant les nombres complexes.

\medskip
\textbf{Rappel (formule d’Euler) :}
\[
\sin x = \frac{e^{ix} - e^{-ix}}{2i}
\]

\medskip
\textbf{Écriture complexe :}
\[
(\sin x)^3 = \left(\frac{e^{ix} - e^{-ix}}{2i}\right)^3 = \frac{1}{8i^3} (e^{ix} - e^{-ix})^3
\]

\medskip
\textbf{Développement :}
\[
(e^{ix} - e^{-ix})^3 = e^{3ix} - 3 e^{ix} + 3 e^{-ix} - e^{-3ix} = (e^{3ix}-e^{-3ix}) + 3(e^{ix}-e^{-ix})
\]

\medskip
\textbf{Regroupement en sinus :}
\[
e^{ikx} - e^{-ikx} = 2i \sin kx
\]

Ainsi,
\[
(e^{ix} - e^{-ix})^3 = 2i \sin 3x + 6i \sin x
\]

\medskip
\textbf{Expression finale :}
\[
(\sin x)^3 = \frac{3}{4} \sin x - \frac{1}{4} \sin 3x
\]

\[
\boxed{(\sin x)^3 = \frac{3}{4} \sin x - \frac{1}{4} \sin 3x}
\]

++++++++++++++++++++++++++++++++++

On linéarise $(\sin x)^4$ en utilisant les nombres complexes.

\medskip
\textbf{Rappel (formule d’Euler) :}
\[
\sin x = \frac{e^{ix} - e^{-ix}}{2i}
\]

\medskip
\textbf{Écriture complexe :}
\[
(\sin x)^4 = \left(\frac{e^{ix} - e^{-ix}}{2i}\right)^4 = \frac{1}{16} (e^{ix} - e^{-ix})^4
\]

\medskip
\textbf{Développement :}
\[
(e^{ix} - e^{-ix})^4 = e^{4ix} - 4 e^{2ix} + 6 - 4 e^{-2ix} + e^{-4ix} = (e^{4ix}+e^{-4ix}) - 4(e^{2ix}+e^{-2ix}) + 6
\]

\medskip
\textbf{Regroupement en cosinus :}
\[
e^{ikx} + e^{-ikx} = 2 \cos kx
\]

Ainsi,
\[
(e^{ix} - e^{-ix})^4 = 2 \cos 4x - 8 \cos 2x + 6
\]

\medskip
\textbf{Expression finale :}
\[
(\sin x)^4 = \frac{1}{16} (2 \cos 4x - 8 \cos 2x + 6) = \frac{3}{8} - \frac{1}{2} \cos 2x + \frac{1}{8} \cos 4x
\]

\[
\boxed{(\sin x)^4 = \frac{3}{8} - \frac{1}{2} \cos 2x + \frac{1}{8} \cos 4x}
\]


\end{document}