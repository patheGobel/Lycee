\documentclass[12pt,a4paper]{article}

% -----------------------------------------------------
% PACKAGES
% -----------------------------------------------------
\usepackage[utf8]{inputenc}
\usepackage[T1]{fontenc}

\usepackage{amsmath,amssymb,amsfonts,mathrsfs}
\usepackage{tikz}
\usetikzlibrary{patterns,arrows,shapes.geometric,fit}

\usepackage{enumitem}
\usepackage{multicol}
\usepackage{lmodern}
\usepackage{times}
\usepackage{pifont}

\usepackage[left=1cm,right=1cm,top=1cm,bottom=1.7cm]{geometry}

\usepackage[colorlinks=true, linkcolor=blue, urlcolor=blue, citecolor=blue]{hyperref}

\usepackage{array,multirow}
\usepackage[most]{tcolorbox}
\tcbuselibrary{skins,hooks}

\usepackage{varwidth}
\usepackage{float}

\usepackage{fancyhdr}
\usepackage{eso-pic}

\usepackage{tkz-tab}
\usepackage{systeme}
\usepackage{verbatim}
\usepackage{color,soul}

% -----------------------------------------------------
% TEXTE EN ARRIÈRE-PLAN (1 seule fois)
% -----------------------------------------------------
\AddToShipoutPicture{
    \AtTextCenter{%
        \makebox[0pt]{\rotatebox{80}{\textcolor[gray]{0.7}{\fontsize{5cm}{5cm}\selectfont PGB}}}
    }
}

% -----------------------------------------------------
% COMMANDES PERSONNALISÉES
% -----------------------------------------------------

% Items
\newcommand\circitem[1]{%
\tikz[baseline=(char.base)]{
\node[circle,draw=gray, fill=red!55,
minimum size=1.2em,inner sep=0] (char) {#1};}}

\newcommand\boxitem[1]{%
\tikz[baseline=(char.base)]{
\node[fill=cyan,
minimum size=1.2em,inner sep=0] (char) {#1};}}

\setlist[enumerate,1]{label=\protect\circitem{\arabic*}}
\setlist[enumerate,2]{label=\protect\boxitem{\alph*}}

\everymath{\displaystyle}

% EXEMPLES
\newcounter{exemple}
\newcommand{\exemple}{%
  \refstepcounter{exemple}%
  \textbf{\textcolor{green}{Exemple \theexemple :}} \ignorespaces
}

% EXERCICES D’APPLICATION
\definecolor{myorange}{rgb}{1.0, 0.8, 0.0}

\newcounter{exerciceapp}
\newcommand{\exerciceapp}{%
  \refstepcounter{exerciceapp}%
  \textbf{\textcolor{myorange}{Exercice d'application \theexerciceapp :}} \ignorespaces
}

% CORRECTIONS
\newcounter{correction}
\newcommand{\correction}{%
  \refstepcounter{correction}%
  \textbf{\textcolor{myorange}{Correction \thecorrection :}} \ignorespaces
}

% Soulignement coloré
\newcommand{\myul}[2][black]{\setulcolor{#1}\ul{#2}\setulcolor{black}}

% Tabulation
\newcommand\tab[1][1cm]{\hspace*{#1}}

% -----------------------------------------------------
% DOCUMENT
% -----------------------------------------------------
\begin{document}

\begin{center}
    \Large\textbf{\underline{Polynômes}}\\[-0.1cm]
    \normalsize\textbf{Prof : M. BA} \hfill \textbf{Classe : 1erL}\\[-0.1cm]
    \textbf{Année scolaire : 2025 -- 2026}
\end{center}

\section*{Introduction aux Polynômes}

\section*{\underline{\textbf{\textcolor{red}{I. Généralités }}}}
\subsection*{\underline{\textbf{\textcolor{red}{1. Monômes}}}} 
\underline{\textbf{\textcolor{red}{Définition et vocabulaire}}}

\underline{\exemple}

\underline{\textbf{\textcolor{red}{Remarque}}}

\subsection*{\underline{\textbf{\textcolor{red}{2. Polynômes}}}} 
\underline{\textbf{\textcolor{red}{Définition}}}

\underline{\exemple}

\underline{\textbf{\textcolor{red}{Remarque}}}

\section*{\underline{\textbf{\textcolor{red}{II. Trinômes du second degré }}}}

\subsection*{\underline{\textbf{\textcolor{red}{1. Définition et exemple}}}} 

\subsection*{\underline{\textbf{\textcolor{red}{2. Factorisation de $ax^{2}+bx+c$}}}} 

\subsection*{\underline{\textbf{\textcolor{red}{3. Signe de $ax^{2}+bx+c$}}}}

\underline{\exemple}

\underline{\exemple}

\section*{\underline{\textbf{\textcolor{red}{III. Théorème fondamental des polynômes }}}}

\subsection*{\underline{\textbf{\textcolor{red}{1. Racine d’un polynôme}}}}

\underline{\textbf{\textcolor{red}{Définition}}}

\underline{\exemple}

\textbf{\exerciceapp}

\subsection*{\underline{\textbf{\textcolor{red}{2. Théorème fondamental}}}}

\underline{\textbf{\textcolor{red}{Activité}}} 

\underline{\textbf{\textcolor{red}{Exploitation de l’activité}}}

\underline{\textbf{\textcolor{red}{Théorème}}}

\subsection*{\underline{\textbf{\textcolor{red}{3. Division euclidienne}}}}

\underline{\exemple}

\textbf{\exerciceapp}

\subsection*{\underline{\textbf{\textcolor{red}{4. Identification des coefficients}}}}
\underline{\exemple}

\underline{\exemple}

\subsection*{\underline{\textbf{\textcolor{red}{5. Tableau de Horner}}}}
\underline{\exemple}

\underline{\exemple}

\section*{\underline{\textbf{\textcolor{red}{VI. Factorisation pour $ n \geq 3 $ }}}}

\underline{\exemple}

\underline{\exemple}

\end{document}
