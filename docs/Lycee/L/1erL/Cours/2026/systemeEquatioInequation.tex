\documentclass[12pt,a4paper]{article}

% -----------------------------------------------------
% PACKAGES
% -----------------------------------------------------
\usepackage[utf8]{inputenc}
\usepackage[T1]{fontenc}

\usepackage{amsmath,amssymb,amsfonts,mathrsfs}
\usepackage{tikz}
\usetikzlibrary{patterns,arrows,shapes.geometric,fit}

\usepackage{enumitem}
\usepackage{multicol}
\usepackage{lmodern}
\usepackage{times}
\usepackage{pifont}

\usepackage[left=1cm,right=1cm,top=1cm,bottom=1.7cm]{geometry}

\usepackage[colorlinks=true, linkcolor=blue, urlcolor=blue, citecolor=blue]{hyperref}

\usepackage{array,multirow}
\usepackage[most]{tcolorbox}
\tcbuselibrary{skins,hooks}

\usepackage{varwidth}
\usepackage{float}

\usepackage{fancyhdr}
\usepackage{eso-pic}

\usepackage{tkz-tab}
\usepackage{systeme}
\usepackage{verbatim}
\usepackage{color,soul}

% -----------------------------------------------------
% TEXTE EN ARRIÈRE-PLAN (1 seule fois)
% -----------------------------------------------------
\AddToShipoutPicture{
    \AtTextCenter{%
        \makebox[0pt]{\rotatebox{80}{\textcolor[gray]{0.7}{\fontsize{5cm}{5cm}\selectfont PGB}}}
    }
}

% -----------------------------------------------------
% COMMANDES PERSONNALISÉES
% -----------------------------------------------------

% Items
\newcommand\circitem[1]{%
\tikz[baseline=(char.base)]{
\node[circle,draw=gray, fill=red!55,
minimum size=1.2em,inner sep=0] (char) {#1};}}

\newcommand\boxitem[1]{%
\tikz[baseline=(char.base)]{
\node[fill=cyan,
minimum size=1.2em,inner sep=0] (char) {#1};}}

\setlist[enumerate,1]{label=\protect\circitem{\arabic*}}
\setlist[enumerate,2]{label=\protect\boxitem{\alph*}}

\everymath{\displaystyle}

% EXEMPLES
\newcounter{exemple}
\newcommand{\exemple}{%
  \refstepcounter{exemple}%
  \textbf{\textcolor{green}{Exemple \theexemple :}} \ignorespaces
}

% EXERCICES D’APPLICATION
\definecolor{myorange}{rgb}{1.0, 0.8, 0.0}

\newcounter{exerciceapp}
\newcommand{\exerciceapp}{%
  \refstepcounter{exerciceapp}%
  \textbf{\textcolor{myorange}{Exercice d'application \theexerciceapp :}} \ignorespaces
}

% CORRECTIONS
\newcounter{correction}
\newcommand{\correction}{%
  \refstepcounter{correction}%
  \textbf{\textcolor{myorange}{Correction \thecorrection :}} \ignorespaces
}

% Soulignement coloré
\newcommand{\myul}[2][black]{\setulcolor{#1}\ul{#2}\setulcolor{black}}

% Tabulation
\newcommand\tab[1][1cm]{\hspace*{#1}}

% -----------------------------------------------------
% DOCUMENT
% -----------------------------------------------------
\begin{document}

\begin{center}
    \Large\textbf{\underline{Systèmes d'équations et d'inéquations}}\\[-0.1cm]
    \normalsize\textbf{Prof : M. BA} \hfill \textbf{Classe : 1erL}\\[-0.1cm]
    \textbf{Année scolaire : 2025 -- 2026}
\end{center}

\section*{Introduction aux Systèmes}

\section*{\underline{\textbf{\textcolor{red}{I. Systèmes de trois équations linéaires à trois inconnues }}}}
\subsection*{\underline{\textbf{\textcolor{red}{1. Exemple}}}} 

Le système 
$
\begin{cases}
x + 10y - 3z = 5 \\
2x - y + 2z = 2 \\
-x + y + z = -3
\end{cases}
$
est formé de trois équations. Chacune d'elle contient trois inconnues $x$ ; $y$ et $z$ avec des exposants tous égaux à 1. On dit qu'on a un système de trois équations linéaires à trois inconnues $x$ ; $y$ et $z$.

\vspace{0.5cm}

Résoudre un tel système c'est trouver tous les triplets $(x, y, z)$ de nombres réels qui vérifient les trois équations du système.

\subsection*{\underline{\textbf{\textcolor{red}{2. Résolution avec la méthode du pivot de Gauss}}}} 
\underline{\exemple}

\underline{\textbf{\textcolor{red}{Résolvons le système suivant en utilisant la méthode du pivot de Gauss}}}

$
\begin{cases}
x + 10y - 3z = 5 \\
2x - y + 2z = 2 \\
-x + y + z = -3
\end{cases}
$

Pour résoudre un tel système, on commence par désigner les 3 équations respectivement par $L_1 ; L_2$ et $L_3$
\[
\begin{cases}
x + 10y - 3z = 5 & (L_1) \\
2x - y + 2z = 2 & (L_2) \\
-x + y + z = -3 & (L_3)
\end{cases}
\]

\textbf{1\textsuperscript{ère} étape :} On fixe l'équation ($L_1$) puis on élimine l'inconnue x dans ($L_2$) en considérant le
\[
\text{sous-système } -2
\begin{cases}
x + 10y - 3z = 5 & (L_1) \\
2x - y + 2z = 2 & (L_2)
\end{cases}
\text{ d'où }
\begin{cases}
-2x - 20y + 6z = -10 \\
\quad \ 2x - y + 2z = 2
\end{cases}
\text{ ainsi on a :}
\]
$-21y + 8z = -8 \quad (L_2')$

\vspace{0.5cm}

\textbf{2\textsuperscript{ème} étape :} On fixe l'équation ($L_1$) puis on élimine l'inconnue x dans ($L_3$) en considérant le
\[
\text{sous-système }
\begin{cases}
x + 10y - 3z = 5 & (L_1) \\
-x + y + z = -3 & (L_3)
\end{cases}
\text{ d'où ainsi on a :}
\]
$11y - 2z = 2 \quad (L_3')$

\vspace{0.5cm}

Ainsi, on obtient le système équivalent suivant :
\[
\begin{cases}
x + 10y - 3z = 5 & (L_1) \\
-21y + 8z = -8 & (L_2') \\
\quad 11y - 2z = -3 & (L_3')
\end{cases}
\]

\noindent \textbf{3\textsuperscript{ème} étape :} On fixe $(L_2')$ puis on élimine y dans $(L_3')$ en considérant le sous-système
\[
\begin{cases}
-21y + 8z = -8 & (L_2') \\
\ \ 11y - 2z = 2 & (L_3')
\end{cases}
\quad \text{Ainsi on a } 46z = -46 \ (L_3'')
\]

On obtient le système équivalent suivant dit système triangulaire
\[
\begin{cases}
x + 10y - 3z = 5 & (L_1) \\
\quad -21y + 8z = -8 & (L_2') \\
\qquad \quad \ \ 46z = -46 & (L_3'')
\end{cases}
\]

\vspace{0.5cm}

\noindent \textbf{4\textsuperscript{ème} étape :} Pour terminer la résolution, on détermine z dans $(L_3'')$, puis on remplace z par cette valeur dans $(L_2')$, on obtient la valeur de y puis on obtient celle de x, en substituant à y et z par leurs valeurs respectives dans $(L_1')$.

\vspace{0.5cm}

Ainsi on a : $S = \{(2; 0; -1)\}$

\underline{\exemple}


\section*{\underline{\textbf{\textcolor{red}{II. Systèmes d'inéquations linéaires à deux inconnues }}}}

\subsection*{\underline{\textbf{\textcolor{red}{1. Inéquations linéaires à deux inconnues }}}} 

\underline{\exemple}

$2x + y - 5 > 0$ est une inéquation linéaire à deux inconnues $x$ et $y$.

\underline{\textbf{\textcolor{red}{Résolution graphique}}}


Pour résoudre graphiquement l'inéquation $2x + y - 5 > 0$, on représente la droite (D) d'équation $2x + y - 5 = 0$ dans un repère orthonormé $(O,I,J)$.

\vspace{0.5cm}

\begin{center}
\begin{tabular}{|c|c|c|}
\hline
$x$ & 0 & 1 \\
\hline
$y$ & 5 & 3 \\
\hline
\end{tabular}
\end{center}

\vspace{0.5cm}

Ensuite, on choisit un point qui n'est pas sur (D) et dont ses coordonnées sont connues.

Par exemple : le point $O(0; 0)$ puis on remplace dans l'inéquation $x$ et $y$ respectivement par les coordonnées de $O$.

Ainsi, on a $2(0) + 0 - 5 > 0$ c'est à dire $-5 > 0$ \textbf{faux} donc les coordonnées de $O$ ne vérifie pas l'inéquation. 

On barre donc le demi-plan de frontière (D) contenant $O$.

\subsection*{\underline{\textbf{\textcolor{red}{2. Systèmes de deux inéquations à deux inconnues}}}} 

\underline{\exemple}

\noindent Le système
$
\begin{cases}
x - 2y + 1 \geq 0 \\
2x + y - 3 < 0
\end{cases}
$
est un système de deux inéquations linéaires à deux inconnues $x$ et $y$

\underline{\textbf{\textcolor{red}{Résolution graphique}}}

\noindent Résolvons graphiquement le système
$
\begin{cases}
x - 2y + 1 \geq 0 \\
2x + y - 3 < 0
\end{cases}
$

On commence par représenter graphiquement les droites $(D_1)$ d'équation $x - 2y + 1 = 0$ et $(D_2)$ d'équation $2x + y - 3 = 0$ dans un repère orthonormé $(O, I, J)$.

\vspace{0.5cm}

\begin{center}
\begin{tabular}{|c|c|c|}
\hline
$x$ & -1 & 1 \\
\hline
$y$ & 0 & 1 \\
\hline
\end{tabular}
\quad % Ajoute un espace horizontal entre les deux tableaux
\begin{tabular}{|c|c|c|}
\hline
$x$ & 0 & 1 \\
\hline
$y$ & 3 & 1 \\
\hline
\end{tabular}
\end{center}

\vspace{0.5cm}

Puis on choisit un point qui n'est ni sur $(D_1)$, ni sur $(D_2)$ et dont les coordonnées sont connues. Par exemple : le point $O(0; 0)$.

\vspace{0.2cm}

En remplaçant $x$ et $y$ par les coordonnées de $O$ dans l'inéquation 1, on a : $0 - 2(0) + 1 \geq 0$ c'est à dire $1 \geq 0$ \textbf{vrai} donc les coordonnées de $O$ vérifie l'inéquation 1. On barre donc le demi-plan de frontière $(D_1)$ ne contenant pas $O$.

\vspace{0.2cm}

En remplaçant $x$ et $y$ par les coordonnées de $O$, dans l'inéquation 2, on a : $2(0) + 0 - 3 < 0$ c'est à dire $-3 < 0$ \textbf{vrai} donc les coordonnées de $O$ vérifie l'inéquation 2. On barre donc le demi-plan de frontière $(D_2)$ ne contenant pas $O$.

\vspace{0.2cm}

NB : le 3. Sera donné comme exercice à faire

\subsection*{\underline{\textbf{\textcolor{red}{3. Systèmes de trois inéquations à deux inconnues }}}}

\underline{\exemple}

\noindent Le système
$
\begin{cases}
-x + y + 2 \geq 0 \\
2x - y - 4 < 0 \\
x - 2y - 1 > 0
\end{cases}
$
est un système de trois inéquations à deux inconnues $x$ et $y$.

\underline{\textbf{\textcolor{red}{Résolution graphique}}}
$
\begin{cases}
-x + y + 2 \geq 0 \\
2x - y - 4 < 0 \\
x - 2y - 1 > 0
\end{cases}
$

\end{document}
