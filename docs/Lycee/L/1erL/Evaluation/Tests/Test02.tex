\documentclass[a4paper,12pt]{article}
\usepackage{graphicx}
\usepackage[a4paper, top=0cm, bottom=2cm, left=2cm, right=2cm]{geometry} % Ajuste les marges
\usepackage{xcolor} % Pour ajouter des couleurs
\usepackage{hyperref} % Pour avoir des références colorées si nécessaire
\usepackage{eso-pic}         % Pour ajouter des éléments en arrière-plan

\usepackage[french]{babel}
\usepackage[T1]{fontenc}
\usepackage{mathrsfs}
\usepackage{amsmath}
\usepackage{amsfonts}
\usepackage{amssymb}
\usepackage{tkz-tab}

\usepackage{tikz}
\usetikzlibrary{arrows, shapes.geometric, fit}
\newcounter{correction} % Compteur pour les questions

% Définir la commande pour afficher une question numérotée
\newcommand{\question}{%
  \refstepcounter{correction}%
  \textbf{\textcolor{black}{Question \thecorrection  :}} \ignorespaces
}
% Commande pour ajouter du texte en arrière-plan
\AddToShipoutPicture{
    \AtTextCenter{%
        \makebox[0pt]{\rotatebox{80}{\textcolor[gray]{0.7}{\fontsize{10cm}{10cm}\selectfont PGB}}}
    }
}
\begin{document}

% En-tête
\begin{center}
    \begin{tabular}{@{} p{5cm} p{5cm} p{5cm} @{}}\\[0.2cm] % 3 colonnes avec largeurs fixées
        Lycée Dindéfélo & Test 1 & 07 Novembre 2025 \\[0.2cm]
    \end{tabular}
    \\[-0.01cm] % Ajuster l'espace vertical entre le tableau et la barre
    \hrule % Barre horizontale
\end{center}
\begin{center}
    \textbf{\Large Calcul Dans $\mathbb{R}$} \\[0.2cm]
    \textbf{\large Professeur : M. BA} \\[0.2cm]
    \textbf{Classe : 2nd S} \\[0.2cm]
    \textbf{\small Durée : 10 minutes} \\[0.2cm]
    \textbf{\small Note :\quad\quad /5}
\end{center}

% Nom de l'élève
\textbf{\small Nom de l'élève :} \underline{\hspace{8cm}} \\[0.5cm]

% Introduction aux questions
%Complétez les exercices suivants en utilisant le cours et vos connaissances sur les asymptotes. \\[0.3cm]

\question \textbf{(2 points)}
Complèter les identités suivantes
\[
(a+t)^{3} = \underline{\hspace{7cm}}, \hspace{2cm} (a-t)^{3} = \underline{\hspace{7cm}}
\]
\[
a^{3}+t^{3} = \underline{\hspace{7cm}}, \hspace{2cm} a^{3}-t^{3} = \underline{\hspace{7cm}}
\]

\question \textbf{(3 points)}
Factoriser les expressions suivantes
\[
 a^2xy + aby^2 + b^2xy + abx^2  = \underline{\hspace{14cm}}
\]
\[
  \hspace{4.2cm} = \underline{\hspace{14cm}}
\]
\[
  \hspace{4.2cm} = \underline{\hspace{14cm}}
\]
\[
  \hspace{4.2cm} = \underline{\hspace{14cm}}
\]
\[
  \hspace{4.2cm} = \underline{\hspace{14cm}}
\]
\[
  \hspace{4.2cm} = \underline{\hspace{14cm}}
\]
\[
  \hspace{4.2cm} = \underline{\hspace{14cm}}
\]
\[
 3a^2 + 3b^2 - 12c^2 - 6ab  = \underline{\hspace{14cm}}
\]
\[
 \hspace{3.42cm} = \underline{\hspace{14cm}}
\]
\[
 \hspace{3.42cm} = \underline{\hspace{14cm}}
\]
\[
 \hspace{3.42cm} = \underline{\hspace{14cm}}
\]
\[
 \hspace{3.42cm} = \underline{\hspace{14cm}}
\]
\[
 \hspace{3.42cm} = \underline{\hspace{14cm}}
\]
\[
 \hspace{3.42cm} = \underline{\hspace{14cm}}
\]
\[
 \hspace{3.42cm} = \underline{\hspace{14cm}}
\]
\[
  y^2 - x^2 + 2x - 1 = \underline{\hspace{14cm}}
\]
\[
 \hspace{2.5cm} = \underline{\hspace{14cm}}
\]
\[
 \hspace{2.5cm} = \underline{\hspace{14cm}}
\]
\[
 \hspace{2.5cm} = \underline{\hspace{14cm}}
\]
\[
 \hspace{2.5cm} = \underline{\hspace{14cm}}
\]
\[
 \hspace{2.5cm} = \underline{\hspace{14cm}}
\]
\[
 \hspace{2.5cm} = \underline{\hspace{14cm}}
\]

\end{document}
