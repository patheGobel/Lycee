\documentclass[a4paper,12pt]{article}
\usepackage{graphicx}
\usepackage[a4paper, top=0cm, bottom=2cm, left=2cm, right=2cm]{geometry} % Ajuste les marges
\usepackage{xcolor} % Pour ajouter des couleurs
\usepackage{hyperref} % Pour avoir des références colorées si nécessaire
\usepackage{eso-pic} % Pour ajouter des éléments en arrière-plan

\usepackage[french]{babel}
\usepackage[T1]{fontenc}
\usepackage{mathrsfs}
\usepackage{amsmath}
\usepackage{amsfonts}
\usepackage{amssymb}
\usepackage{tkz-tab}
\usepackage{tcolorbox} % Pour encadrer avec fond coloré
\usepackage{pgfplots}

\usepackage{tikz}
\usetikzlibrary{arrows, shapes.geometric, fit}
\newcounter{correction} % Compteur pour les questions

% Définir la commande pour afficher une question numérotée
\newcommand{\question}{%
  \refstepcounter{correction}%
  \textbf{\textcolor{black}{Question \thecorrection (1 point) :}} \ignorespaces
}
% Commande pour ajouter du texte en arrière-plan
\AddToShipoutPicture{
    \AtTextCenter{%
        \makebox[50pt]{\rotatebox{80}{\textcolor[gray]{0.5}{\fontsize{10cm}{10cm}\selectfont PGB}}}
    }
}

% Définition de l'encadré adaptatif avec fond jaune
\newtcolorbox{resultbox}{
    colback=red!30, % Fond rouge clair
    colframe=black, % Bordure noire fine
    sharp corners, % Coins nets
    boxrule=0.5pt, % Contour léger
    boxsep=2pt, % Espacement interne
    left=5pt, right=5pt, top=2pt, bottom=2pt, % Marges internes
}

\begin{document}
\hrule % Barre horizontale
% En-tête
\begin{center}
    \begin{tabular}{@{} p{5cm} p{5cm} p{5cm} @{}} % 3 colonnes avec largeurs fixées
        Lycée Dindéfélo & \quad\quad Teste 11 & 07 Mars 2024 \\
    \end{tabular}
    \\[-0.01cm] % Ajuster l'espace vertical entre le tableau et la barre
    \hrule % Barre horizontale
\end{center}
\begin{center}
    \textbf{\Large Nombre Complexe} \\[0.2cm]
    \textbf{\large Professeur : M. BA} \\[0.2cm]
    \textbf{Classe : Terminale S2} \\[0.2cm]
    \textbf{\small Durée : 10 minutes} \\[0.2cm]
    \textbf{\small Note :\quad\quad /5}
\end{center}

% Nom de l'élève
\textbf{\small Nom de l'élève :} \underline{\hspace{8cm}} \\[0.5cm]

\begin{enumerate}
\item Pour tout nombre complexe $z \neq -1 + 2i$, on pose $Z = \frac{z - 2 + 4i}{z + 1 - 2i}$.

Déterminons l'ensemble des points $M$ du plan tels que :

\begin{enumerate}
        \item $|Z| = 1$\\
        	$\begin{aligned}
        	|Z| = 1 &\implies \mid \frac{z - 2 + 4i}{z + 1 - 2i} \mid = 1\\
        					&\implies \frac{\mid z - 2 + 4i \mid}{\mid z + 1 - 2i \mid} = 1\\
        					&\implies \mid z - 2 + 4i \mid = \mid z + 1 - 2i \mid\\
        					&\implies \mid x+iy - 2 + 4i \mid = \mid x+iy + 1 - 2i \mid\\
                            &\implies \mid x-2+i(y+4) \mid = \mid x+1 +i(y-2) \mid\\
        					&\implies \sqrt{ (x-2)^{2} + (4+y)^{2} } = \sqrt{ (x+1)^{2} + (y - 2)^{2} }\\
                            &\implies (x-2)^{2} + (4+y)^{2}=(x+1)^{2} + (y - 2)^{2}\\
                            &\implies (x^2 - 4x + 4) + (16 + 8y + y^2) = (x^2 + 2x + 1) + (y^2 - 4y + 4)\\
                            &\implies x^2 - 4x + 4 + 16 + 8y + y^2 = x^2 + 2x + 1 + y^2 - 4y + 4\\
                            &\implies  x^2 + y^2 - 4x + 8y + 20 = x^2 + y^2 + 2x - 4y + 5\\
                            &\implies -4x + 8y + 20 = 2x - 4y + 5\\
                            &\implies -4x - 2x + 8y + 4y = 5 - 20\\
                            &\implies -6x + 12y = -15\\
                            &\implies 2x - 4y = 5
        	\end{aligned}$
        	
\begin{resultbox}
    \[
    \mathbf{\text{L'ensemble des points $M$ tel que $|Z| = 1$ est la droite d'équation : $2x - 4y = 5$}}
    \]
\end{resultbox}

\textcolor{red}{\textbf{Autre méthode}}

$Z = \frac{z - 2 + 4i}{z + 1 - 2i}$ en posant $z_{M} =  x +iy$, $z_{A} =  2 - 4i$ et $z_{B} = - 1 + 2i$

$Z = \frac{z_{M} - z_{A}}{z_{M} - z_{B}}$

$
\begin{aligned}
|Z| = 1 &\implies \mid \frac{z_{M} - z_{A}}{z_{M} - z_{B}} \mid = 1\\
				&\implies \frac{MA}{MB} = 1\\
				&\implies MA=MB
\end{aligned}
$

\begin{resultbox}
    \[
    \mathbf{\text{L'ensemble des points $M$ tel que $|Z| = 1$ est la Médiatrice  du segment [AB]}}
    \]
\end{resultbox}
\begin{resultbox}
    \[
    \mathbf{E = \left\{ M(x, y) \in \mathbb{R}^2 \mid 2x - 4y = 5 \right\}}
    \]
\end{resultbox}

\newpage

        \item $Z$ soit un réel.
        
        
		$$Z = \frac{x^2 + y^2 - x + 2y - 10 + i(6x + 3y)}{(x+1)^2 + (y-2)^2}$$

Pour que \( Z \) soit un réel, la partie imaginaire du numérateur doit être nulle, c'est-à-dire :

\[
6x + 3y = 0
\]

\[
2x + y = 0
\]
L'ensemble des points $M$ tel que $Z$ soit un réel est la droite d'équation $2x + y = 0$ privé du point $\begin{pmatrix} -1 \\ 2 \end{pmatrix}$

\begin{resultbox}
    \[
    \mathbf{E = \left\{ (x, y) \in \mathbb{R}^2 \mid 2x + y = 0, \quad \begin{pmatrix} x \\ y \end{pmatrix} \neq \begin{pmatrix} -1 \\ 2 \end{pmatrix} \right\}}
    \]
\end{resultbox}

\end{enumerate}

\item Pour tout complexe $z \neq i$, on pose
    \(
    U = \frac{z + i}{z - i}.
    \)
    
    Déterminons l’ensemble des points $M$ d’affixe $z$ tels que :
    
    \(
    \begin{aligned}
     U&= \frac{z + i}{z - i}\\
     	&=\frac{x+iy + i}{x+iy - i}\\
     	&=\frac{x+i(y + 1)}{x+i(y - 1)}\\    
     	&=\frac{\left[ x+i(y + 1) \right]\left[ x-i(y - 1) \right]}{x^{2}+(y - 1)^{2}}\\
     	&=\frac{x\left[ x-i(y - 1) \right]+i(y + 1)\left[ x-i(y - 1) \right]}{x^{2}+(y - 1)^{2}}\\
     	&=\frac{x^{2} - ix(y - 1) + ix(y + 1) - i^{2}(y + 1)(y - 1)}{x^{2}+(y - 1)^{2}}\\   	
     	&=\frac{x^{2} - ixy + ix + ixy + ix + (y + 1)(y - 1)}{x^{2}+(y - 1)^{2}}\\ 
     	&=\frac{x^{2} - ixy + ix + ixy + ix + y^{2}-1}{x^{2}+(y - 1)^{2}}\\ 
     	&=\frac{x^{2} +2ix + y^{2}-1}{x^{2}+(y - 1)^{2}}\\ 
     	&=\frac{x^{2} + y^{2}-1+2ix}{x^{2}+(y - 1)^{2}}\\
    \end{aligned}
    \)
\begin{resultbox}
    \[
    \mathbf{U=\frac{x^{2} + y^{2}-1+2ix}{x^{2}+(y - 1)^{2}}}
    \]
\end{resultbox}
    \begin{enumerate}
        \item $U \in \mathbb{R}^*_{-}$\\
        Pour que \( U \in \mathbb{R}^*_{-} \) soit réel, il faut que \( Im(U)=0 \) et \( Re(U)<0 \) 
        
        \( 
        	\begin{aligned}
        	Im(U)=0  \text{ et }  Re(U)<0 &\implies 2x=0 \text{ et } x^{2} + y^{2}-1<0\\
        	 															&\implies x=0 \text{ et } x^{2} + y^{2}<1\\
        	\end{aligned}
        \)

Ainsi, l'ensemble des points \( M \) d'affixe \( z \) est donné par :


\begin{resultbox}
    \[
    \mathbf{E = \left\{ z \in \mathbb{C} \mid \operatorname{Re}(z) = 0 \text{ et } |z| < 1 \right\}}
    \]
\end{resultbox}

ou, en coordonnées cartésiennes :

\begin{resultbox}
    \[
    \mathbf{E = \left\{ (x, y) \in \mathbb{R}^2 \mid x = 0 \text{ et } x^2 + y^2 < 1 \right\}}
    \]
\end{resultbox}
Cela représente l'intersection de la droite \( x = 0 \) avec l'intérieur du disque de centre \( O(0,0) \) et de rayon \( 1 \), excluant le bord.

\begin{center}
\begin{tikzpicture}
    \begin{axis}[
        axis lines=middle,
        xlabel={$x$}, ylabel={$y$},
        xmin=-1.2, xmax=1.2, ymin=-1.2, ymax=1.2,
        width=8cm, height=8cm,
        xtick=\empty, ytick=\empty,
        samples=100
    ]
        % Disque (bord en pointillés)
        \addplot [domain=0:360, dashed] ({cos(x)}, {sin(x)});
        
        % Segment vertical x=0 (en rouge)
        \addplot [red, thick] coordinates {(0,-1) (0,1)};
        
        % Labels
        \node at (axis cs: 1.1,0) {\small Bord du disque (exclu)};
        \node[red] at (axis cs: 0,0.5) {\small Ensemble \(E\)};
    \end{axis}
\end{tikzpicture}
\end{center}
        \item $U \in i\mathbb{R}$

    Pour que \( U \) soit imaginaire pur, il faut que \( \operatorname{Re}(U) = 0 \).

\begin{resultbox}
    \[
    \mathbf{\operatorname{Re}(U) = \frac{x^2 + y^2 - 1}{x^2 + (y - 1)^2} = 0}
    \]
\end{resultbox}

    Ce qui donne :

    \[
    x^2 + y^2 - 1 = 0
    \]

    Ainsi, l'ensemble des points \( M \) d'affixe \( z \) est donné par :
    \[ E' = \left\{ z \in \mathbb{C} \mid |z| = 1 \right\} \]

\begin{resultbox}
    \[
    \mathbf{ E' = \left\{ z \in \mathbb{C} \mid |z| = 1 \right\}}
    \]
\end{resultbox}

    ou, en coordonnées cartésiennes :

    \[
    E' = \left\{ (x, y) \in \mathbb{R}^2 \mid x^2 + y^2 = 1 \right\}
    \]
\begin{resultbox}
    \[
    \mathbf{ E' = \left\{ (x, y) \in \mathbb{R}^2 \mid x^2 + y^2 = 1 \right\}}
    \]
\end{resultbox}
    Cela représente le **cercle de centre \( O(0,0) \) et de rayon \( 1 \)**.

\end{enumerate}

\begin{center}
\begin{tikzpicture}
    \begin{axis}[
        axis lines=middle,
        xlabel={$x$}, ylabel={$y$},
        xmin=-1.2, xmax=1.2, ymin=-1.2, ymax=1.2,
        width=8cm, height=8cm,
        xtick=\empty, ytick=\empty,
        samples=100
    ]
        % Cercle (solide car il appartient à E')
        \addplot [domain=0:360, blue, thick] ({cos(x)}, {sin(x)});
        
        % Labels
        \node[blue] at (axis cs: 0.8, 0.6) {\small Ensemble \(E'\)};
    \end{axis}
\end{tikzpicture}
\end{center}

\end{enumerate}

\end{document}
