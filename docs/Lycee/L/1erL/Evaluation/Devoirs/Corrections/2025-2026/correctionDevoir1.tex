\documentclass[12pt,a4paper]{article}
\usepackage{amsmath,amssymb,mathrsfs,tikz,times,pifont}
\usepackage{enumitem}
\newcommand\circitem[1]{%
\tikz[baseline=(char.base)]{
\node[circle,draw=gray, fill=red!55,
minimum size=1.2em,inner sep=0] (char) {#1};}}
\newcommand\boxitem[1]{%
\tikz[baseline=(char.base)]{
\node[fill=cyan,
minimum size=1.2em,inner sep=0] (char) {#1};}}
\setlist[enumerate,1]{label=\protect\circitem{\arabic*}}
\setlist[enumerate,2]{label=\protect\boxitem{\alph*}}
%%%::::::by chnini ameur :::::::%%%
\everymath{\displaystyle}
\usepackage[left=1cm,right=1cm,top=1cm,bottom=1.7cm]{geometry}
\usepackage{array,multirow}
\usepackage[most]{tcolorbox}
\usepackage{varwidth}
\tcbuselibrary{skins,hooks}
\usetikzlibrary{patterns}
%%%::::::by chnini ameur :::::::%%%
\newtcolorbox{exa}[2][]{enhanced,breakable,before skip=2mm,after skip=5mm,
colback=yellow!20!white,colframe=black!20!blue,boxrule=0.5mm,
attach boxed title to top left ={xshift=0.6cm,yshift*=1mm-\tcboxedtitleheight},
fonttitle=\bfseries,
title={#2},#1,
% varwidth boxed title*=-3cm,
boxed title style={frame code={
\path[fill=tcbcolback!30!black]
([yshift=-1mm,xshift=-1mm]frame.north west)
arc[start angle=0,end angle=180,radius=1mm]
([yshift=-1mm,xshift=1mm]frame.north east)
arc[start angle=180,end angle=0,radius=1mm];
\path[left color=tcbcolback!60!black,right color = tcbcolback!60!black,
middle color = tcbcolback!80!black]
([xshift=-2mm]frame.north west) -- ([xshift=2mm]frame.north east)
[rounded corners=1mm]-- ([xshift=1mm,yshift=-1mm]frame.north east)
-- (frame.south east) -- (frame.south west)
-- ([xshift=-1mm,yshift=-1mm]frame.north west)
[sharp corners]-- cycle;
},interior engine=empty,
},interior style={top color=yellow!5}}
%%%%%%%%%%%%%%%%%%%%%%%

\usepackage{fancyhdr}
\usepackage{eso-pic}         % Pour ajouter des éléments en arrière-plan
% Commande pour ajouter du texte en arrière-plan
\AddToShipoutPicture{
    \AtTextCenter{%
        \makebox[0pt]{\rotatebox{80}{\textcolor[gray]{0.7}{\fontsize{5cm}{5cm}\selectfont PGB}}}
    }
}
\usepackage{lastpage}
\fancyhf{}
\pagestyle{fancy}
\renewcommand{\footrulewidth}{1pt}
\renewcommand{\headrulewidth}{0pt}
\renewcommand{\footruleskip}{10pt}
\fancyfoot[R]{
\color{blue}\ding{45}\ \textbf{2025}
}
\fancyfoot[L]{
\color{blue}\ding{45}\ \textbf{Prof:M. BA}
}
\cfoot{\bf
\thepage /
\pageref{LastPage}}
\begin{document}
\renewcommand{\arraystretch}{1.5}
\renewcommand{\arrayrulewidth}{1.2pt}
\begin{tikzpicture}[overlay,remember picture]
\node[draw=blue,line width=1.2pt,fill=purple,text=blue,inner sep=3mm,rounded corners,pattern=dots]at ([yshift=-2.5cm]current page.north) {\begingroup\setlength{\fboxsep}{0pt}\colorbox{white}{\begin{tabular}{|*1{>{\centering \arraybackslash}p{0.28\textwidth}} |*2{>{\centering \arraybackslash}p{0.2\textwidth}|} *1{>{\centering \arraybackslash}p{0.19\textwidth}|} }
\hline
\multicolumn{3}{|c|}{$\diamond$$\diamond$$\diamond$\ \textbf{Lycée de Dindéfélo}\ $\diamond$$\diamond$$\diamond$ }& \textbf{A.S. : 2025/2026} \\ \hline
\textbf{Matière: Mathématiques}& \textbf{Niveau : 1er}\textbf{L} &\textbf{Date: 26/11/2025} & \textbf{Durée : 2 heures} \\ \hline
\multicolumn{4}{|c|}{\parbox[c]{10cm}{\begin{center}
\textbf{{\Large\sffamily Correction Devoir n$ ^{\circ} $ 1 Du 1$ ^\text{\bf ère} $ Semestre}}
\end{center}}} \\ \hline
\end{tabular}}\endgroup};
\end{tikzpicture}
\vspace{3cm}

\section*{\underline{Exercice 1 :} 9 pts (Résolution de systèmes triangulaires supérieurs)}
Donnons la solution dans chaque cas

\( 
\begin{aligned}
\begin{cases}
2x + y + z = 12\quad &L1 \\
3y - z = 6\quad &L2\\
z = 3\quad &L3
\end{cases} 
\end{aligned}
\)

\vspace{1cm}
\noindent
\[ (L3) : z = 3 \]

\[
\begin{aligned}
(L2) :\; 3y - z = 6 
&\implies 3y - 3 = 6\\
&\implies 3y = 9\\
&\implies y = \dfrac{9}{3}\\
&\implies y = 3
\end{aligned}
\]

\[
\begin{aligned}
(L1) :\; 2x + y + z = 12 
&\implies 2x + 3 + 3 = 12\\ 
&\implies 2x + 6 = 12\\
&\implies 2x = 6\\
&\implies x = \dfrac{6}{2}\\
&\implies x = 3
\end{aligned}
\]

\[
\boxed{\color{red}{S =\left\lbrace (3,\,3,\,3) \right\rbrace}}
\]

\( 
\begin{aligned}
\begin{cases}
2x + 3y - z = 5\quad  &L1\\
4y + 2z = 6\quad  &L2\\
3z - 9 = 0\quad  &L3
\end{cases} 
\end{aligned}
\)

\vspace{1cm}
\noindent
\[
\begin{aligned}
(L3) :\; 3z - 9 = 0 
&\implies 3z = 9\\
&\implies z = \dfrac{9}{3}\\
&\implies z = 3
\end{aligned}
\]

\[
\begin{aligned}
(L2) :\; 4y + 2z = 6 
&\implies 4y + 2\cdot 3 = 6\\
&\implies 4y + 6 = 6\\
&\implies 4y = 0\\
&\implies y = \dfrac{0}{4}\\
&\implies y = 0
\end{aligned}
\]

\[
\begin{aligned}
(L1) :\; 2x + 3y - z = 5 
&\implies 2x + 3\cdot 0 - 3 = 5\\
&\implies 2x - 3 = 5\\
&\implies 2x = 8\\
&\implies x = \dfrac{8}{2}\\
&\implies x = 4
\end{aligned}
\]

\[
\boxed{\color{red}{S =\left\lbrace (4,\,0,\,3) \right\rbrace}}
\]

\(
\begin{aligned}
\begin{cases}
3x + y + z = 14\\
\quad\quad\quad z = 2\\
2y - z \quad = 4
\end{cases}
\end{aligned}
\)

\vspace{1cm}
\noindent
\[ (L2) : z = 2 \]

\[
\begin{aligned}
(L3) :\; 2y - z = 4 
&\implies 2y - 2 = 4\\
&\implies 2y = 6\\
&\implies y = \dfrac{6}{2}\\
&\implies y = 3
\end{aligned}
\]

\[
\begin{aligned}
(L1) :\; 3x + y + z = 14 
&\implies 3x + 3 + 2 = 14\\
&\implies 3x + 5 = 14\\
&\implies 3x = 9\\
&\implies x = \dfrac{9}{3}\\
&\implies x = 3
\end{aligned}
\]

\[
\boxed{\color{red}{S = \left\lbrace (3,\,3,\,2) \right\rbrace}}
\]


\section*{\underline{Exercice 2 :} 8 pts (Résolution de systèmes par le pivot de Gauss)}
Résoudre les systèmes suivants en utilisant la méthode du pivot de Gauss

\(
\begin{aligned}
\begin{cases}
x + y + z = 9 &L1\\
2x - y + z = 5 &L2\\
x + 2y - z = 4 &L3
\end{cases}
\end{aligned}
\)

Considérons $L1$ comme pivot

\vspace{1cm}
\noindent

\(
\begin{aligned}
\begin{cases}
2\times( x + y + z = 9) &L1\\
\quad 2x - y + z = 5 &L2\\
\end{cases}\implies
\underline{
\begin{cases}
-2x - 2y - 2z = -18 &L1\\
\quad \quad 2x - y + z = 5 &L2
\end{cases}}\\
-3y-z=-13 \quad L'2
\end{aligned}
\)

\vspace{1cm}
\noindent
\(
\begin{aligned}
\begin{cases}
-(x + y + z = 9) &L1\\
x + 2y - z = 4 &L3
\end{cases}\implies
\underline{
\begin{cases}
-x - y - z = -9 &L1\\
\quad x + 2y - z = 4 &L3
\end{cases}}\\
y-2z=-5 \quad L'3
\end{aligned}
\)

\vspace{1cm}
\noindent
\(
\begin{aligned}
\begin{cases}
-3y-z=-13 \quad &L'2\\
3(y-2z=-5) \quad &L'3
\end{cases}\implies
\underline{
\begin{cases}
-3y-z=-13 \quad &L'2\\
\quad 3y-6z=-15 \quad &L'3
\end{cases}}\\
-7z=-28 \quad L''3
\end{aligned}
\)

\vspace{1cm}
\noindent

Le système devient

\(
\begin{aligned}
\begin{cases}
x + y + z = 9 &L1\\
\quad -3y-z=-13  &L'2\\
\quad \quad -7z =-28 &L''3
\end{cases}
\end{aligned}
\)

\[
\begin{aligned}
\begin{cases}
x + y + z = 9 \quad &L1\\
-3y - z = -13 \quad &L'2\\
-7z = -28 \quad &L''3
\end{cases}
\end{aligned}
\]

\noindent
\[
\begin{aligned}
(L'2) :\; -7z = -28
&\implies z = \dfrac{-28}{-7}\\
&\implies z = 4
\end{aligned}
\]

\[
\begin{aligned}
(L'2) :\; -3y - z = -13 
&\implies -3y - 4 = -13\\
&\implies -3y = -9\\
&\implies y = \dfrac{-9}{-3}\\
&\implies y = 3
\end{aligned}
\]

\[
\begin{aligned}
(L1) :\; x + y + z = 9 
&\implies x + 3 + 4 = 9\\
&\implies x + 7 = 9\\
&\implies x = 9 - 7\\
&\implies x = 2
\end{aligned}
\]

\[
\boxed{\textcolor{red}{S = \left\lbrace (2,\,3,\,4) \right\rbrace}}
\]


\(
\begin{aligned}
\begin{cases}
x + 2y + z = 10 &L1\\
2x - y + 3z = 13 &L2\\
x + y - z = 2 &L3
\end{cases}
\end{aligned}
\)

Considérons $L1$ comme pivot

\vspace{0.5cm}
\noindent
\(
\begin{aligned}
\begin{cases}
2\times(x + 2y + z = 10) &L1\\
2x - y + 3z = 13 &L2
\end{cases} \implies
\underline{
\begin{cases}
2x + 4y + 2z = 20 &L1\\
2x - y + 3z = 13 &L2
\end{cases}}\\
\end{aligned}
\)

\vspace{0.5cm}
\noindent
Soustrayons L2 de L1 pour éliminer x :

\(
\begin{aligned}
(2x + 4y + 2z) - (2x - y + 3z) &= 20 - 13\\
5y - z &= 7 \quad L'2
\end{aligned}
\)

\vspace{0.5cm}
\noindent
Pour L3, soustrayons L1 de L3 pour éliminer x :

\(
\begin{aligned}
(x + y - z) - (x + 2y + z) &= 2 - 10\\
-y - 2z &= -8\\
y + 2z &= 8 \quad L'3
\end{aligned}
\)

\vspace{0.5cm}
\noindent
On obtient le système triangulaire :

\(
\begin{aligned}
\begin{cases}
5y - z = 7 &L'2\\
y + 2z = 8 &L'3
\end{cases}
\end{aligned}
\)

\vspace{0.5cm}
\noindent
Résolvons L'3 pour y en fonction de z :

\[
L'3 : y + 2z = 8 \implies y = 8 - 2z
\]

Substituons dans L'2 :

\[
5(8 - 2z) - z = 7 \implies 40 - 10z - z = 7 \implies -11z = -33 \implies z = 3
\]

\[
y = 8 - 2\cdot 3 = 8 - 6 = 2
\]

\[
L1 : x + 2y + z = 10 \implies x + 2\cdot 2 + 3 = 10 \implies x + 7 = 10 \implies x = 3
\]

\[
\boxed{\textcolor{red}{S = \left\lbrace (3,\,2,\,3) \right\rbrace}}
\]


\section*{\underline{Exercice 3 :} 3 pts ( Union et Intersection d'Intervalles )}  

\begin{enumerate}  
\item On considère \( I = [2, 5] \) et \( J = [4, 7] \). Déterminons \( I \cup J \) et \( I \cap J \).  

\( I \cup J = [2,7]\) et \( I \cap J = [4,5] \)

\item On considère \( K = [2, 5] \) et \( L = [6, 7] \). Déterminons \( K \cap L \). 

\( K \cap L = \emptyset \) 
\end{enumerate}

\end{document}
