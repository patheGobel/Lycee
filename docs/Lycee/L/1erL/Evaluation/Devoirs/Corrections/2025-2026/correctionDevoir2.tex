\documentclass[12pt,a4paper]{article}
\usepackage{amsmath,amssymb,mathrsfs,tikz,times,pifont}
\usepackage{enumitem}
\newcommand\circitem[1]{%
\tikz[baseline=(char.base)]{
\node[circle,draw=gray, fill=red!55,
minimum size=1.2em,inner sep=0] (char) {#1};}}
\newcommand\boxitem[1]{%
\tikz[baseline=(char.base)]{
\node[fill=cyan,
minimum size=1.2em,inner sep=0] (char) {#1};}}
\setlist[enumerate,1]{label=\protect\circitem{\arabic*}}
\setlist[enumerate,2]{label=\protect\boxitem{\alph*}}
%%%::::::by chnini ameur :::::::%%%
\everymath{\displaystyle}
\usepackage[left=1cm,right=1cm,top=1cm,bottom=1.7cm]{geometry}
\usepackage{array,multirow}
\usepackage[most]{tcolorbox}
\usepackage{varwidth}
\tcbuselibrary{skins,hooks}
\usetikzlibrary{patterns}
%%%::::::by chnini ameur :::::::%%%
\newtcolorbox{exa}[2][]{enhanced,breakable,before skip=2mm,after skip=5mm,
colback=yellow!20!white,colframe=black!20!blue,boxrule=0.5mm,
attach boxed title to top left ={xshift=0.6cm,yshift*=1mm-\tcboxedtitleheight},
fonttitle=\bfseries,
title={#2},#1,
% varwidth boxed title*=-3cm,
boxed title style={frame code={
\path[fill=tcbcolback!30!black]
([yshift=-1mm,xshift=-1mm]frame.north west)
arc[start angle=0,end angle=180,radius=1mm]
([yshift=-1mm,xshift=1mm]frame.north east)
arc[start angle=180,end angle=0,radius=1mm];
\path[left color=tcbcolback!60!black,right color = tcbcolback!60!black,
middle color = tcbcolback!80!black]
([xshift=-2mm]frame.north west) -- ([xshift=2mm]frame.north east)
[rounded corners=1mm]-- ([xshift=1mm,yshift=-1mm]frame.north east)
-- (frame.south east) -- (frame.south west)
-- ([xshift=-1mm,yshift=-1mm]frame.north west)
[sharp corners]-- cycle;
},interior engine=empty,
},interior style={top color=yellow!5}}
%%%%%%%%%%%%%%%%%%%%%%%

\usepackage{fancyhdr}
\usepackage{eso-pic}         % Pour ajouter des éléments en arrière-plan
% Commande pour ajouter du texte en arrière-plan
\AddToShipoutPicture{
    \AtTextCenter{%
        \makebox[0pt]{\rotatebox{80}{\textcolor[gray]{0.7}{\fontsize{5cm}{5cm}\selectfont PGB}}}
    }
}
\usepackage{lastpage}
\fancyhf{}
\pagestyle{fancy}
\renewcommand{\footrulewidth}{1pt}
\renewcommand{\headrulewidth}{0pt}
\renewcommand{\footruleskip}{10pt}
\fancyfoot[R]{
\color{blue}\ding{45}\ \textbf{2025}
}
\fancyfoot[L]{
\color{blue}\ding{45}\ \textbf{Prof:M. BA}
}
\cfoot{\bf
\thepage /
\pageref{LastPage}}
\begin{document}
\renewcommand{\arraystretch}{1.5}
\renewcommand{\arrayrulewidth}{1.2pt}
\begin{tikzpicture}[overlay,remember picture]
\node[draw=blue,line width=1.2pt,fill=purple,text=blue,inner sep=3mm,rounded corners,pattern=dots]at ([yshift=-2.5cm]current page.north) {\begingroup\setlength{\fboxsep}{0pt}\colorbox{white}{\begin{tabular}{|*1{>{\centering \arraybackslash}p{0.28\textwidth}} |*2{>{\centering \arraybackslash}p{0.2\textwidth}|} *1{>{\centering \arraybackslash}p{0.19\textwidth}|} }
\hline
\multicolumn{3}{|c|}{$\diamond$$\diamond$$\diamond$\ \textbf{Lycée de Dindéfélo}\ $\diamond$$\diamond$$\diamond$ }& \textbf{A.S. : 2025/2026} \\ \hline
\textbf{Matière: Mathématiques}& \textbf{Niveau : 1er}\textbf{L} &\textbf{Date: 20/12/2025} & \textbf{Durée : 2 heures} \\ \hline
\multicolumn{4}{|c|}{\parbox[c]{10cm}{\begin{center}
\textbf{{\Large\sffamily Correction Devoir n$ ^{\circ} $ 2 Du 1$ ^\text{\bf ère} $ Semestre}}
\end{center}}} \\ \hline
\end{tabular}}\endgroup};
\end{tikzpicture}
\vspace{3cm}

\section*{\underline{Exercice 1 :} 5 pts (Résolution de systèmes)}

\[
\begin{aligned}
&\text{Système 1 : } 
\begin{cases}
3x + y + z = 6 \\
2y = 4 \\
z = 1
\end{cases}
\quad\Rightarrow\quad
(x,y,z) = (1,\,2,\,1)
\\[0.5cm]
&\text{Système 2 : } 
\begin{cases}
x + 2y - z = 2 \\
3y + 2z = 8 \\
3y + z = 7
\end{cases}
\quad\Rightarrow\quad
(x,y,z) = (-1,\,2,\,1)
\\[0.5cm]
&\text{Système 3 : } 
\begin{cases}
2x + 3y + z = 9 \\
4y + 2z = 10 \\
3z + 3 = 6
\end{cases}
\quad\Rightarrow\quad
(x,y,z) = (1,\,2,\,1)
\end{aligned}
\]


\section*{\underline{Exercice 5 :} Systèmes 3x3 à Solution Unique (Entière)}

\textbf{Système 1}
\[
\begin{cases}
x + y + z = 3 \\
2x - y + z = 0 \\
x + 2y - 2z = 5
\end{cases}
\qquad \Longrightarrow \qquad
\boxed{(x,y,z) = (1,\,2,\,0)}
\]

\vspace{0.4cm}

\textbf{Système 2}
\[
\begin{cases}
3x + 2y - z = 3 \\
x - 4y + 2z = 1 \\
2x + y + z = 5
\end{cases}
\qquad \Longrightarrow \qquad
\boxed{(x,y,z) = (1,\,1,\,2)}
\]

\vspace{0.4cm}

\textbf{Système 3}
\[
\begin{cases}
x - 2y + z = 0 \\
-x + y + 2z = 5 \\
2x + 3y + z = 5
\end{cases}
\qquad \Longrightarrow \qquad
\boxed{(x,y,z) = (0,\,1,\,2)}
\]

\section*{\underline{Exercice 3 :} 3 pts Racines de Polynômes}  

\noindent Un nombre $a$ est une racine d'un polynôme $P(x)$ si et seulement si l'évaluation $P(a)$ est nulle.

\begin{enumerate}

    \item \textbf{Question :} $2$ est-il une racine de $P(x)=3x^{3}-11x^{2}+17x-14$ ?
    \subitem \textbf{Calcul de $P(2)$ :}
    $$
    \begin{aligned}
    P(2) &= 3(2)^{3}-11(2)^{2}+17(2)-14 \\
    &= 3(8) - 11(4) + 34 - 14 \\
    &= 24 - 44 + 34 - 14 \\
    &= (24 + 34) - (44 + 14) \\
    &= 58 - 58 \\
    &= 0
    \end{aligned}
    $$
    \subitem \textbf{Conclusion :} Puisque $P(2) = 0$, $\mathbf{2 \text{ est une racine}}$.
    
    \vspace{0.5em} % Ajout d'espace vertical

    \item \textbf{Question :} $1$ est-il une racine de $P(x) = x^{4} - 2x^{3} + x^{2} - x + 5$ ?
    \subitem \textbf{Calcul de $P(1)$ :}
    $$
    \begin{aligned}
    P(1) &= (1)^{4} - 2(1)^{3} + (1)^{2} - (1) + 5 \\
    &= 1 - 2(1) + 1 - 1 + 5 \\
    &= 1 - 2 + 1 - 1 + 5 \\
    &= (1 + 1 + 5) - (2 + 1) \\
    &= 7 - 3 \\
    &= 4
    \end{aligned}
    $$
    \subitem \textbf{Conclusion :} Puisque $P(1) \neq 0$, $\mathbf{1 \text{ n'est pas une racine}}$.

    \vspace{0.5em}

    \item \textbf{Question :} $-3$ est-il une racine de $Q(x) = x^{3} + 2x^{2} - 5x - 6$ ?
    \subitem \textbf{Calcul de $Q(-3)$ :}
    $$
    \begin{aligned}
    Q(-3) &= (-3)^{3} + 2(-3)^{2} - 5(-3) - 6 \\
    &= -27 + 2(9) + 15 - 6 \\
    &= -27 + 18 + 15 - 6 \\
    &= (18 + 15) - (27 + 6) \\
    &= 33 - 33 \\
    &= 0
    \end{aligned}
    $$
    \subitem \textbf{Conclusion :} Puisque $Q(-3) = 0$, $\mathbf{-3 \text{ est une racine}}$.

    \vspace{0.5em}

    \item \textbf{Question :} $\frac{1}{2}$ est-il une racine de $R(x) = 2x^{2} + 3x - 2$ ?
    \subitem \textbf{Calcul de $R(\frac{1}{2})$ :}
    $$
    \begin{aligned}
    R\left(\frac{1}{2}\right) &= 2\left(\frac{1}{2}\right)^{2} + 3\left(\frac{1}{2}\right) - 2 \\
    &= 2\left(\frac{1}{4}\right) + \frac{3}{2} - 2 \\
    &= \frac{1}{2} + \frac{3}{2} - 2 \\
    &= \frac{4}{2} - 2 \\
    &= 2 - 2 \\
    &= 0
    \end{aligned}
    $$
    \subitem \textbf{Conclusion :} Puisque $R(\frac{1}{2}) = 0$, $\mathbf{\frac{1}{2} \text{ est une racine}}$.

    \vspace{0.5em}

    \item \textbf{Question :} $3$ est-il une racine de $S(x) = x^{3} - 4x^{2} + 2x + 1$ ?
    \subitem \textbf{Calcul de $S(3)$ :}
    $$
    \begin{aligned}
    S(3) &= (3)^{3} - 4(3)^{2} + 2(3) + 1 \\
    &= 27 - 4(9) + 6 + 1 \\
    &= 27 - 36 + 6 + 1 \\
    &= (27 + 6 + 1) - 36 \\
    &= 34 - 36 \\
    &= -2
    \end{aligned}
    $$
    \subitem \textbf{Conclusion :} Puisque $S(3) \neq 0$, $\mathbf{3 \text{ n'est pas une racine}}$.

    \vspace{0.5em}
    
    \item \textbf{Question :} $0$ est-il une racine de $T(x) = x^{5} - 3x^{2} + 4x$ ?
    \subitem \textbf{Calcul de $T(0)$ :}
    $$
    \begin{aligned}
    T(0) &= (0)^{5} - 3(0)^{2} + 4(0) \\
    &= 0 - 0 + 0 \\
    &= 0
    \end{aligned}
    $$
    \subitem \textbf{Conclusion :} Puisque $T(0) = 0$, $\mathbf{0 \text{ est une racine}}$.

\end{enumerate}

\end{document}
