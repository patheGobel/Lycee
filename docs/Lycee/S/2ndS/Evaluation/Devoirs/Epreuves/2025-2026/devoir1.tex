\documentclass[12pt,a4paper]{article}
\usepackage{amsmath,amssymb,mathrsfs,tikz,times,pifont}
\usepackage{enumitem}
\newcommand\circitem[1]{%
\tikz[baseline=(char.base)]{
\node[circle,draw=gray, fill=red!55,
minimum size=1.2em,inner sep=0] (char) {#1};}}
\newcommand\boxitem[1]{%
\tikz[baseline=(char.base)]{
\node[fill=cyan,
minimum size=1.2em,inner sep=0] (char) {#1};}}
\setlist[enumerate,1]{label=\protect\circitem{\arabic*}}
\setlist[enumerate,2]{label=\protect\boxitem{\alph*}}
%%%::::::by chnini ameur :::::::%%%
\everymath{\displaystyle}
\usepackage[left=1cm,right=1cm,top=1cm,bottom=1.7cm]{geometry}
\usepackage{array,multirow}
\usepackage[most]{tcolorbox}
\usepackage{varwidth}
\tcbuselibrary{skins,hooks}
\usetikzlibrary{patterns}
%%%::::::by chnini ameur :::::::%%%
\newtcolorbox{exa}[2][]{enhanced,breakable,before skip=2mm,after skip=5mm,
colback=yellow!20!white,colframe=black!20!blue,boxrule=0.5mm,
attach boxed title to top left ={xshift=0.6cm,yshift*=1mm-\tcboxedtitleheight},
fonttitle=\bfseries,
title={#2},#1,
% varwidth boxed title*=-3cm,
boxed title style={frame code={
\path[fill=tcbcolback!30!black]
([yshift=-1mm,xshift=-1mm]frame.north west)
arc[start angle=0,end angle=180,radius=1mm]
([yshift=-1mm,xshift=1mm]frame.north east)
arc[start angle=180,end angle=0,radius=1mm];
\path[left color=tcbcolback!60!black,right color = tcbcolback!60!black,
middle color = tcbcolback!80!black]
([xshift=-2mm]frame.north west) -- ([xshift=2mm]frame.north east)
[rounded corners=1mm]-- ([xshift=1mm,yshift=-1mm]frame.north east)
-- (frame.south east) -- (frame.south west)
-- ([xshift=-1mm,yshift=-1mm]frame.north west)
[sharp corners]-- cycle;
},interior engine=empty,
},interior style={top color=yellow!5}}
%%%%%%%%%%%%%%%%%%%%%%%

\usepackage{fancyhdr}
\usepackage{eso-pic}         % Pour ajouter des éléments en arrière-plan
% Commande pour ajouter du texte en arrière-plan
\AddToShipoutPicture{
    \AtTextCenter{%
        \makebox[0pt]{\rotatebox{80}{\textcolor[gray]{0.7}{\fontsize{5cm}{5cm}\selectfont PGB}}}
    }
}
\usepackage{lastpage}
\fancyhf{}
\pagestyle{fancy}
\renewcommand{\footrulewidth}{1pt}
\renewcommand{\headrulewidth}{0pt}
\renewcommand{\footruleskip}{10pt}
\fancyfoot[R]{
\color{blue}\ding{45}\ \textbf{2025}
}
\fancyfoot[L]{
\color{blue}\ding{45}\ \textbf{Prof:M. BA}
}
\cfoot{\bf
\thepage /
\pageref{LastPage}}
\begin{document}
\renewcommand{\arraystretch}{1.5}
\renewcommand{\arrayrulewidth}{1.2pt}
\begin{tikzpicture}[overlay,remember picture]
\node[draw=blue,line width=1.2pt,fill=purple,text=blue,inner sep=3mm,rounded corners,pattern=dots]at ([yshift=-2.5cm]current page.north) {\begingroup\setlength{\fboxsep}{0pt}\colorbox{white}{\begin{tabular}{|*1{>{\centering \arraybackslash}p{0.28\textwidth}} |*2{>{\centering \arraybackslash}p{0.2\textwidth}|} *1{>{\centering \arraybackslash}p{0.19\textwidth}|} }
\hline
\multicolumn{3}{|c|}{$\diamond$$\diamond$$\diamond$\ \textbf{Lycée de Dindéfélo}\ $\diamond$$\diamond$$\diamond$ }& \textbf{A.S. : 2025/2026} \\ \hline
\textbf{Matière: Mathématiques}& \textbf{Niveau : 2nd}\textbf{S} &\textbf{Date: 27/11/2025} & \textbf{Durée : 4 heures} \\ \hline
\multicolumn{4}{|c|}{\parbox[c]{10cm}{\begin{center}
\textbf{{\Large\sffamily Devoir n$ ^{\circ} $ 1 Du 1$ ^\text{\bf ère} $ Semestre}}
\end{center}}} \\ \hline
\end{tabular}}\endgroup};
\end{tikzpicture}
\vspace{3cm}

\section*{\underline{Exercice 1 :} 5 pts (Factoriser les expressions suivantes :) }
\begin{enumerate}
    \item \( a^2xy + aby^2 + b^2xy + abx^2 \)
    \item \( 3a^2 + 3b^2 - 4c^2 - 6ab \)
    \item \( y^2 - x^2 + 2x - 1 \)
    \item \( a^2b^2 - 1 + a^2 - b^2 \)
    \item \( (ab - 1)^2 - (a - b)^2 \)
\end{enumerate}
\section*{\underline{Exercice 2 :} 3 pts }
\begin{enumerate}
    \item Développer $(a + b + c)^2$.
    \item Montrer que si $a + b + c = 0$ alors $a^2 + b^2 + c^2 = -2 (ab + bc + ca)$.
    \item On suppose $a$, $b$ et $c$ sont non nuls. \\
    Montrer que $\frac{1}{a} + \frac{1}{b} + \frac{1}{c} = 0 \implies (a + b + c)^2 = a^2 + b^2 + c^2$.
\end{enumerate}
\section*{\underline{Exercice 3 :} 4 pts } Soit $a, b, c$ trois réels :
\begin{enumerate}
    \item Développer $(a + b + c)(ab + bc + ca)$ puis $(a + b + c)^3$
    \item Démontrer que si $a + b + c = 0$ alors $a^3 + b^3 + c^3 = 3abc$
    \item En déduire que, pour tous réel $x, y, z$ on a :
    $$(x + y)^3 + (y + z)^3 + (z + x)^3 = 3 (x + y) (y + z) (z + x)$$
\end{enumerate}
\section*{\underline{Exercice 2 :} 8 pts }
\begin{enumerate}
\item Simplifier les expressions suivantes (on suppose que tous les dénominateurs sont non nuls).
\[ A = \frac{\frac{x+y}{1-xy} - \frac{x-y}{1+xy}}{1 - \frac{x^2-y^2}{1-x^2y^2}} \quad ; \quad B=\frac{\frac{1}{a} - \frac{1}{b}}{\frac{1}{a} + \frac{1}{b}} \div \frac{a^2-b^2}{(a+b)^2} \quad ; \quad C=\frac{\frac{1}{a} - \frac{1}{b+c}}{\frac{1}{a} + \frac{1}{b+c}} \times \frac{\frac{1}{b} + \frac{1}{a+c}}{\frac{1}{b} - \frac{1}{a+c}} \quad ; \quad D=\frac{\frac{1}{a} + \frac{1}{b+c}}{\frac{1}{a} - \frac{1}{b+c}} \div \frac{a+b+c}{a-b-c}\]

\item Écrire sous la forme $2^m \times 3^n \times 5^p$ (avec $m, n, p$ des entiers relatifs) les réels suivants :
  \[
  A = \dfrac{(0{,}009)^{-3} \times (0{,}016)^2 \times 250}{(0{,}00075)^{-1} \times 810^3 \times 30}
  \quad ; \quad
  B = \dfrac{(-6)^4 \times 30^{-2} \times (-10)^{-3} \times 15^4}{(-25)^2 \times (36)^{-5} \times (-12)^3}
  \]
\end{enumerate}
\end{document}
