\documentclass[12pt,a4paper]{article}
\usepackage{amsmath,amssymb,mathrsfs,tikz,times,pifont}
\usepackage{enumitem}
\newcommand\circitem[1]{%
\tikz[baseline=(char.base)]{
\node[circle,draw=gray, fill=red!55,
minimum size=1.2em,inner sep=0] (char) {#1};}}
\newcommand\boxitem[1]{%
\tikz[baseline=(char.base)]{
\node[fill=cyan,
minimum size=1.2em,inner sep=0] (char) {#1};}}
\setlist[enumerate,1]{label=\protect\circitem{\arabic*}}
\setlist[enumerate,2]{label=\protect\boxitem{\alph*}}
%%%::::::by chnini ameur :::::::%%%
\everymath{\displaystyle}
\usepackage[left=1cm,right=1cm,top=1cm,bottom=1.7cm]{geometry}
\usepackage{array,multirow}
\usepackage[most]{tcolorbox}
\usepackage{varwidth}
\tcbuselibrary{skins,hooks}
\usetikzlibrary{patterns}
%%%::::::by chnini ameur :::::::%%%
\newtcolorbox{exa}[2][]{enhanced,breakable,before skip=2mm,after skip=5mm,
colback=yellow!20!white,colframe=black!20!blue,boxrule=0.5mm,
attach boxed title to top left ={xshift=0.6cm,yshift*=1mm-\tcboxedtitleheight},
fonttitle=\bfseries,
title={#2},#1,
% varwidth boxed title*=-3cm,
boxed title style={frame code={
\path[fill=tcbcolback!30!black]
([yshift=-1mm,xshift=-1mm]frame.north west)
arc[start angle=0,end angle=180,radius=1mm]
([yshift=-1mm,xshift=1mm]frame.north east)
arc[start angle=180,end angle=0,radius=1mm];
\path[left color=tcbcolback!60!black,right color = tcbcolback!60!black,
middle color = tcbcolback!80!black]
([xshift=-2mm]frame.north west) -- ([xshift=2mm]frame.north east)
[rounded corners=1mm]-- ([xshift=1mm,yshift=-1mm]frame.north east)
-- (frame.south east) -- (frame.south west)
-- ([xshift=-1mm,yshift=-1mm]frame.north west)
[sharp corners]-- cycle;
},interior engine=empty,
},interior style={top color=yellow!5}}
%%%%%%%%%%%%%%%%%%%%%%%

\usepackage{fancyhdr}
\usepackage{eso-pic}         % Pour ajouter des éléments en arrière-plan
% Commande pour ajouter du texte en arrière-plan
\AddToShipoutPicture{
    \AtTextCenter{%
        \makebox[0pt]{\rotatebox{80}{\textcolor[gray]{0.7}{\fontsize{5cm}{5cm}\selectfont PGB}}}
    }
}
\usepackage{lastpage}
\fancyhf{}
\pagestyle{fancy}
\renewcommand{\footrulewidth}{1pt}
\renewcommand{\headrulewidth}{0pt}
\renewcommand{\footruleskip}{10pt}
\fancyfoot[R]{
\color{blue}\ding{45}\ \textbf{2025}
}
\fancyfoot[L]{
\color{blue}\ding{45}\ \textbf{Prof:M. BA}
}
\cfoot{\bf
\thepage /
\pageref{LastPage}}

% Définition de l'encadré adaptatif avec fond jaune
\newtcolorbox{resultbox}{
    colback=red!30, % Fond rouge clair
    colframe=black, % Bordure noire fine
    sharp corners, % Coins nets
    boxrule=0.5pt, % Contour léger
    boxsep=2pt, % Espacement interne
    left=5pt, right=5pt, top=2pt, bottom=2pt, % Marges internes
}

\begin{document}
\renewcommand{\arraystretch}{1.5}
\renewcommand{\arrayrulewidth}{1.2pt}
\begin{tikzpicture}[overlay,remember picture]
\node[draw=blue,line width=1.2pt,fill=purple,text=blue,inner sep=3mm,rounded corners,pattern=dots]at ([yshift=-2.5cm]current page.north) {\begingroup\setlength{\fboxsep}{0pt}\colorbox{white}{\begin{tabular}{|*1{>{\centering \arraybackslash}p{0.28\textwidth}} |*2{>{\centering \arraybackslash}p{0.2\textwidth}|} *1{>{\centering \arraybackslash}p{0.19\textwidth}|} }
\hline
\multicolumn{3}{|c|}{$\diamond$$\diamond$$\diamond$\ \textbf{Lycée de Dindéfélo}\ $\diamond$$\diamond$$\diamond$ }& \textbf{A.S. : 2025/2026} \\ \hline
\textbf{Matière: Mathématiques}& \textbf{Niveau : 2nd}\textbf{S} &\textbf{Date: 20/12/2025} & \textbf{Durée : 4 heures} \\ \hline
\multicolumn{4}{|c|}{\parbox[c]{10cm}{\begin{center}
\textbf{{\Large\sffamily Devoir n$ ^{\circ} $ 2 Du 1$ ^\text{\bf ère} $ Semestre}}
\end{center}}} \\ \hline
\end{tabular}}\endgroup};
\end{tikzpicture}
\vspace{3cm}

\section*{\underline{Exercice 1 :} (4,5 points) }

Soient $x>0$ et $y>0$.

\begin{enumerate}
    \item Démontrons que :
    \(
    \frac{1}{x^2+y^2} \leq \frac{1}{2xy}.
    \)

    Comme $(x-y)^2 \geq 0$, on a :
    \(
    x^2 - 2xy + y^2 \geq 0
    \quad \Longrightarrow \quad
    x^2 + y^2 \geq 2xy.
    \)
    
    Les deux membres étant strictement positifs, on peut prendre les inverses :
    \[
    \frac{1}{x^2+y^2} \leq \frac{1}{2xy}.
    \]

    \item
    \begin{enumerate}
        \item Déduisons que, pour tous $x,y \in \mathbb{R}_+^*$,
        \[
        \frac{x+y}{x^2+y^2} \leq \frac{1}{2}\left(\frac{1}{x}+\frac{1}{y}\right).
        \]

        En multipliant l’inégalité précédente par $x+y>0$, on obtient :
        \[
        \begin{aligned}
                \frac{x+y}{x^2+y^2} &\leq \frac{x+y}{2xy}\\
                                    &\leq \frac{1}{2}\left(\frac{x}{xy}+\frac{y}{xy}\right)\\
                                    &\leq \frac{1}{2}\left(\frac{1}{x}+\frac{1}{y}\right)
        \end{aligned}
        \]

        \item En utilisant des inégalités semblables, montrons que pour tous
        $x>0$, $y>0$ et $z>0$, on a :
        \[
        \frac{x+y}{x^2+y^2}
        + \frac{y+z}{y^2+z^2}
        + \frac{z+x}{z^2+x^2}
        \leq
        \frac{1}{x}+\frac{1}{y}+\frac{1}{z}.
        \]

        En effet, on a successivement :
        \[
        \frac{x+y}{x^2+y^2} \leq \frac{1}{2}\left(\frac{1}{x}+\frac{1}{y}\right),
        \]
        \[
        \frac{y+z}{y^2+z^2} \leq \frac{1}{2}\left(\frac{1}{y}+\frac{1}{z}\right),
        \]
        \[
        \frac{z+x}{z^2+x^2} \leq \frac{1}{2}\left(\frac{1}{z}+\frac{1}{x}\right).
        \]

        En additionnant membre à membre, on obtient :
        
\(
\begin{aligned}
\frac{x+y}{x^2+y^2}+ \frac{y+z}{y^2+z^2}+\frac{z+x}{z^2+x^2} &\leq \frac{1}{2}\left[\left(\frac{1}{x}+\frac{1}{y}\right) + \left(\frac{1}{y}+\frac{1}{z}\right) + \left(\frac{1}{z}+\frac{1}{x}\right)\right]\\
&\leq \frac{1}{2}\left(2\frac{1}{x}+2\frac{1}{y}+2\frac{1}{z}\right)\\
&\leq \frac{1}{x}+\frac{1}{y}+\frac{1}{z}
\end{aligned}
\)

    \end{enumerate}
\end{enumerate}  

\section*{\underline{Exercice 2 :}  (4,5 points) }

\begin{enumerate}
\item \textbf{Encadrement du carré  $x-y$}. \textbf{ (1 point)}

Encadrer $x-y$ dans les cas suivants :
\begin{enumerate}
\item $-2 \leq x \leq -1$ et $2 \leq y \leq 3$.
\item En déduire l'amplitude  de l'encadrement de $x-y$
\end{enumerate}
\item \textbf{Encadrement du carré  $xy$}.\textbf{ (1 point)}

Encadrer $xy$ dans les cas suivants :

\begin{enumerate}
    \item $-10 \leq x \leq -7$ et $1 \leq y \leq 2$.
    \item $-2 \leq x \leq 5$ et $2 \leq y \leq 7$.
\end{enumerate}

\item \textbf{Encadrement du carré $x^2$}. \textbf{ (1 point)}

Encadrer $x^2$ dans les cas suivants :

\begin{enumerate}
    \item $-7 \leq x \leq -3$
    \item $-2 \leq x \leq 3$
\end{enumerate}

\item \textbf{Encadrer $\frac{x}{y}$ dans les cas suivants}.\textbf{ (1,5 point)}

\begin{enumerate}
    \item $1 \leq x \leq 2$ et $3 \leq y \leq 7$
    \item $-1 \leq x \leq -3$ et $-7 \leq y \leq 2$
    \item $-5 \leq x \leq -3$ et $3 \leq y \leq -1$
\end{enumerate}
\end{enumerate}   

\section*{\underline{Exercice 3 :}  (2,75 points) } 

\begin{enumerate}
%=========================================
\item On donne :
\[
A=\frac{1}{1+\frac{a}{b+c}},\quad
B=\frac{1}{1+\frac{b}{c+a}},\quad
C=\frac{1}{1+\frac{c}{a+b}}.
\]

\textbf{Vérifions que } $A+B+C=2$.

\[
A=\frac{1}{\frac{a+b+c}{b+c}}=\frac{b+c}{a+b+c},
\quad
B=\frac{1}{\frac{a+b+c}{a+c}}=\frac{a+c}{a+b+c},
\quad
C=\frac{1}{\frac{a+b+c}{a+b}}=\frac{a+b}{a+b+c}.
\]

Ainsi :
\[
A+B+C
=\frac{b+c}{a+b+c}
+\frac{a+c}{a+b+c}
+\frac{a+b}{a+b+c}.
\]

En regroupant :
\[
A+B+C
=\frac{(b+c)+(a+c)+(a+b)}{a+b+c}
=\frac{2(a+b+c)}{a+b+c}.
\]

Donc :
\[
A+B+C=2.
\]

\hfill \textbf{(1,5 point)}

%=========================================
\item Mettons $D$ sous la forme d’un produit de puissances de nombres premiers :
\[
D=\frac{(-25)^3 \times (-16)^3 \times 36^{-3}}
{(-8)^4 \times 48^{-2} \times (-15)^2}.
\]

Décomposons chaque facteur :

\[
(-25)^3=-(5^2)^3=-5^6,\quad
(-16)^3=-(2^4)^3=-2^{12},
\quad
36^{-3}=(2^2\cdot3^2)^{-3}=2^{-6}3^{-6}.
\]

\[
(-8)^4=(2^3)^4=2^{12},\quad
48^{-2}=(2^4\cdot3)^{-2}=2^{-8}3^{-2},\quad
(-15)^2=(3\cdot5)^2=3^25^2.
\]

En remplaçant dans l’expression de \(D\) :
\[
D=\frac{(-1)\cdot(-1)\cdot 2^{12}\cdot2^{-6}\cdot3^{-6}\cdot5^6}
{2^{12}\cdot2^{-8}\cdot3^{-2}\cdot3^2\cdot5^2}.
\]

Les signes négatifs se compensent, donc :
\[
D=2^{(12-6-12+8)}\cdot3^{(-6+2-2)}\cdot5^{(6-2)}.
\]

Ainsi :
\[
D=2^{2}\cdot3^{-6}\cdot5^{4}.
\]

\hfill \textbf{(1,25 point)}

\end{enumerate}

\section*{\underline{Exercice 4 :} (8,25 points) }

\begin{enumerate}
\item Complétons le tableau suivant : \textbf{(2,25 pt)}

\begin{center}
\begin{tabular}{|>{\centering\arraybackslash}p{3cm}|>{\centering\arraybackslash}p{3cm}|>{\centering\arraybackslash}p{3cm}|>{\centering\arraybackslash}p{3cm}|}
\hline
\textbf{Valeur absolue} & \textbf{Distance} & \textbf{Encadrement} & \textbf{Intervalle} \\
\hline
$|x - 3| \leq 1$ & $d(x , 3) \leq 1$ & $2 \leq x \leq 4$ & $x\in [2;4]$ \\
\hline
$|x + 4| < 2$ & $d(x , -4) < 2$ & $-6 < x < -2$ & $x\in ]-6;\,-2[$ \\
\hline
$|x + \tfrac{11}{4}| \leq \tfrac{9}{4}$ & $d\!\left(x,\,-\tfrac{11}{4}\right) \leq \tfrac{9}{4}$ 
& $-5 \leq x \leq -\tfrac{1}{2}$ 
& $x \in \left[-5; -\frac12\right]$ \\
\hline
\end{tabular}
\end{center}

\item Résolvons dans $\mathbb{R}$
\begin{enumerate}
        \item $|2x+3| > 0 $ toujours vrai deonc $ S=\mathbb{R} $ \hfill \textbf{ (1 point)}

 \begin{resultbox}
    \[
    \mathbf{ S=\mathbb{R} }
    \]
\end{resultbox}        
        
        \item $|-2x+3| \geq 6$ \hfill \textbf{ (1 point)}
        
        $\begin{aligned}
        |-2x+3| \geq 6 &\implies  -2x+3 \geq 6 \textbf{ ou } -2x+3 \leq -6\\
        							 &\implies  -2x \geq 3 \textbf{ ou } -2x \leq -9\\
        							 &\implies  x \leq -\dfrac{3}{2} \textbf{ ou } x \geq \dfrac{9}{2}\\
        							 &\implies  x \in \left] -\infty ; -\dfrac{3}{2} \right]  \textbf{ ou } x \in \left] \dfrac{9}{2} ; +\infty \right]\\
        							 &\implies  x \in \left(  \left] -\infty ; -\dfrac{3}{2} \right]  \cup \left] \dfrac{9}{2} ; +\infty \right]\right) \\
        \end{aligned}$

\begin{resultbox}
    \[
    \mathbf{ S= \left(  \left] -\infty ; -\dfrac{3}{2} \right]  \cup \left] \dfrac{9}{2} ; +\infty \right]\right)  }
    \]
\end{resultbox}        
        
        \item $|3x+5| \leq 2$ \hfill \textbf{ (1 point)}

        $\begin{aligned}
        |3x+5| \leq 2 &\implies  -2 \leq 3x+5 \leq 2\\
        							 &\implies  -7 \leq 3x \leq -3\\
        							 &\implies  \dfrac{-7}{3} \leq x \leq \dfrac{-3}{3}\\
        							 &\implies  \dfrac{-7}{3} \leq x \leq -1\\
        \end{aligned}$        
        
\begin{resultbox}
    \[
    \mathbf{ S=\left[ \dfrac{-7}{3} , -1 \right]  }
    \]
\end{resultbox} 
        \item $|3-x| = 4x-3$ \hfill \textbf{ (1 point)}

L'équation n'a de sens que ssi $ 4x-3 \geq 0 $      

        $\begin{aligned}
        4x-3 \geq 0 &\implies  x \geq \dfrac{3}{4}\\
        							 &\implies  x \in \left[\dfrac{3}{4}, +\infty \right[  \\
        \end{aligned}$         

$Dv=\left[\dfrac{3}{4}, +\infty \right[$        

        $\begin{aligned}
        |3-x| = 4x-3 &\implies  3-x = 4x-3 \text{ ou } 3-x = -4x+3\\
        						 &\implies  -5x = -6 \text{ ou } 3x =0\\
        						 &\implies  x = \dfrac{6}{5} \text{ ou } x =0\\
        \end{aligned}$      

$ 0 \notin Dv $  et $\dfrac{6}{5} \in Dv $       

\begin{resultbox}
    \[
    \mathbf{ S= \left\lbrace \dfrac{6}{5} \right\rbrace   }
    \]
\end{resultbox}        
        
        \item $E(|x-3|) = 2$ \hfill \textbf{ (1 point)}

				 $\begin{aligned}
        E(|x-3|) = 2 &\implies  E(|x-3|) \leq |x-3| < E(|x-3|)+1 \\
                     &\implies  2 \leq |x-3| < 3 \\
                     &\implies  
                     \begin{cases}
                     |x-3| \geq 2\\
                     |x-3| < 3
                     \end{cases}\\
                     &\implies  
                     \begin{cases}
                     x-3 \geq 2 \text{ ou } x-3 \leq -2\\
                     -3 < x-3 < 3
                     \end{cases}\\
                     &\implies
                     \begin{cases}
                     x \geq 5 \text{ ou } x \leq 1\\
                     0 < x < 6
                     \end{cases}\\
                     &\implies
                     \begin{cases}
                     x \in ] 5,+\infty [ \text{ ou } x \in ] -\infty, 1[\\
                     x \in  ] 0,6 [
                     \end{cases}\\
                      &\implies
                     \begin{cases}
                     x \in (] 5,+\infty [ \cup ] -\infty, 1[)\\
                     x \in  ] 0,6 [
                     \end{cases}
        \end{aligned}$         

         $\begin{aligned}
         S & = (] 5,+\infty [ \cup ] -\infty, 1[) \cap ] 0,6 [\\
           & = (] 5,+\infty [\cap ] 0,6 [) \cup (] -\infty, 1[\cap ] 0,6 [)\\
           & = (] 5,6 [) \cup (] 0,1 [)\\
           & = ] 0,1 [ \cup ] 5,6 [\\
        \end{aligned}$       

\begin{resultbox}
    \[
    \mathbf{ S= ] 0,1 [ \cup ] 5,6 [   }
    \]
\end{resultbox}    
        
        \item $E(|x-2|) = -2$ \hfill \textbf{ (1 point)}

				$\begin{aligned}
        E(|x-2|) = -2 &\implies  E(|x-2|) \leq |x-2| < E(|x-2|)+1 \\
                     &\implies  -2 \leq |x-2| < -1 \\
                     &\implies  
                     \begin{cases}
                     |x-2| \geq -2\\
                     |x-2| < -1
                     \end{cases}\\
                     &\implies  
                     \begin{cases}
                     x \in \mathbb{R }\\
                     x \in  \emptyset
                     \end{cases}
        \end{aligned}$         
        
         $\begin{aligned}
         S & = \mathbb{R } \cap \emptyset\\
           & = \emptyset
        \end{aligned}$       

\begin{resultbox}
    \[
    \mathbf{ S= \emptyset   }
    \]
\end{resultbox}      
        
\end{enumerate}
\end{enumerate}

\end{document}