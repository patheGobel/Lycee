\documentclass[12pt,a4paper]{article}
\usepackage{amsmath,amssymb,mathrsfs,tikz,times,pifont}
\usepackage{enumitem}
\newcommand\circitem[1]{%
\tikz[baseline=(char.base)]{
\node[circle,draw=gray, fill=red!55,
minimum size=1.2em,inner sep=0] (char) {#1};}}
\newcommand\boxitem[1]{%
\tikz[baseline=(char.base)]{
\node[fill=cyan,
minimum size=1.2em,inner sep=0] (char) {#1};}}
\setlist[enumerate,1]{label=\protect\circitem{\arabic*}}
\setlist[enumerate,2]{label=\protect\boxitem{\alph*}}
%%%::::::by chnini ameur :::::::%%%
\everymath{\displaystyle}
\usepackage[left=1cm,right=1cm,top=1cm,bottom=1.7cm]{geometry}
\usepackage{array,multirow}
\usepackage[most]{tcolorbox}
\usepackage{varwidth}
\tcbuselibrary{skins,hooks}
\usetikzlibrary{patterns}
%%%::::::by chnini ameur :::::::%%%
\newtcolorbox{exa}[2][]{enhanced,breakable,before skip=2mm,after skip=5mm,
colback=yellow!20!white,colframe=black!20!blue,boxrule=0.5mm,
attach boxed title to top left ={xshift=0.6cm,yshift*=1mm-\tcboxedtitleheight},
fonttitle=\bfseries,
title={#2},#1,
% varwidth boxed title*=-3cm,
boxed title style={frame code={
\path[fill=tcbcolback!30!black]
([yshift=-1mm,xshift=-1mm]frame.north west)
arc[start angle=0,end angle=180,radius=1mm]
([yshift=-1mm,xshift=1mm]frame.north east)
arc[start angle=180,end angle=0,radius=1mm];
\path[left color=tcbcolback!60!black,right color = tcbcolback!60!black,
middle color = tcbcolback!80!black]
([xshift=-2mm]frame.north west) -- ([xshift=2mm]frame.north east)
[rounded corners=1mm]-- ([xshift=1mm,yshift=-1mm]frame.north east)
-- (frame.south east) -- (frame.south west)
-- ([xshift=-1mm,yshift=-1mm]frame.north west)
[sharp corners]-- cycle;
},interior engine=empty,
},interior style={top color=yellow!5}}
%%%%%%%%%%%%%%%%%%%%%%%

\usepackage{fancyhdr}
\usepackage{eso-pic}         % Pour ajouter des éléments en arrière-plan
% Commande pour ajouter du texte en arrière-plan
\AddToShipoutPicture{
    \AtTextCenter{%
        \makebox[0pt]{\rotatebox{80}{\textcolor[gray]{0.7}{\fontsize{5cm}{5cm}\selectfont PGB}}}
    }
}
\usepackage{lastpage}
\fancyhf{}
\pagestyle{fancy}
\renewcommand{\footrulewidth}{1pt}
\renewcommand{\headrulewidth}{0pt}
\renewcommand{\footruleskip}{10pt}
\fancyfoot[R]{
\color{blue}\ding{45}\ \textbf{2025}
}
\fancyfoot[L]{
\color{blue}\ding{45}\ \textbf{Prof:M. BA}
}
\cfoot{\bf
\thepage /
\pageref{LastPage}}
\begin{document}
\renewcommand{\arraystretch}{1.5}
\renewcommand{\arrayrulewidth}{1.2pt}
\begin{tikzpicture}[overlay,remember picture]
\node[draw=blue,line width=1.2pt,fill=purple,text=blue,inner sep=3mm,rounded corners,pattern=dots]at ([yshift=-2.5cm]current page.north) {\begingroup\setlength{\fboxsep}{0pt}\colorbox{white}{\begin{tabular}{|*1{>{\centering \arraybackslash}p{0.28\textwidth}} |*2{>{\centering \arraybackslash}p{0.2\textwidth}|} *1{>{\centering \arraybackslash}p{0.19\textwidth}|} }
\hline
\multicolumn{3}{|c|}{$\diamond$$\diamond$$\diamond$\ \textbf{Lycée de Dindéfélo}\ $\diamond$$\diamond$$\diamond$ }& \textbf{A.S. : 2025/2026} \\ \hline
\textbf{Matière: Mathématiques}& \textbf{Niveau : 2nd}\textbf{S} &\textbf{Date: 20/12/2025} & \textbf{Durée : 4 heures} \\ \hline
\multicolumn{4}{|c|}{\parbox[c]{10cm}{\begin{center}
\textbf{{\Large\sffamily Devoir n$ ^{\circ} $ 2 Du 1$ ^\text{\bf ère} $ Semestre}}
\end{center}}} \\ \hline
\end{tabular}}\endgroup};
\end{tikzpicture}
\vspace{3cm}

\section*{\underline{Exercice 4 :} (8,25 points) }

\begin{enumerate}
\item Complétons le tableau suivant : \textbf{(2,25 pt)}

\begin{center}
\begin{tabular}{|>{\centering\arraybackslash}p{3cm}|>{\centering\arraybackslash}p{3cm}|>{\centering\arraybackslash}p{3cm}|>{\centering\arraybackslash}p{3cm}|}
\hline
\textbf{Valeur absolue} & \textbf{Distance} & \textbf{Encadrement} & \textbf{Intervalle} \\
\hline
$|x - 3| \leq 1$ & $d(x , 3) \leq 1$ & $2 \leq x \leq 4$ & $x\in [2;4]$ \\
\hline
$|x + 4| < 2$ & $d(x , -4) < 2$ & $-6 < x < -2$ & $x\in ]-6;\,-2[$ \\
\hline
$|x + \tfrac{11}{4}| \leq \tfrac{9}{4}$ & $d\!\left(x,\,-\tfrac{11}{4}\right) \leq \tfrac{9}{4}$ 
& $-5 \leq x \leq -\tfrac{1}{2}$ 
& $x \in \left[-5; -\frac12\right]$ \\
\hline
\end{tabular}
\end{center}

\item Résolvons dans $\mathbb{R}$
\begin{enumerate}
        \item $|2x+3| > 0 $ toujours vrai deonc $ S=\mathbb{R} $ \hfill \textbf{ (1 point)}
        \item $|-2x+3| \geq 6$ \hfill \textbf{ (1 point)}
        
        $\begin{aligned}
        |-2x+3| \geq 6 &\implies  -2x+3 \geq 6 \textbf{ ou } -2x+3 \leq -6\\
        							 &\implies  -2x \geq 3 \textbf{ ou } -2x \leq -9\\
        							 &\implies  x \leq -\dfrac{3}{2} \textbf{ ou } x \geq \dfrac{9}{2}\\
        							 &\implies  x \in \left] -\infty ; -\dfrac{3}{2} \right]  \textbf{ ou } x \in \left] \dfrac{9}{2} ; +\infty \right]\\
        							 &\implies  x \in \left(  \left] -\infty ; -\dfrac{3}{2} \right]  \cup \left] \dfrac{9}{2} ; +\infty \right]\right) \\
        \end{aligned}$
        
        \item $|3x+5| \leq 2$ \hfill \textbf{ (1 point)}
        \item $|3-x| = 4x-3$ \hfill \textbf{ (1 point)}
        \item $E(|x-3|) = 2$ \hfill \textbf{ (1 point)}
        \item $E(|x-2|) = -2$ \hfill \textbf{ (1 point)}
\end{enumerate}
\end{enumerate}

\end{document}