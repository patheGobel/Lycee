\documentclass[12pt,a4paper]{article}
\usepackage{amsmath,amssymb,mathrsfs,tikz,times,pifont}
\usepackage{enumitem}
\newcommand\circitem[1]{%
\tikz[baseline=(char.base)]{
\node[circle,draw=gray, fill=red!55,
minimum size=1.2em,inner sep=0] (char) {#1};}}
\newcommand\boxitem[1]{%
\tikz[baseline=(char.base)]{
\node[fill=cyan,
minimum size=1.2em,inner sep=0] (char) {#1};}}
\setlist[enumerate,1]{label=\protect\circitem{\arabic*}}
\setlist[enumerate,2]{label=\protect\boxitem{\alph*}}
%%%::::::by chnini ameur :::::::%%%
\everymath{\displaystyle}
\usepackage[left=1cm,right=1cm,top=1cm,bottom=1.7cm]{geometry}
\usepackage{array,multirow}
\usepackage[most]{tcolorbox}
\usepackage{varwidth}
\tcbuselibrary{skins,hooks}
\usetikzlibrary{patterns}
%%%::::::by chnini ameur :::::::%%%
\newtcolorbox{exa}[2][]{enhanced,breakable,before skip=2mm,after skip=5mm,
colback=yellow!20!white,colframe=black!20!blue,boxrule=0.5mm,
attach boxed title to top left ={xshift=0.6cm,yshift*=1mm-\tcboxedtitleheight},
fonttitle=\bfseries,
title={#2},#1,
% varwidth boxed title*=-3cm,
boxed title style={frame code={
\path[fill=tcbcolback!30!black]
([yshift=-1mm,xshift=-1mm]frame.north west)
arc[start angle=0,end angle=180,radius=1mm]
([yshift=-1mm,xshift=1mm]frame.north east)
arc[start angle=180,end angle=0,radius=1mm];
\path[left color=tcbcolback!60!black,right color = tcbcolback!60!black,
middle color = tcbcolback!80!black]
([xshift=-2mm]frame.north west) -- ([xshift=2mm]frame.north east)
[rounded corners=1mm]-- ([xshift=1mm,yshift=-1mm]frame.north east)
-- (frame.south east) -- (frame.south west)
-- ([xshift=-1mm,yshift=-1mm]frame.north west)
[sharp corners]-- cycle;
},interior engine=empty,
},interior style={top color=yellow!5}}
%%%%%%%%%%%%%%%%%%%%%%%

\usepackage{fancyhdr}
\usepackage{eso-pic}         % Pour ajouter des éléments en arrière-plan
% Commande pour ajouter du texte en arrière-plan
\AddToShipoutPicture{
    \AtTextCenter{%
        \makebox[0pt]{\rotatebox{80}{\textcolor[gray]{0.7}{\fontsize{5cm}{5cm}\selectfont PGB}}}
    }
}
\usepackage{lastpage}
\fancyhf{}
\pagestyle{fancy}
\renewcommand{\footrulewidth}{1pt}
\renewcommand{\headrulewidth}{0pt}
\renewcommand{\footruleskip}{10pt}
\fancyfoot[R]{
\color{blue}\ding{45}\ \textbf{2025}
}
\fancyfoot[L]{
\color{blue}\ding{45}\ \textbf{Prof:M. BA}
}
\cfoot{\bf
\thepage /
\pageref{LastPage}}
\begin{document}
\renewcommand{\arraystretch}{1.5}
\renewcommand{\arrayrulewidth}{1.2pt}
\begin{tikzpicture}[overlay,remember picture]
\node[draw=blue,line width=1.2pt,fill=purple,text=blue,inner sep=3mm,rounded corners,pattern=dots]at ([yshift=-2.5cm]current page.north) {\begingroup\setlength{\fboxsep}{0pt}\colorbox{white}{\begin{tabular}{|*1{>{\centering \arraybackslash}p{0.28\textwidth}} |*2{>{\centering \arraybackslash}p{0.2\textwidth}|} *1{>{\centering \arraybackslash}p{0.19\textwidth}|} }
\hline
\multicolumn{3}{|c|}{$\diamond$$\diamond$$\diamond$\ \textbf{Lycée de Dindéfélo}\ $\diamond$$\diamond$$\diamond$ }& \textbf{A.S. : 2025/2026} \\ \hline
\textbf{Matière: Mathématiques}& \textbf{Niveau : 2nd}\textbf{S} &\textbf{Date: 00/11/2025} & \textbf{Durée : 4 heures} \\ \hline
\multicolumn{4}{|c|}{\parbox[c]{10cm}{\begin{center}
\textbf{{\Large\sffamily Devoir n$ ^{\circ} $ 1 Du 1$ ^\text{\bf ère} $ Semestre}}
\end{center}}} \\ \hline
\end{tabular}}\endgroup};
\end{tikzpicture}
\vspace{3cm}

\section*{\underline{Correction Exercice 1 :} 4 pts (Factoriser les expressions suivantes :) }
\begin{enumerate}
\item \( a^2xy + aby^2 + b^2xy + abx^2 \) \\

\[
\begin{aligned}
(a^2xy + abx^2) + (b^2xy + aby^2) &= ax(ax + by) + by(ax + by)\\
  																&=(ax + by) (ax+by)\\
\end{aligned} 
\]
\item \( 3a^2 + 3b^2 - 4c^2 - 6ab \)  

\[
\begin{aligned}
(3a^2 - 6ab + 3b^2) - 4c^2 &= 3(a^2 - 2ab + b^2) - 4c^2\\
  												&=3(a - b)^2 - (2\cdot c)^2\\
  												&=\left[\sqrt{3}(a - b)\right]^2 - (2\cdot c)^2\\
  												&=( \sqrt{3}(a - b) - 2c )( \sqrt{3}(a - b) + 2c )
\end{aligned} 
\] 
  

\item \( y^2 - x^2 + 2x - 1 \)  

\[
\begin{aligned}
y^2 - (x^2 - 2x + 1) &= y^2 - (x - 1)^2\\
  									 &=(y - (x - 1))(y + (x - 1))\\ 
  									 &= (y - x + 1)(y + x - 1)
\end{aligned} 
\] 

\item \( a^2b^2 - 1 + a^2 - b^2 \)  

\[
\begin{aligned}
a^2b^2 - 1 + a^2 - b^2 &=a^2b^2 - b^2+ a^2 -1\\
 												&= b^2(a^2 - 1) + (a^2 - 1) \\
 												&=(a^2 - 1)(b^2 + 1)\\
 												&=(a - 1)(a + 1)(b^2 + 1)
\end{aligned} 
\] 
  

\item \( (ab - 1)^2 - (a - b)^2 \) 

\[
\begin{aligned}
(ab - 1)^2 - (a - b)^2 &=[ab - 1 - (a - b)] [ab - 1 + (a - b)]\\
 											 &=(ab - 1 - a + b)(ab - 1 + a - b)\\
 											 &=[a(b - 1) + (b - 1)][a(b + 1) - (1 + b)]\\
 											 &= [(a + 1)(b - 1)][(a + 1)(b - 1)]\\
 											 &=(a - 1)(a + 1)(b - 1)(b + 1)
\end{aligned} 
\]  

\end{enumerate}
\section*{\underline{Exercice 2 :} 3 pts }
\begin{enumerate}
    \item Développons $(a + b + c)^2$.\\
    $\begin{aligned}
    [(a + b) + c]^2 &= (a + b)^2 + 2c(a+b)+c^2\\
        						&= \big(a^2 + 2ab + b^2\big) + 2ac + 2bc + c^2 \\
    								&= a^2 + b^2 + c^2 + 2ab + 2ac + 2bc.
    \end{aligned}
    $
    \item Montrons que si $a + b + c = 0$ alors $a^2 + b^2 + c^2 = -2 (ab + bc + ca)$.
\[
\text{On suppose que } a+b+c=0.
\]

\[
\text{ On a: }(a+b+c)^2 = a^2 + b^2 + c^2 + 2ab + 2ac + 2bc \text{ donc } (0)^2 = a^2 + b^2 + c^2 + 2ab + 2ac + 2bc
\]

\[
\begin{aligned}
a^2 + b^2 + c^2 + 2ab + 2ac + 2bc =0 &\implies a^2 + b^2 + c^2 = - 2ab - 2ac - 2bc\\ 
																		 &\implies a^2 + b^2 + c^2 = -2 (ab + bc + ca)
\end{aligned}
\]
    \item On suppose $a$, $b$ et $c$ sont non nuls. \\
    Montrons que $\frac{1}{a} + \frac{1}{b} + \frac{1}{c} = 0 \implies (a + b + c)^2 = a^2 + b^2 + c^2$.
   \[
\begin{aligned}
\frac{1}{a} + \frac{1}{b} + \frac{1}{c} = 0 &\implies \frac{bc+ac+ab}{abc} = 0\\
																						&\implies bc+ac+ab = 0
\end{aligned}
\] 					
D'après la 1er question, $(a+b+c)^2 = a^2 + b^2 + c^2 + 2(ab + ac + bc)$

Donc si $bc+ac+ab = 0$ alors $(a+b+c)^2 = a^2 + b^2 + c^2$
\end{enumerate}
\section*{\underline{Exercice 3 :} 4 pts } Soit $a, b, c$ trois réels :
\begin{enumerate}
    \item Développons $(a + b + c)(ab + bc + ca)$ puis $(a + b + c)^3$
    
\[ 
\begin{aligned}
(a+b+c)(ab+bc+ca)&= a(ab+bc+ca) + b(ab+bc+ca) + c(ab+bc+ca) \\[4pt]
    						 &= a^2b + abc + a^2c + ab^2 + b^2c + abc + abc + bc^2 + ac^2 \\[4pt]
                 &= a^2b + a^2c + ab^2 + ac^2 + b^2c + bc^2 + 3abc
\end{aligned}
\]

\textcolor{red}{\boxed{(a+b+c)(ab+bc+ca) =a^2b + a^2c + ab^2 + ac^2 + b^2c + bc^2 + 3abc} } 
    
\[
\begin{aligned}
(a+b+c)^3 &= (a+b)^3 + 3c(a+b)^2 + 3(a+b)c^2 + c^3 \\[4pt]
          &= (a^3 + 3a^2b + 3ab^2 + b^3) 
           + 3c(a^2 + 2ab + b^2) 
           + 3(a c^2 + b c^2) 
           + c^3 \\[4pt]
          &= a^3 + b^3 + c^3 
           + 3a^2b + 3ab^2 
           + 3a^2c + 6abc + 3b^2c 
           + 3ac^2 + 3bc^2 .
\end{aligned}
\]

\textcolor{red}{\boxed{(a+b+c)^3=a^3 + b^3 + c^3 + 3a^2b + 3ab^2 + 3a^2c + 6abc + 3b^2c + 3ac^2 + 3bc^2} } 

    \item Démontrons que si $a + b + c = 0$ alors $a^3 + b^3 + c^3 = 3abc$
    
\(
\underline{
\begin{cases}
(a+b+c)(ab+bc+ca) &= a^2b + a^2c + ab^2 + ac^2 + b^2c + bc^2 + 3abc\\
(a+b+c)^3 &=a^3 + b^3 + c^3 + 3a^2b + 3ab^2 + 3a^2c + 6abc + 3b^2c + 3ac^2 + 3bc^2
\end{cases}}
\)

\vspace{1cm}

\(
\underline{
\begin{cases}
(a+b+c)(ab+bc+ca) &= a^2b + a^2c + ab^2 + ac^2 + b^2c + bc^2 + 3abc\\
(a+b+c)^3 &=3a^2b +3a^2c+ 3ab^2 + 3ac^2+ 3b^2c+ 6abc + 3bc^2+ a^3 + b^3 + c^3
\end{cases}}
\)

\vspace{1cm}

\(
\underline{
\begin{cases}
3(a+b+c)(ab+bc+ca)= 3a^2b + 3a^2c + 3ab^2 + 3ac^2 + 3b^2c + 3bc^2 + 9abc\\
(a+b+c)^3 =3a^2b + 3a^2c + 3ab^2 + 3ac^2 + 3b^2c + 3bc^2+ 9abc-3abc + a^3 + b^3 + c^3
\end{cases}}\\
3(a+b+c)(ab+bc+ca)-(a+b+c)^3 = 3abc-(a^3 + b^3 + c^3)
\)

si $a + b + c = 0$ alors $3(0)(ab+bc+ca)-(0)^3 = 3abc-(a^3 + b^3 + c^3)$

donc $ 3abc-(a^3 + b^3 + c^3) =  0 $ d'où $ a^3 + b^3 + c^3 = 3abc$

    \item Déduisons-en que , pour tous réel $x, y, z$ on a :
    $$(x + y)^3 + (y + z)^3 + (z + x)^3 = 3 (x + y) (y + z) (z + x)$$
    
En Posons\(
\begin{cases}
x+y =a\\
y + z =b\\
z + x =c
\end{cases}
\text{ donc } a^3 + b^3 + c^3 = 3abc \text{ devient } (x + y)^3 + (y + z)^3 + (z + x)^3 = 3 (x + y) (y + z) (z + x)
\) 
\end{enumerate}
\end{document}
