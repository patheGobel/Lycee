\documentclass{article}
\usepackage[utf8]{inputenc}
\usepackage[T1]{fontenc}
\usepackage[french]{babel}
\usepackage{amsmath}
\usepackage{amsfonts}
\usepackage{amssymb}
\usepackage{xcolor}
\usepackage{pgfplots}
\pgfplotsset{compat=1.15}
\usepackage{mathrsfs}
\usetikzlibrary{arrows}
\usepackage{tikz} % Indispensable pour les dessins
\pagestyle{empty}
% Configuration des titres en rouge
\usepackage{titlesec}
\usepackage{verbatim} % <--- Indispensable pour \begin{comment}
\usetikzlibrary{calc}
\titleformat{\section}{\color{red}\normalfont\Large\bfseries}{}{0pt}{}
\titleformat{\subsection}{\color{red}\normalfont\large\bfseries}{}{0pt}{}

\begin{document}

\begin{center}
    \color{red}
    \framebox[0.6\textwidth]{
        \huge \textbf{Chapitre 1 : Vecteurs}
    }
\end{center}

\section*{I - Généralités}
\subsection*{1- Définitions}

\textit{\color{red}Définition 1 :}\\
On appelle \textbf{plan vectoriel} l'ensemble des vecteurs du plan.

\vspace{1.5em}

\textit{\color{red}Définition 2 :}\\
Soient $\vec{u}$ un vecteur du plan vectoriel et $(A, B)$ un couple de points. On dit que le couple $(A, B)$ est un \textbf{représentant} du vecteur $\vec{u}$ si et seulement si $\overrightarrow{AB} = \vec{u}$.

\begin{center}
    \begin{tikzpicture}[>=stealth, thick]
        % Premier vecteur (u)
        \draw[->] (0,0) -- (1.5,1.5) node[midway, right] {\color{red} $\vec{u}$};
        
        % Deuxième vecteur (AB) décalé vers la droite
        % On utilise les mêmes coordonnées relatives pour qu'ils soient parallèles
        \draw[->] (2,0) node[below left] {\color{red} $A$} -- (3.5,1.5) node[below right] {\color{red} $B$};
    \end{tikzpicture}
\end{center}
\textit{\color{red}Remarque 1 :}\\
Tout vecteur du plan a une infinité de représentants.
\begin{center}
    \begin{tikzpicture}[>=stealth, thick]
        % Vecteurs parallèles
        \draw[->] (0,1.2) node[above] {$E$} -- (2,1.2) node[right] {$F$};
        \draw[->] (0,0.8) -- (2,0.8) node[right] {$\vec{u}$};
        \draw[->] (0,0.4) node[left] {$A$} -- (2,0.4) node[right] {$B$};
        \draw[->] (0,0) node[left] {$C$} -- (2,0) node[right] {$D$};
    \end{tikzpicture}
    \quad $\vec{u} = \overrightarrow{AB} = \overrightarrow{CD} = \overrightarrow{EF} = ...$
\end{center}

\vspace{1em}

\textit{\color{red}Remarque 2 : (Caractéristiques d'un vecteur).}\\
Un vecteur est caractérisé par un \textcolor{red}{sens}, une \textcolor{red}{direction} et une \textcolor{red}{norme} :

\begin{itemize}
    \item[$*_{1}$] La \textcolor{red}{direction} est celle de la "droite" dans laquelle est inclus le vecteur.
    \item[$*_{2}$] Le \textcolor{red}{sens} est donné par l'orientation du segment dans lequel est inclus le vecteur ("vers la gauche" ou bien "vers la droite").
    \item[$*_{3}$] La \textcolor{red}{norme} correspond à la longueur du segment sur lequel est inclus le vecteur.
\end{itemize}

Ainsi, on en déduit la propriété suivante :

\subsection*{\underline{\color{red}Propriété :}}
\begin{enumerate}
    \item[i)] $\overrightarrow{AB} = \vec{0} \iff A = B \quad (\text{donc } \overrightarrow{AA} = \vec{0})$
    \item[ii)] $\overrightarrow{AB} = -\overrightarrow{BA}$
\end{enumerate}

\section*{2- Propriétés}
\subsection*{\underline{\color{red}Activité}}
Soient $O$ un point du plan et $\vec{u}$ un vecteur du plan vectoriel.
\begin{enumerate}
    \item[i)] Construire un point $A$ tel que $\overrightarrow{OA} = \vec{u}$. ($A \neq O$)
    \item[ii)] Construire un point $B$ tel que $\overrightarrow{OB} = \vec{u}$. ($B \neq O$)
    \item[iii)] Que peut-on dire des points $A$ et $B$ ?
\end{enumerate}

\subsection*{\underline{\color{red}Correction}}
\begin{enumerate}
    \item[i)] Construisons le point $A$.
    \begin{center}
        \begin{tikzpicture}[>=stealth, thick]
            \draw[->] (0,1) node[above] {$O$} -- (2,1) node[above] {$A$};
            \node at (2,0.7) {$B$};
            \draw[->] (0.5,0.2) -- (2.5,0.2) node[midway, below] {$\vec{u}$};
        \end{tikzpicture}
    \end{center}
    \item[ii)] Construisons le point $B$ (voir la question i).
    \item[iii)] On peut dire que les points $A$ et $B$ sont \textbf{confondus}.
\end{enumerate}
\subsection*{\underline{Propriété 1 : (Propriété fondamentale)}}
Pour tout point $O$ et pour tout vecteur $\vec{u}$, il existe un et un seul (unique) point $M$ tel que $\overrightarrow{OM} = \vec{u}$.

\subsubsection*{\underline{Preuve :}}
$\overrightarrow{OM} = \vec{u}$ signifie que $M$ est l'image de $O$ par la translation du vecteur $\vec{u}$. D'où le point $M$ est l'unique point tel que $\overrightarrow{OM} = \vec{u}$.

\vspace{1em}

\subsection*{\underline{Propriété 2 :}}
Soient $A, B, C$ et $D$ des points du plan. Alors les 3 énoncés suivants sont équivalents :
\begin{enumerate}
    \item[i)] $\overrightarrow{AB} = \overrightarrow{DC}$
    \item[ii)] $ABDC$ est un parallélogramme.
    \item[iii)] Les segments $[AC]$ et $[BD]$ ont le même milieu.
\end{enumerate}
\begin{center}
    \includegraphics[width=3cm]{image1.png}
\end{center}

\begin{comment}
\begin{center}
\begin{tikzpicture}[scale=1]

% Points
\coordinate (A) at (1,3);
\coordinate (B) at (0,0);
\coordinate (C) at (4,0);
\coordinate (D) at (5,3);

% Quadrilatère
\draw[thick] (A) -- (D) -- (C) -- (B) -- cycle;

% Diagonales
\draw[thick] (A) -- (C);
\draw[thick] (B) -- (D);

% Point d'intersection
\coordinate (O) at ($(A)!0.5!(C)$);

% Marques sur AC
\draw ($(A)!0.45!(C)$) ++(-0.1,0.1) -- ++(0.2,-0.2);
\draw ($(A)!0.55!(C)$) ++(-0.1,0.1) -- ++(0.2,-0.2);

% Marques sur BD
\draw ($(B)!0.45!(D)$) ++(-0.1,-0.1) -- ++(0.2,0.2);
\draw ($(B)!0.55!(D)$) ++(-0.1,-0.1) -- ++(0.2,0.2);

% Noms des points
\node[left] at (A) {$A$};
\node[left] at (B) {$B$};
\node[right] at (C) {$C$};
\node[right] at (D) {$D$};

\end{tikzpicture}
\end{center}
\end{comment}
\begin{comment}
Cette section traite de la définition et des propriétés 
élémentaires de la norme d'un vecteur.
\end{comment}

\subsection*{3- \underline{\color{red}Norme d'un vecteur}}

\subsubsection*{\underline{\color{red}Définition :}}
Soit $\vec{u}$ un vecteur du plan de représentant $(A, B)$, ($\overrightarrow{AB} = \vec{u}$).\\
On appelle \textcolor{red}{norme} du vecteur notée $||\vec{u}||$ la distance $AB$ :
$ ||\vec{u}|| = AB $

\subsubsection*{\underline{\color{red}Propriété :}}
Soit $\vec{u}$ un vecteur du plan vectoriel :
\begin{enumerate}
    \item[i)] $||\vec{u}|| = 0 \iff \vec{u} = \vec{0}$
    \item[ii)] $||-\vec{u}|| = ||\vec{u}||$
\end{enumerate}
\subsubsection*{\underline{\color{red}Preuve :}}
\begin{enumerate}
    \item[i)] Montrons que $||\vec{u}|| = 0 \iff \vec{u} = \vec{0}$
    \begin{itemize}
        \item[$*_{1}$] $\vec{u} = \vec{0} \implies \vec{u} = \overrightarrow{AA}$ or $||\overrightarrow{AA}|| = 0$ d'où $||\vec{u}|| = 0$.
        \item[$*_{2}$] $||\vec{u}|| = 0$ donc si $(A, B)$ est un représentant de $\vec{u}$ ($\overrightarrow{AB} = \vec{u}$) alors $AB = 0$. Donc $A = B \iff \overrightarrow{AB} = \vec{0}$. D'où $\vec{u} = \vec{0}$.
    \end{itemize}
    \item[ii)] Montrons que $||-\vec{u}|| = ||\vec{u}||$ \\
    $(A, B)$ est un représentant du vecteur $\vec{u}$ équivaut à $(B, A)$ est un représentant du vecteur $-\vec{u}$.\\
    Donc $||\vec{u}|| = AB$ et $||-\vec{u}|| = BA$.\\
    Comme $AB = BA$ alors on en déduit que $||\vec{u}|| = ||-\vec{u}||$.
\end{enumerate}

\begin{comment}
Cette section explique que l'égalité des normes n'implique 
pas nécessairement l'égalité des vecteurs.
\end{comment}

\subsection*{\underline{\color{red}Remarque 1 :}}
Deux vecteurs de même norme ne sont pas nécessairement égaux.

\vspace{0.5em}

\noindent \underline{\color{red}Par exemple :} Dans $\mathbb{R}^2$, on prend :
$\vec{u} = \begin{pmatrix} 2 \\ 1 \end{pmatrix}$ et $\vec{v} = \begin{pmatrix} 1 \\ 2 \end{pmatrix}$

\vspace{0.5em}

\noindent On a :
\begin{align*}
||\vec{u}|| &= \sqrt{2^2 + 1^2} = \sqrt{5} \\
||\vec{v}|| &= \sqrt{1^2 + 2^2} = \sqrt{5}
\end{align*}

\noindent Donc $||\vec{u}|| = ||\vec{v}||$ mais $\vec{u} \neq \vec{v}$.

\begin{comment}
Cette section présente la conclusion sur la comparaison des vecteurs 
par leurs normes ainsi que les propriétés liées à la somme de vecteurs.
\end{comment}

\noindent \underline{Conclusion} : \\
$\|\vec{u}\| = \|\vec{v}\| \not\implies \vec{u} = \vec{v}$

\subsection*{\underline{\color{red}Remarque 2 :}}
Soient $\vec{u}$ et $\vec{v}$ deux vecteurs.
\begin{enumerate}
    \item[i)] \textbf{En général}, on a : \\
    $\|\vec{u} + \vec{v}\| \leq \|\vec{u}\| + \|\vec{v}\|$
    
    \item[ii)] \textbf{Par contre}, si $\vec{u}$ et $\vec{v}$ sont orthogonaux ($\vec{u} \perp \vec{v}$), alors : \\
    $\|\vec{u} + \vec{v}\|^2 = \|\vec{u}\|^2 + \|\vec{v}\|^2$
\end{enumerate}

\noindent Cette propriété est le \underline{théorème de Pythagore}.

\noindent \underline{\color{red}Exemple :} \\
i) $\vec{u} = \begin{pmatrix} 2 \\ -1 \end{pmatrix}$ et $\vec{v} = \begin{pmatrix} 2 \\ 1 \end{pmatrix}$ \\
On a : \\
$\vec{u} + \vec{v} = \begin{pmatrix} 4 \\ 0 \end{pmatrix}$ \\
$\|\vec{u} + \vec{v}\| = \sqrt{4^2 + 0^2} = 4 \quad (*)$ \\
$\begin{cases} 
\|\vec{u}\| = \sqrt{2^2 + (-1)^2} = \sqrt{5} \\ 
\|\vec{v}\| = \sqrt{2^2 + 1^2} = \sqrt{5} 
\end{cases}$ \\
$\|\vec{u}\| + \|\vec{v}\| = 2\sqrt{5} \quad (**)$ \\
D'après $(*)$ et $(**)$, on a : \\
$\|\vec{u} + \vec{v}\| < \|\vec{u}\| + \|\vec{v}\|$
\end{document}