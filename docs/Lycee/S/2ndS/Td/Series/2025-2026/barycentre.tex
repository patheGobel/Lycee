\documentclass[12pt,a4paper]{article}
\usepackage[T1]{fontenc}
\usepackage{amsmath,amssymb,mathrsfs,tikz,times,pifont}
\usepackage{enumitem}
\usepackage{multicol}
\usepackage{lmodern}
\usetikzlibrary{trees}
\newcommand\circitem[1]{%
\tikz[baseline=(char.base)]{
\node[circle,draw=gray, fill=red!55,
minimum size=1.2em,inner sep=0] (char) {#1};}}
\newcommand\boxitem[1]{%
\tikz[baseline=(char.base)]{
\node[fill=cyan,
minimum size=1.2em,inner sep=0] (char) {#1};}}
\setlist[enumerate,1]{label=\protect\circitem{\arabic*}}
\setlist[enumerate,2]{label=\protect\boxitem{\alph*}}
%%%::::::by chnini ameur :::::::%%%
\everymath{\displaystyle}
\usepackage[left=1cm,right=1cm,top=1cm,bottom=1.7cm]{geometry}
\usepackage[colorlinks=true, linkcolor=blue, urlcolor=blue, citecolor=blue]{hyperref}
\usepackage{array,multirow}
\usepackage[most]{tcolorbox}
\usepackage{varwidth}
\usepackage{float} %pour utiliser l'option [H] qui force l'image à apparaître exactement à l'endroit où elle est placée dans le code.
\tcbuselibrary{skins,hooks}
\usetikzlibrary{patterns}
%%%::::::by chnini ameur :::::::%%%
\newtcolorbox{exa}[2][]{enhanced,breakable,before skip=2mm,after skip=5mm,
colback=yellow!20!white,colframe=black!20!blue,boxrule=0.5mm,
attach boxed title to top left ={xshift=0.6cm,yshift*=1mm-\tcboxedtitleheight},
fonttitle=\bfseries,
title={#2},#1,
% varwidth boxed title*=-3cm,
boxed title style={frame code={
\path[fill=tcbcolback!30!black]
([yshift=-1mm,xshift=-1mm]frame.north west)
arc[start angle=0,end angle=180,radius=1mm]
([yshift=-1mm,xshift=1mm]frame.north east)
arc[start angle=180,end angle=0,radius=1mm];
\path[left color=tcbcolback!60!black,right color = tcbcolback!60!black,
middle color = tcbcolback!80!black]
([xshift=-2mm]frame.north west) -- ([xshift=2mm]frame.north east)
[rounded corners=1mm]-- ([xshift=1mm,yshift=-1mm]frame.north east)
-- (frame.south east) -- (frame.south west)
-- ([xshift=-1mm,yshift=-1mm]frame.north west)
[sharp corners]-- cycle;
},interior engine=empty,
},interior style={top color=yellow!5}}
%%%%%%%%%%%%%%%%%%%%%%%
\usepackage{fancyhdr}
\usepackage{eso-pic}         % Pour ajouter des éléments en arrière-plan
% Commande pour ajouter du texte en arrière-plan
\usepackage{tkz-tab}
\AddToShipoutPicture{
    \AtTextCenter{%
        \makebox[0pt]{\rotatebox{80}{\textcolor[gray]{0.7}{\fontsize{5cm}{5cm}\selectfont PGB}}}
    }
}
\usepackage{lastpage}
\fancyhf{}
\pagestyle{fancy}
\renewcommand{\footrulewidth}{1pt}
\renewcommand{\headrulewidth}{0pt}
\renewcommand{\footruleskip}{10pt}
\fancyfoot[R]{
\color{blue}\ding{45}\ \textbf{2025}
}
\fancyfoot[L]{
\color{blue}\ding{45}\ \textbf{Prof:M. BA}
}
\cfoot{\bf
\thepage /
\pageref{LastPage}}
% Création du compteur pour les exercices
\newcounter{exercice}
\renewcommand{\theexercice}{\arabic{exercice}}  % Définit l'affichage du compteur en chiffres arabes

% Définir la commande \exo
\newcommand{\exo}{\refstepcounter{exercice}\textbf{Exercice \theexercice} }
\begin{document}
\renewcommand{\arraystretch}{1.5}
\renewcommand{\arrayrulewidth}{1.2pt}
\begin{tikzpicture}[overlay,remember picture]
    \node[draw=blue,line width=1.2pt,fill=purple,text=blue,inner sep=3mm,rounded corners,pattern=dots]at ([yshift=-2.5cm]current page.north) {\begingroup\setlength{\fboxsep}{0pt}\colorbox{white}{\begin{tabular}{|*1{>{\centering \arraybackslash}p{0.28\textwidth}} |*2{>{\centering \arraybackslash}p{0.2\textwidth}|} *1{>{\centering \arraybackslash}p{0.19\textwidth}|} }
                \hline
                \multicolumn{3}{|c|}{$\diamond$$\diamond$$\diamond$\ \textbf{Lycée de Dindéfélo}\ $\diamond$$\diamond$$\diamond$ } & \textbf{A.S. : 2025/2026}                                              \\ \hline
                \textbf{Matière: Mathématiques}                                                                                    & \textbf{Niveau : 2nd}\textbf{S} & \textbf{Date: 29/12/2025} & \textbf{} \\ \hline
                \multicolumn{4}{|c|}{\parbox[c]{10cm}{\begin{center}
                                                                  \textbf{{\Large\sffamily Td Barycentre}}
                                                              \end{center}}}                                                                                                        \\ \hline
            \end{tabular}}\endgroup};
\end{tikzpicture}
\vspace{3cm}
\begin{multicols}{2}
\setlength{\columnseprule}{0.1mm} % La largeur de la ligne verticale entre les colonnes
\fbox{\textbf{\exo}}
Déterminer deux nombres réels $a$ et $b$ tels que $C$ soit le barycentre du système $\{(A, a); (B, b)\}$ dans des cas suivants :

\begin{enumerate}
    \item $3\overrightarrow{AB} - 2\overrightarrow{CA} = \overrightarrow{CB}$
    \item $\overrightarrow{BA} = \frac{3}{8}\overrightarrow{BC}$
    \item $\overrightarrow{AB} = \frac{1}{4}\overrightarrow{CA}$
    \item $\overrightarrow{AB} = \frac{5}{6}\overrightarrow{BC}$
    \item $2\overrightarrow{AB} + 2\overrightarrow{CB} = \overrightarrow{CA}$
\end{enumerate}

\fbox{\textbf{\exo}}

$A$ et $B$ sont deux points distants de $4$ cm.

\begin{enumerate}
    \item Ecrire comme barycentre des points $A$ et $B$ :
    \begin{enumerate}
        \item Le symétrique $I$ de $A$ par rapport à $B$.
        \item Le symétrique $E$ du milieu $K$ de $[AB]$ par rapport à $A$.
    \end{enumerate}
    \item Déterminer et construire l'ensemble des points $M$ du plan tels que :
    \[ \left\| 3\overrightarrow{MA} - \overrightarrow{MB} \right\| = 6 \]
\end{enumerate}

\fbox{\textbf{\exo}}

Soit $ABC$ un triangle tel que $AB = 4$ et $BC = 5$. On note
\[ I = bar\{(A, 3), (B, -2)\} \]
\[ J = bar\{(B, -2), (C, 5)\} \]

\begin{enumerate}
    \item Construire $I$ et $J$.
    \item Soit $G$ tel que : $3\overrightarrow{GA} - 2\overrightarrow{GB} + 5\overrightarrow{GC} = \overrightarrow{0}$. \\
    Montrer que $G = bar\{(I, 1), (C, 5)\}$.
    \item Déterminer et construire l'ensemble des points $M$ du plan tels que :
    \begin{enumerate}
        \item $\left\| 3\overrightarrow{MA} - 2\overrightarrow{MB} \right\| = \left\| -2\overrightarrow{MB} + 3\overrightarrow{MC} \right\|$.
        \item $\left\| 18\overrightarrow{MA} - 12\overrightarrow{MB} \right\| = \left\| \overrightarrow{MI} + 5\overrightarrow{MC} \right\|$.
    \end{enumerate}
\end{enumerate}

\fbox{\textbf{\exo}}
Soit $ABC$ un triangle tel que : $AB = 10$ ; $AC = 12$ et $BC = 8$ et placer le barycentre $G$ du système $\{(A, 1), (B, 2), (C, 1)\}$.
\begin{enumerate}
    \item Faire la figure que l'on complétera au fur et à mesure.
    \item Déterminer et construire l'ensemble $(E_1)$ des points $M$ du plan tels que : $\left\| \overrightarrow{MA} + 2\overrightarrow{MB} + \overrightarrow{MC} \right\| = AC$.
    \item Soit $(E)$ l'ensemble $(E_2)$ des points $M$ du plan tels que :\\ $\left\| \overrightarrow{MA} + 2\overrightarrow{MB} + \overrightarrow{MC} \right\| = \left\| \overrightarrow{BA} + \overrightarrow{BC} \right\|$.
    \begin{enumerate}
        \item Montrer que $B \in (E)$.
        \item Déterminer et représenter l'ensemble $(E_2)$.
    \end{enumerate}
    \item Déterminer et construire l'ensemble des points $M$ du plan tels que : $\overrightarrow{MA} + 2\overrightarrow{MB} + \overrightarrow{MC}$ soit colinéaire à $\overrightarrow{AB}$.
\end{enumerate}

\fbox{\textbf{\exo}} % Combiné d'après les captures
\begin{enumerate}
    \item Construire le barycentre \\$G$ de $(A, 3)$ et $(B, 3)$\\ $E$ de $(B, 3)$ et $(C, 1)$ \\ $F$ de $(A, 3)$ et $(C, 1)$.
    \item Soit $I$ le barycentre de $(A, 3), (B, 3)$ et $(C, 1)$. Démontrer que :
    \begin{itemize}
        \item Les points $A, I$ et $E$ sont alignés.
        \item Les points $B, I$ et $F$ sont alignés.
        \item Les points $C, I$ et $G$ sont alignés.
    \end{itemize}
    \item Que peut-on en déduire pour les droites $(AE), (BF)$ et $(CG)$ ?
\end{enumerate}

\fbox{\textbf{\exo}}

Soit $ABC$ un triangle. On considère :
\begin{itemize}
    \item Le barycentre $I$ de $(A, 2)$ et $(C, 1)$.
    \item Le barycentre $J$ de $(A, 1)$ et $(B, 2)$.
    \item Le barycentre $K$ de $(C, 1)$ et $(B, -4)$.
\end{itemize}
\begin{enumerate}
    \item Démontrer que $B$ est le barycentre de $(K, 3)$ et $(C, 1)$.
    \item En déduire le barycentre de $(A, 2), (K, 3)$ et $(C, 1)$.
    \item Montrer que $J$ est le milieu de $[IK]$.
\end{enumerate}
\fbox{\textbf{\exo}}
Soit $ABC$ un triangle tel que $AB=6$ cm, $AC=4$ cm et $BC=3$ cm. $G$ le barycentre des points pondérés $(A; 2),(B; 3)$ et $(C; 5)$, $I$ barycentre des points pondérés $(A; 2)$ et $(B; 3)$, $J$ barycentre des points pondérés $(A; 2)$ et $(C; 5)$ et $H$ le milieu du segment $[AC]$.

\begin{enumerate}
    \item Construire les points $I, J$ et $H$.
    \item Montrer que $G = \text{bar} \{(I, 1); (C, 1)\}$.
    \item Montrer que les points $B, J$ et $G$ sont alignés.
    \item Déterminer et construire les ensembles suivants :
    \begin{enumerate}
        \item $(\Delta)$ : l’ensemble des points $M$ du plan vérifiant
        
         $\|2\overrightarrow{MA} + 3\overrightarrow{MB} + 5\overrightarrow{MC}\| = 5\|\overrightarrow{MA} + \overrightarrow{MC}\|$
        \item $(\Gamma)$ : l’ensemble des points $M$ du plan vérifiant
        
         $\|2\overrightarrow{MA} + 3\overrightarrow{MB}\| = \|5\overrightarrow{MA} - 5\overrightarrow{MB}\|$.
    \end{enumerate}
\end{enumerate}
\fbox{\textbf{\exo}}
Soit $ABC$ un triangle tel que

 $AB = 4\text{cm}$, $AC = 2\text{cm}$, $BC = 3\text{cm}$ ; $J$ le barycentre du système $B; 1$, $C; 2$ et $H$ le point tel que $\overrightarrow{AH} = 2\overrightarrow{AB}$.

\begin{enumerate}
    \item Construire le triangle $ABC$ et placer les points $J$ et $H$.
    \item Justifier que le système $\{(B, 1), (C, 2)\}$ admet un barycentre que l'on notera $G$.
    \item Exprimer $\overrightarrow{AG}$ en fonction de $\overrightarrow{AB}$ et $\overrightarrow{AC}$.
    \item Ecrire $H$ comme barycentre des points $A$ et $B$ affectés de coefficients que l'on précisera.
    \item Montrer que les droites $(AJ)$ et $(CH)$ se coupent en $G$.
    \item Déterminer et construire l'ensemble des points $M$ du plan tels que :
    \begin{enumerate}
        \item $\|-\overrightarrow{MA} + 2\overrightarrow{MB} + 4\overrightarrow{MC}\| = 10$
        \item $-\overrightarrow{MA} + 2\overrightarrow{MB} + 4\overrightarrow{MC}$ est colinéaire à $\overrightarrow{AC}$.
    \end{enumerate}
\end{enumerate}
\fbox{\textbf{\exo}}
Soit $ABC$ un triangle tel que $AB=6$ cm, $AC=4$ cm et $BC=3$ cm. $G$ le barycentre des points pondérés $(A; 2),(B; 3)$ et $(C; 5)$, $I$ barycentre des points pondérés $(A; 2)$ et $(B; 3)$, $J$ barycentre des points pondérés $(A; 2)$ et $(C; 5)$ et $H$ le milieu du segment $[AC]$.

\begin{enumerate}
    \item Construire les points $I, J$ et $H$.
    \item Montrer que $G = \text{bar} \{(I, 5); (C, 5)\}$.
    \item Montrer que les points $B, J$ et $G$ sont alignés.
    \item Déterminer et construire les ensembles suivants :
    \begin{enumerate}
        \item $(\Delta)$ : l'ensemble des points $M$ du plan vérifiant
        
         $\|2\overrightarrow{MA} + 3\overrightarrow{MB} + 5\overrightarrow{MC}\| = 5\|\overrightarrow{MA} + \overrightarrow{MC}\|$
        \item $(\Gamma)$ : l'ensemble des points $M$ du plan vérifiant
        
         $\|2\overrightarrow{MA} + 3\overrightarrow{MB}\| = \|5\overrightarrow{MA} - 5\overrightarrow{MB}\|$.
    \end{enumerate}
\end{enumerate}
\fbox{\textbf{\exo}}
Soit $ABC$ un triangle.
\begin{enumerate}
    \item Construire les points $P$, $Q$ et $R$ tels que :
    $\overrightarrow{CP} = \frac{3}{8}\overrightarrow{CA}$ ; $\overrightarrow{AQ} = \frac{1}{4}\overrightarrow{AB}$ et $\overrightarrow{BR} = \frac{5}{6}\overrightarrow{BC}$
    \item Démontrer que les droites $(AR)$, $(BP)$ et $(CQ)$ sont concourantes.
\end{enumerate}

\fbox{\textbf{\exo}}
$ABC$ est un triangle équilatéral et $ABDC$ est un parallélogramme. Construire le point $G$ vérifiant :
$3\overrightarrow{GA} - \overrightarrow{AB} + 2\overrightarrow{AC} = \vec{0}$

\begin{enumerate}
    \item Montrer que $G$ est le barycentre du système $\{(A, 2), (B, -1), (C, 2)\}$.
    \item Soit $I$ milieu de $[AC]$. Démontrer que $G$ est barycentre de $B$ et $I$ affectés de coefficients que l'on déterminera. En déduire que $G$ appartient à la médiatrice de $[AC]$.
    \item En désignant par $H$ l'isobarycentre de $G$ et $D$. Déterminer puis construire l'ensemble $(\Gamma)$ des points $M$ tels que :
    \[ \|2\overrightarrow{MA} - \overrightarrow{MB} + 2\overrightarrow{MC} + 3\overrightarrow{MD}\| = 6AB \]
\end{enumerate}

\fbox{\textbf{\exo}}
Soit $RTS$ un triangle quelconque.
\begin{enumerate}
    \item Construire les points $A$ ; $B$ ; $C$ et $D$ définis par : $D$ est le milieu de $[RS]$ ; $B$ celui de $[RT]$ ; $A$ le symétrique de $D$ par rapport à $R$ et $\overrightarrow{CT} = \frac{1}{4}\overrightarrow{ST}$.
    \item Déterminer les réels $a$ ; $b$ ; $s$ et $t$ tels que :
    \begin{itemize}
        \item[] $R = bar\{(A, a); (S, s)\}$ ;
        \item[] $C = bar\{(T, t); (S, b)\}$
    \end{itemize}
    \item Démontrer que : $B = bar\{(T, 3); (A, 2); (S, 1)\}$
    \item En déduire que les points $A$ ; $B$ et $C$ sont alignés.
\end{enumerate}

\fbox{\textbf{\exo}}
Soit $ABC$ un triangle. $\forall m \in \mathbb{R}$ considérons le point\\ $G_m = bar\{(A, m); (B, 2m); (C, 1)\}$
\begin{enumerate}
    \item Pour quelles valeurs de $m$ le point $G_m$ existe-t-il ?
    \item Soit $D$ barycentre de $(A, 1)$ $(B, 2)$.
    \item Démontrer que $G_m \in (CD)$.
    \item Pour quelles valeurs de $m$, $G_m$ est un point de $[CD]$.
\end{enumerate}

\fbox{\textbf{\exo}}

Soit $ABC$ un triangle isocèle en $A$. $O, I$ et $J$ les milieux respectifs des côtés $[AB]$, $[AC]$ et $[BC]$.

\begin{enumerate}
    \item 
    \begin{enumerate}
        \item Montrer que pour tout $M$ du plan on a : \\
        $\overrightarrow{MA} + \overrightarrow{MB} + \overrightarrow{MC} = \overrightarrow{MO} + \overrightarrow{MI} + \overrightarrow{MJ}$.
        \item En déduire que les triangles $ABC$ et $OIJ$ ont même centre de gravité.
    \end{enumerate}
    \item Montrer que $B$ est le barycentre $\{(J; 1), (O; 1), (I; -1)\}$.
    \item On donne $H = bar\{(J; 3), (C; -1)\}$. \\
    Montrer que les droites $(OH)$ et $(AJ)$ sont parallèles.
    \item Soit $K = bar\{(A; 2), (B; 1), (C; 1)\}$.
    \begin{enumerate}
        \item Construire $K$ puis montrer que est le milieu $[AJ]$.
        \item Montrer que $OAIJ$ est un losange.
    \end{enumerate}
    \item On considère l'ensemble $(E)$ des points $M$ du plan tels que :
    \[ \left\| 2\overrightarrow{AM} + \overrightarrow{BM} + \overrightarrow{CM} + 4\overrightarrow{KI} \right\| = 2\left\| \overrightarrow{MA} + \overrightarrow{MB} \right\|. \]
    \begin{enumerate}
        \item Vérifier que $A \in (E)$.
        \item Montrer que $(E)$ est la médiatrice du segment $[IO]$.
    \end{enumerate}
\end{enumerate}

\vspace{1cm}

\fbox{\textbf{\exo}}

Dans le plan, on considère le triangle $ABC$ isocèle en $A$ de hauteur $AH$ tel que $AH = BC = 4$ cm.

\begin{enumerate}
    \item En justifiant la construction, placer le point $G$ barycentre de $\{(A, 2) ; (B, 1), (C, 1)\}$.
    \item On désigne par $M$ un point quelconque du plan.
    \begin{enumerate}
        \item[a-] Montrer que $\vec{V} = 2\overrightarrow{MA} - \overrightarrow{MB} - \overrightarrow{MC}$ est vecteur constant et $\|\vec{V}\| = 8$.
    \end{enumerate}
    \item Déterminer et construire l'ensemble $\mathbf{F}$ des points $M$ du plan tels que :
    \[ \left\| 2\overrightarrow{MA} + \overrightarrow{MB} + \overrightarrow{MC} \right\| = \|\vec{V}\| \]
    \item On considère le système $\{(A, 2) ; (B, n) ; (C, n)\}$ où $n$ est un entier naturel fixé.
    \begin{enumerate}
        \item Montrer que $G_n$ le barycentre de ce système existe pour tout $n$.
        \item Construire les points $G_0, G_1$ et $G_2$.
        \item Montrer que $G_n$ appartient au segment $[AH]$.
    \end{enumerate}
    \item Soit $\mathbf{E}_n$ l'ensemble des points $M$ du plan tels que :
    \[ \left\| 2\overrightarrow{MA} + n\overrightarrow{MB} + n\overrightarrow{MC} \right\| = n\|\vec{V}\|. \]
    Montrer que $\mathbf{E}_n$ est un cercle qui passe par $A$. \\
    Construire $\mathbf{E}_n$.
\end{enumerate}
\fbox{\textbf{\exo}}
% --- APPROFONDISSEMENT ---

Soit $ABCD$ est un parallélogramme, on construit les points $P, Q$ et $R$ définis par : $\overrightarrow{AP} = \frac{2}{3}\overrightarrow{AB}$, $\overrightarrow{AR} = \frac{3}{4}\overrightarrow{AD}$ et $Q$ est tel que $PARQ$ est un parallélogramme.
\begin{enumerate}
    \item Faire une figure soignée.
    \item Trouver les réels $x_1$ et $x_2$ tels que $P$ soit le barycentre de $\{(A, x_1); (B, x_2)\}$.
    \item Trouver les réels $y_1$ et $y_2$ tels que $R$ soit le barycentre de $\{(A, y_1); (D, y_2)\}$.
    \item Montrer que $(BR)$ et $(DP)$ sont sécantes en $I$ barycentre de $ABD$ dont on précisera.
    \item Prouver que $Q = \{(A, -5); (B, 8); (D, 9)\}$.
    \item Montrer que les points $I, Q$ et $C$ sont alignés.
    \item Préciser la position du point $Q$ sur la droite $(IC)$.
    \item Démontrer que $(BR), (CQ)$ et $(DP)$ sont concourantes.
\end{enumerate}
\fbox{\textbf{\exo}}
\begin{enumerate}
    \item Construire un triangle $ABC$ tels que :
    
     $AC = 12$, $AB = 10$ et $BC = 8$ et placer le barycentre $G$ du système de points pondérés $\{(A; 1), (B; 2), (C; 1)\}$.
    
    \item Déterminer et représenter l'ensemble des points $M$ tels que :
     
    $\|\overrightarrow{MA} + 2\overrightarrow{MB} + \overrightarrow{MC}\| = AC$.
    
    \item Soit $\mathcal{E}$ l'ensemble des points $N$ tels que :
     
    $\|\overrightarrow{NA} + 2\overrightarrow{NB} + \overrightarrow{NC}\| = \|\overrightarrow{BA} + \overrightarrow{BC}\|$.
    \begin{enumerate}
        \item Montrer que $B$ appartient à $\mathcal{E}$.
        \item Déterminer et représenter l'ensemble $\mathcal{E}$.
    \end{enumerate}
    
    \item Déterminer et représenter l'ensemble des points $P$ du plan tels que :
    
    $\|\overrightarrow{PA} + 2\overrightarrow{PB} + \overrightarrow{PC}\| = \|3\overrightarrow{PA} + \overrightarrow{PC}\|$.
\end{enumerate}
\fbox{\textbf{\exo}}

Soit $A, B$ et $C$ trois points du plan, $a, b$ et $c$ trois réels tels que : $a + b + c \neq 0$. Soit $G$ le barycentre de $(A, a), (B, b)$ et $(C, c)$.

\begin{enumerate}
    \item Démontrer que le système 
    
    $\{(A, 2a + 1), (B, 2b - 2), (C, 2c + 1)\}$ admet un barycentre que l'on notera $K$.
    \item Déterminer une relation vectorielle qui définit $K$.
    \item En déduire que : 
    
    $a\overrightarrow{KA} + b\overrightarrow{KB} + c\overrightarrow{KC} = \frac{\overrightarrow{AB} + \overrightarrow{AC}}{2}$.
    \item En déduire que $G$ et $K$ sont confondus si $B$ est le milieu de $[AC]$.
    \item En supposant que les points $A, B$ et $C$ ne soient pas alignés. Soit $E$ le point vérifiant que $ABCE$ soit un parallélogramme.
    \begin{enumerate}
        \item Montrer que : $\overrightarrow{GK} = \frac{\overrightarrow{BE}}{2(a + b + c)}$ en utilisant la question 2).
        \item Construire $G$ et $K$ pour $a = c = \frac{1}{2}$ et $b = 2$.
    \end{enumerate}
\end{enumerate}

\fbox{\textbf{\exo}}
Soit $A, B$ et $C$ trois points non alignés, $\alpha, \beta$ et $\gamma$ trois réels vérifiant la condition d'existence des barycentres suivants : \\
$G = \text{bar}\{(A; \alpha), (B; \beta), (C; \gamma)\}$ \\
$G_1 = \text{bar}\{(A; -\alpha), (B; \beta), (C; \gamma)\}$ \\
$G_2 = \text{bar}\{(A; \alpha), (B; -\beta), (C; \gamma)\}$ \\
$G_3 = \text{bar}\{(A; \alpha), (B; \beta), (C; -\gamma)\}$ \\
Démontrer que les droites $(AG_1)$, $(BG_2)$ et $(CG_3)$ sont concourantes en $G$.

\fbox{\textbf{\exo}}

\noindent Soit $ABC$ un triangle quelconque.

\begin{enumerate}
    \item \begin{enumerate}
    \item Placer les points $I$ et $J$ tels que :
    \[ \overrightarrow{AI} = \frac{1}{3} \overrightarrow{AB} \quad \text{et} \quad \overrightarrow{BJ} = \frac{3}{4} \overrightarrow{BC}. \]
    \item Ecrire le point $I$ comme barycentre de $A$ et $B$ et $J$ comme barycentre de $B$ et $C$ en précisant les coefficients des systèmes.
    \end{enumerate}
    \item
\begin{enumerate}
    \item Placer sur la figure le point 
    
    $K = \text{bar} \{(A; -2), (B; -3)\}$.
    
    \item Soit $G$ le barycentre du système $\{(A; -2), (B; -1), (C; -3)\}$. Montrer que les droites $(AJ)$, $(CI)$ et $(BK)$ sont concourantes en $G$.
\end{enumerate}

    \item Déterminer et représenter l'ensemble des points $M$ du plan tels que :
    \begin{enumerate}
        \item $(\mathcal{C}) : 2\overrightarrow{MA} + \overrightarrow{MB}$ soit colinéaire à $\overrightarrow{BC}$.
        \item $(\mathcal{F}) : \| 2\overrightarrow{MA} + 3\overrightarrow{MC} \| = 5MA$.
        \item $(\mathcal{G}) : \overrightarrow{MB} + 3\overrightarrow{MC}$ soit orthogonal à $\overrightarrow{AB}$.
        \item $(\mathcal{H}) : \| 2\overrightarrow{MA} + \overrightarrow{MB} + 3\overrightarrow{MC} \| = 6AB$.
        \item $(\mathcal{L}) : 6 \le \| 2\overrightarrow{MA} + \overrightarrow{MB} + 3\overrightarrow{MC} \| \le 6\sqrt{2}$.
    \end{enumerate}
\end{enumerate}

\end{multicols}

\end{document}