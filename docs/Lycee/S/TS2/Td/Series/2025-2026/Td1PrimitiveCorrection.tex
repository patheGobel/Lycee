\documentclass[12pt,a4paper]{article}
\usepackage[T1]{fontenc}
\usepackage{amsmath,amssymb,mathrsfs,tikz,times,pifont}
\usepackage{enumitem}
\usepackage{multicol}
\usepackage{lmodern}
\usetikzlibrary{trees}
\newcommand\circitem[1]{%
\tikz[baseline=(char.base)]{
\node[circle,draw=gray, fill=red!55,
minimum size=1.2em,inner sep=0] (char) {#1};}}
\newcommand\boxitem[1]{%
\tikz[baseline=(char.base)]{
\node[fill=cyan,
minimum size=1.2em,inner sep=0] (char) {#1};}}
\setlist[enumerate,1]{label=\protect\circitem{\arabic*}}
\setlist[enumerate,2]{label=\protect\boxitem{\alph*}}
%%%::::::by chnini ameur :::::::%%%
\everymath{\displaystyle}
\usepackage[left=1cm,right=1cm,top=1cm,bottom=1.7cm]{geometry}
\usepackage[colorlinks=true, linkcolor=blue, urlcolor=blue, citecolor=blue]{hyperref}
\usepackage{array,multirow}
\usepackage[most]{tcolorbox}
\usepackage{varwidth}
\usepackage{float} %pour utiliser l'option [H] qui force l'image à apparaître exactement à l'endroit où elle est placée dans le code.
\tcbuselibrary{skins,hooks}
\usetikzlibrary{patterns}
%%%::::::by chnini ameur :::::::%%%
\newtcolorbox{exa}[2][]{enhanced,breakable,before skip=2mm,after skip=5mm,
colback=yellow!20!white,colframe=black!20!blue,boxrule=0.5mm,
attach boxed title to top left ={xshift=0.6cm,yshift*=1mm-\tcboxedtitleheight},
fonttitle=\bfseries,
title={#2},#1,
% varwidth boxed title*=-3cm,
boxed title style={frame code={
\path[fill=tcbcolback!30!black]
([yshift=-1mm,xshift=-1mm]frame.north west)
arc[start angle=0,end angle=180,radius=1mm]
([yshift=-1mm,xshift=1mm]frame.north east)
arc[start angle=180,end angle=0,radius=1mm];
\path[left color=tcbcolback!60!black,right color = tcbcolback!60!black,
middle color = tcbcolback!80!black]
([xshift=-2mm]frame.north west) -- ([xshift=2mm]frame.north east)
[rounded corners=1mm]-- ([xshift=1mm,yshift=-1mm]frame.north east)
-- (frame.south east) -- (frame.south west)
-- ([xshift=-1mm,yshift=-1mm]frame.north west)
[sharp corners]-- cycle;
},interior engine=empty,
},interior style={top color=yellow!5}}
%%%%%%%%%%%%%%%%%%%%%%%
\usepackage{fancyhdr}
\usepackage{eso-pic}         % Pour ajouter des éléments en arrière-plan
% Commande pour ajouter du texte en arrière-plan
\usepackage{tkz-tab}
\AddToShipoutPicture{
    \AtTextCenter{%
        \makebox[0pt]{\rotatebox{80}{\textcolor[gray]{0.7}{\fontsize{5cm}{5cm}\selectfont PGB}}}
    }
}
\usepackage{lastpage}
\fancyhf{}
\pagestyle{fancy}
\renewcommand{\footrulewidth}{1pt}
\renewcommand{\headrulewidth}{0pt}
\renewcommand{\footruleskip}{10pt}
\fancyfoot[R]{
\color{blue}\ding{45}\ \textbf{2025}
}
\fancyfoot[L]{
\color{blue}\ding{45}\ \textbf{Prof:M. BA}
}
\cfoot{\bf
\thepage /
\pageref{LastPage}}
% Création du compteur pour les exercices
\newcounter{exercice}
\renewcommand{\theexercice}{\arabic{exercice}}  % Définit l'affichage du compteur en chiffres arabes

% Définir la commande \exo
\newcommand{\exo}{\refstepcounter{exercice}\textbf{Exercice \theexercice} }
\begin{document}
\renewcommand{\arraystretch}{1.5}
\renewcommand{\arrayrulewidth}{1.2pt}
\begin{tikzpicture}[overlay,remember picture]
    \node[draw=blue,line width=1.2pt,fill=purple,text=blue,inner sep=3mm,rounded corners,pattern=dots]at ([yshift=-2.5cm]current page.north) {\begingroup\setlength{\fboxsep}{0pt}\colorbox{white}{\begin{tabular}{|*1{>{\centering \arraybackslash}p{0.28\textwidth}} |*2{>{\centering \arraybackslash}p{0.2\textwidth}|} *1{>{\centering \arraybackslash}p{0.19\textwidth}|} }
                \hline
                \multicolumn{3}{|c|}{$\diamond$$\diamond$$\diamond$\ \textbf{Lycée de Dindéfélo}\ $\diamond$$\diamond$$\diamond$ } & \textbf{A.S. : 2025/2026}                                              \\ \hline
                \textbf{Matière: Mathématiques}                                                                                    & \textbf{Niveau : T}\textbf{S2} & \textbf{Date: 25/12/2025} & \textbf{} \\ \hline
                \multicolumn{4}{|c|}{\parbox[c]{10cm}{\begin{center}
                                                                  \textbf{{\Large\sffamily Correction Td Primitives}}
                                                              \end{center}}}                                                                                                        \\ \hline
            \end{tabular}}\endgroup};
\end{tikzpicture}
\vspace{3cm}
\begin{multicols}{2}
\setlength{\columnseprule}{0.1mm} % La largeur de la ligne verticale entre les colonnes
\fbox{\textbf{\exo}} 
\begin{enumerate}
    \item Fonctions polynomiales et puissances simples

Dans cette section, on utilise principalement la formule de base : 
$ \int x^n \, dx = \dfrac{x^{n+1}}{n+1} + C \quad (\text{pour } n \neq -1) $

\begin{enumerate}
    \item $f(x) = x^3 - 2x + 1$ \\
    $F(x) = \dfrac{x^4}{4} - 2\left(\dfrac{x^2}{2}\right) + x + C$ \\
    $\mathbf{F(x) = \dfrac{1}{4}x^4 - x^2 + x + C}$

    \item $f(x) = (x-1)(x-2) = x^2 - 3x + 2$ \\
    $F(x) = \dfrac{x^3}{3} - 3\left(\dfrac{x^2}{2}\right) + 2x + C$ \\
    $\mathbf{F(x) = \dfrac{1}{3}x^3 - \dfrac{3}{2}x^2 + 2x + C}$

    \item $f(x) = 4x^4 - 2x^2 + 5x$ \\
    $F(x) = 4\left(\dfrac{x^5}{5}\right) - 2\left(\dfrac{x^3}{3}\right) + 5\left(\dfrac{x^2}{2}\right) + C$ \\
    $\mathbf{F(x) = \dfrac{4}{5}x^5 - \dfrac{2}{3}x^3 + \dfrac{5}{2}x^2 + C}$

    \item $f(x) = x + \dfrac{1}{x^2} = x + x^{-2}$ \\
    $F(x) = \dfrac{x^2}{2} + \dfrac{x^{-1}}{-1} + C$ \\
    $\mathbf{F(x) = \dfrac{1}{2}x^2 - \dfrac{1}{x} + C}$

    \item $f(x) = (x+1)^3$ \\
    On peut développer ou utiliser la forme $(ax+b)^n$. \\
    $\mathbf{F(x) = \dfrac{1}{4}(x+1)^4 + C}$

    \item $f(x) = x + \dfrac{1}{\sqrt{x}} = x + x^{-1/2}$ \\
    $F(x) = \dfrac{x^2}{2} + \dfrac{x^{1/2}}{1/2} + C$ \\
    $\mathbf{F(x) = \dfrac{1}{2}x^2 + 2\sqrt{x} + C}$

    \item $f(x) = \dfrac{2}{x^2} - \dfrac{3}{x^3} = 2x^{-2} - 3x^{-3}$ \\
    $F(x) = 2\left(\dfrac{x^{-1}}{-1}\right) - 3\left(\dfrac{x^{-2}}{-2}\right) + C$ \\
    $\mathbf{F(x) = -\dfrac{2}{x} + \dfrac{3}{2x^2} + C}$

    \item $f(x) = \dfrac{x^4+x^2+1}{x^2} = \dfrac{x^4}{x^2} + \dfrac{x^2}{x^2} + \dfrac{1}{x^2} = x^2 + 1 + x^{-2}$ \\
    $F(x) = \dfrac{x^3}{3} + x - \dfrac{1}{x} + C$ \\
    $\mathbf{F(x) = \dfrac{1}{3}x^3 + x - \dfrac{1}{x} + C}$
\end{enumerate}
\item Forme $u'u^n$\textit{Formule : $\int u'u^n = \frac{u^{n+1}}{n+1}$}
\begin{enumerate}
    \item $f(x) = (2x-1)(x^2-x)^2$
    \item $f(x) = 2x(x^2-1)^5$
    \item $f(x) = \sin^2 x \cos x$
    \item $f(x) = \sin x \cos^3 x$ %\textit{(forme $-u'u^3$)}
\end{enumerate}

\item Forme $\dfrac{u'}{u^n}$ ($n \geq 2$)\\ \textit{Formule : $\int \frac{u'}{u^n} = -\frac{1}{(n-1)u^{n-1}}$}

\begin{enumerate}
    \item $f(x) = \dfrac{1}{(x+1)^3}$
    \item $f(x) = \dfrac{2x+1}{(x^2+x+1)^2}$
    \item $f(x) = \dfrac{2x}{(x^2+1)^2}$
    \item $f(x) = \dfrac{4x+3}{(2x^2+3x+1)^3}$
    \item $f(x) = \dfrac{1-x}{(x^2-2x+3)^2}$ %\textit{(forme $-\frac{1}{2}\frac{u'}{u^2}$)}
    \item $f(x) = \dfrac{\cos x}{\sin^2 x}$
\end{enumerate}

\item Forme $\dfrac{u'}{\sqrt{u}}$\\ \textit{Formule : $\int \frac{u'}{\sqrt{u}} = 2\sqrt{u}$}
\begin{enumerate}
    \item $f(x) = \dfrac{1}{\sqrt{x+1}}$
    \item $f(x) = \dfrac{3x}{\sqrt{x^2+1}}$
    \item $f(x) = \dfrac{x}{\sqrt{x^2-1}}$
    \item $f(x) = \dfrac{2x+3}{\sqrt{x^2+3x+2}}$
\end{enumerate}

\item Fonctions Trigonométriques\\ \textit{Formes : $\sin(ax+b)$, $u'\cos(u)$, linéarisation ou dérivées de $\tan(x)$}
\begin{enumerate}
    \item $f(x) = 3\sin\dfrac{\pi x}{2}$
    \item $f(x) = \sin 3x + \cos(2x+3)$
    \item $f(x) = \sin 2x - 2\cos 2x$
    \item $f(x) = x \cos x^2$ %\textit{(forme $\frac{1}{2}u'\cos u$)}
    \item $f(x) = \dfrac{\cos \sqrt{x}}{\sqrt{x}}$ %\textit{(forme $2u'\cos u$)}
    \item $f(x) = \dfrac{1}{x^2} \sin \dfrac{1}{x}$ %\textit{(forme $-u'\sin u$)}
    \item $f(x) = \dfrac{\tan x}{\cos^2 x}$ %\textit{(forme $u'u$ avec $u=\tan x$)}
    \item $f(x) = \tan^2 x$ %\textit{(à écrire $(1+\tan^2 x) - 1$)}
    \item $f(x) = \tan x + \tan^3 x$ %\textit{(forme $\tan x(1+\tan^2 x)$)}
    \item $f(x) = 1 + \dfrac{1}{\tan^2 x}$
    \item $f(x) = \sin^2 x$ %\textit{(linéarisation)}
    \item $f(x) = \cos^3 x$ %\textit{(linéarisation)}
\end{enumerate}

\end{enumerate}

\end{multicols}

\begin{enumerate}
\item \textit{Dans chacun des cas suivants, déterminer une primitive $F$ de $f$ sur $I$ après avoir effectuée la transformation d'écriture indiquée.}

\begin{enumerate}
    \item $f(x) = \dfrac{x^2 - 2x}{(x - 1)^2} \quad I = ]1 ; +\infty[ \quad \text{Indication : Mettre } f(x) \text{ sous la forme } a + \dfrac{b}{(x - 1)^2}$
    
    \item $f(x) = \dfrac{3x^2 + 12x - 1}{(x + 2)^2}, \quad I = ]-2 ; +\infty[ \quad \text{Indication : Mettre } f(x) \text{ sous la forme } a + \dfrac{b}{(x + 2)^2}$
    
    \item $f(x) = \dfrac{2x^3 + 13x^2 + 24x + 2}{(x + 3)^2}, \quad I = ]-3 ; +\infty[$ \\
    \textit{Indication : Mettre $f(x)$ sous la forme $ax + b + \dfrac{c}{(x + 3)^2}$}
    
    \item $f(x) = \dfrac{x(x^2 + 3)}{(x^2 - 1)^3}, \quad I = ]-1 ; 1[ \quad \text{Indication : Mettre } f(x) \text{ sous la forme } \dfrac{a}{(x - 1)^3} + \dfrac{b}{(x + 1)^3}$
\end{enumerate}

\begin{comment}

 \item  \textit{Déterminer une primitive $F$ de la fonction $f$ définie sur $K$.}
\begin{enumerate}


    \item $f(x) = \dfrac{\ln x}{x} \quad K = ]0 ; +\infty[$
    
    \item $f(x) = \dfrac{1}{x \ln x} \quad K = ]0 ; +\infty[$
    
    \item $f(x) = |x| + |x - 1|, \quad K = \mathbb{R}$
    
    \item $f(x) = \dfrac{2x - 1}{x^2 (x - 1)^2} \quad K = ]0 ; 1[$ \\
    \textit{Indication : Mettre $f(x)$ sous la forme $\dfrac{a}{x^2} + \dfrac{b}{(x - 1)^2}$}
    
    \item $f(x) = \dfrac{2x + 1}{x^2 - 2x + 1} \quad K = ]1 ; +\infty[$ \\
    \textit{Indication : Mettre $f(x)$ sous la forme $\dfrac{a}{x - 1} + \dfrac{b}{(x - 1)^2}$}
    
    \item $f(x) = \dfrac{2x^3 + x^2 - 2x + 1}{x^2 - 1} \quad K = ]1 ; +\infty[$ \\
    \textit{Indication : Mettre $f(x)$ sous la forme $ax + b + \dfrac{c}{x + 1} + \dfrac{d}{x - 1}$}

\end{enumerate}

\end{comment}

\end{enumerate}

\end{document}



\begin{comment}
\begin{multicols}{2}
\setlength{\columnseprule}{0.1mm} % La largeur de la ligne verticale entre les colonnes
    \begin{enumerate}
        \item 
    \end{enumerate}
\end{multicols}
\end{comment}