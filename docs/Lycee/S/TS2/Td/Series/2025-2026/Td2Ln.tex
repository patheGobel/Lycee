\documentclass[12pt,a4paper]{article}
\usepackage[T1]{fontenc}
\usepackage{amsmath,amssymb,mathrsfs,tikz,times,pifont}
\usepackage{enumitem}
\usepackage{multicol}
\usepackage{lmodern}
\usetikzlibrary{trees}
\newcommand\circitem[1]{%
\tikz[baseline=(char.base)]{
\node[circle,draw=gray, fill=red!55,
minimum size=1.2em,inner sep=0] (char) {#1};}}
\newcommand\boxitem[1]{%
\tikz[baseline=(char.base)]{
\node[fill=cyan,
minimum size=1.2em,inner sep=0] (char) {#1};}}
\setlist[enumerate,1]{label=\protect\circitem{\arabic*}}
\setlist[enumerate,2]{label=\protect\boxitem{\alph*}}
%%%::::::by chnini ameur :::::::%%%
\everymath{\displaystyle}
\usepackage[left=1cm,right=1cm,top=1cm,bottom=1.7cm]{geometry}
\usepackage[colorlinks=true, linkcolor=blue, urlcolor=blue, citecolor=blue]{hyperref}
\usepackage{array,multirow}
\usepackage[most]{tcolorbox}
\usepackage{varwidth}
\usepackage{float} %pour utiliser l'option [H] qui force l'image à apparaître exactement à l'endroit où elle est placée dans le code.
\tcbuselibrary{skins,hooks}
\usetikzlibrary{patterns}
%%%::::::by chnini ameur :::::::%%%
\newtcolorbox{exa}[2][]{enhanced,breakable,before skip=2mm,after skip=5mm,
colback=yellow!20!white,colframe=black!20!blue,boxrule=0.5mm,
attach boxed title to top left ={xshift=0.6cm,yshift*=1mm-\tcboxedtitleheight},
fonttitle=\bfseries,
title={#2},#1,
% varwidth boxed title*=-3cm,
boxed title style={frame code={
\path[fill=tcbcolback!30!black]
([yshift=-1mm,xshift=-1mm]frame.north west)
arc[start angle=0,end angle=180,radius=1mm]
([yshift=-1mm,xshift=1mm]frame.north east)
arc[start angle=180,end angle=0,radius=1mm];
\path[left color=tcbcolback!60!black,right color = tcbcolback!60!black,
middle color = tcbcolback!80!black]
([xshift=-2mm]frame.north west) -- ([xshift=2mm]frame.north east)
[rounded corners=1mm]-- ([xshift=1mm,yshift=-1mm]frame.north east)
-- (frame.south east) -- (frame.south west)
-- ([xshift=-1mm,yshift=-1mm]frame.north west)
[sharp corners]-- cycle;
},interior engine=empty,
},interior style={top color=yellow!5}}
%%%%%%%%%%%%%%%%%%%%%%%
\usepackage{fancyhdr}
\usepackage{eso-pic}         % Pour ajouter des éléments en arrière-plan
% Commande pour ajouter du texte en arrière-plan
\usepackage{tkz-tab}
\AddToShipoutPicture{
    \AtTextCenter{%
        \makebox[0pt]{\rotatebox{80}{\textcolor[gray]{0.7}{\fontsize{5cm}{5cm}\selectfont PGB}}}
    }
}
\usepackage{lastpage}
\fancyhf{}
\pagestyle{fancy}
\renewcommand{\footrulewidth}{1pt}
\renewcommand{\headrulewidth}{0pt}
\renewcommand{\footruleskip}{10pt}
\fancyfoot[R]{
\color{blue}\ding{45}\ \textbf{2025}
}
\fancyfoot[L]{
\color{blue}\ding{45}\ \textbf{Prof:M. BA}
}
\cfoot{\bf
\thepage /
\pageref{LastPage}}
% Création du compteur pour les exercices
\newcounter{probleme}
\renewcommand{\theprobleme}{\arabic{probleme}}  % Définit l'affichage du compteur en chiffres arabes

% Définir la commande \exo
\newcommand{\exo}{\refstepcounter{probleme}\textbf{Problème \theprobleme} }
\begin{document}
\renewcommand{\arraystretch}{1.5}
\renewcommand{\arrayrulewidth}{1.2pt}
\begin{tikzpicture}[overlay,remember picture]
    \node[draw=blue,line width=1.2pt,fill=purple,text=blue,inner sep=3mm,rounded corners,pattern=dots]at ([yshift=-2.5cm]current page.north) {\begingroup\setlength{\fboxsep}{0pt}\colorbox{white}{\begin{tabular}{|*1{>{\centering \arraybackslash}p{0.28\textwidth}} |*2{>{\centering \arraybackslash}p{0.2\textwidth}|} *1{>{\centering \arraybackslash}p{0.19\textwidth}|} }
                \hline
                \multicolumn{3}{|c|}{$\diamond$$\diamond$$\diamond$\ \textbf{Lycée de Dindéfélo}\ $\diamond$$\diamond$$\diamond$ } & \textbf{A.S. : 2025/2026}                                              \\ \hline
                \textbf{Matière: Mathématiques}                                                                                    & \textbf{Niveau : T}\textbf{S2} & \textbf{Date: 29/12/2025} & \textbf{} \\ \hline
                \multicolumn{4}{|c|}{\parbox[c]{10cm}{\begin{center}
                                                                  \textbf{{\Large\sffamily Td Limites ln}}
                                                              \end{center}}}                                                                                                        \\ \hline
            \end{tabular}}\endgroup};
\end{tikzpicture}

\vspace{3cm}
\fbox{\textbf{\exo}}

\underline{\textbf{Partie A}}
Soit $g$ la fonction numérique définie sur $]0; +\infty[$ par $g(x) = x(1 + \ln x)^2 - 1$.
\begin{enumerate}
    \item On admettra que $g$ est dérivable sur $]0; +\infty[$.
    \begin{enumerate}
        \item Calculer $g'(x)$ pour tout $x \in ]0; +\infty[$.
        \item Vérifier que pour tout $x \in ]0; +\infty[$, $g'(x) = (1 + \ln x)(3 + \ln x)$.
        \item Étudier le signe de $g'(x)$ suivant les valeurs de $x$ et en déduire le sens de variation de $g$.
    \end{enumerate}
    \item Dresser le tableau de variation de $g$. (On ne calculera pas les limites en $0$ et $+\infty$).
    \item 
    \begin{enumerate}
        \item Calculer $g(1)$.
        \item En déduire que pour tout $x \in ]0; 1[$, $g(x) < 0$ et pour tout $x \in ]1; +\infty[$, $g(x) > 0$.
    \end{enumerate}
\end{enumerate}

\underline{\textbf{Partie B}}
Soit $f$ la fonction définie par :
\[
\begin{cases} 
f(x) = x + \frac{1}{1 + \ln x} - 1 & \text{si } x \in ]0; \frac{1}{e}[ \cup ]\frac{1}{e}; +\infty[ \\
f(0) = -1
\end{cases}
\]
$(C)$ est la représentation graphique dans le repère orthonormé $(O, I, J)$ (unités : 2cm).
\begin{enumerate}
    \item 
    \begin{enumerate}
        \item Montrer que $f$ est continue en 0.
        \item Étudier la dérivabilité de $f$ en 0.
        \item En déduire la tangente à $(C)$ au point $A(0; -1)$.
    \end{enumerate}
    \item 
    \begin{enumerate}
        \item Calculer $\lim\limits_{x \to \frac{1}{e}^-} f(x)$, $\lim\limits_{x \to \frac{1}{e}^+} f(x)$ et $\lim\limits_{x \to +\infty} f(x)$.
        \item Montrer que la droite $(\Delta)$ d'équation $y = x - 1$ est une asymptote à $(C)$.
        \item Étudier les positions relatives de $(C)$ et $(\Delta)$.
        \item Préciser l'autre asymptote à la courbe $(C)$ de $f$.
    \end{enumerate}
    \item 
    \begin{enumerate}
        \item Montrer que pour tout $x \in ]0; +\infty[ \setminus \{\frac{1}{e}\}$, $f'(x) = \frac{g(x)}{x(1 + \ln x)^2}$.
        \item En déduire le sens de variation de $f$.
        \item Dresser son tableau de variation.
    \end{enumerate}
\end{enumerate}

\underline{\textbf{Partie C}}
\begin{enumerate}
    \item Calculer la dérivée de la fonction $h$ définie sur $]\frac{1}{e}; +\infty[$ par $h(x) = \ln(1 + \ln x)$.
    \item 
    \begin{enumerate}
        \item En déduire les primitives sur $]\frac{1}{e}; +\infty[$ de la fonction $k : x \mapsto \frac{f(x)}{x}$.
        \item Déterminer la primitive de $k$ qui prend la valeur $-1$ en 1.
    \end{enumerate}
\end{enumerate}

\fbox{\textbf{\exo}}

Le plan est muni d'un repère orthonormé $(O, I, J)$ d'unité graphique 2cm.

\underline{\textbf{Partie A : Étude d'une fonction auxiliaire}}

Soit $g$ la fonction définie sur l'intervalle $]0; +\infty[$ par $g(x) = x^2 - 1 + \ln x$.
\begin{enumerate}
    \item Calculer $g'(x)$ pour tout réel $x$ appartenant à l'intervalle $]0; +\infty[$. \\
    En déduire le sens de variation de la fonction $g$ sur l'intervalle $]0; +\infty[$.
    \item Calculer $g(1)$ et en déduire l'étude du signe de $g(x)$ pour $x$ appartenant à $]0; +\infty[$.
\end{enumerate}

\underline{\textbf{Partie B : Détermination de l'expression de la fonction $f$}}

On admet qu'il existe deux constantes réelles $a$ et $b$ telles que, pour tout nombre réel $x$ appartenant à $]0; +\infty[$, $f(x) = ax + b - \frac{\ln x}{x}$.
\begin{enumerate}
    \item On désigne par $f'$ la fonction dérivée de la fonction $f$. \\
    Calculer $f'(x)$ pour tout réel $x$ appartenant à l'intervalle $]0; +\infty[$.
    \item Sachant que la courbe $(C)$ passe par le point de coordonnées $(1; 0)$ et qu'elle admet en ce point une tangente horizontale, déterminer les nombres $a$ et $b$.
\end{enumerate}

\underline{\textbf{Partie C : Étude de la fonction $f$}}

On admet désormais que, pour tout nombre réel $x$ appartenant à l'intervalle $]0; +\infty[$, $f(x) = x - 1 - \frac{\ln x}{x}$.
\begin{enumerate}
    \item 
    \begin{enumerate}
        \item Déterminer la limite de la fonction $f$ en $0$ et donner une interprétation graphique de cette limite.
        \item Déterminer la limite de la fonction $f$ en $+\infty$.
    \end{enumerate}
    \item 
    \begin{enumerate}
        \item Vérifier que, pour tout réel $x$ appartenant à l'intervalle $]0; +\infty[$, $f'(x) = \frac{g(x)}{x^2}$.
        \item Établir le tableau de variation de la fonction $f$ sur l'intervalle $]0; +\infty[$.
        \item En déduire le signe de $f(x)$ pour $x$ appartenant à l'intervalle $]0; +\infty[$.
    \end{enumerate}
    \item On considère la droite $(D)$ d'équation $y = x - 1$.
    \begin{enumerate}
        \item Justifier que la droite $(D)$ est asymptote à la courbe $(C)$.
        \item Étudier les positions relatives de la courbe $(C)$ et de la droite $(D)$.
        \item Tracer la droite $(D)$ et la courbe $(C)$ dans le plan $P$ muni du repère $(O; \vec{i}, \vec{j})$.
    \end{enumerate}
\end{enumerate}

\underline{\textbf{Partie D : Calcul d'aire}}

On note $A$ la mesure, exprimée en $cm^2$, de l'aire de la partie du plan $P$ comprise entre la courbe $C$, l'axe des abscisses, et les droites d'équations $x = 1$ et $x = e$.
\begin{enumerate}
    \item On considère la fonction $H$ définie sur l'intervalle $]0; +\infty[$ par $H(x) = (\ln x)^2$. \\
    On désigne par $H'$ la fonction dérivée de la fonction $H$.
    \begin{enumerate}
        \item Calculer $H'(x)$ pour tout réel $x$ appartenant à l'intervalle $]0; +\infty[$.
        \item En déduire une primitive de la fonction $f$ sur l'intervalle $]0; +\infty[$.
    \end{enumerate}
    \item Calculer $A$ et donner sa valeur arrondie au $mm^2$ près.
\end{enumerate}

\fbox{\textbf{\exo}}

\underline{\textbf{Partie A}}
On considère $g$ la fonction numérique de la variable réelle $x$ définie par :
\[ g(x) = 1 + x(2\ln|x| + 1) \]

\begin{enumerate}
    \item 
    \begin{enumerate}
        \item Justifier que $g$ est définie sur $]-\infty; 0[ \cup ]0; +\infty[$.
        \item Déterminer les limites de $g$ aux bornes de son ensemble de définition.
    \end{enumerate}
    \item 
    \begin{enumerate}
        \item Étudier le sens de variation de $g$.
        \item Dresser le tableau de variation de $g$.
    \end{enumerate}
    \item 
    \begin{enumerate}
        \item Calculer l'image de $-1$ par $g$.
        \item Déterminer l'image $J$ par $g$ de l'intervalle $I$ tel que : $I = ]-\infty; -e^{-\frac{3}{2}}]$.
        \item Démontrer que la restriction $h$ de $g$ sur l'intervalle $I$ est une bijection de $I$ sur $J$.
        \item En déduire l'ensemble des solutions de l'équation : $x \in \mathbb{R}, g(x) = 0$.
    \end{enumerate}
    \item Déduire de tout ce qui précède que : \\
    $\forall x \in ]-\infty; -1[, g(x) < 0$ et $\forall x \in ]-1; 0[ \cup ]0; +\infty[, g(x) > 0$.
\end{enumerate}

\underline{\textbf{Partie B}}
On considère $f$ la fonction numérique de la fonction de la variable réelle $x$ définie par :
\[
\begin{cases}
f(x) = x(x\ln|x| + 1) & \text{si } x \neq 0 \\
f(0) = 0
\end{cases}
\]
et $(C)$ sa courbe représentative dans le plan muni du repère orthonormé $(O; \vec{i}, \vec{j})$. (Unité : 5cm)

\begin{enumerate}
    \item Démontrer que $f$ est continue en 0.
    \item 
    \begin{enumerate}
        \item Donner l'ensemble de définition de $f'$ et déterminer les limites de $f$ aux bornes de son ensemble de définition.
        \item Étudier la dérivabilité de $f$ en 0.
        \item Déterminer la fonction dérivée $f'$ et déterminer le tableau de variation de $f$.
    \end{enumerate}
    \item 
    \begin{enumerate}
        \item Écrire une équation de la tangente $(D)$ à la courbe $(C)$ au point $O$.
        \item Démontrer que $(D)$ coupe $(C)$ en deux points $E$ et $F$ et calculer leurs coordonnées.
        \item Étudier la position de $(C)$ par rapport à $(D)$.
    \end{enumerate}
    \item Démontrer que $(C)$ coupe l'axe $(OI)$ en un point $K$ d'abscisse $\beta$ tel que $-1,8 < \beta < -1,7$.
    \item Construire $(C)$.
\end{enumerate}

\underline{\textbf{Partie C}}
\begin{enumerate}
    \item Soit $\alpha$ un réel appartenant à $]0; 1[$. \\
    À l'aide d'une intégration par parties, calculer $\int_{\alpha}^{1} x^2 \ln(x) \, dx$.
    \item 
    \begin{enumerate}
        \item Calculer l'aire $A(\alpha)$ de la partie du plan limitée par $(C)$, la droite $(D)$ et les droites d'équation $x = 1$ et $x = \alpha$.
        \item Calculer $\lim\limits_{\alpha \to 0} A(\alpha)$.
    \end{enumerate}
\end{enumerate}

\begin{center}
    On prendra : $\ln(2) \approx 0,7$ ; $\ln(3) \approx 1,1$ ; $\ln(5) \approx 1,6$ ; $\ln(17) \approx 2,9$ ; $e \approx 2,7$ ; $\sqrt{e} \approx 1,6$.
\end{center}
\fbox{\textbf{\exo}}

\noindent \underline{\textbf{Partie A}} \\
On considère la fonction $g$ dérivable et définie sur $\mathbb{R}^*$ par :
\[ g(x) = -x^3 + x + 1 - \ln|x| \]
\begin{enumerate}
    \item Montrer que $-1$ est un zéro de la fonction polynôme $P$ définie par : $P(x) = -3x^3 + x - 2$.
    \item \begin{enumerate}
        \item Déterminer le signe de $P(x)$ suivant les valeurs de $x$.
        \item Calculer les limites de $g$ aux bornes des intervalles de son ensemble de définition.
        \item Étudier les variations de $g$.
        \item Montrer que l'équation $g(x) = 0$ admet une solution unique. On note $\alpha$ cette solution. \\
        Montrer que : $1,2 < \alpha < 1,3$.
        \item En déduire que : $\forall x \in ]-\infty; 0[ \cup ]0; \alpha[, g(x) > 0$ ; $\forall x \in ]\alpha; +\infty[, g(x) < 0$.
    \end{enumerate}
\end{enumerate}

\vspace{0.5cm}
\noindent \underline{\textbf{Partie B}} \\
On considère la fonction $f$ dérivable et définie sur $\mathbb{R}^*$ par :
\[ f(x) = -x + 1 - \frac{x - \ln|x|}{x^2} \]
On appelle $(C)$ la courbe représentative de $f$ dans un repère orthonormé $(O, I, J)$. (Unité graphique : 2cm)
\begin{enumerate}
    \item Calculer les limites de $f$ aux bornes des intervalles de son ensemble de définition.
    \item \begin{enumerate}
        \item Démontrer que pour tout $x \in \mathbb{R}^*$, $f'(x) = \frac{g(x)}{x^3}$.
        \item Étudier les variations de $f$ et dresser son tableau de variation.
    \end{enumerate}
    \item Déterminer une équation de la tangente $(T)$ au point d'abscisse $-1$.
    \item Déterminer que la droite $(D)$ d'équation $y = -x + 1$ est asymptote à la courbe $(C)$.
    \item \begin{enumerate}
        \item Étudier les variations de la fonction $h$ définie sur $\mathbb{R}^*$ par : $h(x) = x - \ln|x|$.
        \item En déduire que $(D)$ coupe $(C)$ en un point unique d'abscisse $\beta$ vérifiant : $\ln(-\beta) = \beta$.
        \item Montrer que : $-0,57 < \beta < 0,56$.
        \item Déterminer la position de $(C)$ par rapport à $(D)$.
    \end{enumerate}
    \item Construire $(T)$, $(D)$ et $(C)$.
    \item Démontrer que la fonction numérique $F$ définie sur $]0; +\infty[$ par : \\
    $F(x) = -\frac{1}{2}x^2 + x - \frac{1}{x} - \ln x - \frac{(\ln x)^2}{x}$ est une primitive de $f$ sur $]0; +\infty[$.
\end{enumerate}

\begin{center}
    \fbox{\textbf{\exo}}
\end{center}

\vspace{0.5cm}

\noindent \underline{\textbf{Partie A}} \\
Soit $g$, la fonction de $\mathbb{R}$ vers $\mathbb{R}$ définie par $g(x) = (x+2)^2 + \ln|x+2|$.
\begin{enumerate}
    \item \begin{enumerate}
        \item Calculer les limites de $g$ au borne de son ensemble de définition.
        \item Etudier les variations de $g$ sur $]-2; +\infty[$.
    \end{enumerate}
    \item \begin{enumerate}
        \item Montrer que l'équation $g(x) = 0$ admet une solution unique $\beta$ dans $]-2; +\infty[$.
        \item Montrer que $\beta$ vérifie $-1,35 < \beta < -1,34$.
        \item Déterminer le signe de $g(x)$ sur $]-2; +\infty[$.
    \end{enumerate}
\end{enumerate}

\vspace{0.5cm}

\noindent \underline{\textbf{Partie B}} \\
Soit $f$ la fonction définie par $f(x) = -x - 1 + \frac{1 + \ln(x+2)}{x+2}$ et $(\varepsilon)$ sa courbe représentative dans le plan muni du repère orthonormé $(O, I, J)$. Unité 2 centimètres.
\begin{enumerate}
    \item \begin{enumerate}
        \item Déterminer l'ensemble de définition de $f$ et la limite de $f$ en $+\infty$.
        \item Déterminer la limite de $f$ à droite en $-2$. Interpréter graphiquement le résultat.
        \item Montrer que, $\forall x \in ]-2; +\infty[ ; f'(x) = \frac{-g(x)}{(x+2)^2}$. En déduire le signe de $f'(x)$ et dresser le tableau de variation de $f$.
        \item Montrer que $f(\beta) = -2\beta - 3 + \frac{1}{\beta+2}$.
    \end{enumerate}
    \item \begin{enumerate}
        \item Montre que la droite $(D) : y = -x - 1$ est asymptote à $(\varepsilon)$.
        \item Déterminer les coordonnées de $A$ intersection de $(\varepsilon)$ et de $(D)$.
        \item Etudier la position de $(\varepsilon)$ par rapport à $(D)$.
    \end{enumerate}
    \item Construire $(\varepsilon)$ et $(D)$ sur le même graphique.
    \item Déterminer $G$, la primitive de $g(x)$ tel que $g(x) = (x+2)^2 + \ln(x+2)$ sur $]-2; +\infty[$ et s'annule en $-1$. \\
    Sachant que la primitive sur $]-2; +\infty[$ de $\ln(x+2)$ est $(x+2)\ln(x+2) - (x+2)$.
    \item \begin{enumerate}
        \item Si $h$ est la restriction de $f$ à l'intervalle $[\beta; +\infty[$, montrer que $h$ est une bijection de $[\beta; +\infty[$ sur une partie $K$ que l'on déterminera.
        \item Calculer $h(-1)$.
        \item Montrer $h^{-1}$ est dérivable en 1 et calculer $(h^{-1})'(1)$.
        \item Construire $(\varepsilon')$, la représentation graphique de $h^{-1}$, bijection réciproque de $h$ sur le graphique précédent.
    \end{enumerate}
\end{enumerate}

\begin{center}
    \fbox{\textbf{\exo}}
\end{center}

\vspace{0.5cm}

\noindent \underline{\textbf{Partie A :}} Etude d'une fonction numérique de la variable réelle $x$ définie par :
\[ f(x) = \left(x - \frac{1}{2}\right)e^{2x} - 4(x - 1)e^x - 2 \]

\begin{enumerate}
    \item \begin{enumerate}
        \item Calculer la limite de $f$ en $-\infty$.
        \item Montrer que $f(x) = xe^{2x} \left( 1 - \frac{1}{2x} - \frac{4}{e^x} + \frac{4}{xe^x} - \frac{2}{xe^{2x}} \right)$ et en déduire la limite de $f$ en $+\infty$.
    \end{enumerate}
    \item Etudier les variations de $f$.
    \item Soit $(C_f)$ la courbe représentative de $f$ dans le repère orthonormé, unité graphique 2cm.
    \begin{enumerate}
        \item Etudier les branches infinies de $(C_f)$.
        \item Montrer que $(C)$ coupe l'axe des abscisses en un point, dont l'abscisse $\alpha$ appartient à l'intervalle $[-2; -1]$.
        \item Tracer $(C_f)$. On prendra $\ln 2 \approx 0,7$.
    \end{enumerate}
    \item Montrer que $F(x) = \left(\frac{x}{2} - \frac{1}{2}\right)e^{2x} + 4(2 - x)e^x$ est une primitive de $f(x) + 2$.
\end{enumerate}

\vspace{0.8cm}

\noindent \underline{\textbf{Partie B :}} Etude d'une nouvelle fonction numérique de variable réelle $x$ définie par :
\[
\begin{cases} 
g(x) = (x^2 - 4x)\ln x - \dfrac{1}{2}(x^2 - 8x + 4) & \forall x > 0 \\
g(0) = -2 
\end{cases}
\]

\begin{enumerate}
    \item Montrer que : $g(x) = f(\ln x), \forall x > 0$.
    \item Etudier la continuité et la dérivabilité de $g$ à droite en 0.
    \item Calculer les limites aux bornes de son domaine.
    \item Etudier les variations de $g$.
    \item Soit $(C_g)$ la courbe représentative de $g$ dans un repère orthonormé, unité 2cm.
    \begin{enumerate}
        \item Etudier la branche infinie de $(C_g)$.
    \end{enumerate}
\end{enumerate}

\begin{center}
    \fbox{\textbf{\exo}}
\end{center}

\noindent Soit $f$ la fonction numérique à variable réelle définie par : $f(x) = \dfrac{1}{1-xe^{-x}}$ \\
On note $(C)$ sa courbe représentative dans un repère orthonormé direct $(O, I, J)$. Unité graphique : 2cm

\vspace{0.5cm}

\noindent \underline{\textbf{Partie A}} \\
Soit $g$ la fonction définie sur $\mathbb{R}$ par : $g(x) = 1 - xe^{-x}$
\begin{enumerate}
    \item Démontrer que : $\forall x \in \mathbb{R}, g'(x) = (x - 1)e^{-x}$
    \item Déterminer le sens de variation de $g$ puis, dresser le tableau de variation de $g$ (sans les limites aux bornes).
    \item Démontrer que : $\forall x \in \mathbb{R}, g(x) > 0$.
\end{enumerate}

\vspace{0.8cm}

\noindent \underline{\textbf{Partie B}}
\begin{enumerate}
    \item Démontrer que l'ensemble de définition de $f$ est $\mathbb{R}$.
    \item Calculer la limite de $f$ en $-\infty$ et en $+\infty$. En donner une interprétation graphique.
    \item \begin{enumerate}
        \item Démontrer que : $\forall x \in \mathbb{R}, f'(x) = \dfrac{(1-x)e^{-x}}{(1-xe^{-x})^2}$.
        \item Déterminer le sens de variation de $f$ puis dresser le tableau de variation de $f$.
    \end{enumerate}
    \item \begin{enumerate}
        \item Démontrer qu'une équation de la tangente $(T)$ au point d'abscisse 0 est $y = x + 1$.
        \item Démontrer que : $\forall x \in \mathbb{R}, f(x) - x - 1 = \dfrac{(x+1-e^x)xe^{-x}}{1-xe^{-x}}$.
        \item Déterminer le signe de la fonction $h$ telle que : $h(x) = x + 1 - e^{-x}$.
        \item Déduire de la question précédente les positions relatives de $(C)$ et de $(T)$.
    \end{enumerate}
    \item Construire $(T)$ et $(C)$.
\end{enumerate}
\begin{center}
    \fbox{\textbf{\exo}}
\end{center}

\noindent \underline{\textbf{Partie A :}} (Etude d'une fonction auxiliaire $g$) \\
Soit $g$ la fonction numérique définie sur $]0; +\infty[$ par : 
\( g(x) = x - 2 - 2x\ln x \)
\begin{enumerate}
    \item Justifier que l'ensemble des solutions de l'inéquation : 
    $-2\ln x - 1 \ge 0$ est $S = \left] 0; e^{-\frac{1}{2}} \right]$
    \item \begin{enumerate}
        \item Calculer $g'(x)$.
        \item Etudier les variations de $g$.
    \end{enumerate}
    \item \begin{enumerate}
        \item Etablir le tableau de variation de $g$. (On ne cherchera pas à calculer les limites de $g$).
        \item En déduire que : $\forall x \in ]0; +\infty[, g(x) < 0$.
    \end{enumerate}
\end{enumerate}

\vspace{0.8cm}

\noindent \underline{\textbf{Partie B :}} (Etude et représentation graphique d'une fonction $f$) \\
Soit $f$ la fonction numérique définie sur $[0; +\infty[$ par : 
$f(x) = (x - 1)^2 - x^2\ln x$ si $x \in ]0; +\infty[$ et $f(0) = 1$. \\
On note $(C)$ la courbe représentative de $f$ dans le plan muni du repère orthonormé $(O, I, J)$. \\
L'unité graphique est 2cm.
\begin{enumerate}
    \item \begin{enumerate}
        \item Calculer la limite de $f$ en $+\infty$.
        \item Justifier que la courbe $(C)$ admet en $+\infty$ une branche parabolique dont on précisera la direction.
    \end{enumerate}
    \item \begin{enumerate}
        \item Justifier que $f$ est continue en 0.
        \item Démontrer que $\lim\limits_{x \to 0} \dfrac{f(x) - f(0)}{x - 0} = -2$.
    \end{enumerate}
    \item Soit $(T)$ la droite d'équation $y = -2x + 1$ et $d$ la fonction définie sur $]0; +\infty[$ par $d(x) = f(x) - y$.
    \begin{enumerate}
        \item Justifier que $(T)$ est la tangente à la courbe $(C)$ au point d'abscisse 0.
        \item Vérifier que pour tout $x \in ]0; +\infty[, d(x) = x^2 - x^2\ln x$.
        \item Etudier la position relative de $(C)$ par rapport à $(T)$.
    \end{enumerate}
    \item \begin{enumerate}
        \item Démontrer que : $\forall x \in ]0; +\infty[, f'(x) = g(x)$.
        \item Etablir le tableau de variation de la fonction $f$.
    \end{enumerate}
    \item \begin{enumerate}
        \item Démontrer que l'équation $f(x) = 0$ admet une unique solution $\alpha$ dans $]0; +\infty[$.
        \item Vérifier que $\alpha = 1$ et démontrer que $f(x) \ge 0$ si $0 \le x \le 1$ et $f(x) \le 0$ si $x \ge 1$.
    \end{enumerate}
    \item Construire $(T)$ et $(C)$.
\end{enumerate}

\vspace{0.8cm}

\noindent \underline{\textbf{Partie C :}} (Etude d'une primitive $F$ d'une restriction de la fonction $f$) \\
Soit $F$ la primitive de $f$ sur $[1; +\infty[$ qui s'annule en $e$. (On ne cherchera pas à déterminer $F$).
\begin{enumerate}
    \item Déterminer $F(e)$ et $F'(x)$. (On justifiera chaque réponse).
    \item Démontrer que $F$ est une bijection de $]1; +\infty[$ vers $F(]1; +\infty[)$.
    \item Soit $F^{-1}$ la réciproque de $F$. Calculer $(F^{-1})'(0)$.
\end{enumerate}


\begin{center}
    \fbox{\textbf{\exo}}
\end{center}

\noindent On désigne par $(C)$ la courbe de la fonction $f$ ci-dessous dérivable sur $\mathbb{R} \setminus \{0; 1\}$ dans le plan d'un repère orthonormé $(O; I, J)$. L'unité de longueur : 2 cm.

\vspace{0.2cm}

\noindent \underline{\textbf{Partie A :}} On considère la fonction $f$ définie par :
\( f(x) = x - 1 - \frac{2}{x} - \ln \left| 1 - \frac{1}{x} \right| \)

\begin{enumerate}
    \item Montrer que $f$ est définie sur $\mathbb{R} \setminus \{0; 1\}$.
    \item Montrer que :
    \begin{itemize}
        \item Pour tout $x \in ]-\infty; 0[ \cup ]1; +\infty[ ; f(x) = x - 1 - \frac{2}{x} - \ln\left( 1 - \frac{1}{x}\right) $
        \item Pour tout $x \in ]0; 1[ ; f(x) = x - 1 - \frac{2}{x} - \ln\left( \frac{1}{x} - 1\right) $
    \end{itemize}
    \item Déterminer les limites de $f$ en $-\infty$ et en $+\infty$.
    \item Calculer la limite de $f$ en 1. Interpréter graphiquement ce résultat.
    \item \begin{enumerate}
        \item Calculer les limites de $f$ à gauche en 0 et à droite en 0.
        \item Interpréter graphiquement ces résultats.
    \end{enumerate}
\end{enumerate}


\noindent \underline{\textbf{Partie B :}} Soit la fonction $g$ définie sur $\mathbb{R}$ par : $g(x) = x^3 - x^2 + x - 2$.
\begin{enumerate}
    \item Calculer $g'(x)$, $g'$ étant la fonction dérivée de $g$.
    \item Déterminer le sens de variation de $g$.
    \item \begin{enumerate}
        \item Montrer que l'équation $g(x) = 0$ admet une solution unique $\alpha$ dans $\mathbb{R}$ et que $1 < \alpha < 1,5$.
        \item En déduire un encadrement de $\alpha$ par deux décimaux consécutifs d'ordre 1.
    \end{enumerate}
    \item Montrer que pour tout $x \in ]-\infty; \alpha[ ; g(x) < 0$ et pour tout $x \in ]\alpha; +\infty[ ; g(x) > 0$.
    \item Montrer que pour tout nombre réel $x$ élément de $\mathbb{R} \setminus \{0; 1\}$, $f'(x) = \dfrac{g(x)}{x^2(x-1)}$.
    \item \begin{enumerate}
        \item Déterminer le signe de $f'(x)$.
        \item Dresser le tableau de variation de $f$.
    \end{enumerate}
\end{enumerate}


\noindent \underline{\textbf{Partie C}}
\begin{enumerate}
    \item Soit $(D)$ la droite d'équation $y = x - 1$. \\
    Montrer que la droite $(D)$ est une asymptote à $(C)$ en $-\infty$ et en $+\infty$.
    \item Calculer $f(-2) ; f(-\frac{1}{2})$ et $f(\frac{1}{2})$.
    \item Construire dans le repère $(O; I, J)$ la courbe $(C)$ et ses asymptotes. On prendra $\alpha \approx 1,3$.
\end{enumerate}

\noindent \underline{\textbf{Partie D}}\\
On considère les fonctions $h$ et $k$ définies sur $]2; +\infty[$ par $h(x) = (x - 1) \ln(x - 1) - x \ln x$ et on admettra que pour tout $x$ élément de $]2; +\infty[$, $f(x) - x + 1 < 0$.
\begin{enumerate}
    \item Calculer $h'(x)$, $h'$ étant la dérivée de $h$.
    \item Calculer l'aire de la partie limitée par la courbe $(C)$, la droite $(D)$ et les droites d'équations $x = 2$ et $x = 4$.
\end{enumerate}

\begin{center}
    \fbox{\textbf{\exo}}
\end{center}

\noindent Le plan est muni d'un repère orthonormé $(O, I, J)$. Unité : $OI = 2\text{cm}$ et $OJ = 1\text{cm}$.

\vspace{0.5cm}

\noindent \underline{\textbf{Partie A}} \\
Soit $g$ la fonction de $\mathbb{R}$ vers $\mathbb{R}$ définie par : $g(0) = -1$ et $g(x) = \dfrac{x}{(\ln(x))^2} - 1$ et $(C_g)$ sa courbe.

\begin{enumerate}
    \item \begin{enumerate}
        \item Déterminer l'ensemble de définition de $g$.
        \item Calculer les limites de $g$ en 1 et en $+\infty$.
        \item Calculer $\lim\limits_{x \to +\infty} \dfrac{g(x)}{x}$ puis interpréter graphiquement ce résultat.
    \end{enumerate}
    \item \begin{enumerate}
        \item Démontrer que $g$ est continue en 0.
        \item Étudier la dérivabilité de $g$ en 0 et interpréter graphiquement ce résultat.
    \end{enumerate}
    \item On admet que $g$ est dérivable sur $[0; 1[ \cup ]1; +\infty[$.
    \begin{enumerate}
        \item Démontrer que $\forall x \in ]0; 1[ \cup ]1; +\infty[ ; g'(x) = \dfrac{\ln(x)(\ln(x) - 2)}{(\ln(x))^4}$.
        \item Déterminer le sens de variation de $g$ puis dresser son tableau de variation.
    \end{enumerate}
    \item \begin{enumerate}
        \item Démontrer que l'équation $\forall x \in \mathbb{R}, g(x) = 0$ admet une unique solution $\alpha$ dans $]0; 1[$.
        \item Vérifier que $0,4 < \alpha < 0,5$.
        \item En déduire que : $\forall x \in [0; \alpha[ ; g(x) < 0$ \\
        $\forall x \in ]\alpha; 1[ \cup ]1; +\infty[ ; g(x) > 0$.
    \end{enumerate}
\end{enumerate}

\vspace{0.5cm}

\noindent \underline{\textbf{Partie B}} \\
On considère la fonction numérique $f$ définie sur $]0; 1[ \cup ]1; +\infty[$ par : $f(x) = \dfrac{1}{x} - \dfrac{1}{\ln(x)}$ et $(C)$, sa courbe.

\begin{enumerate}
    \item \begin{enumerate}
        \item Calculer les limites de $f$ en 0, en 1 et en $+\infty$.
        \item Interpréter graphiquement les résultats précédents.
    \end{enumerate}
    \item On admet que $f$ est dérivable sur $]0; 1[ \cup ]1; +\infty[$.
    \begin{enumerate}
        \item Démontrer que $\forall x \in ]0; 1[ \cup ]1; +\infty[ , f'(x) = \dfrac{g(x)}{x^2}$.
        \item Étudier le sens de variation de $f$ et dresser son tableau de variation de $f$.
    \end{enumerate}
    \item Démontrer que $f(\alpha) = \dfrac{1 - \sqrt{\alpha}}{\alpha}$. En déduire le signe de $f(x)$ sur $]0; 1[ \cup ]1; +\infty[$.
     \item Construire la courbe $(C)$. On prendra $\alpha \simeq 0,5$.
\end{enumerate}

\vspace{0.5cm}

\noindent \underline{\textbf{Partie C}} \\
Soit $h$ la restriction de $f$ à $]1; +\infty[$.

\begin{enumerate}
    \item Démontrer que $h$ est la bijection de $]1; +\infty[$ sur un intervalle $K$ que l'on déterminera.
    \item On note $h^{-1}$ la bijection réciproque de $h$ et $(\Gamma)$ sa représentation graphique.
    \begin{enumerate}
        \item Dresser le tableau de variation de $h^{-1}$.
        \item Calculer $h(e)$ ; $h^{-1}\left( \dfrac{1-e}{e}\right) $ et $(h^{-1})'\left( \dfrac{1-e}{e}\right) $.
        \item Déterminer une équation de la tangente $(T)$ à $(\Gamma)$ au point d'abscisse $\dfrac{1-e}{e}$.
    \end{enumerate}
    \item Construire $(\Gamma)$ dans le même repère que $(C)$.
\end{enumerate}

\begin{center}
    \fbox{\textbf{\exo}}
\end{center}

\vspace{0.5cm}

\noindent \underline{\textbf{Partie A}} \\
Soit $g$ la fonction définie sur $[0; +\infty[$ par : 
$\begin{cases} \forall x > 0, g(x) = 1 - x\ln x \\ g(0) = 1 \end{cases}$

\begin{enumerate}
    \item \begin{enumerate}
        \item Etudier la continuité de $g$ en 0.
        \item Etudier la dérivabilité de $g$ en 0. Interpréter graphiquement le résultat.
    \end{enumerate}
    \item Calculer la limite de $g$ en $+\infty$.
    \item Etudier les variations de $g$ et dresser son tableau de variations.
    \item \begin{enumerate}
        \item Montrer que l'équation $g(x)=0$ admet une unique solution $\alpha \in ]e^{-1}; +\infty[$.
        \item Justifier que : $1,7 < \alpha < 1,8$.
    \end{enumerate}
    \item Démontrer que : 
    $\begin{cases} \forall x \in [0; \alpha[, g(x) > 0 \\ \forall x \in [\alpha; +\infty[, g(x) < 0 \end{cases}$
\end{enumerate}

\vspace{0.8cm}

\noindent \underline{\textbf{Partie B}} \\
Soit $f$ la fonction définie sur $]0; +\infty[$ par : $f(x) = e^{3-x}\ln x$ \\
$(C_f)$ désigne sa représentation graphique dans le plan est muni d'un repère orthonormé $(O; I; J)$ Unité graphique : 2cm

\begin{enumerate}
    \item Calculer la limite de $f$ en 0. Interpréter graphiquement le résultat.
    \item \begin{enumerate}
        \item Vérifier que : $\forall x \in ]0; +\infty[ : f(x) = \left(\dfrac{\ln x}{x}\right) \left(\dfrac{x}{e^x}\right) e^3$
        \item En déduire la limite de $f$ en $+\infty$. Interpréter graphiquement le résultat.
    \end{enumerate}
    \item \begin{enumerate}
        \item Montrer que : $\forall x \in ]0; +\infty[, f'(x) = \dfrac{e^{3-x}}{x} g(x)$.
        \item En déduire le sens de variations de $f$.
    \end{enumerate}
    \item \begin{enumerate}
        \item Montrer que : $f(\alpha) = \dfrac{e^3}{\alpha e^{\alpha}}$.
        \item En déduire un encadrement de $f(\alpha)$.
        \item Dresser le tableau de variations de $f$.
    \end{enumerate}
     \item Construire $(C_f)$.
     \item Soit $h$ la restriction de $f$ à l'intervalle $]0; \alpha[$.
     \begin{enumerate}
            \item Montrer que $h$ admet une bijection réciproque $h^{-1}$ dont on précisera l'ensemble de départ, l'arrivée d'arrivée.
            \item Dresser le tableau de variation de $h^{-1}$.
            \item Calculer $(h^{-1})'(0)$.
      \end{enumerate}
\end{enumerate}

\begin{center}
    \fbox{\textbf{\exo}}
\end{center}

\vspace{0.3cm}

\noindent \underline{\textbf{Partie A}} \\
On considère l'équation différentielle (1) : $y' - 2y = xe^x$
\begin{enumerate}
    \item Résoudre l'équation différentielle (2) : $y' - 2y = 0$
    \item Soient $a$ et $b$ deux réels et soit $u$ la fonction définie sur $\mathbb{R}$ par $u(x) = (ax + b)e^x$.
    \begin{enumerate}
        \item Déterminer $a$ et $b$ pour que $u$ soit solution de l'équation (1).
        \item Montrer que $v$ est une solution de l'équation (1) si et seulement si $u+v$ est une solution de l'équation (2).
        \item En déduire l'ensemble des solutions de l'équation (1).
    \end{enumerate}
    \item Déterminer la solution $f$ de l'équation (1) qui s'annule en 0.
\end{enumerate}

\vspace{0.8cm}

\noindent \underline{\textbf{Partie B : Etude d'une fonction auxiliaire}} \\
Soit $g$ la fonction définie sur $\mathbb{R}$ par : $g(x) = 2e^x - x - 2$
\begin{enumerate}
    \item Déterminer les limites de $g$ en $-\infty$ et $+\infty$.
    \item Étudier le sens de variation de $g$, puis dresser son tableau de variation.
    \item \begin{enumerate}
        \item Montrer que l'équation $g(x)=0$ admet exactement deux solutions réelles : 0 et $\alpha$ telle que $-1,6 < \alpha < -1,5$.
        \item Déterminer le signe de $g(x)$ suivant les valeurs de $x$.
    \end{enumerate}
\end{enumerate}

\vspace{0.3cm}

\noindent \underline{\textbf{Partie C : Etude de la fonction principale $f$}} \\
Soit $f$ la fonction définie sur $\mathbb{R}$ par : $f(x) = e^{2x} - (x + 1)e^x$ \\
On désigne par $(C)$ la courbe représentative de $f$ dans un repère orthonormé $(O; I; J)$. (Unité : 4cm)
\begin{enumerate}
    \item Déterminer les limites de $f$ en $-\infty$ et $+\infty$.
    \item \begin{enumerate}
        \item Montrer que $\forall x \in \mathbb{R}, f'(x) = e^x g(x)$.
        \item En déduire les variations de $f$ et dresser le tableau de variation de $f$.
    \end{enumerate}
    \item Montrer que $f(\alpha) = -\dfrac{\alpha^2 + 2\alpha}{4}$. En déduire un encadrement de $f(\alpha)$ à $10^{-2}$ près.
    \item Tracer $(C)$.
\end{enumerate}

\vspace{0.3cm}

      \noindent \underline{\textbf{Partie D : Calcul d'aire}} \\
Soit $m$ un réel négatif.
\begin{enumerate}
    \item Interpréter graphiquement l'intégrale $I = \int_{m}^{0} f(x) \, dx$
    \begin{enumerate}
        \item Calculer $\int_{m}^{0} xe^x \, dx$ à l'aide d'une intégration par parties.
        \item En déduire la valeur de $I$.
    \end{enumerate}
    \item Calculer la limite de $I$ lorsque $m$ tend vers $-\infty$.
\end{enumerate}

\begin{center}
    \fbox{\textbf{\exo}}
\end{center}

\vspace{0.3cm}

\noindent L'objet de ce problème est la fonction définie par : 
$\begin{cases} f(x) = x(x - (\ln x)^2) \text{ si } x \neq 0 \\ f(0) = 0 \end{cases}$ \\
On désigne par $(C)$ la courbe représentative de $f$ dans le plan muni d'un repère orthonormé $(O; I, J)$. Unité graphique 5cm.

\vspace{0.3cm}

\noindent \underline{\textbf{Partie A}} \\
On considère la fonction $h$ dérivable sur $]0; +\infty[$ et définie par $h(x) = 2 - \dfrac{2}{x} - 2\dfrac{\ln x}{x}$.
\begin{enumerate}
    \item \begin{enumerate}
        \item Calculer $h'(x)$ et étudier les variations de $h$.
        \item En déduire que $\forall x \in ]0; +\infty[, h(x) \ge 0$.
    \end{enumerate}
    \item On considère la fonction $g$ dérivable sur $]0; +\infty[$ et définie par $g(x) = 2x - 2\ln(x) - (\ln x)^2$.
    \begin{enumerate}
        \item Calculer $\lim\limits_{x \to 0^+} g(x)$ et montrer que $\lim\limits_{x \to +\infty} g(x) = +\infty$.
        \item Calculer $g'(x)$ et montrer que $g'(x) = h(x)$.
        \item Étudier les variations de $g$.
        \item Montrer que l'équation $g(x) = 0$ admet une solution unique $\alpha$ et vérifier que $0,1 < \alpha < 0,2$.
        \item Démontrer que $\begin{cases} \forall x \in ]0; \alpha[, g(x) < 0 \\ \forall x \in ]\alpha; +\infty[, g(x) > 0 \end{cases}$
    \end{enumerate}
\end{enumerate}

\vspace{0.3cm}

\noindent \underline{\textbf{Partie B}}
\begin{enumerate}
    \item Déterminer l'ensemble de définition de $f$.
    \item Montrer que $f$ est continue en 0.
    \item \begin{enumerate}
        \item Étudier la dérivabilité de $f$ en 0.
        \item Interpréter graphiquement ce résultat.
    \end{enumerate}
    \item \begin{enumerate}
        \item Calculer $\lim\limits_{x \to +\infty} f(x)$ et $\lim\limits_{x \to +\infty} \dfrac{f(x)}{x}$.
        \item Interpréter graphiquement les résultats.
        \item Démontrer que $f(\alpha) = \alpha(-\alpha + 2\ln(\alpha))$.
    \end{enumerate}
    \item \begin{enumerate}
        \item On suppose que $f$ est dérivable sur $]0; +\infty[$.
        \item On suppose que $f$ est dérivable sur $]0; +\infty[$. Calculer $f'(x)$ et montrer que $\forall x \in ]0; +\infty[, f'(x) = g(x)$.
        \item Étudier les variations de $f$ et dresser son tableau de variation.
    \end{enumerate}
\end{enumerate}

\vspace{0.3cm}

\noindent \underline{\textbf{Partie C}}
\begin{enumerate}
    \item Déterminer une équation de la tangente $(T)$ à $(C)$ au point d'abscisse 1.
    \item Soit la fonction $k(x) = f(x) - (2x - 1)$.
    \begin{enumerate}
        \item Vérifier que $k'(x) = g(x) - 2$ et que $k''(x) = h(x)$.
        \item En déduire le sens de variation de $k'$. Calculer $k'(1)$ puis donner le signe de $k'$.
        \item Dresser le tableau de variation de $k$ puis donner le signe de $k$. (On ne calculera pas de limites).
        \item En déduire la position relative de $(C)$ et de la droite $(T)$.
    \end{enumerate}
    \item Tracer $(C)$ et $(T)$. On prendra $\alpha = 0,1$.
    \item Soit la fonction $q$, restriction de $f$ à l'intervalle $[\alpha; +\infty[$.
    \begin{enumerate}
        \item Montrer que $q$ admet une bijection réciproque notée $q^{-1}$ dont on précisera les ensembles de départ et d'arrivée.
        \item Dresser le tableau de variation de $q^{-1}$.
        \item Calculer $q(1)$, $q^{-1}(1)$ et $(q^{-1})'(1)$.
        \item Construire la courbe de $q^{-1}$ dans le même repère que $(C)$.
    \end{enumerate}
\end{enumerate}

\begin{center}
    \fbox{\textbf{\exo}}
\end{center}

\vspace{0.3cm}

\noindent \underline{\textbf{Partie A : Etude d'une fonction auxiliaire.}} \\
Soit $g$ la fonction définie sur $\mathbb{R}$ par $g(x) = e^x(1 - x) + 1$.
\begin{enumerate}
    \item Étudier le sens de variation de $g$.
    \item Démontrer que l'équation $g(x) = 0$ admet une unique solution dans l'intervalle $[1,27; 1,28]$. On note $\alpha$ cette solution.
    \item Déterminer le signe de $g(x)$ sur $]-\infty; 0]$. \\
    Montrer que $g(x) > 0$ sur $[0; \alpha[$ et $g(x) < 0$ sur $]\alpha; +\infty[$.
\end{enumerate}

\vspace{0.3cm}

\noindent \underline{\textbf{Partie B :}} \\
Étude de la fonction $f$ définie sur $\mathbb{R}$ par : $f(x) = \dfrac{x}{e^x + 1} + 2$. \\
On désigne par $(C_f)$ la courbe représentative de $f$ dans un repère orthogonal $(O; \vec{i}, \vec{j})$ ; unités graphiques : 1cm sur l'axe des abscisses et 2cm sur l'axe des ordonnées.
\begin{enumerate}
    \item \begin{enumerate}
        \item Déterminer la limite de $f$ en $-\infty$.
        \item Démontrer que la courbe $(C_f)$ admet une asymptote oblique $(d)$ dont on déterminera l'équation.
    \end{enumerate}
    \item Étudier la position de $(C_f)$ par rapport à $(d)$.
    \item \begin{enumerate}
        \item Montrer que la fonction dérivée de $f$ a même signe que la fonction $g$ étudiée dans la \textbf{Partie A}.
        \item Montrer qu'il existe deux entiers $p$ et $q$ tels que $f(\alpha) = p\alpha + q$.
        \item Dresser le tableau des variations de la fonction $f$.
    \end{enumerate}
    \item Tracer la courbe $(C_f)$ dans le repère $(O; \vec{i}, \vec{j})$ avec ses asymptotes et sa tangente au point d'abscisse $\alpha$.
\end{enumerate}

\vspace{0.3cm}

\noindent \underline{\textbf{Partie C : Encadrement d'aire}} \\
Pour tout entier naturel $n$, tel que $n \ge 2$, on donne $D_n$, l'ensemble des points $M(x; y)$ du plan, dont les coordonnées vérifient $2 \le x \le n$ et $2 \le y \le f(x)$, et on appelle $A_n$ son aire, exprimée en unité d'aire.
\begin{enumerate}
    \item Faire apparaître $D_n$ sur la figure.
\end{enumerate}

\begin{center}
    \fbox{\textbf{\exo}}
\end{center}

\noindent On considère la fonction $f$ définie par $f(x) = e^{1-x} + \ln |x|$ et $(C)$ sa courbe représentative dans le plan muni du repère orthonormé $(O ; I ; J)$ d'unité 2cm.

\vspace{0.3cm}

\noindent \underline{\textbf{Partie A}} (Etude d'une fonction auxiliaire) \\
Soit $g$ la fonction numérique définie par : $g(x) = x e^{1-x}$
\begin{enumerate}
    \item Calculer $\lim\limits_{x \to -\infty} g(x)$ et $\lim\limits_{x \to +\infty} g(x)$.
    \item Démontrer que $\forall x \in \mathbb{R}, g'(x) = (1-x)e^{1-x}$.
    \item Déterminer les variations de $g$ puis dresser son tableau de variation.
    \item Démontrer que : $\forall x \in \mathbb{R}, g(x) \le 1$.
\end{enumerate}

\vspace{0.3cm}

\noindent \underline{\textbf{Partie B}} (Etude de la fonction $f$) \\
Soit $f(x) = e^{1-x} + \ln |x|$
\begin{enumerate}
    \item Déterminer $D_f$ l'ensemble de définition de $f$.
    \item Calculer la limite de $f$ en 0. \\
    En déduire une interprétation graphique du résultat.
    \item a) Calculer la limite de $f$ en $-\infty$ et en $+\infty$. \\
    b) Calculer la limite de $\dfrac{f(x)}{x}$ en $-\infty$ et en $+\infty$. \\
    Interpréter graphiquement les résultats.
    \item a) Démontrer que $\forall x \in \mathbb{R}^*, f'(x) = \dfrac{1-g(x)}{x}$. \\
    b) En déduire le signe de $f'(x)$ suivant les valeurs de $x$.
    \item Déterminer les variations de $f$ puis dresser son tableau de variation.
    \item a) Démontrer que l'équation $f(x) = 0$ admet deux solutions $\alpha$ et $\beta$ tels que $0,08 < \alpha < 0,09$ et $-0,06 < \beta < -0,05$. \\
    b) Donner une valeur approchée de $\alpha$ à $10^{-3}$ près. \\
    c) Construire $(C)$ avec précision.
\end{enumerate}

\vspace{0.3cm}

\noindent \underline{\textbf{Partie C}}
\begin{enumerate}
    \item Soit $(U_n)_{n \ge 1}$ la suite définie par $U_n = \int_{n}^{n+1} e^{-x} dx$ et $S_n = U_1 + U_2 + \dots + U_n$
    \begin{enumerate}
        \item Démontrer que $(U_n)_{n \ge 1}$ est une suite géométrique dont on précisera la raison et le premier terme.
        \item Exprimer $U_n$ puis $S_n$ en fonction de $n$.
    \end{enumerate}
    \item a) Démontrer que la suite $(S_n)_{n \ge 1}$ est croissante. \\
    b) Démontrer que $(S_n)_{n \ge 1}$ est majorée par $\dfrac{1}{e}$. \\
    c) Les suites $(U_n)_{n \ge 1}$ et $(S_n)_{n \ge 1}$ sont-elles convergentes ? Justifier votre réponse.
    \item Soit $(V_n)_{n \ge 1}$ la suite définie par $V_n = \int_{1}^{n+1} \ln |x| dx$
\end{enumerate}

\begin{comment}
\begin{center}
    \fbox{\textbf{\exo}}
\end{center}
\begin{center}
    \fbox{\textbf{\exo}}
\end{center}
\begin{center}
    \fbox{\textbf{\exo}}
\end{center}
\begin{center}
    \fbox{\textbf{\exo}}
\end{center}
\begin{center}
    \fbox{\textbf{\exo}}
\end{center}
\begin{center}
    \fbox{\textbf{\exo}}
\end{center}
\begin{center}
    \fbox{\textbf{\exo}}
\end{center}
\begin{center}
    \fbox{\textbf{\exo}}
\end{center}
\begin{center}
    \fbox{\textbf{\exo}}
\end{center}
\begin{center}
    \fbox{\textbf{\exo}}
\end{center}
\begin{center}
    \fbox{\textbf{\exo}}
\end{center}
\begin{center}
    \fbox{\textbf{\exo}}
\end{center}
\begin{center}
    \fbox{\textbf{\exo}}
\end{center}
\begin{center}
    \fbox{\textbf{\exo}}
\end{center}
\begin{center}
    \fbox{\textbf{\exo}}
\end{center}
\begin{center}
    \fbox{\textbf{\exo}}
\end{center}
\begin{center}
    \fbox{\textbf{\exo}}
\end{center}
\begin{center}
    \fbox{\textbf{\exo}}
\end{center}
\begin{center}
    \fbox{\textbf{\exo}}
\end{center}
\begin{center}
    \fbox{\textbf{\exo}}
\end{center}
\begin{center}
    \fbox{\textbf{\exo}}
\end{center}
\begin{center}
    \fbox{\textbf{\exo}}
\end{center}
\begin{center}
    \fbox{\textbf{\exo}}
\end{center}
\begin{center}
    \fbox{\textbf{\exo}}
\end{center}
\begin{center}
    \fbox{\textbf{\exo}}
\end{center}
\begin{center}
    \fbox{\textbf{\exo}}
\end{center}
\begin{center}
    \fbox{\textbf{\exo}}
\end{center}
\begin{center}
    \fbox{\textbf{\exo}}
\end{center}
\begin{center}
    \fbox{\textbf{\exo}}
\end{center}
\begin{center}
    \fbox{\textbf{\exo}}
\end{center}
\begin{center}
    \fbox{\textbf{\exo}}
\end{center}
\begin{center}
    \fbox{\textbf{\exo}}
\end{center}
\begin{center}
    \fbox{\textbf{\exo}}
\end{center}
\begin{center}
    \fbox{\textbf{\exo}}
\end{center}
\begin{center}
    \fbox{\textbf{\exo}}
\end{center}
\begin{center}
    \fbox{\textbf{\exo}}
\end{center}
\begin{center}
    \fbox{\textbf{\exo}}
\end{center}
\begin{center}
    \fbox{\textbf{\exo}}
\end{center}
\begin{center}
    \fbox{\textbf{\exo}}
\end{center}
\end{comment}
\end{document}