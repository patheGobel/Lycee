\documentclass[12pt]{article}
\usepackage{lmodern} % Pour une police plus nette
\usepackage{stmaryrd}
\usepackage{graphicx} % Pour l'insertion d'images
\usepackage{float}    % Pour contrôler précisément le placement
\usepackage[utf8]{inputenc}
\usepackage[french]{babel}
\usepackage[T1]{fontenc}
\usepackage{hyperref}
\usepackage{verbatim}
\usepackage{color, soul}
\usepackage{pgfplots}
\pgfplotsset{compat=1.18} % Version plus récente de pgfplots
\usepackage{mathrsfs}
\usepackage{amsmath}
\usepackage{amsfonts}
\usepackage{amssymb}
\usepackage{tkz-tab}
%\author{Destiné aux élèves de Terminale S\\Lycée de Dindéfelo\\Présenté par M. BA}
%\title{\textbf{Rappels et compléments sur les fonctions numériques}}
%\date{\today}
\usepackage{tikz}
\usetikzlibrary{arrows, shapes.geometric, fit}
% Commande pour la couleur d'accentuation
\newcommand{\myul}[2][black]{\setulcolor{#1}\ul{#2}\setulcolor{black}}
\newcommand\tab[1][1cm]{\hspace*{#1}}
\usepackage[margin=2.5cm]{geometry} % Ajustement des marges
\usepackage{eso-pic} % Pour ajouter des éléments en arrière-plan

% Commande pour ajouter du texte en arrière-plan, centré au milieu de chaque page
\AddToShipoutPicture{
    \AtPageCenter{%
        \makebox(0,0)[c]{\rotatebox{60}{\textcolor[gray]{0.9}{\fontsize{2cm}{2cm}\selectfont PGB}}}
    }
}

\begin{document}

\noindent
\begin{minipage}[t]{0.48\textwidth}
\raggedright
\textbf{Ministère de l'Éducation Nationale}\\
Inspection Académique de Kédougou\\
Lycée Dindéfelo\\
Cellule de Mathématiques
\end{minipage}
\hfill
\begin{minipage}[t]{0.48\textwidth}
\raggedleft
\textbf{Année scolaire 2025-2026}\\
Date : 24/10/2025\\
Classe : Terminale S2\\
Professeur : M. BA
\end{minipage}

\vspace{1cm}
\section*{Exercice 1}

\begin{center}
\begin{tikzpicture}
\tkzTabInit[
    espcl=2.2,
    lgt=1.5
]{$x$/1,$f'(x)$/1,$f$/2}
{$-\infty$,$-2$,$0$,$1$,$2$,$+\infty$}

\tkzTabLine{,+,z,-,d,+,d,-,d,+}

\tkzTabVar{
-/$1$,
+/$\frac{4}{3}$,
-/$0$,
+D+/$+\infty$,
-/$0$,
+/$+\infty$
}
\end{tikzpicture}
\end{center}

\medskip

\textbf{Informations :}

\begin{itemize}
\item \textbf{Branches infinies :}
\begin{itemize}
\item Asymptote verticale : $x = 1$ en $+\infty$
\item Asymptote horizontale : $y = 1$ en $-\infty$
\item Asymptote oblique : $y = x - \dfrac{1}{2}$ en $+\infty$
\end{itemize}

\item \textbf{Demi-tangentes :}
\begin{itemize}
\item En $0^-$ on a une demi-tangente d’équation $y = 0$ (demi-tangente horizontale)
\item En $0^+$ on a une demi-tangente d’équation $y = \sqrt{2x}$
\item En $2$ on a une demi-tangente verticale dirigée vers le haut d’équation $x = 2$
\end{itemize}
\end{itemize}

\medskip

\begin{center}
\begin{tikzpicture}
\tkzTabInit[
    espcl=2.3,
    lgt=1.5
]{$x$/1,$f'(x)$/1,$f$/2}
{$-\infty$,$\alpha$,$1$,$3$,$+\infty$}

\tkzTabLine{,+,,+,d,-,z,+}

\tkzTabVar{
-/$-\infty$,
+/$0$,
+D+/$+\infty$,
-/$6$,
+/$+\infty$
}
\end{tikzpicture}
\end{center}

\medskip

\textbf{Informations :}

\begin{itemize}
\item \textbf{Branches infinies :}
\begin{itemize}
\item Asymptote verticale : $x = 1$ en $+\infty$
\item Asymptote oblique : $y = x + 2$ en $+\infty$
\end{itemize}

\item \textbf{Points d’intersection avec les axes du repère :}
\begin{itemize}
\item Avec l’axe $(Ox)$ le point $A(\alpha,0)$. Prendre $\alpha = -2,5$
\item Avec l’axe $(Oy)$ le point $B(0,6)$
\end{itemize}

\item Construire $(C_{f^{-1}})$ sur $I = ]1,3[$
\end{itemize}


\begin{center}
\begin{tikzpicture}
\tkzTabInit[
    espcl=2.3,
    lgt=1.5
]{$x$/1,$f'(x)$/1,$f$/2}
{$-\infty$,$0$,$1$,$\alpha$,$+\infty$}

\tkzTabLine{,+,d,-,d,-,+}

\tkzTabVar{
-/$-4$,
+/$-3$,
-D+/$+\infty$,
-/$3\alpha$,
+/$+\infty$
}
\end{tikzpicture}
\end{center}

\textbf{On donne $\alpha$ = 1,3}

\medskip

\textbf{Informations :}

\begin{itemize}
\item \textbf{Branches infinies :}
\begin{itemize}
\item Asymptote verticale : $x = 1$ en $+\infty$ et en $-\infty$
\item Asymptote horizontale : $y = -4$ en $-\infty$
\item Asymptote oblique : $y = 2x$ en $+\infty$
\end{itemize}

\item \textbf{Demi-tangentes :}
\begin{itemize}
\item En $0^-$ on a une demi-tangente verticale dirigée vers le bas d’équation $x = 0$
\item En $0^+$ on a une demi-tangente horizontale d’équation $y = -3$
\end{itemize}

\item Construire $(C_{f^{-1}})$ sur $I = ]\alpha,+\infty[$
\end{itemize}


\end{document}
