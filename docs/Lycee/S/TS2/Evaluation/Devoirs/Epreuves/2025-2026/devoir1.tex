\documentclass[12pt,a4paper]{article}
\usepackage{amsmath,amssymb,mathrsfs,tikz,times,pifont}
\usepackage{enumitem}
\newcommand\circitem[1]{%
\tikz[baseline=(char.base)]{
\node[circle,draw=gray, fill=red!55,
minimum size=1.2em,inner sep=0] (char) {#1};}}
\newcommand\boxitem[1]{%
\tikz[baseline=(char.base)]{
\node[fill=cyan,
minimum size=1.2em,inner sep=0] (char) {#1};}}
\setlist[enumerate,1]{label=\protect\circitem{\arabic*}}
\setlist[enumerate,2]{label=\protect\boxitem{\alph*}}
%%%::::::by chnini ameur :::::::%%%
\everymath{\displaystyle}
\usepackage[left=1cm,right=1cm,top=1cm,bottom=1.7cm]{geometry}
\usepackage{array,multirow}
\usepackage[most]{tcolorbox}
\usepackage{varwidth}
\tcbuselibrary{skins,hooks}
\usetikzlibrary{patterns}
%%%::::::by chnini ameur :::::::%%%
\newtcolorbox{exa}[2][]{enhanced,breakable,before skip=2mm,after skip=5mm,
colback=yellow!20!white,colframe=black!20!blue,boxrule=0.5mm,
attach boxed title to top left ={xshift=0.6cm,yshift*=1mm-\tcboxedtitleheight},
fonttitle=\bfseries,
title={#2},#1,
% varwidth boxed title*=-3cm,
boxed title style={frame code={
\path[fill=tcbcolback!30!black]
([yshift=-1mm,xshift=-1mm]frame.north west)
arc[start angle=0,end angle=180,radius=1mm]
([yshift=-1mm,xshift=1mm]frame.north east)
arc[start angle=180,end angle=0,radius=1mm];
\path[left color=tcbcolback!60!black,right color = tcbcolback!60!black,
middle color = tcbcolback!80!black]
([xshift=-2mm]frame.north west) -- ([xshift=2mm]frame.north east)
[rounded corners=1mm]-- ([xshift=1mm,yshift=-1mm]frame.north east)
-- (frame.south east) -- (frame.south west)
-- ([xshift=-1mm,yshift=-1mm]frame.north west)
[sharp corners]-- cycle;
},interior engine=empty,
},interior style={top color=yellow!5}}
%%%%%%%%%%%%%%%%%%%%%%%

\usepackage{fancyhdr}
\usepackage{eso-pic}         % Pour ajouter des éléments en arrière-plan
% Commande pour ajouter du texte en arrière-plan
\AddToShipoutPicture{
    \AtTextCenter{%
        \makebox[0pt]{\rotatebox{80}{\textcolor[gray]{0.7}{\fontsize{5cm}{5cm}\selectfont PGB}}}
    }
}
\usepackage{lastpage}
\fancyhf{}
\pagestyle{fancy}
\renewcommand{\footrulewidth}{1pt}
\renewcommand{\headrulewidth}{0pt}
\renewcommand{\footruleskip}{10pt}
\fancyfoot[R]{
\color{blue}\ding{45}\ \textbf{2025}
}
\fancyfoot[L]{
\color{blue}\ding{45}\ \textbf{Prof:M. BA}
}
\cfoot{\bf
\thepage /
\pageref{LastPage}}
\begin{document}
\renewcommand{\arraystretch}{1.5}
\renewcommand{\arrayrulewidth}{1.2pt}
\begin{tikzpicture}[overlay,remember picture]
\node[draw=blue,line width=1.2pt,fill=purple,text=blue,inner sep=3mm,rounded corners,pattern=dots]at ([yshift=-2.5cm]current page.north) {\begingroup\setlength{\fboxsep}{0pt}\colorbox{white}{\begin{tabular}{|*1{>{\centering \arraybackslash}p{0.28\textwidth}} |*2{>{\centering \arraybackslash}p{0.2\textwidth}|} *1{>{\centering \arraybackslash}p{0.19\textwidth}|} }
\hline
\multicolumn{3}{|c|}{$\diamond$$\diamond$$\diamond$\ \textbf{Lycée de Dindéfélo}\ $\diamond$$\diamond$$\diamond$ }& \textbf{A.S. : 2024/2025} \\ \hline
\textbf{Matière: Mathématiques}& \textbf{Niveau : T}\textbf{S2} &\textbf{Date: 26/11/2024} & \textbf{Durée : 4 heures} \\ \hline
\multicolumn{4}{|c|}{\parbox[c]{10cm}{\begin{center}
\textbf{{\Large\sffamily Devoir n$ ^{\circ} $ 1 Du 1$ ^\text{\bf er} $ Semestre}}
\end{center}}} \\ \hline
\end{tabular}}\endgroup};
\end{tikzpicture}
\vspace{3cm}

\section*{\underline{Exercice 1 :} 0,5 $\times $ 10 = 5 points}
\begin{enumerate}
\item Énoncer le théorème des valeurs intermédiaires et son corollaire.
\item Énoncer Théorème de l'inégalité des accroissements finis (TIAF).
\item Énoncer le théorème de l’inégalité des accroissements finis (IAF).
\item Théorème de la bijection.
\item \(\text{Si }\lim_{x \to x_0} \frac{f(x) - f(x_0)}{x - x_0} = a \ (a \neq 0) \text{ alors ... }\)
\item \(\text{Si }\lim_{x \to x_0^-} \frac{f(x) - f(x_0)}{x - x_0} = +\infty \text{ alors ... }\)
\item \(\text{Si }\lim_{x \to x_0} \frac{f(x) - f(x_0)}{x - x_0} = 0 \text{ alors ... }\)
\item \(\text{Si }\lim_{x \to +\infty} f(x) =+\infty \text{ et }\lim_{x \to +\infty}\frac{f(x)}{x}=\beta \in\mathbb{R}^{*}\text{ et }\lim_{x \to +\infty}[f(x)-\beta x]=+\infty\text{ alors ...}\)
\item Si \(f\) est continue et strictement décroissante sur \( ]-\infty; b] \), alors \( f(]-\infty; b]) = ... \)
\end{enumerate}

\section*{\underline{Exercice 2 :} 5,5 points}

\begin{enumerate}
    \item Calculer les limites suivantes : \textbf{(4 × 1 pt)}
    \[
    \lim_{x \to 0} \frac{\sqrt{1+\sin x} - 1}{\sin 2x} \; ; \quad
    \lim_{x \to 0} \frac{\cos x - 1}{x^3 + x^2} \; ; \quad
    \lim_{x \to 1} \frac{\sqrt{x + 3} - \sqrt{5 - x}}{\sqrt{2x + 7} - \sqrt{10 - x}}\; ; \quad
    \lim_{x \to 0} \dfrac{\cos x - 1}{x^3 + x^2}
    \]
    \item Donner les primitives des fonctions \(f\) et \(g\) respectivement sur \(\mathbb{R}\) et \(\mathbb{R} \setminus \{1; 2\}\). \textbf{(3 × 0,5 pt)}
    \[
    f(x)=2\cos(x)-3\sin(x)\; ; \quad
    g(x) = (3x-1)(3x^2-2x+3)^3 \; ; \quad
    h(x) = \frac{1-x^2}{(x^3-3x+2)^3}.
    \]
\end{enumerate}

\section*{\underline{Problème :} 9,5 points}
\underline{\textbf{Partie A :}}
Soit la fonction $f$ définie par :
\[
f(x)=
\begin{cases}
\dfrac{x^2-2x}{x-1} & \text{si } x<0,\\[4mm]
x + \sqrt{x^2+x} & \text{si } x\ge 0,
\end{cases}
\qquad
\text{et } (\mathcal{C}_f) \text{ sa courbe représentative dans un repère orthonormé } (O,\vec{i},\vec{j}).
\]

\begin{enumerate}
    \item Déterminer l'ensemble de définition $\mathcal{D}_f$ de $f$. \hspace{1cm} \textbf{(0,5 pt)}
    \item Déterminer les limites aux bornes de $\mathcal{D}_f$. \hspace{1cm} \textbf{(0,5 pt)}
    \item Étudier la continuité et la dérivabilité de $f$ en $0$. Interpréter les résultats obtenus. \hspace{1cm} \textbf{(1,5 pt)}
    \item 
      \begin{enumerate}
        \item Montrer que $(\mathcal{C}_f)$ admet en $-\infty$ une asymptote oblique $\Delta_1$ dont on déterminera l’équation. \textbf{(0,5 pt)}
        \item Étudier la position relative de $(\mathcal{C}_f)$ par rapport à $\Delta_1$ sur $(-\infty;0[$. \hspace{1cm} \textbf{(0,5 pt)}
      \end{enumerate}
    \item Étudier la nature de la branche infinie en $+\infty$. \hspace{1cm} \textbf{(0,5 pt)}
    \item Préciser l'ensemble de dérivabilité de $f$ puis calculer $f'(x)$ sur chaque intervalle où $f$ est dérivable.\\ \textbf{(0,5 pt)}
    \item Dresser le tableau de variation de $f$. \hspace{1cm} \textbf{(0,5 pt)}
    \item Préciser les points d'intersection de $(\mathcal{C}_f)$ avec les axes du repère. \textbf{(0,25 pt)}
    \item Construire la courbe $(\mathcal{C}_f)$. \hspace{1cm} \textbf{(1,5 pt)}
\end{enumerate}

\underline{\textbf{Partie B :}}

Soit $h$ la restriction de $f$ à l’intervalle $I = [0; +\infty[$.

\begin{enumerate}

    \item Montrer que $h$ réalise une bijection de $I$ dans un intervalle $J$ à préciser.
    \hspace{1cm} \textbf{(0,25 pt)}

    \item La bijection réciproque $h^{-1}$ est-elle dérivable sur $J$ ?
    \hspace{1cm} \textbf{(0,25 pt)}

    \item Calculer $h\!\left(\dfrac{4}{5}\right)$ puis $(h^{-1})'(2)$.
   \hspace{1cm} \textbf{(0,5 pt)}

    \item Construire $(\mathcal{C}_{h^{-1}})$ la courbe représentative de $h^{-1}$ dans le même repère.
    \hspace{1cm} \textbf{(0,5 pt)}

    \item Exprimer $h^{-1}(x)$ pour tout $x \in J$.
    \hspace{1cm} \textbf{(0,5 pt)}

\end{enumerate}

\end{document}
