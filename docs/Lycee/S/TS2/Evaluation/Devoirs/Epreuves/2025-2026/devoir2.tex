\documentclass[12pt,a4paper]{article}
\usepackage{amsmath,amssymb,mathrsfs,tikz,times,pifont}
\usepackage{enumitem}
\newcommand\circitem[1]{%
\tikz[baseline=(char.base)]{
\node[circle,draw=gray, fill=red!55,
minimum size=1.2em,inner sep=0] (char) {#1};}}
\newcommand\boxitem[1]{%
\tikz[baseline=(char.base)]{
\node[fill=cyan,
minimum size=1.2em,inner sep=0] (char) {#1};}}
\setlist[enumerate,1]{label=\protect\circitem{\arabic*}}
\setlist[enumerate,2]{label=\protect\boxitem{\alph*}}
%%%::::::by chnini ameur :::::::%%%
\everymath{\displaystyle}
\usepackage[left=1cm,right=1cm,top=1cm,bottom=1.7cm]{geometry}
\usepackage{array,multirow}
\usepackage[most]{tcolorbox}
\usepackage{varwidth}
\tcbuselibrary{skins,hooks}
\usetikzlibrary{patterns}
%%%::::::by chnini ameur :::::::%%%
\newtcolorbox{exa}[2][]{enhanced,breakable,before skip=2mm,after skip=5mm,
colback=yellow!20!white,colframe=black!20!blue,boxrule=0.5mm,
attach boxed title to top left ={xshift=0.6cm,yshift*=1mm-\tcboxedtitleheight},
fonttitle=\bfseries,
title={#2},#1,
% varwidth boxed title*=-3cm,
boxed title style={frame code={
\path[fill=tcbcolback!30!black]
([yshift=-1mm,xshift=-1mm]frame.north west)
arc[start angle=0,end angle=180,radius=1mm]
([yshift=-1mm,xshift=1mm]frame.north east)
arc[start angle=180,end angle=0,radius=1mm];
\path[left color=tcbcolback!60!black,right color = tcbcolback!60!black,
middle color = tcbcolback!80!black]
([xshift=-2mm]frame.north west) -- ([xshift=2mm]frame.north east)
[rounded corners=1mm]-- ([xshift=1mm,yshift=-1mm]frame.north east)
-- (frame.south east) -- (frame.south west)
-- ([xshift=-1mm,yshift=-1mm]frame.north west)
[sharp corners]-- cycle;
},interior engine=empty,
},interior style={top color=yellow!5}}
%%%%%%%%%%%%%%%%%%%%%%%

\usepackage{fancyhdr}
\usepackage{eso-pic}         % Pour ajouter des éléments en arrière-plan
% Commande pour ajouter du texte en arrière-plan
\AddToShipoutPicture{
    \AtTextCenter{%
        \makebox[0pt]{\rotatebox{80}{\textcolor[gray]{0.7}{\fontsize{5cm}{5cm}\selectfont PGB}}}
    }
}
\usepackage{lastpage}
\fancyhf{}
\pagestyle{fancy}
\renewcommand{\footrulewidth}{1pt}
\renewcommand{\headrulewidth}{0pt}
\renewcommand{\footruleskip}{10pt}
\fancyfoot[R]{
\color{blue}\ding{45}\ \textbf{2025}
}
\fancyfoot[L]{
\color{blue}\ding{45}\ \textbf{Prof:M. BA}
}
\cfoot{\bf
\thepage /
\pageref{LastPage}}
\begin{document}
\renewcommand{\arraystretch}{1.5}
\renewcommand{\arrayrulewidth}{1.2pt}
\begin{tikzpicture}[overlay,remember picture]
\node[draw=blue,line width=1.2pt,fill=purple,text=blue,inner sep=3mm,rounded corners,pattern=dots]at ([yshift=-2.5cm]current page.north) {\begingroup\setlength{\fboxsep}{0pt}\colorbox{white}{\begin{tabular}{|*1{>{\centering \arraybackslash}p{0.28\textwidth}} |*2{>{\centering \arraybackslash}p{0.2\textwidth}|} *1{>{\centering \arraybackslash}p{0.19\textwidth}|} }
\hline
\multicolumn{3}{|c|}{$\diamond$$\diamond$$\diamond$\ \textbf{Lycée de Dindéfélo}\ $\diamond$$\diamond$$\diamond$ }& \textbf{A.S. : 2025/2026} \\ \hline
\textbf{Matière: Mathématiques}& \textbf{Niveau : T}\textbf{S2} &\textbf{Date: 15/12/2025} & \textbf{Durée : 4 heures} \\ \hline
\multicolumn{4}{|c|}{\parbox[c]{10cm}{\begin{center}
\textbf{{\Large\sffamily Devoir n$ ^{\circ} $ 2 Du 1$ ^\text{\bf er} $ Semestre}}
\end{center}}} \\ \hline
\end{tabular}}\endgroup};
\end{tikzpicture}
\vspace{3cm}

\section*{\underline{Exercice 1 :} (03 points $\approx$ 36 mns)}
\begin{enumerate}
    \item Déterminer une primitive $F$ de la fonction $f$ sur $I$.
    \begin{enumerate}
        \item $f(x) = (6x - 3)(4x^2 - 4x + 2)^3 \quad ; I = \mathbb{R}$ \hfill $(0,5pt)$
        \item $f(x) = \dfrac{\sin x}{\sqrt{3 + \cos x}} \quad ; I = \mathbb{R}$ \hfill $(0,5pt)$
        \item $f(x) = 2\cos 3x - 3\sin 2x \quad ; I = \mathbb{R}$ \hfill $(0,5pt)$
        \item $f(x) = \dfrac{3x}{\sqrt{x^2 - 1}} + \sin x \sin 2x \quad ; I = ]-\infty; -1[$ \hfill $(0,5pt)$
    \end{enumerate}
    \item Soit $k$ la fonction définie sur $\mathbb{R} \setminus \{-2\}$ par : $k(x) = \dfrac{3x - 2}{(x+2)^3}$.
    \begin{enumerate}
        \item Déterminer les $a$ et $b$ tels que $\forall x \in \mathbb{R} \setminus \{-2\}$, $k(x) = \dfrac{a}{(x+2)^2} + \dfrac{b}{(x+2)^3}$. \hfill $(0,5pt)$
        \item En déduire la primitive $K$ de $k$ qui prend la valeur 2 en -3. \hfill $(0,5pt)$
    \end{enumerate}
\end{enumerate}

\section*{\underline{Exercice 2 :} (05 points $\approx$ 60 mns)}
\begin{enumerate}
    \item 
    \begin{enumerate}
        \item Écrivez le complexe $z = -1 + i\sqrt{3}$ sous forme trigonométrique puis sous forme exponentielle.\hfill $(1pt)$
        \item Donner le module et un argument de $z = (1 - \sqrt{2})e^{i\frac{\pi}{4}}$.\hfill $(1pt)$
    \end{enumerate}
    \item 
    \begin{enumerate}
        \item Donner le module et un argument de $z_1 = \dfrac{\sqrt{6}-i\sqrt{2}}{2}$ et $z_2 = 1 - i$. \hfill $(1pt)$
        \item Déduis en le module et un argument de chacun des complexes $u = \dfrac{z_1}{z_2}$ et $u^5$. \hfill $(1pt)$
        \item Déduis en les valeurs exactes de $\cos\left(\frac{\pi}{12}\right)$ et $\sin\left(\frac{\pi}{12}\right)$.\hfill $(1pt)$
    \end{enumerate}
\end{enumerate}

\section*{\underline{Problème :} ( 12 points $\approx$ 144 mns)}
On note $(C_f)$ la courbe représentative de la fonction $f$ dans un repère orthonormé direct $(O;\vec{i},\vec{j})$ d'unité $1cm$ avec
$$
f(x) =
\begin{cases}
\dfrac{x^3 - 2x^2}{x^2 - 1} & \text{; si } x \leq 0 \\
x + \sqrt{x^2 + x} & \text{; si } x > 0
\end{cases}
$$

\underline{\textbf{Partie A : Étude d'une fonction auxiliaire :}}

Soit $g$ la fonction numérique définie par : $g(x) = -x^3 + 3x - 4$.

\begin{enumerate}
    \item Étudier les variations de $g$ puis dresser son tableau de variations. \hfill $(0,75pt)$
    \item 
    \begin{enumerate}
        \item Montrer que l'équation $g(x)=0$ admet une unique solution $\alpha \in \mathbb{R}$. \hfill $(0,5pt)$
        \item Donner un encadrement de $\alpha$ à $10^{-1}$. \hfill $(0,5pt)$
    \end{enumerate}
    \item En déduire le signe de $g(x)$ sur $]-\infty; 0]$. \hfill $(0,25pt)$
\end{enumerate}

\hrule
\vskip 0.5cm

% --- Partie B ---
\underline{\textbf{Partie B : Étude de la fonction $f$}}
\begin{enumerate}
    \item Déterminer l'ensemble de définition $D_f$ de $f$. \hfill $(0,5pt)$
    \item 
    \begin{enumerate}
        \item Calculer les limites de $f$ aux bornes de $D_f$ et en déduire l'existence d'une asymptote dont on précisera. \hfill $(01pt)$
        \item Pour $x \leq 0$, déterminer les réels $a, b, c$ et $d$ tels que : 
        $f(x)=ax+b+\dfrac{cx+d}{x^2-1}$
        et en déduire la nature de la branche infinie en $-\infty$. \hfill $(0,75pt)$
        \item Donner la nature de la branche infinie en $+\infty$. \hfill $(0,5pt)$
    \end{enumerate}
    \item 
    \begin{enumerate}
        \item Montrer que $f$ est continue sur $D_f$. \hfill $(0,75pt)$
        \item Étudier la dérivabilité de $f$ en $0$ puis interpréter le résultat obtenu. \hfill $(01pt)$
    \end{enumerate}
    \item Démontrer que : $f(\alpha) = \dfrac{3}{2}\alpha - 2 $. \hfill $(0,5pt)$
    \item 
    \begin{enumerate}
        \item Montrer que $\forall x \in ]-\infty; -1[ \cup ]-1; 0[$, on a : 
        $f'(x) = \dfrac{-xg(x)}{(x^2-1)^2}$ \hfill $(0,5pt)$
        \item Calculer $f'(x)$ sur $]0; +\infty[$ et étudier son signe. \hfill $(0,75pt)$
        \item Dresser le tableau de variation de $f$ sur $D_f$. \hfill $(0,75pt)$
    \end{enumerate}
    \item Construire soigneusement la courbe $(C_f)$. \hfill $(01pt)$
\end{enumerate}

\underline{\textbf{Partie C : Bijection}}

Soit $h$ la restriction de $f$ sur $I = ]0; +\infty[$.
\begin{enumerate}
    \item Montrer que $h$ réalise une bijection de $I$ vers un intervalle $J$ à préciser. \hfill $(0,5pt)$
    \item Calculer $h^{-1}(1)$ et $(h^{-1})'(\sqrt{2}+1)$. \hfill $(0,5pt)$
    \item Expliciter $h^{-1}(x)$. \hfill $(0,5pt)$
    \item Construire $(C_{h^{-1}})$ dans le repère précédent. \hfill $(0,5pt)$
\end{enumerate}

\end{document}
