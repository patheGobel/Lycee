\documentclass[a4paper,12pt]{article}
\usepackage{graphicx}
\usepackage[a4paper, top=0cm, bottom=2cm, left=2cm, right=2cm]{geometry} % Ajuste les marges
\usepackage{xcolor} % Pour ajouter des couleurs
\usepackage{hyperref} % Pour avoir des références colorées si nécessaire
\usepackage{eso-pic}         % Pour ajouter des éléments en arrière-plan

\usepackage[french]{babel}
\usepackage[T1]{fontenc}
\usepackage{mathrsfs}
\usepackage{amsmath}
\usepackage{amsfonts}
\usepackage{amssymb}
\usepackage{tkz-tab}

\usepackage{tikz}
\usetikzlibrary{arrows, shapes.geometric, fit}
\newcounter{correction} % Compteur pour les questions

% Définir la commande pour afficher une question numérotée
\newcommand{\question}{%
  \refstepcounter{correction}%
  \textbf{\textcolor{black}{Question \thecorrection (1 point) :}} \ignorespaces
}
% Commande pour ajouter du texte en arrière-plan
\AddToShipoutPicture{
    \AtTextCenter{%
        \makebox[0pt]{\rotatebox{80}{\textcolor[gray]{0.7}{\fontsize{10cm}{10cm}\selectfont PGB}}}
    }
}
\begin{document}

% En-tête
\begin{center}
    \begin{tabular}{@{} p{5cm} p{5cm} p{5cm} @{}}\\[0.2cm] % 3 colonnes avec largeurs fixées
        Lycée Dindéfélo & Test 2 & 07 Novembre 2025 \\
    \end{tabular}
    \\[-0.01cm] % Ajuster l'espace vertical entre le tableau et la barre
    \hrule % Barre horizontale
\end{center}
\begin{center}
    \textbf{\Large Asymptotes et Branches Infinies} \\[0.2cm]
    \textbf{\large Professeur : M. BA} \\[0.2cm]
    \textbf{Classe : Terminale S2} \\[0.2cm]
    \textbf{\small Durée : 10 minutes} \\[0.2cm]
    \textbf{\small Note :\quad\quad /5}
\end{center}


\question
On dit que la droite d'équation \( y = d \) est une asymptote horizontale à la courbe représentative de \( h \) si :

\[
\lim_{x \to +\infty} h(x) = \underline{d\hspace{0.5cm}}, \quad \lim_{x \to -\infty} h(x) = \underline{d\hspace{0.5cm}}
\]

\question\
Complétez la phrase suivante : La droite d'équation \( x = c \) est une asymptote verticale à la courbe représentative de \( h \) si \underline{$\lim\limits_{x \to +\infty} h(x) = \infty $\hspace{0.5cm}}\\[0.3cm]

\question\\
Soit \( h(x) = \frac{3x + 1}{-x + 2} \). \\
Déterminez les limites de \( h(x) \) en \( x \to +\infty \), \( x \to -\infty \) et \( x \to 2 \) :
\[
\lim_{x \to +\infty} h(x) = \underline{-3\hspace{1cm}}, \quad \lim_{x \to -\infty} h(x) = \underline{-3\hspace{1cm}}, \quad \lim_{x \to 2^-} h(x) = \underline{ +\infty \hspace{1cm}}, \quad 
\]
\[\lim_{x \to 2^+} h(x) = \underline{ -\infty \hspace{1cm}}.\]

\question\\
Montrez que la droite \( y = 3x - 1 \) est une asymptote oblique de la fonction \( h(x) = \frac{3x^2 + 2x}{x+1} \) en \( +\infty \). \\[0.2cm]

\vspace{0.5em}

Pour prouver que la droite $\mathcal{D}$ est une asymptote oblique à la courbe de $h$ en $+\infty$, nous devons montrer que :
$$ \lim_{x \to +\infty} \left[ h(x) - (3x - 1) \right] = 0 $$

\vspace{0.5em}

\textbf{Calcul de la différence $h(x) - (3x - 1)$ :}
\begin{align*}
    h(x) - (3x - 1) &= \frac{3x^2 + 2x}{x+1} - (3x - 1) \\
    &= \frac{3x^2 + 2x}{x+1} - \frac{(3x - 1)(x+1)}{x+1} \\
    &= \frac{3x^2 + 2x - (3x^2 + 3x - x - 1)}{x+1} \\
    &= \frac{3x^2 + 2x - (3x^2 + 2x - 1)}{x+1} \\
    &= \frac{3x^2 + 2x - 3x^2 - 2x + 1}{x+1} \\
    &= \frac{1}{x+1}
\end{align*}

\vspace{0.5em}

\textbf{Calcul de la limite :}
Nous calculons la limite de cette différence lorsque $x \to +\infty$ :
$$ \lim_{x \to +\infty} [h(x) - (3x - 1)] = \lim_{x \to +\infty} \frac{1}{x+1} $$
Lorsque $x \to +\infty$, le dénominateur $(x+1)$ tend vers $+\infty$. Par conséquent :
$$ \lim_{x \to +\infty} \frac{1}{x+1} = 0 $$

\vspace{0.5em}

\textbf{Conclusion :}
Puisque $\lim_{x \to +\infty} [h(x) - (3x - 1)] = 0$, la droite d'équation $y = 3x - 1$ est bien une \textbf{asymptote oblique} de la fonction $h(x)$ en $+\infty$.

\question\\

$$\text{Si } \lim_{x \to -\infty} h(x) =-\infty \text{ et } \lim_{x \to -\infty}\frac{h(x)}{x}= +\infty \text{ alors } (C_{h}):$$

$$ \textcolor{red}{\underline{\text{admet une \textbf{branche parabolique} de \textbf{direction l'axe des ordonnées} (l'axe } Oy\text{) en } -\infty}} $$

$$ \text{Si } \lim_{x \to +\infty} h(x) =+\infty \text{ et } \lim_{x \to +\infty}\frac{h(x)}{x}=\gamma \in\mathbb{R}^{*}\text{ et }\lim_{x \to +\infty}[h(x)-\gamma x]=+\infty\text{ alors } (C_{h}) $$

$$ \textcolor{red}{ \underline{\text{admet une \textbf{branche parabolique} de \textbf{direction} la droite } y = \gamma x \text{ en } +\infty} } $$

\end{document}
