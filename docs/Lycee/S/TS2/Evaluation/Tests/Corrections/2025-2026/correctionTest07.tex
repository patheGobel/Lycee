\documentclass[a4paper,12pt]{article}
\usepackage{graphicx}
\usepackage[a4paper, top=0cm, bottom=2cm, left=2cm, right=2cm]{geometry}

\usepackage{xcolor}
\usepackage{hyperref}
\usepackage{eso-pic}

\usepackage[french]{babel}
\usepackage[T1]{fontenc}
\usepackage{amsmath, amsfonts, amssymb, mathrsfs}

% Filigrane
\AddToShipoutPicture{
    \AtTextCenter{%
        \makebox[0pt]{\rotatebox{80}{\textcolor[gray]{0.9}{\fontsize{10cm}{10cm}\selectfont PGB}}}
    }
}

\begin{document}

\hrule
\begin{center}
    \begin{tabular}{@{} p{5cm} p{5cm} p{5cm} @{}} 
        Lycée Dindéfélo & \quad\quad \textbf{Correction – Test 7} & 02 Décembre 2025 \\
    \end{tabular}
    \\[-0.01cm]
    \hrule
\end{center}

\begin{center}
    \textbf{\Large Correction : Nombres Complexes et Propriétés} \\[0.2cm]
    \textbf{\large Professeur : M. BA} \\[0.3cm]
\end{center}

% --- CORRECTIONS ---

\section*{Correction de l’épreuve}

\textbf{Question 1. Définir le conjugué d’un nombre complexe.}

Pour un nombre complexe \( z = a + ib \) (avec \( a,b \in \mathbb{R} \)),  
son conjugué est :
\[
\overline{z} = a - ib.
\]

\textbf{Propriétés :}
\[
z + \overline{z} = 2a, \qquad z\,\overline{z} = a^2 + b^2 = |z|^2.
\]

\vspace{0.5cm}

\textbf{Question 2. Définir le module et son interprétation géométrique.}

Pour \( z = a + ib \), le module est :
\[
|z| = \sqrt{a^2 + b^2}.
\]

\textbf{Interprétation géométrique :}  
C’est la distance du point associé à \( z \) au point \( O(0,0) \) dans le plan complexe.

\vspace{0.5cm}

\textbf{Question 3. Définir l’argument d’un complexe non nul.}

Pour \( z = a + ib \neq 0 \), un argument de \( z \) est un réel \( \theta \) tel que :
\[
z = |z|\big(\cos\theta + i\sin\theta\big).
\]

\textbf{Ensemble des arguments :}
\[
\arg(z) = \theta + 2k\pi,\quad k \in \mathbb{Z}.
\]

\vspace{0.5cm}

\textbf{Question 4. Citer trois propriétés du module et du conjugué.}

\[
|z| = |\overline{z}|
\]
\[
|z\,w| = |z|\;|w|
\]
\[
|z + w| \le |z| + |w| \quad \text{(inégalité triangulaire)}
\]

\[
\overline{z+w} = \overline{z} + \overline{w}, \qquad 
\overline{zw} = \overline{z}\,\overline{w}
\]

\vspace{0.5cm}

\textbf{Question 5. Citer trois propriétés de l’argument.}

\[
\arg(zw) \equiv \arg(z) + \arg(w) \; [2\pi]
\]
\[
\arg\left(\frac{z}{w}\right) \equiv \arg(z) - \arg(w) \; [2\pi]
\]
\[
\arg(\overline{z}) \equiv -\arg(z) \; [2\pi]
\]

Autre propriété utile :
\[
\arg\big(\cos\theta + i\sin\theta\big) = \theta.
\]

\end{document}
