\documentclass[a4paper,12pt]{article}
\usepackage{graphicx}
\usepackage[a4paper, top=0cm, bottom=2cm, left=2cm, right=2cm]{geometry}

\usepackage{xcolor}
\usepackage{hyperref}
\usepackage{eso-pic}

\usepackage[french]{babel}
\usepackage[T1]{fontenc}
\usepackage{amsmath, amsfonts, amssymb, mathrsfs}

\newcounter{correction}

\newcommand{\question}{%
  \refstepcounter{correction}%
  \textbf{\textcolor{black}{Question \thecorrection\ (1 point) :}} \ignorespaces
}

% Filigrane
\AddToShipoutPicture{
    \AtTextCenter{%
        \makebox[0pt]{\rotatebox{80}{\textcolor[gray]{0.6}{\fontsize{10cm}{10cm}\selectfont PGB}}}
    }
}

\begin{document}

\hrule
\begin{center}
    \begin{tabular}{@{} p{5cm} p{5cm} p{5cm} @{}} 
        Lycée Dindéfélo & \quad\quad Test 8 & 05 Décembre 2025 \\
    \end{tabular}
    \\[-0.01cm]
    \hrule
\end{center}

\begin{center}
    \textbf{\Large Nombres Complexes et Propriétés} \\[0.2cm]
    \textbf{\large Professeur : M. BA} \\[0.2cm]
    \textbf{Classe : Terminale S2} \\[0.2cm]
    \textbf{\small Durée : 10 minutes}\\[0.2cm]
    \textbf{\small Note : \quad /5}
\end{center}

\textbf{\small Nom de l'élève :} \underline{\hspace{8cm}} \\[0.7cm]

% --- Début évaluation ---

\textbf{Rappeler les principales notions sur les nombres complexes :} \\[0.3cm]

\question\ Donner le conjugué  . \\[0.3cm]
\( z_{1} = 2i \left( \dfrac{1}{\sqrt{2}}-\dfrac{1}{\sqrt{2}}i \right) \)
\underline{\hspace{20cm}}\\[0.3cm]
\underline{\hspace{20cm}}\\[0.3cm]
\underline{\hspace{20cm}}\\[0.3cm]
\underline{\hspace{20cm}}\\[0.3cm]


\(z_{2} = (1 + i)(1 - 2i)(1 + 3i) \)
\underline{\hspace{20cm}}\\[0.3cm]
\underline{\hspace{20cm}}\\[0.3cm]
\underline{\hspace{20cm}}\\[0.3cm]
\underline{\hspace{20cm}}\\[0.3cm]

\question\ Donner la forme algébrique  . \\[0.3cm]
\( z_{1} = 2i \left( \dfrac{1}{\sqrt{2}}-\dfrac{1}{\sqrt{2}}i \right) \)
\underline{\hspace{20cm}}\\[0.3cm]
\underline{\hspace{20cm}}\\[0.3cm]
\underline{\hspace{20cm}}\\[0.3cm]
\underline{\hspace{20cm}}\\[0.3cm]
\underline{\hspace{20cm}}\\[0.3cm]
\underline{\hspace{20cm}}\\[0.3cm]

\(z_{2} = (1 + i)(1 - 2i)(1 + 3i) \)
\underline{\hspace{20cm}}\\[0.3cm]
\underline{\hspace{20cm}}\\[0.3cm]
\underline{\hspace{20cm}}\\[0.3cm]
\underline{\hspace{20cm}}\\[0.3cm]
\underline{\hspace{20cm}}\\[0.3cm]
\underline{\hspace{20cm}}\\[0.3cm]

\question\ Donner le module  . \\[0.3cm]
\(z_{2} = (1 + i)(1 - 2i)(1 + 3i) \) 
\underline{\hspace{20cm}}\\[0.3cm]
\underline{\hspace{20cm}}\\[0.3cm]
\underline{\hspace{20cm}}\\[0.3cm]
\underline{\hspace{20cm}}\\[0.3cm]
\underline{\hspace{20cm}}\\[0.3cm]
\underline{\hspace{20cm}}\\[0.3cm]

\( z_{3} = \left( \dfrac{1}{2}-\dfrac{\sqrt{3}}{2}i \right)^{4} \)
\underline{\hspace{20cm}}\\[0.3cm]
\underline{\hspace{20cm}}\\[0.3cm]
\underline{\hspace{20cm}}\\[0.3cm]
\underline{\hspace{20cm}}\\[0.3cm]
\underline{\hspace{20cm}}\\[0.3cm]
\underline{\hspace{20cm}}\\[0.3cm]


\question\ Donner l'argument  . \\[0.3cm]
\( z_{1} = 2i \left( \dfrac{1}{\sqrt{2}}-\dfrac{1}{\sqrt{2}}i \right) \)
\underline{\hspace{20cm}}\\[0.3cm]
\underline{\hspace{20cm}}\\[0.3cm]
\underline{\hspace{20cm}}\\[0.3cm]
\underline{\hspace{20cm}}\\[0.3cm]
\underline{\hspace{20cm}}\\[0.3cm]
\underline{\hspace{20cm}}\\[0.3cm]

\( z_{3} = \left( \dfrac{1}{2}-\dfrac{\sqrt{3}}{2}i \right)^{4} \)
\underline{\hspace{20cm}}\\[0.3cm]
\underline{\hspace{20cm}}\\[0.3cm]
\underline{\hspace{20cm}}\\[0.3cm]
\underline{\hspace{20cm}}\\[0.3cm]
\underline{\hspace{20cm}}\\[0.3cm]
\underline{\hspace{20cm}}\\[0.3cm]

\end{document}
