\documentclass[a4paper,12pt]{article}

\usepackage[a4paper, margin=0.5cm]{geometry} % marges minimales
\usepackage{amsmath}
\usepackage{amssymb}
\usepackage{lmodern} % police plus nette
\usepackage{bold-extra} % permet du texte plus foncé
\renewcommand{\seriesdefault}{b} % rend tout plus foncé

\begin{document}

\tiny

\textbf{Exercice D'application} \\[2pt]

Soit $f$ la fonction définie par :
\(
f(x) =
\begin{cases} 
2x\sqrt{1 - x^2} & \text{si } x > 0, \\[4pt]
-x + \sqrt{x^2 - 2x} & \text{si } x \leq 0.
\end{cases}
\)

\begin{enumerate}
    \item Déterminer $D_f$, les limites aux bornes et préciser les asymptotes et branches infinies éventuelles.

    \item Étudier la dérivabilité de $f$ en 0 et 1 ; interpréter géométriquement les résultats obtenus.

    \item Calculer $f'(x)$ là où $f$ est définie, puis dresser le tableau de variation de $f$.

    \item Tracer la courbe de $f$.

    \item Soit $h$ la restriction de $f$ à l’intervalle $]-\infty ; 0]$.
    \begin{enumerate}
        \item Montrer que $h$ admet une bijection réciproque $h^{-1}$ dont on précisera l’ensemble de définition, l’ensemble de dérivabilité et le tableau de variation.

        \item Sans utiliser l’expression de $h^{-1}(x)$, calculer $(h^{-1})'(2)$.

        \item Déterminer explicitement $h^{-1}$.

        \item Tracer la courbe de $h^{-1}$ dans le même repère que celle de $f$.
    \end{enumerate}
\end{enumerate}

\end{document}
