\documentclass[12pt,a4paper]{article}
\usepackage{amsmath,amssymb,mathrsfs,tikz,times,pifont}
\usepackage{enumitem}
\usepackage{multicol}
\usepackage{lmodern}
\usepackage{diagbox}
\newcommand\circitem[1]{%
\tikz[baseline=(char.base)]{
\node[circle,draw=gray, fill=red!55,
minimum size=1.2em,inner sep=0] (char) {#1};}}
\newcommand\boxitem[1]{%
\tikz[baseline=(char.base)]{
\node[fill=cyan,
minimum size=1.2em,inner sep=0] (char) {#1};}}
\setlist[enumerate,1]{label=\protect\circitem{\arabic*}}
\setlist[enumerate,2]{label=\protect\boxitem{\alph*}}
\everymath{\displaystyle}
\usepackage[left=1cm,right=1cm,top=1cm,bottom=1.7cm]{geometry}
\usepackage[colorlinks=true, linkcolor=blue, urlcolor=blue, citecolor=blue]{hyperref}
\usepackage{array,multirow}
\usepackage[most]{tcolorbox}
\usepackage{varwidth}
\usepackage{float}
\tcbuselibrary{skins,hooks}
\usetikzlibrary{patterns}

\newtcolorbox{exa}[2][]{enhanced,breakable,before skip=2mm,after skip=5mm,
colback=yellow!20!white,colframe=black!20!blue,boxrule=0.5mm,
attach boxed title to top left ={xshift=0.6cm,yshift*=1mm-\tcboxedtitleheight},
fonttitle=\bfseries,
title={#2},#1,
boxed title style={frame code={
\path[fill=tcbcolback!30!black]
([yshift=-1mm,xshift=-1mm]frame.north west)
arc[start angle=0,end angle=180,radius=1mm]
([yshift=-1mm,xshift=1mm]frame.north east)
arc[start angle=180,end angle=0,radius=1mm];
\path[left color=tcbcolback!60!black,right color = tcbcolback!60!black,
middle color = tcbcolback!80!black]
([xshift=-2mm]frame.north west) -- ([xshift=2mm]frame.north east)
[rounded corners=1mm]-- ([xshift=1mm,yshift=-1mm]frame.north east)
-- (frame.south east) -- (frame.south west)
-- ([xshift=-1mm,yshift=-1mm]frame.north west)
[sharp corners]-- cycle;
},interior engine=empty,
},interior style={top color=yellow!5}}

\usepackage{fancyhdr}
\usepackage{eso-pic}
\usepackage{tkz-tab}
\AddToShipoutPicture{
    \AtTextCenter{%
        \makebox[0pt]{\rotatebox{80}{\textcolor[gray]{0.7}{\fontsize{5cm}{5cm}\selectfont PGB}}}
    }
}
\usepackage{lastpage}
\fancyhf{}
\pagestyle{fancy}
\renewcommand{\footrulewidth}{1pt}
\renewcommand{\headrulewidth}{0pt}
\renewcommand{\footruleskip}{10pt}
\fancyfoot[R]{\color{blue}\ding{45}\ \textbf{2025}}
\fancyfoot[L]{\color{blue}\ding{45}\ \textbf{Prof : M. BA}}
\cfoot{\bf \thepage / \pageref{LastPage}}

\newcommand{\exo}[1]{%
        \textbf{\underline{Exercice #1}}
}

\begin{document}
\renewcommand{\arraystretch}{1.5}
\renewcommand{\arrayrulewidth}{1.2pt}
\begin{tikzpicture}[overlay,remember picture]
    \node[draw=blue,line width=1.2pt,fill=purple,text=blue,inner sep=3mm,rounded corners,pattern=dots]at ([yshift=-2.5cm]current page.north) {\begingroup\setlength{\fboxsep}{0pt}\colorbox{white}{\begin{tabular}{|*1{>{\centering \arraybackslash}p{0.28\textwidth}} |*2{>{\centering \arraybackslash}p{0.2\textwidth}|} *1{>{\centering \arraybackslash}p{0.19\textwidth}|} }
                \hline
                \multicolumn{3}{|c|}{$\diamond$$\diamond$$\diamond$\ \textbf{Lycée de Dindéfélo}\ $\diamond$$\diamond$$\diamond$ } & \textbf{A.S. : 2024/2025} \\ \hline
                \textbf{Matière : Mathématiques} & \textbf{Niveau : T S2} & \textbf{Date : 29/05/2025} & \textbf{} \\ \hline
                \multicolumn{4}{|c|}{\parbox[c]{10cm}{\begin{center}
                  \textbf{{\Large\sffamily TD : Statistiques}}
                \end{center}}} \\ \hline
            \end{tabular}}\endgroup};
\end{tikzpicture}
\vspace{3cm}
\noindent
\section*{\underline{Exercice 1}(BAC 2005)}
Une entreprise a mis au point un nouveau produit et cherche à fixer le prix de vente.\\
Une enquête est réalisée auprès des clients potentiels ; les résultats sont donnés dans le tableau suivant où \( y_i \) représente le nombre d’exemplaires du produit que les clients sont disposés à acheter si le prix de vente, exprimé en milliers de francs, est \( x_i \) :

\vspace{0.3cm}
\begin{center}
\begin{tabular}{|c|cccccccc|}
\hline
\( x_i \) & 60 & 80 & 100 & 120 & 140 & 160 & 180 & 200 \\
\hline
\( y_i \) & 952 & 805 & 630 & 522 & 510 & 324 & 205 & 84 \\
\hline
\end{tabular}
\end{center}

\vspace{0.4cm}

\begin{enumerate}
    \item \textbf{\textcolor{red}{Calculer le coefficient de corrélation linéaire de \( y \) et \( x \).}}\\
    La valeur trouvée justifie-t-elle la recherche d’un ajustement linéaire ?

    \item \textbf{\textcolor{red}{Déterminer l’équation de la droite de régression de \( y \) en \( x \).}}

    \item \textbf{\textcolor{red}{Les frais de conception du produit se sont élevés à 28 millions de francs. Le prix de fabrication de chaque produit est de 25 000 francs.}}

    \begin{enumerate}
        \item \textbf{Déduire de la précédente question que le bénéfice \( z \) en fonction du prix de vente \( x \) est donné par l’égalité :}
        \[
        z = -5{,}95x^2 + 1426{,}25x - 59937{,}5
        \]
        où \( x \) et \( z \) sont exprimés en milliers de francs.

        \item \textbf{Déterminer le prix de vente \( x \) permettant de réaliser un bénéfice maximum et calculer ce bénéfice.}
    \end{enumerate}
\end{enumerate}

\vspace{0.4cm}
\textcolor{red}{\textbf{NB}} : Prendre 2 chiffres après la virgule sans arrondir.

\vspace{0.2cm}
\textcolor{red}{\textbf{Rappel}} : Bénéfice = Prix de vente – prix de revient.
\section*{\underline{Exercice 2}(04,5 points)(BAC 2008)}

Dans cet exercice, le détail des calculs n’est pas exigé. On donnera les formules utilisées pour répondre aux questions. Les résultats seront donnés à \( 10^{-1} \) près.\\
Le tableau ci-dessous donne le poids moyen (\( y \)) d’un enfant en fonction de son âge (\( x \)).

\vspace{0.3cm}

\begin{center}
\begin{tabular}{|c|c|c|c|c|c|c|c|}
\hline
\( x \) (années) & 0 & 1 & 2 & 4 & 7 & 11 & 12 \\
\hline
\( y \) (kg) & 3{,}5 & 6{,}5 & 9{,}5 & 14 & 21 & 32{,}5 & 34 \\
\hline
\end{tabular}
\end{center}

\vspace{0.5cm}

\begin{enumerate}
    \item Représenter le nuage de points de cette série statistique dans le plan muni du repère orthonormal.

    \hspace{0.5cm} \textit{Unité graphique : en abscisse 1 cm pour 1 année et en ordonnée 1 cm pour 2 kg.} \hfill \textbf{(01 point)}

    \item Déterminer les coordonnées du point moyen \( G \) puis placer \( G \). \hfill \textbf{(0,5 point)}

    \item
    \begin{enumerate}
        \item Déterminer le coefficient de corrélation linéaire \( r \). \hfill \textbf{(0,5 point)}
        \item Interpréter votre résultat. \hfill \textbf{(0,5 point)}
    \end{enumerate}

    \item Donner une équation de la droite de régression (D) de \( y \) en \( x \).\\
    \hspace*{0.5cm} Tracer (D). \hfill \textbf{(0,5 point + 0,5 point)}

    \item
    \begin{enumerate}
        \item Déterminer graphiquement, à partir de quel âge le poids sera supérieur à 15 kg. Expliciter votre raisonnement. \hfill \textbf{(0,5 point)}
        \item Retrouver ce résultat par le calcul. \hfill \textbf{(0,5 point)}
    \end{enumerate}
\end{enumerate}

\section*{\underline{Exercice 3}(03 points)(BAC 2009)}

\vspace{0.3cm}

\begin{enumerate}
    \item (\(X, Y\)) est une série statistique double. Soit \( (D_1) \) la droite de régression de \( Y \) en \( X \).\\

    Soit \( (D_2) \) la droite de régression de \( X \) en \( Y \). On suppose que :
    \[
    (D_1) : y = a x + b \quad \text{et} \quad (D_2) : x = a' y + b'
    \]
    
    Soit \( r \) le coefficient de corrélation linéaire entre \( X \) et \( Y \).\\
    
    Établir que \( r^2 = aa' \). \hfill \textbf{(01 point)}
    
    \item Dans une entreprise, une étude simultanée portant sur deux caractères \( X \) et \( Y \) donne les résultats suivants :
    \begin{itemize}
        \item la droite de régression de \( Y \) en \( X \) a pour équation : \( 2{,}4x - y = 0 \)
        \item la droite de régression de \( X \) en \( Y \) a pour équation : \( 3{,}5y - 9x + 24 = 0 \)
    \end{itemize}

    \begin{enumerate}
        \item Calculer le coefficient de corrélation linéaire entre \( X \) et \( Y \), sachant que leur covariance est positive. \hfill \textbf{(0,5 point)}
        
        \item Calculer la moyenne de chacun des caractères \( X \) et \( Y \). \hfill \textbf{(0,75 + 0,75 point)}
    \end{enumerate}
\end{enumerate}
\section*{\underline{Exercice 4}(03 points)(BAC 2010)}
Une étude sur le nombre d’années d’exercice \( X \), des ouvriers d’une entreprise et leur salaire mensuel \( Y \) en milliers de francs, a donné les résultats indiqués dans le tableau ci-dessous avec des données manquantes désignées par \( a \) et \( b \).

\vspace{0.3cm}

\begin{center}
\begin{tabular}{|c|c|c|c|c|c|c|}
\hline
\diagbox[width=4em]{\( Y \)}{\( X \)} & 2 & 6 & 10 & 14 & 18 & 22 \\
\hline
75 & \( a \) & 5 & 0 & 0 & 0 & 0 \\
\hline
125 & 0 & 7 & 1 & 0 & 2 & 0 \\
\hline
175 & 2 & 0 & 9 & 1 & 5 & 4 \\
\hline
225 & 0 & 1 & 0 & 3 & \( b \) & 1 \\
\hline
\end{tabular}
\end{center}

\vspace{0.5cm}

\begin{enumerate}
    \item Déterminer \( a \) et \( b \) pour que la moyenne de la série marginale de \( X \) soit égale à \( \frac{596}{59} \) et celle de la série marginale de \( Y \) soit \( \frac{8450}{59} \). \hfill \textbf{(0,25 + 0,25 pt)}
    
    \item Dans la suite, on suppose que \( a = 40 \) et \( b = 20 \). À chaque valeur \( x_i \) de \( X \), on associe la moyenne \( m_i \) de la série conditionnelle : \( Y/X = x_i \).\\
    On obtient ainsi la série double \( (X, M) \) définie par le tableau ci-dessous. Les calculs se feront à deux chiffres après la virgule.
    
    \begin{center}
    \begin{tabular}{|c|c|c|c|c|c|c|}
    \hline
    \( X \) & 2 & 6 & 10 & 14 & 18 & 22 \\
    \hline
    \( M \) & 80 & 113 & 170 & 189 & 199 & 185 \\
    \hline
    \end{tabular}
    \end{center}

    \begin{enumerate}
        \item Calculer le coefficient de corrélation de \( X \) et \( M \) puis interpréter le résultat. \hfill \textbf{(1,75 pt)}
        
        \item Déterminer l’équation de la droite de régression de \( M \) en \( X \). \hfill \textbf{(0,5 pt)}
        
        \item Quelle serait le salaire moyen d’un ouvrier de l’entreprise si son ancienneté était 30 ans, si cette tendance se poursuit. \hfill \textbf{(0,25 pt)}
    \end{enumerate}
\end{enumerate}
\section*{\underline{Exercice 5}(05 points)(BAC 2013)}
\noindent
Le tableau statistique ci-dessous donne le degré de salinité \( Y_i \) du Lac Rose pendant le \( i^\text{ème} \) mois de pluie, noté \( X_i \).

\vspace{0.4cm}

\begin{center}
\begin{tabular}{|c|c|c|c|c|c|}
\hline
\( X_i \) & 0 & 1 & 2 & 3 & 4 \\
\hline
\( Y_i \) & 4{,}26 & 3{,}4 & 2{,}01 & 1{,}16 & 1{,}01 \\
\hline
\end{tabular}
\end{center}

\vspace{0.5cm}

Dans ce qui suit il faudra rappeler chaque formule le cas échéant, avant de faire les calculs. On donnera les valeurs approchées par excès des résultats à \( 10^{-3} \) près.

\vspace{0.5cm}

\begin{enumerate}
    \item 
    \begin{enumerate}
        \item Déterminer le coefficient de corrélation linéaire de cette série \((X, Y)\) et interpréter le résultat.
        \hfill \textbf{(01,5 point = 0,25pt + 1,25pt)}

        \item Quelle est l’équation de la droite de régression de \( Y \) en \( X \). 
        \hfill \textbf{(0,5 pt = 0,25pt + 0,25pt)}

        \item Cette équation permet-elle d’estimer le degré de salinité du lac au \(6^\text{ième}\) mois de pluie, le cas échéant ?\\
        Justifier la réponse.
        \hfill \textbf{(0,25pt)}
    \end{enumerate}

    \item On pose \( Z = \ln(Y - 1) \).
    \begin{enumerate}
        \item Donner le tableau correspondant à la série \((X, Z)\). Les résultats seront arrondis au millième près.
        \hfill \textbf{(0,5 pt)}

        \item Donner le coefficient de corrélation linéaire de cette série \((X, Z)\).
        \hfill \textbf{(01,5 point = 0,25pt + 1,25pt)}

        \item Donner l’équation de la droite de régression de \( Z \) en \( X \), puis exprimer \( Y \) en fonction de \( X \).
        \hfill \textbf{(0,5 pt = 0,25pt + 0,25pt)}

        \item Utiliser cette équation pour répondre à la question 1(c).
        \hfill \textbf{(0,25pt)}
    \end{enumerate}
\end{enumerate}

\section*{\underline{Exercice 6}(02,5 points)(BAC 2015)}
\vspace{0.3cm}

Au Sénégal, une entreprise veut vérifier l’efficacité de son service de publicité. Elle a relevé chaque mois durant une période de 6 mois les sommes \( X \) consacrées à la publicité et le chiffre d’affaire constaté \( Y \) (en milliards de FCFA).\\

On donne le tableau ci-dessous :

\vspace{0.3cm}

\begin{center}
\begin{tabular}{|c|c|c|c|c|c|c|}
\hline
Rang du mois & 1 & 2 & 3 & 4 & 5 & 6 \\
\hline
\( X \) & 1{,}2 & 0{,}5 & 1 & 1 & 1{,}5 & 1{,}8 \\
\hline
\( Y \) & 19 & 49 & 100 & 125 & 148 & 181 \\
\hline
\end{tabular}
\end{center}

\vspace{0.4cm}

Les résultats seront donnés au centième près.\\
Le détail des calculs n’est pas indispensable. On précisera les formules utilisées.

\vspace{0.3cm}

\begin{enumerate}
    \item Calculer le coefficient de corrélation linéaire de \( X \) et \( Y \). \hfill \textbf{(01 pt)}

    \item 
    \begin{enumerate}
        \item Déterminer l’équation de la droite de régression de \( Y \) en \( X \). \hfill \textbf{(01 pt)}

        \item Déterminer la somme qu’il faut investir en publicité si l’on désire avoir un chiffre d’affaire de 300 milliards si cette tendance se poursuit. \hfill \textbf{(0,5 pt)}
    \end{enumerate}
\end{enumerate}

\end{document}