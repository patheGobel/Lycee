\documentclass[a4paper,12pt]{article}
\usepackage{graphicx}
\usepackage[a4paper, top=0cm, bottom=2cm, left=2cm, right=2cm]{geometry} % Ajuste les marges
\usepackage{xcolor} % Pour ajouter des couleurs
\usepackage{hyperref} % Pour avoir des références colorées si nécessaire
\usepackage{eso-pic}         % Pour ajouter des éléments en arrière-plan

\usepackage[french]{babel}
\usepackage[T1]{fontenc}
\usepackage{mathrsfs}
\usepackage{amsmath}
\usepackage{amsfonts}
\usepackage{amssymb}
\usepackage{tkz-tab}

\usepackage{tikz}
\usetikzlibrary{arrows, shapes.geometric, fit}
\newcounter{correction} % Compteur pour les questions

% Définir la commande pour afficher une question numérotée
\newcommand{\question}{%
  \refstepcounter{correction}%
  \textbf{\textcolor{black}{Question \thecorrection (1 point) :}} \ignorespaces
}
% Commande pour ajouter du texte en arrière-plan
\AddToShipoutPicture{
    \AtTextCenter{%
        \makebox[0pt]{\rotatebox{80}{\textcolor[gray]{0.9}{\fontsize{10cm}{10cm}\selectfont Pathé Gobel BA}}}
    }
}
\begin{document}
\hrule % Barre horizontale
% En-tête
\begin{center}
    \begin{tabular}{@{} p{5cm} p{5cm} p{5cm} @{}} % 3 colonnes avec largeurs fixées
        Lycée Dindéfélo & Test 4 & 04 Novembre 2024 \\
    \end{tabular}
    \\[-0.01cm] % Ajuster l'espace vertical entre le tableau et la barre
    \hrule % Barre horizontale
\end{center}
\begin{center}
    \textbf{\Large Limites et Continuité} \\[0.2cm]
    \textbf{\large Professeur : M. BA} \\[0.2cm]
    \textbf{Classe : Terminale S2} \\[0.2cm]
    \textbf{\small Durée : 10 minutes} \\[0.2cm]
    \textbf{\small Note :\quad\quad /5}
\end{center}

% Nom de l'élève
\textbf{\small Nom de l'élève :} \underline{\hspace{8cm}} \\[0.5cm]

% Introduction aux questions
Complétez les exercices suivants en utilisant le cours et vos connaissances sur la continuité des fonctions. \\[0.3cm]

\question Enoncer le théorème des valeurs intermediaries.\\
\underline{\hspace{20cm}}\\
\underline{\hspace{20cm}}\\
\underline{\hspace{20cm}}\\
\question Enoncer le théorème d’existence et d’unicité d’une solution\\
\underline{\hspace{20cm}}\\
\underline{\hspace{20cm}}\\
\underline{\hspace{20cm}}\\
\question\
Soit \(f(x)=2x+3\) calculer \( f^{-1}(5) \) sans déterminer l'expression de \( f^{-1} \) \underline{\hspace{15cm}}. \\[0.3cm]
\underline{\hspace{20cm}}

\question\\
\( f \) est dérivable en \( x_{0} \) si et seulment si
\underline{\hspace{20cm}}\\[0.3cm]
\underline{\hspace{20cm}}\\[0.3cm]
Dans ce cas donner l'équation de la tangente
\underline{\hspace{20cm}}\\[0.3cm]
\underline{\hspace{20cm}}\\[0.3cm]
\question Interprétation géométrique du nombre dérivé\\
\[\text{Si }\lim_{x \to x_0} \frac{f(x) - f(x_0)}{x - x_0} = a \ (a \neq 0) \text{ alors  } \underline{\hspace{15cm}}\] 
\( \underline{\hspace{15cm}} \)\\
\[\text{Si }\lim_{x \to x_0^-} \frac{f(x) - f(x_0)}{x - x_0} = +\infty \text{ alors  } \underline{\hspace{15cm}}\] 
\( \underline{\hspace{15cm}} \)\\
\end{document}