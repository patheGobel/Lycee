\documentclass[12pt,a4paper]{article}
\usepackage{amsmath,amssymb,mathrsfs,tikz,times,pifont}
\usepackage{enumitem}
\newcommand\circitem[1]{%
\tikz[baseline=(char.base)]{
\node[circle,draw=gray, fill=red!55,
minimum size=1.2em,inner sep=0] (char) {#1};}}
\newcommand\boxitem[1]{%
\tikz[baseline=(char.base)]{
\node[fill=cyan,
minimum size=1.2em,inner sep=0] (char) {#1};}}
\setlist[enumerate,1]{label=\protect\circitem{\arabic*}}
\setlist[enumerate,2]{label=\protect\boxitem{\alph*}}
%%%::::::by chnini ameur :::::::%%%
\everymath{\displaystyle}
\usepackage[left=1cm,right=1cm,top=1cm,bottom=1.7cm]{geometry}
\usepackage{array,multirow}
\usepackage[most]{tcolorbox}
\usepackage{varwidth}
\tcbuselibrary{skins,hooks}
\usetikzlibrary{patterns}
%%%::::::by chnini ameur :::::::%%%
\newtcolorbox{exa}[2][]{enhanced,breakable,before skip=2mm,after skip=5mm,
colback=yellow!20!white,colframe=black!20!blue,boxrule=0.5mm,
attach boxed title to top left ={xshift=0.6cm,yshift*=1mm-\tcboxedtitleheight},
fonttitle=\bfseries,
title={#2},#1,
% varwidth boxed title*=-3cm,
boxed title style={frame code={
\path[fill=tcbcolback!30!black]
([yshift=-1mm,xshift=-1mm]frame.north west)
arc[start angle=0,end angle=180,radius=1mm]
([yshift=-1mm,xshift=1mm]frame.north east)
arc[start angle=180,end angle=0,radius=1mm];
\path[left color=tcbcolback!60!black,right color = tcbcolback!60!black,
middle color = tcbcolback!80!black]
([xshift=-2mm]frame.north west) -- ([xshift=2mm]frame.north east)
[rounded corners=1mm]-- ([xshift=1mm,yshift=-1mm]frame.north east)
-- (frame.south east) -- (frame.south west)
-- ([xshift=-1mm,yshift=-1mm]frame.north west)
[sharp corners]-- cycle;
},interior engine=empty,
},interior style={top color=yellow!5}}
%%%%%%%%%%%%%%%%%%%%%%%

\usepackage{fancyhdr}
\usepackage{eso-pic}         % Pour ajouter des éléments en arrière-plan
% Commande pour ajouter du texte en arrière-plan
\AddToShipoutPicture{
    \AtTextCenter{%
        \makebox[0pt]{\rotatebox{80}{\textcolor[gray]{0.7}{\fontsize{5cm}{5cm}\selectfont PGB}}}
    }
}
\usepackage{lastpage}
\fancyhf{}
\pagestyle{fancy}
\renewcommand{\footrulewidth}{1pt}
\renewcommand{\headrulewidth}{0pt}
\renewcommand{\footruleskip}{10pt}
\fancyfoot[R]{
\color{blue}\ding{45}\ \textbf{2025}
}
\fancyfoot[L]{
\color{blue}\ding{45}\ \textbf{Prof:M. BA}
}
\cfoot{\bf
\thepage /
\pageref{LastPage}}
\begin{document}
\renewcommand{\arraystretch}{1.5}
\renewcommand{\arrayrulewidth}{1.2pt}
\begin{tikzpicture}[overlay,remember picture]
\node[draw=blue,line width=1.2pt,fill=purple,text=blue,inner sep=3mm,rounded corners,pattern=dots]at ([yshift=-2.5cm]current page.north) {\begingroup\setlength{\fboxsep}{0pt}\colorbox{white}{\begin{tabular}{|*1{>{\centering \arraybackslash}p{0.28\textwidth}} |*2{>{\centering \arraybackslash}p{0.2\textwidth}|} *1{>{\centering \arraybackslash}p{0.19\textwidth}|} }
\hline
\multicolumn{3}{|c|}{$\diamond$$\diamond$$\diamond$\ \textbf{Lycée de Dindéfélo}\ $\diamond$$\diamond$$\diamond$ }& \textbf{A.S. : 2024/2025} \\ \hline
\textbf{Matière: Mathématiques}& \textbf{Niveau : T}\textbf{S2} &\textbf{Date: 19/03/2025} & \textbf{Durée : 4 heures} \\ \hline
\multicolumn{4}{|c|}{\parbox[c]{10cm}{\begin{center}
\textbf{{\Large\sffamily Devoir n$ ^{\circ} $ 1 Du 2$ ^\text{\bf nd} $ Semestre}}
\end{center}}} \\ \hline
\end{tabular}}\endgroup};
\end{tikzpicture}
\vspace{3cm}

\section*{\underline{Exercice 1 :} 6 points }
\subsection*{A) Questions de cours}

\begin{enumerate}
    \item Rappeler les formes algébrique, exponentielle et trigonométrique d’un nombre complexe $z$ non nul.\\ \hfill \textbf{(0,75 pt)}
    \item Donner l’écriture complexe de la rotation $r$ de centre $K(z_0)$, d’angle $\theta$. \hfill \textbf{(0,5 pt)}
\end{enumerate}

\subsection*{B)On donne $z_0 = 1 - i\sqrt{3}$.}

\begin{enumerate}
    \item Donner une écriture trigonométrique de $z_0$. \hfill \textbf{(0,5 pt)}
    \item Montrer que : $z_0^4 = -8 + 8i\sqrt{3}$. \hfill \textbf{(0,25 pt)}
    \item Résoudre dans $\mathbb{C}$ l’équation $Z^4 = 1$. \hfill \textbf{(0,5 pt)}
    \item En déduire les solutions de (E) : $z^4 = -8 + 8i\sqrt{3}$ sous la forme algébrique et sous la forme trigonométrique. \hfill \textbf{(1 pt)}
    
    \medskip
    On peut remarquer que (E) équivaut à :
$\left( \frac{z}{1 - i\sqrt{3}} \right)^4 = 1$
    \item Dans le plan complexe muni d’un repère orthonormal direct $(O, \vec{u}, \vec{v})$, unité graphique 2 cm, placer les points $A$, $B$, $C$ et $D$ d’affixes respectives $z_A = 1 - i\sqrt{3}$, $z_B = -1 + i\sqrt{3}$, $z_C = \sqrt{3} + i$ et $z_D = -\sqrt{3} - i$. \hfill \textbf{(0,75 pt)}
    
    \item Donner une écriture complexe de la rotation $r$ de centre $O$ et d’angle $\frac{\pi}{2}$. \hfill \textbf{(0,5 pt)}
    
    \item Vérifier que : $r(A) = C$ ; $r(C) = B$ et $r(B) = D$. \hfill \textbf{(0,75 pt)}
    
    \item En déduire que les points $A$, $B$, $C$ et $D$ sont situés sur un même cercle dont on précisera le centre et le rayon. \hfill \textbf{(0,5 pt)}
\end{enumerate}

\section*{\underline{Exercice 2 :} 2,25 points }
Déterminer les limites suivantes :
\[
\textbf{1.} \lim\limits_{x \to +\infty} \ln\left[ \frac{x+1}{x^2 + x + 1}\right]  \quad 
\textbf{2.} \lim\limits_{x \to +\infty} \frac{\ln(x+2)}{\ln(x+1)} \quad 
\textbf{3.} \lim\limits_{x \to 0^+} \frac{\ln x+2}{\ln x+1} \quad
\textbf{4.} \lim\limits_{x \to +\infty} \frac{\ln x}{\sqrt{x}}
\]

\section*{\underline{Problème :} 11,75 points }
\subsection*{\underline{\textbf{Partie A}}:\textbf{ 2,75 pts}}

Soit \( g(x) = 2x \ln(-x) + x + 1 \).

\begin{enumerate}
    \item Déterminer l’ensemble de définition \( D_g \) de \( g \).\hfill \textbf{(0,5 pt)}
    \item Calculer les limites aux bornes de \( D_g \).\hfill \textbf{(0,5 pt)}
    \item Étudier les variations de \( g \).\hfill \textbf{(1 pt)}
    \item Calculer \( g(-1) \) puis en déduire le signe de \( g(x) \).\hfill \textbf{(0,75 pt)}
\end{enumerate}

\subsection*{\underline{\textbf{Partie B}}:\textbf{ 7 pts}}

On considère la fonction \( f \) définie par :

\[
    f(x) =
    \begin{cases}
        x^2 \ln(-x) + x + 1 & \text{si } x < 0 \\
        x \ln(x)^2 + x + 1  & \text{si } x > 0 \\
        1                   & \text{si } x = 0
    \end{cases}
\]

On note \( (C_f) \) sa courbe représentative dans un repère orthonormé.

\begin{enumerate}
    \item Justifier que \( f \) est définie sur \( \mathbb{R} \).\hfill \textbf{(0,5 pt)}
    \item Étudier la continuité et la dérivabilité de \( f \) en 0. Interprétez graphiquement les résultats.\hfill \textbf{(1,5 pt)}
    \item Donner le domaine de dérivabilité de \( f \) puis montrer que
          \(
              f'(x) =
              \begin{cases}
                  g(x)          & \text{si } x < 0 \\
                  (1 + \ln x)^2 & \text{si } x > 0
              \end{cases}
          \)\hfill \textbf{(0,5$\times$3 pts)}
    \item Calculer les limites de \( f \) aux bornes de son domaine de définition.\hfill \textbf{(0,5 pt)}
    \item Étudier les branches infinies de \( (C_f) \).\hfill \textbf{(0,5 pt)}
    \item Dresser le tableau de variations de \( f \).\hfill \textbf{(1 pt)}
    \item Montrer que dans \( ]-\infty; -1[ \), l’équation \( f(x) = 1 \)

          admet une unique solution \( \alpha \) puis vérifier que \(-1,8 < \alpha < -1,7\).\hfill \textbf{(0,75 pt)}
    \item Construire \( (C_f) \) (unité 2 cm) (on précisera la tangente au point d’abscisse \( e^{-1} \) et on placera le point d’abscisse 1).\hfill \textbf{(0,75 pt)}
\end{enumerate}

\subsection*{\underline{\textbf{Partie C}}:\textbf{ 2 pts}}

Soit \( h \) la restriction de \( f \) à \( I =]0; +\infty[ \).

    \begin{enumerate}
        \item Montrer que \( h \) admet une bijection réciproque \( h^{-1} \) définie sur un intervalle \( J \) à préciser.\hfill \textbf{(0,5 pt)}
        \item Étudier la dérivabilité de \( h^{-1} \) sur \( J \).\hfill \textbf{(0,25 pt)}
        \item
              \begin{enumerate}
                  \item Calculer \( h(1) \).\hfill \textbf{(0,25 point)}
                  \item Calculer \( (h^{-1})'(2) \).\hfill \textbf{(0,5 pt)}
              \end{enumerate}
        \item Construire la courbe de \( h^{-1} \).\hfill \textbf{(0,5 pt)}
    \end{enumerate}
\end{document}
