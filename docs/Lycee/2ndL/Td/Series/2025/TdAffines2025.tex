\documentclass[12pt,a4paper]{article}
\usepackage{amsmath,amssymb,mathrsfs,tikz,times,pifont}
\usepackage{enumitem}
\usepackage{multicol}
\usepackage{lmodern}
\newcommand\circitem[1]{%
\tikz[baseline=(char.base)]{
\node[circle,draw=gray, fill=red!55,
minimum size=1.2em,inner sep=0] (char) {#1};}}
\newcommand\boxitem[1]{%
\tikz[baseline=(char.base)]{
\node[fill=cyan,
minimum size=1.2em,inner sep=0] (char) {#1};}}
\setlist[enumerate,1]{label=\protect\circitem{\arabic*}}
\setlist[enumerate,2]{label=\protect\boxitem{\alph*}}
%%%::::::by chnini ameur :::::::%%%
\everymath{\displaystyle}
\usepackage[left=1cm,right=1cm,top=1cm,bottom=1.7cm]{geometry}
\usepackage[colorlinks=true, linkcolor=blue, urlcolor=blue, citecolor=blue]{hyperref}
\usepackage{array,multirow}
\usepackage[most]{tcolorbox}
\usepackage{varwidth}
\usepackage{float} %pour utiliser l'option [H] qui force l'image à apparaître exactement à l'endroit où elle est placée dans le code.
\tcbuselibrary{skins,hooks}
\usetikzlibrary{patterns}
%%%::::::by chnini ameur :::::::%%%
\newtcolorbox{exa}[2][]{enhanced,breakable,before skip=2mm,after skip=5mm,
colback=yellow!20!white,colframe=black!20!blue,boxrule=0.5mm,
attach boxed title to top left ={xshift=0.6cm,yshift*=1mm-\tcboxedtitleheight},
fonttitle=\bfseries,
title={#2},#1,
% varwidth boxed title*=-3cm,
boxed title style={frame code={
\path[fill=tcbcolback!30!black]
([yshift=-1mm,xshift=-1mm]frame.north west)
arc[start angle=0,end angle=180,radius=1mm]
([yshift=-1mm,xshift=1mm]frame.north east)
arc[start angle=180,end angle=0,radius=1mm];
\path[left color=tcbcolback!60!black,right color = tcbcolback!60!black,
middle color = tcbcolback!80!black]
([xshift=-2mm]frame.north west) -- ([xshift=2mm]frame.north east)
[rounded corners=1mm]-- ([xshift=1mm,yshift=-1mm]frame.north east)
-- (frame.south east) -- (frame.south west)
-- ([xshift=-1mm,yshift=-1mm]frame.north west)
[sharp corners]-- cycle;
},interior engine=empty,
},interior style={top color=yellow!5}}
%%%%%%%%%%%%%%%%%%%%%%%

\usepackage{fancyhdr}
\usepackage{eso-pic}         % Pour ajouter des éléments en arrière-plan
% Commande pour ajouter du texte en arrière-plan
\usepackage{tkz-tab}
\AddToShipoutPicture{
    \AtTextCenter{%
        \makebox[0pt]{\rotatebox{80}{\textcolor[gray]{0.7}{\fontsize{5cm}{5cm}\selectfont PGB}}}
    }
}
\usepackage{lastpage}
\fancyhf{}
\pagestyle{fancy}
\renewcommand{\footrulewidth}{1pt}
\renewcommand{\headrulewidth}{0pt}
\renewcommand{\footruleskip}{10pt}
\fancyfoot[R]{
\color{blue}\ding{45}\ \textbf{2025}
}
\fancyfoot[L]{
\color{blue}\ding{45}\ \textbf{Prof:M. BA}
}
\cfoot{\bf
\thepage /
\pageref{LastPage}}
\begin{document}
\renewcommand{\arraystretch}{1.5}
\renewcommand{\arrayrulewidth}{1.2pt}
\begin{tikzpicture}[overlay,remember picture]
    \node[draw=blue,line width=1.2pt,fill=purple,text=blue,inner sep=3mm,rounded corners,pattern=dots]at ([yshift=-2.5cm]current page.north) {\begingroup\setlength{\fboxsep}{0pt}\colorbox{white}{\begin{tabular}{|*1{>{\centering \arraybackslash}p{0.28\textwidth}} |*2{>{\centering \arraybackslash}p{0.2\textwidth}|} *1{>{\centering \arraybackslash}p{0.19\textwidth}|} }
                \hline
                \multicolumn{3}{|c|}{$\diamond$$\diamond$$\diamond$\ \textbf{Lycée de Dindéfélo}\ $\diamond$$\diamond$$\diamond$ } & \textbf{A.S. : 2024/2025}                                              \\ \hline
                \textbf{Matière: Mathématiques}                                                                                    & \textbf{Niveau : }\textbf{2ndL} & \textbf{Date: 15/04/2025} & \textbf{} \\ \hline
                \multicolumn{4}{|c|}{\parbox[c]{10cm}{\begin{center}
                                                                  \textbf{{\Large\sffamily Td Fonctions affines et droites du plan}}
                                                              \end{center}}}                                                                                                        \\ \hline
            \end{tabular}}\endgroup};
\end{tikzpicture}
\vspace{3cm}

\begin{multicols}{2}
\setlength{\columnseprule}{0.1mm} % La largeur de la ligne verticale entre les colonnes
\section*{Exercice 1}
Indique si $f$ est une application affine ou non. Si oui, préciser les valeurs de $a$ et $b$ telles que : $f(x) = ax + b$.
\begin{enumerate}
    \item $f(x) = \frac{4}{3}x - 3$
    \item $f(x) = \sqrt{2}x - 4$
    \item $f(x) = 3x + 1$
    \item $f(x) = 2(x + 2), 5$
    \item $f(x) = x^2 - 4$
    \item $f(x) = (4x - 1) - 4(x - 2)$
    \item $f(x) = (2x + 1)^2 - (2x - 2)^2$
\end{enumerate}

\section*{Exercice 2}
Soit $f$ une application affine définie dans $\mathbb{R}$ par : 
$$ f(x) = -3x + \frac{4}{3} $$
\begin{enumerate}
    \item Calculer les images par $f$ des nombres : $-2$, $0$, $1$ et $\frac{3}{4}$.
    \item Calculer les antécédents par $f$ des nombres : $-2$, $0$, $3$.
    \item Déterminer les applications affines $g$ et $h$ telles que : $g(-1) = 1$ et $g(-3) = 1$ ; $h(2) = 2$ et $h(1) = 1$
    \item Préciser le sens de variation de chacune des applications affines définies dans $\mathbb{R}$ par : $f(x) = 2x + 3$ ; $m(x) = -5x$, $n(x) = 7$ et $q(x) = (1 + \sqrt{5})x + 3$.
\end{enumerate}

\section*{Exercice 3}
Soit l’application affine définie par : $f(1) = -1$ et $f(2) = -3$.
\begin{enumerate}
    \item Montrer que : $f(x) = -2x + 1$
    \item Le Plan avec Deux Droites $(D) : y = -2x + 1$ et $(D') : y = x + 2$
    \begin{enumerate}
        \item Montrer que $(D)$ et $(D')$ sont sécantes.
        \item Trace les droites $(D)$ et $(D')$.
        \item Déterminer le point d'intersection de $(D)$ et $(D')$.
    \end{enumerate}
\end{enumerate}

\section*{Exercice 4}
Pour financer une sortie pédagogique, une école décide de vendre les tomates de son jardin. Le client peut payer de la quantité de tomates achetées avec une somme forfaitaire pour le transport.
\begin{enumerate}
    \item Un commerçant qui a acheté 300 kg a versé au gestionnaire une somme totale de 125.000F.
    \item Un membre de l'association des parents d'élèves a acheté 100 kg et a payé 40.000F.
    \item Calculer le prix d'un kilogramme de tomate en fonction de la quantité de tomates achetées.
    \item Soit $p$ la somme totale, en francs, payée pour l'achat de $x$ kg de tomates. Déterminer $p(x)$.
\end{enumerate}

\section*{Exercice 5}
Dans un repère orthonormé, construire l'ensemble des points $M(x; y)$ en prenant 1 cm pour 50 kg en abscisses et 1 cm pour 10.000F en ordonnées.
\begin{enumerate}
    \item Déterminer la somme totale à payer pour un achat de 75 kg de tomates.
\end{enumerate}

\section*{Exercice 6}
Le plan est muni d’un repère orthonormé $(O ; \vec{i}, \vec{j})$. Soit la fonction définie par : $f(x) = |x| - 2$.
\begin{enumerate}
    \item Montrer que $f$ est une application affine par intervalle.
    \item Calculer $f(1)$ et $f(3)$.
    \item Construire la représentation graphique de $f$.
\end{enumerate}

\section*{Exercice 7}
Une entreprise fabrique des coquetiers en bois qu’elle vend ensuite à des artistes-peintres. Elle leur propose deux tarifs, auxquels sont associés les conditions suivantes :
\begin{itemize}
    \item Tarif n°1 : 25F le coquetier
    \item Tarif n°2 : Un forfait de 400F et 15F le coquetier
\end{itemize}
\begin{enumerate}
    \item Calculer le prix de 30 coquetiers et celui de 50 coquetiers au tarif n°1 et au tarif n°2.
    \item On note $x$ le nombre de coquetiers achetés et $P_1(x)$ et $P_2(x)$ les prix payés respectivement pour le tarif n°1 et le tarif n°2.
    \item Construire dans un même repère orthonormé, les droites $(D_1)$ et $(D_2)$ qui représentent les fonctions $P_1$ et $P_2$.
    \item Par simple lecture graphique, répondre aux trois questions suivantes :
    \begin{enumerate}
        \item Quel est le prix des grand nombre de coquetiers ? Quand le client peut acheter avec 1200F ?
        \item Quel nombre de coquetiers les prix des deux coquetiers sont-ils égaux ?
        \item Quelle condition le tarif n°2 est-il le plus avantageux ?
    \end{enumerate}
\end{enumerate}

\section*{Exercice 8 : Fonctions affines}

\begin{enumerate}
    \item Donner les expressions des fonctions affines \( f_1, f_2, f_3 \) dont les représentations graphiques sont respectivement les droites \( d_1, d_2, d_3 \) tracées dans le repère \( (O, I, J) \) ci-dessous.
    \item Tracer la droite \( d_4 \) représentative de la fonction \( f_4 \) définie par \( f_4(x) = \frac{1}{2}x - 1 \) dans le même repère.
    \item Donner le sens de variation de chacune des fonctions \( f_1, f_2, f_3, f_4 \).
    \item Calculer l'expression de la fonction affine \( k \) sachant que \( k(3) = 1 \) et \( k(2) = 5 \).
    % Inclusion de l'image
\begin{figure}[H]
    \centering
    \includegraphics[width=0.6\textwidth]{Screenshot_from_2025-04-21_11-51-00.png}
    \caption{Graphique des fonctions affines.}
\end{figure}
\end{enumerate}
\end{multicols}

\end{document}