\documentclass[12pt,a4paper]{article}
\usepackage[utf8]{inputenc} % inutile avec XeLaTeX/LuaLaTeX
\usepackage[T1]{fontenc}
\usepackage{amsmath,amssymb,mathrsfs,tikz,times,pifont}
\usepackage{enumitem}
\usepackage{multicol}
\usepackage{lmodern}
\newcommand\circitem[1]{%
\tikz[baseline=(char.base)]{
\node[circle,draw=gray, fill=red!55,
minimum size=1.2em,inner sep=0] (char) {#1};}}
\newcommand\boxitem[1]{%
\tikz[baseline=(char.base)]{
\node[fill=cyan,
minimum size=1.2em,inner sep=0] (char) {#1};}}
\setlist[enumerate,1]{label=\protect\circitem{\arabic*}}
\setlist[enumerate,2]{label=\protect\boxitem{\alph*}}
\everymath{\displaystyle}
\usepackage[left=1cm,right=1cm,top=1cm,bottom=1.7cm]{geometry}
\usepackage[colorlinks=true, linkcolor=blue, urlcolor=blue, citecolor=blue]{hyperref}
\usepackage{array,multirow}
\usepackage[most]{tcolorbox}
\usepackage{varwidth}
\usepackage{float}
\tcbuselibrary{skins,hooks}
\usetikzlibrary{patterns}

\newtcolorbox{exa}[2][]{enhanced,breakable,before skip=2mm,after skip=5mm,
colback=yellow!20!white,colframe=black!20!blue,boxrule=0.5mm,
attach boxed title to top left ={xshift=0.6cm,yshift*=1mm-\tcboxedtitleheight},
fonttitle=\bfseries,
title={#2},#1,
boxed title style={frame code={
\path[fill=tcbcolback!30!black]
([yshift=-1mm,xshift=-1mm]frame.north west)
arc[start angle=0,end angle=180,radius=1mm]
([yshift=-1mm,xshift=1mm]frame.north east)
arc[start angle=180,end angle=0,radius=1mm];
\path[left color=tcbcolback!60!black,right color = tcbcolback!60!black,
middle color = tcbcolback!80!black]
([xshift=-2mm]frame.north west) -- ([xshift=2mm]frame.north east)
[rounded corners=1mm]-- ([xshift=1mm,yshift=-1mm]frame.north east)
-- (frame.south east) -- (frame.south west)
-- ([xshift=-1mm,yshift=-1mm]frame.north west)
[sharp corners]-- cycle;
},interior engine=empty,
},interior style={top color=yellow!5}}

\usepackage{fancyhdr}
\usepackage{eso-pic}
\usepackage{tkz-tab}
\AddToShipoutPicture{
    \AtTextCenter{%
        \makebox[0pt]{\rotatebox{80}{\textcolor[gray]{0.7}{\fontsize{5cm}{5cm}\selectfont PGB}}}
    }
}

\usepackage{verbatim}

\usepackage{color,soul}

\usepackage{amsmath}
\usepackage{amsfonts}
\usepackage{amssymb}
\usepackage{systeme}
\usepackage{tkz-tab}
\usepackage{tikz}
\usetikzlibrary{arrows}
\newcounter{exemple} % Compteur pour les questions

% Définir la commande pour afficher une question numérotée
\newcommand{\exemple}{%
  \refstepcounter{exemple}%
  \textbf{\textcolor{green}{Exemple \theexemple :}} \ignorespaces
}
%---------------------------------------
\definecolor{myorange}{rgb}{1.0, 0.8, 0.0}

% Définir un compteur pour les exercices d'application
\newcounter{exerciceapp}

% Définir la commande pour afficher un exercice d'application numéroté
\newcommand{\exerciceapp}{%
  \refstepcounter{exerciceapp}%
  \textbf{\textcolor{myorange}{Exercice d'application \theexerciceapp :}} \ignorespaces
}
%--------------------------------------
% Définir un compteur pour les exercices d'application
\newcounter{correction}

% Définir la commande pour afficher un correction exercice d'application numéroté
\newcommand{\correction}{%
  \refstepcounter{correction}%
  \textbf{\textcolor{myorange}{Correction \thecorrection :}} \ignorespaces
}
%--------------------------------------
% Commande pour ajouter du texte en arrière-plan
\usepackage{fancyhdr}
\usepackage{eso-pic}
%\usepackage{tkz-tab}
\AddToShipoutPicture{
    \AtTextCenter{%
        \makebox[0pt]{\rotatebox{80}{\textcolor[gray]{0.7}{\fontsize{5cm}{5cm}\selectfont PGB}}}
    }
}
%This command takes a colour as an optional argument; the default colour is black.
\usetikzlibrary{shapes.geometric,fit}
\newcommand{\myul}[2][black]{\setulcolor{#1}\ul{#2}\setulcolor{black}}
\newcommand\tab[1][1cm]{\hspace*{#1}}

\begin{document}
% En-tête personnalisée
\begin{center}
    \Large\textbf{\underline{Distances}}\\[-0.1cm]
    \normalsize\textbf{Prof : M. BA} \hfill \textbf{Classe : 4eme}\\[-0.1cm]
    \textbf{Année scolaire : 2025 -- 2026}
\end{center}

\begin{center}
\textbf{Introduction à la Distance}
\end{center}



\section*{\underline{\textbf{\textcolor{red}{I. POSITION RELATIVE DE DEUX CERCLES : }}}}

\subsection*{\underline{\textbf{\textcolor{red}{1) Cercles tangents : }}}}
\underline{\textbf{\textcolor{red}{a)Activité 1 :}}}

\underline{\textbf{\textcolor{red}{b) Vocabulaire et configuration :}}}

\subsection*{\underline{\textbf{\textcolor{red}{2) Cercles sécants : }}}}
\underline{\textbf{\textcolor{red}{a)Activité  :}}}

\underline{\textbf{\textcolor{red}{b) Vocabulaire et configuration :}}}

\subsection*{\underline{\textbf{\textcolor{red}{3) Cercles disjoints : }}}}
\underline{\textbf{\textcolor{red}{a)Activité  :}}}

\underline{\textbf{\textcolor{red}{b) Vocabulaire et configuration :}}}


\section*{\underline{\textbf{\textcolor{red}{II. APPLICATION AU TRIANGLE : }}}}
\underline{\textbf{\textcolor{red}{Activité  :}}}
\subsection*{\underline{\textbf{\textcolor{red}{1)Condition d’existence d’un triangle :}}}}
\textbf{\exerciceapp}

\textbf{\correction}

\section*{\underline{\textbf{\textcolor{red}{III. REGIONNEMENT PLAN ET CARACTERISATION D’UN DEMI-PLAN : }}}}
\underline{\textbf{\textcolor{red}{Activité  :}}}

\subsection*{\underline{\textbf{\textcolor{red}{1) Régionnement du plan :}}}}
\subsection*{\underline{\textbf{\textcolor{red}{2) Propriétés de caractérisation d’un demi-plan : }}}}
\subsection*{\underline{\textbf{\textcolor{red}{ }}}}
\subsection*{\underline{\textbf{\textcolor{red}{ }}}}
\textbf{\exerciceapp}

\textbf{\correction}

\underline{\exemple}

\underline{\textbf{\textcolor{red}{Remarque}}}\\

\end{document}