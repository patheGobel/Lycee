\documentclass[12pt,a4paper]{article}
\usepackage[utf8]{inputenc} % inutile avec XeLaTeX/LuaLaTeX
\usepackage[T1]{fontenc}
\usepackage{amsmath,amssymb,mathrsfs,tikz,times,pifont}
\usepackage{enumitem}
\usepackage{multicol}
\usepackage{lmodern}
\newcommand\circitem[1]{%
\tikz[baseline=(char.base)]{
\node[circle,draw=gray, fill=red!55,
minimum size=1.2em,inner sep=0] (char) {#1};}}
\newcommand\boxitem[1]{%
\tikz[baseline=(char.base)]{
\node[fill=cyan,
minimum size=1.2em,inner sep=0] (char) {#1};}}
\setlist[enumerate,1]{label=\protect\circitem{\arabic*}}
\setlist[enumerate,2]{label=\protect\boxitem{\alph*}}
\everymath{\displaystyle}
\usepackage[left=1cm,right=1cm,top=1cm,bottom=1.7cm]{geometry}
\usepackage[colorlinks=true, linkcolor=blue, urlcolor=blue, citecolor=blue]{hyperref}
\usepackage{array,multirow}
\usepackage[most]{tcolorbox}
\usepackage{varwidth}
\usepackage{float}
\tcbuselibrary{skins,hooks}
\usetikzlibrary{patterns}

\newtcolorbox{exa}[2][]{enhanced,breakable,before skip=2mm,after skip=5mm,
colback=yellow!20!white,colframe=black!20!blue,boxrule=0.5mm,
attach boxed title to top left ={xshift=0.6cm,yshift*=1mm-\tcboxedtitleheight},
fonttitle=\bfseries,
title={#2},#1,
boxed title style={frame code={
\path[fill=tcbcolback!30!black]
([yshift=-1mm,xshift=-1mm]frame.north west)
arc[start angle=0,end angle=180,radius=1mm]
([yshift=-1mm,xshift=1mm]frame.north east)
arc[start angle=180,end angle=0,radius=1mm];
\path[left color=tcbcolback!60!black,right color = tcbcolback!60!black,
middle color = tcbcolback!80!black]
([xshift=-2mm]frame.north west) -- ([xshift=2mm]frame.north east)
[rounded corners=1mm]-- ([xshift=1mm,yshift=-1mm]frame.north east)
-- (frame.south east) -- (frame.south west)
-- ([xshift=-1mm,yshift=-1mm]frame.north west)
[sharp corners]-- cycle;
},interior engine=empty,
},interior style={top color=yellow!5}}

\usepackage{fancyhdr}
\usepackage{eso-pic}
\usepackage{tkz-tab}
\AddToShipoutPicture{
    \AtTextCenter{%
        \makebox[0pt]{\rotatebox{80}{\textcolor[gray]{0.7}{\fontsize{5cm}{5cm}\selectfont PGB}}}
    }
}

\usepackage{verbatim}

\usepackage{color,soul}

\usepackage{amsmath}
\usepackage{amsfonts}
\usepackage{amssymb}
\usepackage{systeme}
\usepackage{tkz-tab}
\usepackage{tikz}
\usetikzlibrary{arrows}
\newcounter{exemple} % Compteur pour les questions

% Définir la commande pour afficher une question numérotée
\newcommand{\exemple}{%
  \refstepcounter{exemple}%
  \textbf{\textcolor{green}{Exemple \theexemple :}} \ignorespaces
}
%---------------------------------------
\definecolor{myorange}{rgb}{1.0, 0.8, 0.0}

% Définir un compteur pour les exercices d'application
\newcounter{exerciceapp}

% Définir la commande pour afficher un exercice d'application numéroté
\newcommand{\exerciceapp}{%
  \refstepcounter{exerciceapp}%
  \textbf{\textcolor{myorange}{Exercice d'application \theexerciceapp :}} \ignorespaces
}
%--------------------------------------
% Définir un compteur pour les exercices d'application
\newcounter{correction}

% Définir la commande pour afficher un correction exercice d'application numéroté
\newcommand{\correction}{%
  \refstepcounter{correction}%
  \textbf{\textcolor{myorange}{Correction \thecorrection :}} \ignorespaces
}
%--------------------------------------
% Commande pour ajouter du texte en arrière-plan
\usepackage{fancyhdr}
\usepackage{eso-pic}
%\usepackage{tkz-tab}
\AddToShipoutPicture{
    \AtTextCenter{%
        \makebox[0pt]{\rotatebox{80}{\textcolor[gray]{0.7}{\fontsize{5cm}{5cm}\selectfont PGB}}}
    }
}
%This command takes a colour as an optional argument; the default colour is black.
\usetikzlibrary{shapes.geometric,fit}
\newcommand{\myul}[2][black]{\setulcolor{#1}\ul{#2}\setulcolor{black}}
\newcommand\tab[1][1cm]{\hspace*{#1}}

\begin{document}
% En-tête personnalisée
\begin{center}
    \Large\textbf{\underline{Distances}}\\[-0.1cm]
    \normalsize\textbf{Prof : M. BA} \hfill \textbf{Classe : 4eme}\\[-0.1cm]
    \textbf{Année scolaire : 2025 -- 2026}
\end{center}

\begin{center}
\textbf{Introduction à la Distance}
\end{center}



\section*{\underline{\textbf{\textcolor{red}{I. POSITION RELATIVE DE DEUX CERCLES : }}}}

\subsection*{\underline{\textbf{\textcolor{red}{1) Cercles tangents : }}}}
\underline{\textbf{\textcolor{red}{a)Activité 1 :}}}

\noindent \underline{ } \\

\begin{enumerate}
    \item Soit $(C)$ un cercle de centre $O$ et de rayon $r=3,5$ cm.
    \begin{enumerate}
        \item Tracer $(C)$ puis placer le point $O'$ tel que $OO' = 5$ cm.
        \item Tracer le cercle $(C')$ de centre $O'$ et de rayon $r'=1,5$ cm.
        \item Comparer $OO'$ et $r + r'$ puis donner la position relative de ces deux cercles.
    \end{enumerate}

    \item Soit $(C) (O ; 3)$ et $(C') (O' ; 2)$ tel que $OO' = 1$ cm.
    \begin{enumerate}
        \item Tracer la figure.
        \item Comparer les distances $OO'$ et $|r - r'|$ puis donner la position relative de $(C)$ et $(C')$.
    \end{enumerate}
\end{enumerate}

\underline{\textbf{\textcolor{red}{b) Vocabulaire et configuration :}}}

\begin{itemize}
\item \textbf{cercles sont tangents extérieurement}

Si la distance des centres est égale à la somme des rayons, alors les deux cercles sont tangents
extérieurement

\definecolor{xdxdff}{rgb}{0.49019607843137253,0.49019607843137253,1}
\definecolor{wewdxt}{rgb}{0.43137254901960786,0.42745098039215684,0.45098039215686275}
\begin{tikzpicture}[line cap=round,line join=round,>=triangle 45,x=1cm,y=1cm]
\clip(-5.6,-5.28) rectangle (13.96,5.96);
\draw [line width=2pt] (0,0) circle (3cm);
\draw [line width=2pt] (5,0) circle (2cm);
\draw [line width=2pt] (0,0)-- (-0.9547754343499091,2.8440119321061124);
\draw [line width=2pt] (5,0)-- (6.612160968398339,-1.1836118502165018);
\begin{scriptsize}
\draw [fill=wewdxt] (0,0) circle (2pt);
\draw[color=wewdxt] (0.16,0.39) node {$O$};
\draw[color=black] (-2.04,3.01) node {$(C)$};
\draw [fill=xdxdff] (5,0) circle (2.5pt);
\draw[color=xdxdff] (5.16,0.43) node {$O'$};
\draw[color=black] (5.46,2.43) node {$(C')$};
\draw[color=wewdxt] (-0.14,1.4) node {$r$};
\draw[color=wewdxt] (5.84,-0.08)node {$r'$};
\end{scriptsize}
\end{tikzpicture}

\begin{center}
\textcolor{red}{\boxed{\textbf{OO’ = r + r’.}}}
\end{center}

\item \textbf{cercles sont tangents intérieurement}

Si la distance des centres est égale à la différence des rayons, alors les deux cercles sont tangents
intérieurement.

\definecolor{ududff}{rgb}{0.30196078431372547,0.30196078431372547,1}
\definecolor{wewdxt}{rgb}{0.43137254901960786,0.42745098039215684,0.45098039215686275}
\begin{tikzpicture}[line cap=round,line join=round,>=triangle 45,x=1cm,y=1cm]
\clip(-5.6,-5.28) rectangle (13.96,5.96);
\draw [line width=2pt] (0,0) circle (3cm);
\draw [line width=2pt] (0,0)-- (-0.9547754343499091,2.8440119321061124);
\draw [line width=2pt] (1.66,1.24) circle (0.9375551229017804cm);
\draw [line width=2pt] (1.66,1.24)-- (0.7303199546208459,1.3612626146146725);
\begin{scriptsize}
\draw [fill=ududff] (0,0) circle (2pt);
\draw[color=ududff] (0.16,0.2) node {$O$};
\draw[color=black] (-0.16,2.19) node {$r$};
\draw[color=black] (-2.04,3.01) node {$(C)$};
\draw [fill=ududff] (1.66,1.24) circle (2.5pt);
\draw[color=ududff] (1.82,1.67) node {$O'$};
\draw[color=black] (1.3,1.6) node {$r'$};
\draw[color=black] (1.82,0) node {$(C')$};
\end{scriptsize}
\end{tikzpicture}

\begin{center}
\textcolor{red}{\boxed{\textbf{OO’ = |r - r’|.}}}
\end{center}

\end{itemize}

\subsection*{\underline{\textbf{\textcolor{red}{2) Cercles disjoints : }}}}
\underline{\textbf{\textcolor{red}{a)Activité  :}}}

\underline{\textbf{\textcolor{red}{b) Vocabulaire et configuration :}}}

\begin{itemize}
\item \textbf{Cercles disjoints extérieurement :} 

Si la distance des centres est supérieure à la somme des rayons, alors les deux cercles sont disjoints
extérieurement

\definecolor{ududff}{rgb}{0.30196078431372547,0.30196078431372547,1}
\definecolor{wewdxt}{rgb}{0.43137254901960786,0.42745098039215684,0.45098039215686275}
\begin{tikzpicture}[line cap=round,line join=round,>=triangle 45,x=1cm,y=1cm]
\clip(-5.6,-5.28) rectangle (13.96,5.96);
\draw [line width=2pt] (0,0) circle (3cm);
\draw [line width=2pt] (0,0)-- (-0.9547754343499091,2.8440119321061124);
\draw [line width=2pt] (7.96,0.32) circle (2cm);
\draw [line width=2pt] (7.96,0.32)-- (6.445887709166324,1.6266996482537204);
\begin{scriptsize}
\draw [fill=ududff] (0,0) circle (2pt);
\draw[color=ududff] (0.16,0.39) node {$O$};
\draw[color=black] (-2.04,3.01) node {$(C)$};
\draw[color=ududff] (-0.16,2.11) node {$r$};
\draw [fill=ududff] (7.96,0.32) circle (2.5pt);
\draw[color=ududff] (8.12,0.75) node {$O'$};
\draw[color=black] (8,2.7) node {$(C')$};
\draw[color=black] (7.48,1.45) node {$r'$};
\end{scriptsize}
\end{tikzpicture}

\begin{center}
\textcolor{red}{\boxed{\textbf{OO’ > r + r’.}}}
\end{center}

\item \textbf{Cercles disjoints intérieurement :} 

\definecolor{ududff}{rgb}{0.30196078431372547,0.30196078431372547,1}
\definecolor{wewdxt}{rgb}{0.43137254901960786,0.42745098039215684,0.45098039215686275}
\begin{tikzpicture}[line cap=round,line join=round,>=triangle 45,x=1cm,y=1cm]
\clip(-7.02,-5.8) rectangle (12.54,5.44);
\draw [line width=2pt] (0,0) circle (4cm);
\draw [line width=2pt] (0,0)-- (-1.2730339124665455,3.7920159094748165);
\draw [line width=2pt] (0.18,-1.74) circle (2cm);
\draw [line width=2pt] (0.18,-1.74)-- (-1.596716109990185,-2.658302708534253);
\begin{scriptsize}
\draw [fill=ududff] (0,0) circle (2pt);
\draw[color=ududff] (0.2,0.1) node {$O$};
\draw[color=black] (-2.54,3.85) node {$(C)$};
\draw[color=black] (-0.16,2.11) node {$r$};
\draw [fill=ududff] (0.18,-1.74) circle (2.5pt);
\draw[color=ududff] (0.34,-1.31) node {$O'$};
\draw[color=black] (-1.7,-0.37) node {$(C')$};
\draw[color=black] (-0.78,-1.67) node {$r'$};
\end{scriptsize}
\end{tikzpicture}

\begin{center}
\textcolor{red}{\boxed{\textbf{OO’ < |r - r’|.}}}
\end{center}

\end{itemize}

\subsection*{\underline{\textbf{\textcolor{red}{3) Cercles sécants : }}}}
\underline{\textbf{\textcolor{red}{a)Activité  :}}}

\underline{\textbf{\textcolor{red}{b) Vocabulaire et configuration :}}}

Si la distance des centres est comprise entre la différence et
la somme des rayons, alors les deux cercles sont sécants.
Ils ont deux points en commun.

\definecolor{ududff}{rgb}{0.30196078431372547,0.30196078431372547,1}
\definecolor{wewdxt}{rgb}{0.43137254901960786,0.42745098039215684,0.45098039215686275}
\begin{tikzpicture}[line cap=round,line join=round,>=triangle 45,x=1cm,y=1cm]
\clip(-7.02,-5.8) rectangle (12.54,5.44);
\draw [line width=2pt] (0,0) circle (4cm);
\draw [line width=2pt] (0,0)-- (-1.2730339124665455,3.7920159094748165);
\draw [line width=2pt] (5.52,0.3) circle (3cm);
\draw [line width=2pt] (5.52,0.3)-- (8.502385976830519,0.6246134396550225);
\begin{scriptsize}
\draw [fill=ududff] (0,0) circle (2pt);
\draw[color=ududff] (0.16,0.39) node {$O$};
\draw[color=black] (-2.54,3.85) node {$(C)$};
\draw[color=black] (-0.16,2.11) node {$r$};
\draw [fill=ududff] (5.52,0.3) circle (2.5pt);
\draw[color=ududff] (5.68,0.73) node {$O$};
\draw[color=black] (5.04,3) node {$(C')$};
\draw[color=black] (7.12,0.3) node {$r'$};
\end{scriptsize}
\end{tikzpicture}

\begin{center}
\textcolor{red}{\boxed{\textbf{|r - r’| < OO’ < r+r'.}}}
\end{center}

\section*{\underline{\textbf{\textcolor{red}{II. APPLICATION AU TRIANGLE : }}}}
\underline{\textbf{\textcolor{red}{Activité  :}}}
\subsection*{\underline{\textbf{\textcolor{red}{1)Condition d’existence d’un triangle :}}}}
\textbf{\exerciceapp}

\textbf{\correction}

\section*{\underline{\textbf{\textcolor{red}{III. REGIONNEMENT PLAN ET CARACTERISATION D’UN DEMI-PLAN : }}}}
\underline{\textbf{\textcolor{red}{Activité  :}}}

\subsection*{\underline{\textbf{\textcolor{red}{1) Régionnement du plan :}}}}
\subsection*{\underline{\textbf{\textcolor{red}{2) Propriétés de caractérisation d’un demi-plan : }}}}
\subsection*{\underline{\textbf{\textcolor{red}{ }}}}
\subsection*{\underline{\textbf{\textcolor{red}{ }}}}
\textbf{\exerciceapp}

\textbf{\correction}

\underline{\exemple}

\underline{\textbf{\textcolor{red}{Remarque}}}\\

\end{document}