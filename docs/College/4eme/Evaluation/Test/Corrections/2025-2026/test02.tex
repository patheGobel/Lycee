\documentclass[a4paper,12pt]{article}
\usepackage{graphicx}
\usepackage[a4paper, top=0cm, bottom=2cm, left=2cm, right=2cm]{geometry} % Ajuste les marges
\usepackage{xcolor} % Pour ajouter des couleurs
\usepackage{hyperref} % Pour avoir des références colorées si nécessaire
\usepackage{eso-pic}         % Pour ajouter des éléments en arrière-plan

\usepackage[french]{babel}
\usepackage[T1]{fontenc}
\usepackage{mathrsfs}
\usepackage{amsmath}
\usepackage{amsfonts}
\usepackage{amssymb}
\usepackage{tkz-tab}

\usepackage{tikz}
\usetikzlibrary{arrows, shapes.geometric, fit}
\newcounter{correction} % Compteur pour les questions

% Définir la commande pour afficher une question numérotée
\newcommand{\question}{%
  \refstepcounter{correction}%
  \textbf{\textcolor{black}{Question \thecorrection :}} \ignorespaces
}
% Commande pour ajouter du texte en arrière-plan
\AddToShipoutPicture{
    \AtTextCenter{%
        \makebox[0pt]{\rotatebox{80}{\textcolor[gray]{0.6}{\fontsize{10cm}{10cm}\selectfont PGB}}}
    }
}
\begin{document}
\hrule % Barre horizontale
% En-tête
\begin{center}
    \begin{tabular}{@{} p{5cm} p{5cm} p{5cm} @{}} % 3 colonnes avec largeurs fixées
        Lycée Dindéfélo & Test 2 & 17 Novembre 2025 \\
    \end{tabular}
    \\[-0.01cm] % Ajuster l'espace vertical entre le tableau et la barre
    \hrule % Barre horizontale
\end{center}
\begin{center}
    \textbf{\Large Calcul Littéral } \\[0.2cm]
    \textbf{\large Professeur : M. BA} \\[0.2cm]
    \textbf{Classe : 4 M2} \\[0.2cm]
    \textbf{\small Durée : 10 minutes} \\[0.2cm]
    \textbf{\small Note :\quad\quad /5}
\end{center}

% Nom de l'élève
\textbf{\small Nom de l'élève :} \underline{\hspace{8cm}} \\[0.5cm]

\question (1,5 points)\\
\[
\begin{aligned}
A&=-2+6-2+5\\
&= (-2-2) + (6+5)\\
&= -4 + 11\\
&= \mathbf{7}
\end{aligned}
\quad\quad\quad
\begin{aligned}
B&=-12-11-8+12\\
&= (-12+12) + (-11-8)\\
&= 0 + (-19)\\
&= \mathbf{-19}
\end{aligned}
\quad\quad\quad
\begin{aligned}
C&=9-15+6-21\\
&= (9+6) + (-15-21)\\
&= 15 + (-36)\\
&= \mathbf{-21}
\end{aligned}
\]
\\[1cm]
\question (1,5 points)\\
\(
\begin{aligned}
D&=(-2)\times(-7)+5\\
&= 14 + 5\\
&= \mathbf{19}\\
&\\
&\\
\end{aligned}
\quad\quad\quad
\begin{aligned}
E&=-15+(5)\times(-9)+9\\
&= -15 + (-45) + 9\\
&= -15 - 45 + 9\\
&= -60 + 9\\
&= \mathbf{-51}
\end{aligned}
\quad
\begin{aligned}
F&=2-(-11)\times(7)-15+2+(-10)\times(4)\\
&= 2 - (-77) - 15 + 2 + (-40)\\
&= 2 + 77 - 15 + 2 - 40\\
&= (2+77+2) + (-15-40)\\
&= 81 - 55\\
&= \mathbf{26}
\end{aligned}
\)
\\[1cm]
\question (2 points)\\
\[\begin{aligned}
G &= -\frac{7}{5} + \frac{9}{10} - \frac{12}{5} + \frac{4}{10} \\
   &= \left(-\frac{7}{5} - \frac{12}{5}\right) + \left(\frac{9}{10} + \frac{4}{10}\right) \quad \text{(Regroupement des fractions par dénominateur)}\\
   &= -\frac{7+12}{5} + \frac{9+4}{10} \\
   &= -\frac{19}{5} + \frac{13}{10} \quad \text{(Mettre au dénominateur commun 10)}\\
   &= -\frac{19 \times 2}{5 \times 2} + \frac{13}{10}\\ 
   &= -\frac{38}{10} + \frac{13}{10}\\ 
   &= \frac{-38+13}{10}\\ 
   &= -\frac{25}{10}\\ 
   &= \mathbf{-\frac{5}{2}}
\end{aligned} 
\]

\vspace{1cm}

\[
\begin{aligned}
H &= -\frac{7}{3} \times \frac{9}{2} - \frac{12}{4} \times \frac{4}{9} \\
   &= -\frac{7 \times 9}{3 \times 2} - \frac{12 \times 4}{4 \times 9} \quad \text{(Multiplications de fractions)}\\
   &= -\frac{7 \times (3 \times 3)}{3 \times 2} - \frac{12}{9} \quad \text{(Simplification par 3 et par 4)}\\
   &= -\frac{7 \times 3}{2} - \frac{4 \times 3}{3 \times 3} \\
   &= -\frac{21}{2} - \frac{4}{3} \quad \text{(Mettre au dénominateur commun 6)}\\
   &= -\frac{21 \times 3}{2 \times 3} - \frac{4 \times 2}{3 \times 2} \\
   &= -\frac{63}{6} - \frac{8}{6}\\ 
   &= \frac{-63-8}{6} \\
   &= \mathbf{-\frac{71}{6}}
\end{aligned}
\]
\end{document}