\documentclass[12pt,a4paper]{article}
\usepackage{amsmath,amsfonts,amssymb,mathrsfs,tikz,times,pifont}
\usepackage{enumitem}
\newcommand\circitem[1]{%
\tikz[baseline=(char.base)]{
\node[circle,draw=gray, fill=red!55,
minimum size=1.2em,inner sep=0] (char) {#1};}}
\newcommand\boxitem[1]{%
\tikz[baseline=(char.base)]{
\node[fill=cyan,
minimum size=1.2em,inner sep=0] (char) {#1};}}
\setlist[enumerate,1]{label=\protect\circitem{\arabic*}}
\setlist[enumerate,2]{label=\protect\boxitem{\alph*}}
%%%::::::by chnini ameur :::::::%%%
\everymath{\displaystyle}
\usepackage[left=1cm,right=1cm,top=1cm,bottom=1.7cm]{geometry}
\usepackage{array,multirow}
\usepackage[most]{tcolorbox}
\usepackage{varwidth}
\usepackage{mathrsfs} % pour \mathscr
\usepackage{calligra}
\newcommand{\Dcalligra}{\text{\calligra D}}
\tcbuselibrary{skins,hooks}
\usetikzlibrary{patterns}
%%%::::::by chnini ameur :::::::%%%
\newtcolorbox{exa}[2][]{enhanced,breakable,before skip=2mm,after skip=5mm,
colback=yellow!20!white,colframe=black!20!blue,boxrule=0.5mm,
attach boxed title to top left ={xshift=0.6cm,yshift*=1mm-\tcboxedtitleheight},
fonttitle=\bfseries,
title={#2},#1,
% varwidth boxed title*=-3cm,
boxed title style={frame code={
\path[fill=tcbcolback!30!black]
([yshift=-1mm,xshift=-1mm]frame.north west)
arc[start angle=0,end angle=180,radius=1mm]
([yshift=-1mm,xshift=1mm]frame.north east)
arc[start angle=180,end angle=0,radius=1mm];
\path[left color=tcbcolback!60!black,right color = tcbcolback!60!black,
middle color = tcbcolback!80!black]
([xshift=-2mm]frame.north west) -- ([xshift=2mm]frame.north east)
[rounded corners=1mm]-- ([xshift=1mm,yshift=-1mm]frame.north east)
-- (frame.south east) -- (frame.south west)
-- ([xshift=-1mm,yshift=-1mm]frame.north west)
[sharp corners]-- cycle;
},interior engine=empty,
},interior style={top color=yellow!5}}
%%%%%%%%%%%%%%%%%%%%%%%

\usepackage{fancyhdr}
\usepackage{eso-pic}         % Pour ajouter des éléments en arrière-plan
% Commande pour ajouter du texte en arrière-plan
\AddToShipoutPicture{
    \AtTextCenter{%
        \makebox[0pt]{\rotatebox{80}{\textcolor[gray]{0.7}{\fontsize{5cm}{5cm}\selectfont PGB}}}
    }
}
\usepackage{lastpage}
\fancyhf{}
\pagestyle{fancy}
\renewcommand{\footrulewidth}{1pt}
\renewcommand{\headrulewidth}{0pt}
\renewcommand{\footruleskip}{10pt}
\fancyfoot[R]{
\color{blue}\ding{45}\ \textbf{2025}
}
\fancyfoot[L]{
\color{blue}\ding{45}\ \textbf{Prof:M. BA et DIALLO}
}
\cfoot{\bf
\thepage /
\pageref{LastPage}}
\begin{document}
\renewcommand{\arraystretch}{1.5}
\renewcommand{\arrayrulewidth}{1.2pt}
\begin{tikzpicture}[overlay,remember picture]
\node[draw=blue,line width=1.2pt,fill=purple,text=blue,inner sep=3mm,rounded corners,pattern=dots]at ([yshift=-2.5cm]current page.north) {\begingroup\setlength{\fboxsep}{0pt}\colorbox{white}{\begin{tabular}{|*1{>{\centering \arraybackslash}p{0.28\textwidth}} |*2{>{\centering \arraybackslash}p{0.2\textwidth}|} *1{>{\centering \arraybackslash}p{0.19\textwidth}|} }
\hline
\multicolumn{3}{|c|}{$\diamond$$\diamond$$\diamond$\ \textbf{Lycée de Dindéfélo}\ $\diamond$$\diamond$$\diamond$ }& \textbf{A.S. : 2025/2026} \\ \hline
\textbf{Matière: Mathématiques}& \textbf{Niveau : 4}\textbf{ème} &\textbf{Date: 25/11/2025} & \textbf{Durée : 2 heures} \\ \hline
\multicolumn{4}{|c|}{\parbox[c]{10cm}{\begin{center}
\textbf{{\Large\sffamily Devoir n$ ^{\circ} $ 1 Du 1$ ^\text{\bf er} $ Semestre}}
\end{center}}} \\ \hline
\end{tabular}}\endgroup};
\end{tikzpicture}
\vspace{3cm}

\section*{\underline{Exercice 1 :} 7,5 points }
\begin{enumerate}
\item \textbf{Compléter par $\in$ ou $\notin$. (2,5 pts)}

\noindent Nous utilisons les définitions des ensembles de nombres :
($\mathbb{N}$: Naturels, $\mathbb{Z}$: Entiers relatifs, $\mathbb{D}$: Décimaux, $\mathbb{Q}$: Rationnels).

$$
\dfrac{7}{5}\; \boldsymbol{\notin}\; \mathbb{Z} \quad ; \quad \dfrac{7}{3}\; \boldsymbol{\notin}\; \mathbb{D} \quad ; \quad -\dfrac{14}{7}\; \boldsymbol{\notin}\; \mathbb{N} \quad ; \quad -\dfrac{5}{7}\; \boldsymbol{\notin}\; \mathbb{D} \quad ; \quad -\dfrac{5}{7}\; \boldsymbol{\in}\; \mathbb{Q}
$$
\begin{itemize}
    \item $\dfrac{7}{5} = 1,4$. $1,4$ n'est pas un entier relatif ($\notin \mathbb{Z}$).
    \item $\dfrac{7}{3} = 2,333\ldots$ Cette fraction n'a pas de développement décimal fini, elle n'est donc pas décimale ($\notin \mathbb{D}$).
    \item $-\dfrac{14}{7} = -2$. $-2$ n'est pas un nombre naturel ($\notin \mathbb{N}$).
    \item $-\dfrac{5}{7}$ n'a pas de développement décimal fini ($\notin \mathbb{D}$).
    \item $-\dfrac{5}{7}$ est un quotient d'entiers, c'est donc un nombre rationnel ($\in \mathbb{Q}$).
\end{itemize}


\item \textbf{Répondre par Vrai ou Faux. (2,5 pts)}
\begin{enumerate}
    \item Les fractions $\dfrac{7}{5}$ et $\dfrac{70}{50}$ sont égales. \hfill \textbf{Vrai} ($\dfrac{70}{50} = \dfrac{70 \div 10}{50 \div 10} = \dfrac{7}{5}$)
    \item L'inverse de $-\dfrac{5}{7}$ est $\dfrac{7}{5}$. \hfill \textbf{Faux} (L'inverse est $-\dfrac{7}{5}$)
    \item Si $a \leq b$ alors $a +c \leq b+c$. \hfill \textbf{Vrai} (Propriété de l'addition des inégalités)
    \item $ \left(\dfrac{a}{b} \right)^{n} \times \left(\dfrac{a}{b} \right)^{m} = \left(\dfrac{b}{a} \right)^{m+n} $. \hfill \textbf{Faux} (Le résultat est $\left(\dfrac{a}{b} \right)^{m+n}$)
    \item $\dfrac{7}{5} < \dfrac{4}{5}$ et $\dfrac{1}{5} < \dfrac{4}{3}$. \hfill \textbf{Faux} (Car $\dfrac{7}{5}$ n'est pas inférieur à $\dfrac{4}{5}$)
\end{enumerate}


\item \textbf{Soit un cercle $\mathscr{C}(O, r)$ et $\mathscr{C'}(O', r')$. Compléter : (2,5 pts)}
\begin{enumerate}
    \item Les cercles $(\mathscr{C})$ et $(\mathscr{C'})$ sont \textbf{tangents extérieurement} si et seulement si $OO' = r + r'$.
    \item Les cercles $(\mathscr{C})$ et $(\mathscr{C'})$ sont disjoints extérieurement si et seulement si $\boldsymbol{OO' > r + r'}$.
    \item Les cercles $(\mathscr{C})$ et $(\mathscr{C'})$ sont sécants si et seulement si $\boldsymbol{|r - r'| < OO' < r + r'}$.
    \item Si $OO' > r + r'$, alors $(\mathscr{C})$ et $(\mathscr{C'})$ sont \textbf{disjoints extérieurement}.
    \item Si $\boldsymbol{OO' = |r - r'|}$, alors $(\mathscr{C})$ et $(\mathscr{C'})$ sont tangents intérieurement.
\end{enumerate}
\end{enumerate}

\section*{\underline{Correction de l'Exercice 2 :} 5 points (Positions relatives de deux cercles)}

\noindent \textit{Note : La valeur de $O_1O_2$ dans la dernière colonne a été ajustée de 24 à 10 pour illustrer le cas des cercles disjoints intérieurement.}

\vspace{0.3cm}

\begin{center}
\begin{tabular}{|>{\centering\arraybackslash}m{4.5cm}|c|c|c|c|c|}
\hline
\textbf{$R_1$} & 9 & 8,2 & 6,4 & 10 & 5 \\
\hline
\textbf{$R_2$} & 14 & 7,5 & 4,9 & 23 & 18 \\
\hline
\textbf{$O_1O_2$} & 12 & 15,7 & 15,6 & 13 & \textbf{10} \\
\hline
\textbf{$R_1 + R_2$} & \textbf{23} & \textbf{15,7} & \textbf{11,3} & \textbf{33} & \textbf{23} \\
\hline
\textbf{$|R_1 - R_2|$} & \textbf{5} & \textbf{0,7} & \textbf{1,5} & \textbf{13} & \textbf{13} \\
\hline
\textbf{Position relative de $(C_1)$ et $(C_2)$} & \textbf{Sécants} & \textbf{Tangents ext.} & \textbf{Disjoints ext.} & \textbf{Tangents int.} & \textbf{Disjoints int.} \\
\hline
\end{tabular}
\end{center}

\vspace{0.5cm}
\noindent \textbf{Justifications :}
\begin{enumerate}
    \item $O_1O_2 = 12$. On a $|R_1 - R_2|=5$ et $R_1 + R_2=23$. Puisque $5 < 12 < 23$, les cercles sont \textbf{Sécants}. 

    \item $O_1O_2 = 15,7$. On a $R_1 + R_2=15,7$. Puisque $O_1O_2 = R_1 + R_2$, les cercles sont \textbf{Tangents extérieurement}.
    \item $O_1O_2 = 15,6$. On a $R_1 + R_2=11,3$. Puisque $O_1O_2 > R_1 + R_2$ ($15,6 > 11,3$), les cercles sont \textbf{Disjoints extérieurement}.
    \item $O_1O_2 = 13$. On a $|R_1 - R_2|=13$. Puisque $O_1O_2 = |R_1 - R_2|$, les cercles sont \textbf{Tangents intérieurement}.
    \item $O_1O_2 = 10$. On a $|R_1 - R_2| = 13$. Puisque $O_1O_2 < |R_1 - R_2|$ ($10 < 13$), les cercles sont \textbf{Disjoints intérieurement}.
\end{enumerate}


\section*{\underline{Correction de l'Exercice 3 :} 4 points (Condition d'existence d'un triangle)}

\noindent Un triangle $DEF$ existe si, et seulement si, la longueur du plus grand côté est strictement inférieure à la somme des longueurs des deux autres côtés. 

\begin{enumerate}
    \item $\mathbf{1^{\text{er}} \text{ cas } :}$ $DE = 500cm \quad EF = 200cm\quad DF = 250cm$
    \begin{itemize}
        \item Le plus grand côté est : $DE = 500$.
        \item Somme des deux autres côtés : $EF + DF = 200 + 250 = 450$.
        \item Comparaison : $500 > 450$.
        \item Conclusion : L'inégalité triangulaire n'est pas vérifiée ($DE > EF + DF$). \textbf{Le triangle $DEF$ n'existe pas}.
    \end{itemize}

    \item $\mathbf{2^{\text{ème}} \text{ cas } :}$ $DE = 7500cm \quad EF = 5000cm \quad DF = 4000cm$
    \begin{itemize}
        \item Le plus grand côté est : $DE = 7500$.
        \item Somme des deux autres côtés : $EF + DF = 5000 + 4000 = 9000$.
        \item Comparaison : $7500 < 9000$.
        \item Conclusion : L'inégalité triangulaire est vérifiée ($DE < EF + DF$). \textbf{Le triangle $DEF$ existe}.
    \end{itemize}

    \item $\mathbf{3^{\text{ème}} \text{ cas } :}$ $DE = 14200cm \quad EF = 19000cm \quad DF = 4200cm$
    \begin{itemize}
        \item Le plus grand côté est : $EF = 19000$.
        \item Somme des deux autres côtés : $DE + DF = 14200 + 4200 = 18400$.
        \item Comparaison : $19000 > 18400$.
        \item Conclusion : L'inégalité triangulaire n'est pas vérifiée ($EF > DE + DF$). \textbf{Le triangle $DEF$ n'existe pas}.
    \end{itemize}

    \item $\mathbf{4^{\text{ème}} \text{ cas } :}$ $DE = 105600cm \quad EF = 104600cm \quad DF = 102400cm$
    \begin{itemize}
        \item Le plus grand côté est : $DE = 105600$.
        \item Somme des deux autres côtés : $EF + DF = 104600 + 102400 = 207000$.
        \item Comparaison : $105600 < 207000$.
        \item Conclusion : L'inégalité triangulaire est vérifiée ($DE < EF + DF$). \textbf{Le triangle $DEF$ existe}.
    \end{itemize}
\end{enumerate}

\section*{\underline{Correction de l'Exercice 4 :} 3 points (Calcul de Fractions et Priorité des Opérations)}

\noindent \textbf{Consigne :} Calculer les expressions suivantes et donner le résultat sous forme de fraction irréductible.

\vspace{0.5cm}

\begin{enumerate}
    \item Calculer l'expression $A$.
    \[
    \begin{aligned}
    A &= \left(\frac{3}{4} - \frac{5}{6}\right) \div \left(\frac{1}{2} + \frac{1}{3}\right) \\
    A &= \left(\frac{3 \times 3}{12} - \frac{5 \times 2}{12}\right) \div \left(\frac{1 \times 3}{6} + \frac{1 \times 2}{6}\right) \quad & \text{(Réduction au même dénominateur)} \\
    A &= \left(\frac{9}{12} - \frac{10}{12}\right) \div \left(\frac{3}{6} + \frac{2}{6}\right) \\
    A &= \left(-\frac{1}{12}\right) \div \left(\frac{5}{6}\right) \\
    A &= -\frac{1}{12} \times \frac{6}{5} \quad & \text{(Multiplication par l'inverse)} \\
    A &= -\frac{1 \times 6}{2 \times 6 \times 5} \quad & \text{(Simplification par 6)} \\
    A &= -\frac{1}{10}
    \end{aligned}
    \]

    \item Calculer l'expression $B$ en simplifiant au maximum.
    \[
    \begin{aligned}
    B &= \left(-\frac{7}{5} \times \frac{25}{21}\right) + \frac{5}{3} - \frac{1}{4} \\
    B &= \left(-\frac{7 \times (5 \times 5)}{5 \times (7 \times 3)}\right) + \frac{5}{3} - \frac{1}{4} \quad & \text{(Simplification de la multiplication)} \\
    B &= -\frac{5}{3} + \frac{5}{3} - \frac{1}{4} \\
    B &= 0 - \frac{1}{4} \\
    B &= -\frac{1}{4}
    \end{aligned}
    \]

    \item Calculer l'expression $C$ qui inclut une puissance négative.
    \[
    \begin{aligned}
    C &= \frac{2^3 - 3^2}{4} + \left(\frac{1}{5}\right)^{-2} \\
    C &= \frac{8 - 9}{4} + \left(\frac{5}{1}\right)^{2} \quad & \text{(Calcul des puissances et inverse)} \\
    C &= \frac{-1}{4} + 25 \\
    C &= -\frac{1}{4} + \frac{25 \times 4}{4} \quad & \text{(Réduction au même dénominateur)} \\
    C &= -\frac{1}{4} + \frac{100}{4} \\
    C &= \frac{99}{4}
    \end{aligned}
    \]
\end{enumerate}

\end{document}
