\documentclass[a4paper,11pt]{article}
\usepackage{lmodern} % Pour une police plus nette
\usepackage{stmaryrd}
\usepackage{graphicx} % Pour l'insertion d'images
\usepackage{float}    % Pour contrôler précisément le placement
\usepackage[utf8]{inputenc}
\usepackage[french]{babel}
\usepackage[T1]{fontenc}
\usepackage{hyperref}
\usepackage{verbatim}
\usepackage{color, soul}
\usepackage{pgfplots}
\pgfplotsset{compat=1.18} % Version plus récente de pgfplots
\usepackage{mathrsfs}
\usepackage{amsfonts}
\usepackage{tkz-tab}
\usepackage{enumitem}
\usepackage{array}
\usepackage{multirow}
\usepackage{tabularx}
\usepackage{amsmath,amssymb}
\usepackage{makecell}
\usepackage{tikz}
\usetikzlibrary{arrows, shapes.geometric, fit}
% Commande pour la couleur d'accentuation
\newcommand{\myul}[2][black]{\setulcolor{#1}\ul{#2}\setulcolor{black}}
\newcommand\tab[1][1cm]{\hspace*{#1}}
\usepackage[margin=2.5cm]{geometry} % Ajustement des marges
\usepackage{eso-pic} % Pour ajouter des éléments en arrière-plan

% Commande pour ajouter du texte en arrière-plan, centré au milieu de chaque page
\AddToShipoutPicture{
    \AtPageCenter{%
        \makebox(0,0)[c]{\rotatebox{60}{\textcolor[gray]{0.9}{\fontsize{2cm}{2cm}\selectfont PGB}}}
    }
}
\newcounter{exercice}

% Définir la commande \exemple pour afficher un exemple numéroté
\newcommand{\exercice}{%
  \refstepcounter{exercice}% Incrémenter le compteur
  \textbf{\textcolor{black}{Exercice \theexercice  }} \ignorespaces
}

\begin{document}

\noindent
\begin{minipage}[t]{0.48\textwidth}
\raggedright
\textbf{Ministère de l'Éducation Nationale}\\
Inspection Académique de Kédougou\\
Lycée Dindéfelo\\
Cellule de Mathématiques
\end{minipage}
\hfill
\begin{minipage}[t]{0.48\textwidth}
\raggedleft
\textbf{Année scolaire 2025-2026}\\
Date : 12/11/2025\\
Classe : 4M2\\
Professeur : M. BA
\end{minipage}

\vspace{0.5cm}

\begin{center}
\textbf{Nombres Rationnels}
\end{center}

\exercice « Opérations sur les nombres décimaux »

\textbf{1. Calculer les additions suivantes.}

\[ A = (+7,5) + (+13,5) \quad;\quad B = (-13,5) + (-10)\quad;\quad C = (-13) + (+20) \quad;\quad D = (+103,5) + (+10)\]

\[
E = (-103,5) + (+13,5) \quad;\quad F = -7,5 + 1,5
\]

\textbf{2. Calculer les soustractions suivantes.}

\[ A = (+7,5) - (-13,5) \quad;\quad B = (-6,5) - (+13,5)\quad;\quad C = (-7,5) - (+1,3) \quad;\quad D = (-8,5) - (-4,10)\]

\exercice : « Somme algébrique »

\textbf{Calculer chacune des expressions suivantes.}
\[ A = -16 + 1,5 - 18,1 - 0,9 + 5,5 \quad;\quad B = +60 - 40,5 - 18,5 + 0,5 - 30 \]
\[
C = 28,5 - 16,5 + 12,9 - 0,90 - 13,5 - 7,5\quad;\quad D = (-84) + (+75) + (-5) + (+18)\quad;\quad 
\]

\exercice : « Somme algébrique »

\textbf{1. Calculer chacune des expressions suivantes en utilisant la règle de la suppression des parenthèses.}
\[ A = (-13) + (-4) - (-7) - (+2) + (+8) \quad;\quad B = (+3,5) - (+13) + (+12) - (-7,5) \]
\[
C = (+14) - (+13) - (+6) - (-8) - (+18)\quad;\quad D = -(-84) - (+75) - (-5) + (+18)
\]

\exercice : « Somme algébrique »

\textbf{1. Calculer chacune des expressions suivantes de deux manières différentes.}

\[ A = (-7,5) - (-17,5) + (-14) - (+2) \quad;\quad B = (-10,5) - (+10,15) + (+0,15) - (+9,5) \]
\[
C = (+140) - (-14,5) - (+4,5) + (+7) - (+18) \quad;\quad D = -(-4) - (+75) - (-5) + (+18) 
\]

\exercice :

\textbf{1. Compléter les pointillés par : } \( \in \) \text{ ou } \( \notin \)
\[ \dfrac{21}{3}\; \ldots\; \mathbb{N} \quad;\quad \dfrac{41}{3}\; \ldots\; \mathbb{N} \quad;\quad \dfrac{21}{11}\; \ldots\; \mathbb{D}\quad ;\quad \dfrac{40}{12}\; \ldots\; \mathbb{Q}\quad ;\quad \dfrac{125}{375}\; \ldots\; \mathbb{Q}^+ \]
\[
\dfrac{365}{73}\; \ldots\; \mathbb{Z} \quad ;\quad \dfrac{121}{11}\; \ldots\; \mathbb{Q}\quad ;\quad \dfrac{42}{6}\; \ldots\; \mathbb{I D} \quad ;\quad -15,5\; \ldots\; \mathbb{Q}\quad ;\quad \dfrac{41}{3}\; \ldots\; \mathbb{N}\quad ;\quad \dfrac{45}{6}\; \ldots\; \mathbb{N}
\]

\textbf{2. Compléter les pointillés par : } \( \subset \) \text{ ou } \( \not\subset \)

\[
\mathbb{N}\; \ldots\; \mathbb{Q}
\quad ;\quad 
\mathbb{Z}\; \ldots\; \mathbb{N}
\quad ;\quad 
\mathbb{D}\; \ldots\; \mathbb{Q}
\quad ;\quad 
\mathbb{I D}\; \ldots\; \mathbb{Q}
\quad ;\quad 
\mathbb{Q}\; \ldots\; \mathbb{I D}
\]

\exercice : \textbf{Le PGCD et le PPCM}

\[
1.\quad \text{Calculer } \; \text{PGCD}(504; 492) \; \text{et} \; \text{PGCD}(888; 777)
\]

\[
\text{Puis simplifier la fraction :} \quad 
A = \dfrac{504}{492} 
\quad ; \quad 
B = \dfrac{888}{777}
\]

\exercice : Écrire un nombre rationnel sous plusieurs formes.

1) \textit{On peut remplacer :}\\
\( \text{ a) }\dfrac{6}{4} \text{ par } \dfrac{3}{2} \text{ b) } \dfrac{3}{5} \text{ par } \dfrac{12}{20} \text{ c) } \dfrac{25}{45} \text{ par } \dfrac{5}{9} \textbf{ Pourquoi ?}\)

\medskip

2) \textit{On peut remplacer :}\\
\( \text{ a) }\dfrac{1}{4} \text{ par } \dfrac{8}{32} \text{ b) } \dfrac{2}{5} \text{ par } \dfrac{4}{10} \text{ c) } \dfrac{5}{4} \text{ par } \dfrac{35}{28} \textbf{ Pourquoi ?}\) 

\noindent\textbf{3) Simplifie l'écriture des nombres rationnels suivants :}

\[
\dfrac{56}{720} \; ; \quad 
\dfrac{-45}{75} \; ; \quad 
\dfrac{42}{-1050} \; ; \quad 
\dfrac{-126}{270} \; ; \quad 
\dfrac{-378}{440}
\]

\exercice : Addition et soustraction des nombres rationnels.


\textbf{1. Calculer les sommes suivantes puis simplifier :}

\[ \text{A}= \dfrac{3}{4} + \dfrac{5}{-3} \quad;\quad \text{B}= 7 + \dfrac{-3}{2} \quad;\quad \text{C}= \left(-\dfrac{2}{13}\right) + \left(-\dfrac{7}{13}\right) \quad;\quad \text{D} = \dfrac{-6}{7} + \dfrac{5}{7}\]

\[ \text{E}= \dfrac{-3}{4} + \dfrac{5}{12} \quad;\quad \text{F}= \dfrac{-2}{7} + \dfrac{-3}{14} \quad;\quad \text{G}= \left(-\dfrac{2}{3}\right) + \left(-\dfrac{7}{27}\right) \quad;\quad \text{H} = \dfrac{3}{4} + \dfrac{-23}{20} \]

\textbf{2. Calculer les différences suivantes puis simplifier :}

\[
1.)\quad \dfrac{12}{11} - \dfrac{25}{11} \quad
2.)\quad  \dfrac{42}{37} - \dfrac{19}{37}\quad
3.)\quad  \dfrac{10}{3} - \dfrac{-9}{8} \quad
4.)\quad  \dfrac{-17}{10} - \dfrac{8}{5} \quad
5.)\quad  \dfrac{14}{25} - 1.
\]

\exercice : Produits et quotients.  

\textbf{1) Calcule les produits suivants :}

\[
a)\quad \dfrac{9}{5} \times \dfrac{4}{5}
\quad b)\quad -\dfrac{3}{7} \times \dfrac{2}{7}
\quad c)\quad \dfrac{10}{13} \times \dfrac{-12}{5}
\quad d)\quad \dfrac{-1}{8} \times \dfrac{-16}{3}
\quad e)\quad -6 \times \dfrac{8}{5}
\]

\textbf{2) Calcule les quotients :}

\[
a)\quad \dfrac{\dfrac{6}{13}}{\dfrac{3}{5}}
\quad b)\quad \dfrac{\dfrac{-1}{7}}{\dfrac{-1}{3}}
\quad c)\quad \dfrac{\dfrac{-3}{22}}{\dfrac{15}{22}}
\quad d)\quad \dfrac{\dfrac{12}{11}}{9}
\quad e)\quad \dfrac{\dfrac{-14}{7}}{\dfrac{7}{2}}
\]

\exercice : Calculer

\[
A = \left( \dfrac{5}{2} + \dfrac{3}{4} \right)
\times \left( \dfrac{-3}{10} + \dfrac{4}{25} \right) \quad;\quad B = \left( \dfrac{1}{9} - \dfrac{5}{6} \right)
\div \left( -3 - \dfrac{3}{8} \right)
\times \left( \dfrac{1}{9} - \dfrac{5}{6} \right)
\]

\[
C = \dfrac{\dfrac{3}{5}-2}{1+\dfrac{4}{3}} \quad;\quad D = \dfrac{\dfrac{3}{5}-\dfrac{2}{3}}{\dfrac{1}{4}+\dfrac{2}{3}} \quad;\quad E = \left(3-\dfrac{9}{4}\right)\times\dfrac{4}{5} \quad;\quad F = \left[\dfrac{2}{5}-\left(\dfrac{3}{2}-2+\dfrac{1}{5}\right)\right]\div \left( \dfrac{1}{2} - \dfrac{1}{5} \right)\]

\exercice :
\begin{enumerate}
\item Compare par la méthode de ton choix les rationnels : \( \dfrac{5}{6} \text{ et } \dfrac{7}{9} \)

\item Déduis-en une comparaison de : \( \dfrac{5}{6} - \dfrac{3}{2}  \text{ et } \dfrac{7}{9} - \dfrac{3}{2}  \)
\end{enumerate}

\end{document}