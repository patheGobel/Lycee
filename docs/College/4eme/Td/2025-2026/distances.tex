\documentclass[a4paper,11pt]{article}
\usepackage{lmodern} % Pour une police plus nette
\usepackage{stmaryrd}
\usepackage{graphicx} % Pour l'insertion d'images
\usepackage{float}    % Pour contrôler précisément le placement
\usepackage[utf8]{inputenc}
\usepackage[french]{babel}
\usepackage[T1]{fontenc}
\usepackage{hyperref}
\usepackage{verbatim}
\usepackage{color, soul}
\usepackage{pgfplots}
\pgfplotsset{compat=1.18} % Version plus récente de pgfplots
\usepackage{mathrsfs}
\usepackage{amsfonts}
\usepackage{tkz-tab}
\usepackage{enumitem}
\usepackage{array}
\usepackage{multirow}
\usepackage{tabularx}
\usepackage{amsmath,amssymb}
\usepackage{makecell}
\usepackage{tikz}
\usetikzlibrary{arrows, shapes.geometric, fit}
% Commande pour la couleur d'accentuation
\newcommand{\myul}[2][black]{\setulcolor{#1}\ul{#2}\setulcolor{black}}
\newcommand\tab[1][1cm]{\hspace*{#1}}
\usepackage[margin=2.5cm]{geometry} % Ajustement des marges
\usepackage{eso-pic} % Pour ajouter des éléments en arrière-plan

% Commande pour ajouter du texte en arrière-plan, centré au milieu de chaque page
\AddToShipoutPicture{
    \AtPageCenter{%
        \makebox(0,0)[c]{\rotatebox{60}{\textcolor[gray]{0.6}{\fontsize{2cm}{2cm}\selectfont PGB}}}
    }
}
\newcounter{exercice}

% Définir la commande \exemple pour afficher un exemple numéroté
\newcommand{\exercice}{%
  \refstepcounter{exercice}% Incrémenter le compteur
  \textbf{\textcolor{black}{Exercice \theexercice  }} \ignorespaces
}

\begin{document}

\begin{minipage}[t]{0.48\textwidth}
\raggedright
\textbf{Ministère de l'Éducation Nationale}\\
Inspection Académique de Kédougou\\
Lycée Dindéfelo\\
Cellule de Mathématiques
\end{minipage}
\hfill
\begin{minipage}[t]{0.48\textwidth}
\raggedleft
\textbf{Année scolaire 2025-2026}\\
Date : 20/11/2025\\
Classe : 4M2\\
Professeur : M. BA
\end{minipage}

\vspace{0.5cm}

\begin{center}
\textbf{Distances}
\end{center}

\exercice «Inégalité triangulaire »

Sans faire la figure, dites dans chacun des cas ci-dessous si les points $A\;,\ B$ et $C$ sont alignés (Préciser l'ordre de l'alignement des points).

$1^{er}$ cas : $AB=12\qquad AC=5\qquad BC=7$

$2^{ième}$ cas : $AB=7.6\qquad AC=2.5\qquad BC=10.2$

$3^{ième}$ cas : $AB=200\qquad AC=10\qquad BC=210$

$4^{ième}$ cas : $AB=0.5\qquad AC=1.06\qquad BC=0.56$


\exercice : «Inégalité triangulaire»

Dans chacun des cas ci-dessous sans faire la figure dite si le triangle $DEF$ existe.

$1^{er}$ cas : $DE=5\qquad EF=2\qquad DF=2.5$

$2^{ième}$ cas : $DE=7.5\qquad EF=5\qquad DF=4$

$3^{ième}$ cas : $DE=14.2\qquad EF=19\qquad DF=4.2$

$4^{ième}$ cas : $DE=105.6\qquad EF=104.6\qquad DF=102.4$


\exercice : Distance entre 2 Droites Parallèles

Trace deux droites parallèles $(D)$ et $(D')$.\\
Marque un point $A$ sur $(D)$ et construis la perpendiculaire à $(D)$ passant par $A$.\\
Mesure la distance entre $(D)$ et $(D')$.

\exercice : Position Relative de 2 Cercles
\begin{enumerate}
    \item Marque 2 points $O$ et $O'$, avec : $OO' = 6 \text{ cm}$.
    \item Construis $(\mathcal{C})$ et $(\mathcal{C'})$, cercles de centres $O$ et $O'$ (respectivement), de même rayon $2 \text{ cm}$.
    \item Construis un cercle $(\mathcal{C''})$ tangent à $(\mathcal{C})$ et à $(\mathcal{C'})$ ; précise le rayon de $(\mathcal{C''})$.
\end{enumerate}

\exercice : Distance d'un Point à une Droite

On donne un point $O$. Construis une droite $(D)$ située à $2 \text{ cm}$ de $O$ et une autre droite $(D')$ située à $3,5 \text{ cm}$ de $O$ et telle que $(D)$ et $(D')$ soient sécantes.

\exercice : Position Relative Droite / Cercle

Construis un cercle $(\mathcal{C})$ et une droite $(D)$ qui ne coupe pas $(\mathcal{C})$ ; puis les droites parallèles à $(D)$ qui sont tangentes à $(\mathcal{C})$.

\exercice : Distance d'un Point à une Droite

Trace une droite $(d)$ et un point $M$ hors de $(d)$.\\
Utilise ton équerre et ta règle pour mesurer la distance du point $M$ à la droite $(d)$.

\exercice : Distance entre 2 Droites Parallèles

Trace deux droites parallèles $(D)$ et $(D')$.\\
Marque un point $A$ sur $(D)$ et construis la perpendiculaire à $(D)$ passant par $A$.\\
Mesure la distance entre $(D)$ et $(D')$.

\exercice : Position Relative de 2 Cercles
\begin{enumerate}
    \item Marque 2 points $O$ et $O'$, avec : $OO' = 6 \text{ cm}$.
    \item Construis $(\mathcal{C})$ et $(\mathcal{C'})$, cercles de centres $O$ et $O'$ (respectivement), de même rayon $2 \text{ cm}$.
    \item Construis un cercle $(\mathcal{C''})$ tangent à $(\mathcal{C})$ et à $(\mathcal{C'})$ ; précise le rayon de $(\mathcal{C''})$.
\end{enumerate}

\exercice :
\begin{enumerate}
    \item Marque deux points $O$ et $O'$, avec $OO' = 6 \text{ cm}$.
    \item Construis les cercles $\mathcal{C}$ et $\mathcal{C'}$ de centre respectifs $O$ et $O'$ et de même rayon $2 \text{ cm}$.
    \item Construis un cercle $\mathcal{C''}$ tangent à $\mathcal{C}$ et à $\mathcal{C'}$.
\end{enumerate}

\exercice : Distance d'un Point à une Droite
On donne un point $O$. Construis une droite $(D)$ située à $2 \text{ cm}$ de $O$ et une autre droite $(D')$ située à $3,5 \text{ cm}$ de $O$ et telle que $(D)$ et $(D')$ soient sécantes.

\exercice : Position Relative Droite / Cercle
Construis un cercle $(\mathcal{C})$ et une droite $(D)$ qui ne coupe pas $(\mathcal{C})$ ; puis les droites parallèles à $(D)$ qui sont tangentes à $(\mathcal{C})$.

\exercice :
On donne une droite $(D)$ et un point $B$ situé à $1 \text{ cm}$ de $(D)$.\\
Construis les droites $(D_1)$ et $(D_2)$ parallèles à $(D)$ et situées à $2 \text{ cm}$ du point $B$.\\
Quelle est la distance des droites $(D_1)$ et $(D_2)$ ?\\
Quelle est la distance de $(D)$ à chacune des droites $(D_1)$ et $(D_2)$ ?

\exercice :
On donne un carré $ABCD$ de centre $O$ et le point $E$, milieu du côté $[BC]$.\\
Trace le cercle $(\mathcal{C})$ de centre $O$ passant par $E$.\\
Quelle est la position relative de chacune des droites $(AB)$, $(BC)$, $(CD)$ et $(DA)$ par rapport au cercle $(\mathcal{C})$ ?

\exercice :
\begin{enumerate}
    \item Trace un cercle $(C)$ de centre $O$ et de rayon $3\text{cm}$ et marque les points $A$ et $B$ diamétralement opposés sur $(C)$.
    \vspace{1ex}
    \begin{itemize}
        \item Place le point $D$ sur le cercle $(C)$ tel que $AD = 4\text{cm}$.
    \end{itemize}
    \item Montre que le triangle $ADB$ est rectangle en $D$.
    \item
    \begin{enumerate}
        \item Quelle est la distance du point $A$ à la droite $(BD)$ ? 
        \item Détermine la distance du point $D$ à la droite $(AB)$. 
    \end{enumerate}
    \item Construis la droite $(D)$ médiatrice du segment $[AB]$. Elle coupe le cercle en $E$ et $F$. 
    \vspace{1ex}
    \begin{itemize}
        \item Justifie que $AE = BE$. 
    \end{itemize}
    \item Place un point $I$ sur le cercle situé à $2\text{cm}$ de la droite $(D)$.
    \item Construis le cercle $(C')$ de centre $I$ et de rayon $2\text{cm}$.
    \begin{enumerate}
        \item Quelle est la position relative des cercles $(C)$ et $(C')$ ?
        \item Quelle est la position relative de $(D)$ et $(C')$ ?
    \end{enumerate}
\end{enumerate}

\exercice :
\begin{enumerate}
    \item Sur un segment $[KJ]$ de longueur $9 \text{ cm}$, placer les points $O$ et $I$ tels que : $KO = 4 \text{ cm}$ et $KI = 7 \text{ cm}$.
    \item Construire le cercle $(C1)$ de centre $O$ et de rayon $4 \text{ cm}$.
    \item Tracer les droites $(D_1), (D_2), (D_3)$ perpendiculaires à $(KJ)$ respectivement en $K$, $I$ et $J$.
    \item Quelle est la position relative de $(C1)$ et $(D_1)$ ; $(C1)$ et $(D_3)$ ? Justifie tes réponses.
    \item Trace le cercle $(C2)$ de centre $J$ et de rayon $4 \text{ cm}$.
    \item Quelle est la position relative de $(C1)$ et $(C2)$ ? Justifie ta réponse.
\end{enumerate}

\exercice :
On donne un carré $ABCD$ de centre $O$ et le point $E$, milieu du côté $[BC]$.
Trace le cercle $(C)$ de centre $O$ passant par $E$.
\vspace{1ex}

Quelle est la position relative de chacune des droites $(AB)$, $(BC)$, $(CD)$ et $(DA)$ par rapport au cercle $(C)$.
\end{document}