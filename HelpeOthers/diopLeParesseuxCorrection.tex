\documentclass[12pt]{article}
\usepackage[utf8]{inputenc}
\usepackage[T1]{fontenc}
\usepackage[french]{babel}
\usepackage{amsmath, amssymb}
\usepackage{enumitem}
\usepackage{geometry}
\geometry{margin=2.5cm}
\usepackage{xcolor} % Required for \textcolor

\begin{document}
\section*{\underline{PROBLEME} \hfill (10pt)}

\subsection*{PARTIE A}

Soit \( h \) la fonction dérivable sur \( \mathbb{R} \) définie par \( h(x) = 1 + e^{2x - 4} \) et \( K = \left[ 1 ; \frac{5}{4} \right] \).

\begin{enumerate}
    \item 
    \begin{enumerate}
        \item[a)] Calcul de \( h'(x) \) :
        \(
        h(x) = 1 + e^{2x - 4} \Rightarrow h'(x) =  \left(1 + e^{2x - 4} \right)' = 2e^{2x - 4}
        \)
        
        Comme \( e^{2x - 4} > 0 \) pour tout \( x \in \mathbb{R} \), on a :
        \(
        h'(x) > 0 \quad \text{pour tout } x \in \mathbb{R}
        \)
        Donc, la fonction \( h \) est **strictement croissante** sur \( \mathbb{R} \). \hfill \textbf{(0,25pt + 0,25pt)}

        \item[b)] Montrons que \( h(K) \subset K \) :

        Comme \( h \) est croissante et \( K = \left[1 ; \frac{5}{4}\right] \), on a :
        \[
        h(K) = \left[ h(1) ; h\left(\frac{5}{4}\right) \right]
        \]

        Calculons :
        \[
        h(1) = 1 + e^{2(1) - 4} = 1 + e^{-2} \approx 1 + 0{,}1353 = 1{,}1353
        \]
        \[
        h\left(\frac{5}{4}\right) = 1 + e^{2 \cdot \frac{5}{4} - 4} = 1 + e^{-1{,}5} \approx 1 + 0{,}2231 = 1{,}2231
        \]

        Donc :
        \[
        h(K) = \left[1{,}1353 ; 1{,}2231\right] \subset \left[1 ; \frac{5}{4}\right]
        \]

        Ainsi, \( h(K) \subset K \). \hfill \textbf{(0,5pt)}
    \end{enumerate}
        \item 
    \begin{enumerate}
        \item[a)] Resoudre \( h(x) = x \)  revient à resoudre \( h(x) - x = 0 \)
        
        On définit la fonction \( \phi(x) = h(x) - x = 1 + e^{2x - 4} - x \).
        
        \textcolor{red}{\textbf{Existance}}
        
        \( \phi \) est continue sur \( K = \left[1 ; \frac{5}{4} \right] \).\\
        Calculons :
        \[
        \phi(1) = 1 + e^{-2} - 1 = e^{-2} > 0 \quad ; \quad \phi\left( \frac{5}{4} \right) = 1 + e^{-1.5} - \frac{5}{4} \approx 1.2231 - 1.25 < 0
        \]

        Donc, \( \phi(1) > 0 \) et \( \phi\left( \frac{5}{4} \right) < 0 \), par le théorème des valeurs intermédiaires, il existe un \( \lambda \in \left] 1 ; \frac{5}{4} \right[ \) tel que \( \phi(\lambda) = 0 \), soit \( h(\lambda) = \lambda \).\\

        \textcolor{red}{\textbf{Unicité}}

        \[
\phi'(x) = h'(x) - 1 = 2e^{2x - 4} - 1
\]

Supposons que \(\phi'(x)<0\)

\(
\begin{aligned}
\phi'(x) < 0 &\Longleftrightarrow 2e^{2x - 4} - 1 < 0 \\
&\Longleftrightarrow e^{2x - 4} < \frac{1}{2} \\
&\Longleftrightarrow 2x - 4 < \ln\left(\frac{1}{2}\right) \\
&\Longleftrightarrow 2x < 4 - \ln(2) \\
&\Longleftrightarrow x < 2 - \frac{\ln(2)}{2}\\
&\Longleftrightarrow x < 1,7
\end{aligned}
\)

Donc si \( x\in ]-\infty ;1,7[ \) alors \(\phi'(x) < 0\)

Comme \( K = \left[1 ; \frac{5}{4} \right] \subset ]-\infty ;1,7[ \) donc \( \forall x \in K\, , \phi'(x) < 0\)

Donc \( \phi'(x) < 0 \) sur \( K \), donc \( \phi \) est strictement décroissante sur \( K \).\\

Or, une fonction continue et strictement monotone sur un intervalle admet **au plus une** racine. Comme on a déjà montré l’existence d’un \( \lambda \), on en déduit que :

\[
\text{L’équation } h(x) = x \text{ admet une \textbf{unique solution} } \lambda \in K.
\] \hfill \textbf{(0,5pt)}

        \item[b)] On a : \( h'(x) = 2e^{2x - 4} \)

Encadrons \( x \in K = \left[1 ; \frac{5}{4} \right] \) :

\begin{equation*}
\begin{aligned}
x \in K &\implies 1 \leq x \leq \frac{5}{4} \\
       &\implies 2 \leq 2x \leq \frac{5}{2} \\
       &\implies -2 \leq 2x - 4 \leq -\frac{3}{2} \\
       &\implies e^{-2} \leq e^{2x - 4} \leq e^{-1.5} \\
       &\implies 2e^{-2} \leq 2e^{2x - 4} \leq 2e^{-1.5} \\
       &\implies 0 \leq 2e^{-2} \leq 2e^{2x - 4} \leq 2e^{-1.5} \leq \frac{1}{2} \\
       &\implies 0 \leq 2e^{2x - 4} \leq \frac{1}{2}
\end{aligned}
\end{equation*}

Donc : \( \forall x \in K, \quad 0 < h'(x) < \frac{1}{2} \) \hfill \textbf{(0,25pt)}
        \item[c)] Soit \( x \in K \), et \( \lambda \in K \) l’unique solution de \( h(\lambda) = \lambda \).

        D’après l’inégalité des accroissements finis (ou le théorème de la moyenne) appliquée à \( h \) sur \( K \), il existe \( c \in [x ; \lambda] \subset K \) tel que :
        \[
        h(x) - h(\lambda) = h(x) - \lambda = h'(c)(x - \lambda)
        \]
        Donc :
        \[
        |h(x) - \lambda| = |h'(c)| \times |x - \lambda| \leq \frac{1}{2} |x - \lambda|
        \]
        \hfill \textbf{(0,25pt)}
    \end{enumerate}
    
    \item[3.a)] Montrons par récurrence que \( \forall n \in \mathbb{N},\ W_n \in K \).

\textcolor{red}{\textbf{Initialisation :}}\\
On a \( W_0 = 1 \in K = \left[1 ; \frac{5}{4} \right] \). L’assertion est vraie au rang \( n = 0 \).

\textcolor{red}{\textbf{Hérédité :}}\\
Supposons que pour un entier \( n \in \mathbb{N} \), on ait \( W_n \in K \).\\
Alors par définition :
\[
W_{n+1} = h(W_n)
\]
Or à la question 1.b), on a démontré que \( h(K) \subset K \).\\
Donc comme \( W_n \in K \), on a \( W_{n+1} \in h(K) \subset K \).\\

\textcolor{red}{\textbf{Conclusion :}}\\
Par le principe de récurrence, on a :
\[
\forall n \in \mathbb{N},\quad W_n \in K
\]
\hfill \textbf{(0,5pt)}

\item[b)] On veut montrer que :
\[
|W_{n+1} - \lambda| \leq \frac{1}{2}|W_n - \lambda| \quad \text{et} \quad |W_n - \lambda| \leq \left( \frac{1}{2} \right)^n, \quad \forall n \in \mathbb{N}
\]

\textcolor{red}{\textbf{1) Inégalité de récurrence :}}\\
On sait que \( W_{n+1} = h(W_n) \) et que \( \lambda \) est l’unique solution de \( h(\lambda) = \lambda \).\\
D’après la question 2.c), on a pour tout \( x \in K \) :
\[
|h(x) - \lambda| \leq \frac{1}{2} |x - \lambda|
\]
Or, à la question 3.a), on a montré que \( \forall n, \ W_n \in K \). Donc :
\[
|W_{n+1} - \lambda| = |h(W_n) - \lambda| \leq \frac{1}{2} |W_n - \lambda|
\]

\textcolor{red}{\textbf{2) Majoration par \( \left( \frac{1}{2} \right)^n \) par récurrence :}}\\

\item[3.b)] On a \( W_{n+1} = h(W_n) \) et \( h(\lambda) = \lambda \).\\
D’après la question 2.c), pour tout \( x \in K \), on a :
\[
|h(x) - \lambda| \leq \frac{1}{2} |x - \lambda|
\]
Or, à la question 3.a), on a montré que \( W_n \in K \) pour tout \( n \in \mathbb{N} \), donc on peut appliquer cette inégalité à chaque itération :

\begin{equation*}
\begin{aligned}
|W_{1} - \lambda| &\leq \frac{1}{2} |W_0 - \lambda| \\
|W_{2} - \lambda| &\leq \frac{1}{2} |W_1 - \lambda| \\
|W_{3} - \lambda| &\leq \frac{1}{2} |W_2 - \lambda| \\
&\ \vdots \\
|W_k - \lambda| &\leq \frac{1}{2} |W_{k-1} - \lambda|
\end{aligned}
\end{equation*}

En multipliant ces inégalités **membre à membre**, on obtient :

\begin{equation*}
|W_k - \lambda| \leq \left( \frac{1}{2} \right)^k |W_0 - \lambda|
\end{equation*}

\[
\forall k \in \mathbb{N},\quad |W_k - \lambda| \leq \left( \frac{1}{2} \right)^k
\]

\hfill \textbf{(0,5pt + 0,25pt)}
\item[c)] D’après la question précédente, on a :
\[
|W_n - \lambda| \leq \left( \frac{1}{2} \right)^n
\]

Or \( \left( \frac{1}{2} \right)^n \to 0 \) quand \( n \to +\infty \), donc :
\[
|W_n - \lambda| \to 0
\quad \text{ce qui équivaut à} \quad
W_n \to \lambda
\quad \text{quand } n \to +\infty
\]

Ainsi, la suite \( (W_n) \) **converge vers le réel \( \lambda \)**, qui est l’unique solution de l’équation \( h(x) = x \) dans l’intervalle \( K \). \hfill \textbf{(0,25pt)}

\end{enumerate}

\subsection*{PARTIE B}

Soit \( f \) la fonction définie par :
\[
f(x) =
\begin{cases}
\ln\left( \left| \dfrac{x - 1}{x + 1} \right| \right) & \text{si } x \in [0 ; +\infty[ \\
\displaystyle \frac{e^x - 1}{e^x + 1} & \text{si } x \in ]-\infty ; 0[
\end{cases}
\]

\begin{enumerate}
    \item Déterminons le domaine de définition \( D_f \) de \( f \).\\

    \textcolor{red}{\textbf{Sur } \( [0 ; +\infty[ \)} : on considère l’expression
    \[
    f(x) = \ln\left( \left| \frac{x - 1}{x + 1} \right| \right)
    \]
    Cette expression est définie si :
    \begin{itemize}
        \item \( x + 1 \neq 0 \Rightarrow x \neq -1 \) (toujours vrai car \( x \geq 0 \))
        \item \( \left| \frac{x - 1}{x + 1} \right| > 0 \Rightarrow \frac{x - 1}{x + 1} \neq 0 \Rightarrow x \neq 1 \)
    \end{itemize}
    Donc sur \( [0 ; +\infty[ \), la fonction est définie sauf en \( x = 1 \).\\

    \textcolor{red}{\textbf{Sur } \( ]-\infty ; 0[ \)} : on considère
    \[
    f(x) = \frac{e^x - 1}{e^x + 1}
    \]
    Cette expression est définie pour tout \( x \in \mathbb{R} \), car le dénominateur \( e^x + 1 > 0 \) pour tout \( x \).\\

    Donc la fonction \( f \) est définie sur :
    \[
    D_f = ]-\infty ; 1[ \cup ]1 ; +\infty[
    \]
    \hfill \textbf{(0,5pt)}
\end{enumerate}
\end{document}