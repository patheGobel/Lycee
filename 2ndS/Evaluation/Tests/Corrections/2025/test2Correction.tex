\documentclass[a4paper,12pt]{article}
\usepackage{graphicx}
\usepackage[a4paper, top=0cm, bottom=2cm, left=2cm, right=2cm]{geometry} % Ajuste les marges
\usepackage{xcolor} % Pour ajouter des couleurs
\usepackage{hyperref} % Pour avoir des r\'ef\'erences color\'ees si n\'ecessaire
\usepackage{eso-pic} % Pour ajouter des \'el\'ements en arri\`ere-plan
\usepackage[french]{babel}
\usepackage[T1]{fontenc}
\usepackage{mathrsfs}
\usepackage{amsmath}
\usepackage{amsfonts}
\usepackage{amssymb}
\usepackage{tkz-tab}
\usepackage{tcolorbox} % Pour encadrer avec fond color\'e
\usepackage{tikz}
\usetikzlibrary{arrows, shapes.geometric, fit}

% D\'efinition de l'encadr\'e adaptatif avec fond jaune
\newtcolorbox{resultbox}{
    colback=red!30, % Fond rouge clair
    colframe=black, % Bordure noire fine
    sharp corners, % Coins nets
    boxrule=0.5pt, % Contour l\'eger
    boxsep=2pt, % Espacement interne
    left=5pt, right=5pt, top=2pt, bottom=2pt, % Marges internes
}
% Commande pour ajouter du texte en arrière-plan
\AddToShipoutPicture{
    \AtTextCenter{%
        \makebox[0pt]{\rotatebox{80}{\textcolor[gray]{0.4}{\fontsize{8cm}{13cm}\selectfont PGB}}}
    }
}
\begin{document}

\hrule % Barre horizontale
% En-t\^ete
\begin{center}
    {\Large \textbf{Correction du Test 2}}
\end{center}
\hrule

\[
    A(x) = 7x^2 - 4x - 3
\]

\begin{enumerate}
\item \textbf{Calcul du discriminant} 
 
Le discriminant \(\Delta\) est donné par :

\[
\begin{aligned}
 \Delta &= b^2 - 4ac \\
        &= (-4)^2 - 4 \times 7 \times (-3) \\
        &= 16 + 84 \\
        &= 100
\end{aligned}
\]

Ainsi, le discriminant est : 

\begin{resultbox}
    \[
    \mathbf{\Delta = 100}
    \]
\end{resultbox}
    
    \item \textbf{Forme canonique :} \newline
    La forme canonique d'un trinome est :
    
$
\begin{aligned}
    A(x)&= a \left[ \left(x + \frac{b}{2a}\right)^2 - \frac{b^{2}-4ac}{4a^{2}}\right]\\
				&= 7 \left[ \left( x + \frac{-4}{2 \times 7} \right)^2 - \frac{(-4)^2 - 4 \times 7 \times (-3)}{4 \times 7^2} \right] \\
         &= 7 \left[ \left( x - \frac{2}{7} \right)^2 - \frac{16 + 84}{4 \times 49} \right] \\
         &= 7 \left[ \left( x - \frac{2}{7} \right)^2 - \frac{100}{196} \right] \\
         &= 7 \left[ \left( x - \frac{2}{7} \right)^2 - \frac{25}{49} \right]
\end{aligned}
$

Ainsi, la forme canonique est :

\begin{resultbox}
    \[
    \mathbf{A(x) = 7 \left[\left( x - \frac{2}{7} \right)^2 - \frac{25}{49}\right]}
    \]
\end{resultbox}
    
\item \textbf{Factorisation du trinôme}  

Comme \(\Delta > 0\), le trinôme admet deux racines réelles distinctes \( x_1 \) et \( x_2 \) :

Les racines sont donc données par :

\[
\begin{aligned}
        x_1 &= \frac{-b - \sqrt{\Delta}}{2a} \\
            &= \frac{-(-4) - \sqrt{100}}{2 \times 7} \\
            &= \frac{4 - 10}{14} \\
            &= \frac{-6}{14} \\
            &= -\frac{3}{7}
\end{aligned}
\]

\begin{resultbox}
    \[
    \mathbf{x_1 = -\frac{3}{7}}
    \]
\end{resultbox}

\[
\begin{aligned}
        x_2 &= \frac{-b + \sqrt{\Delta}}{2a} \\
            &= \frac{-(-4) + \sqrt{100}}{2 \times 7} \\
            &= \frac{4 + 10}{14} \\
            &= \frac{14}{14} \\
            &= 1
\end{aligned}
\]

\begin{resultbox}
    \[
    \mathbf{x_2 = 1}
    \]
\end{resultbox}

La factorisation est donc :

\begin{resultbox}
    \[
    \mathbf{A(x) = 7 \left( x - 1 \right) \left( x + \frac{3}{7} \right)}
    \]
\end{resultbox}
\end{enumerate}

\end{document}
