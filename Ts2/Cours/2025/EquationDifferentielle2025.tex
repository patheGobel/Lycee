\documentclass[12pt]{article}
\usepackage[utf8]{inputenc}
\usepackage[T1]{fontenc}
\usepackage[french]{babel}
\usepackage{amsmath, amssymb}
\usepackage{xcolor}
\usepackage{tcolorbox} % Pour les encadrés
\usepackage{mathrsfs}  % Pour une jolie écriture mathématique si besoin
\usepackage{lmodern}   % Police plus moderne

\begin{document}

\section*{\textcolor{blue}{Chapitre III} \hfill \textcolor{red}{Équation Différentielle (\(\neq\))}}

\subsection*{\underline{\textcolor{red}{I°) Équation du type :}} \quad \( y' + ay = 0 \quad (a \in \mathbb{R}) \)}

\subsubsection*{\textcolor{red}{1/ Définition :}}

Toute équation \((E):\quad y' + ay = 0 \quad (a \in \mathbb{R})\) est appelée \\
\textcolor{red}{équation différentielle linéaire du premier ordre sans second membre.}

\vspace{0.5cm}

\textbf{\textcolor{red}{Exemple :}} \\
\((E_1):\quad y' + 2y = 0\) est une équation différentielle linéaire du premier ordre sans second membre.

\subsection*{\underline{\textcolor{red}{I-2) Résolution}}}

On considère l'équation différentielle \((E) :\quad y' + ay = 0 \quad (a \in \mathbb{R})\)

\vspace{0.3cm}

L'\textbf{ensemble des solutions} de \((E)\) est l'\textbf{ensemble des fonctions dérivables sur \(\mathbb{R}\)} vérifiant :
\begin{center}
    \begin{tcolorbox}[colback=white, colframe=red, sharp corners=southwest, boxrule=0.7pt]
        \( f'(x) + af(x) = 0 \)
    \end{tcolorbox}
\end{center}

\vspace{0.3cm}

Soit \((E):\quad y' + ay = 0 \quad (a \in \mathbb{R})\)

\begin{itemize}
    \item La fonction nulle est solution de \((E)\)
    \item \( y \neq 0 \) avec \( y \) solution de \((E)\) sur \(\mathbb{R}\)
\end{itemize}

\[
y' + ay = 0 \quad \Leftrightarrow \quad y' = -ay
\]

\[
\frac{y'}{y} = -a
\]

\vspace{0.3cm}

On intègre :

\begin{align*}
\int \frac{y'}{y} \, dx &= \int -a \, dx \\
\ln |y| &= -ax + c \quad \text{(avec \( c \in \mathbb{R} \))} \\
|y| &= e^{-ax + c} = e^{c} \cdot e^{-ax}
\end{align*}

D’où :

\[
y_1 = ke^{-ax} \quad \text{avec } k = e^c \in \mathbb{R}^{*}
\]

\textbf{Conclusion :} L'ensemble des solutions est :
\[
\boxed{y(x) = ke^{-ax},\quad k \in \mathbb{R}}
\]

\subsection*{\underline{\textcolor{red}{Réciproquement :}}}

Soit \textcolor{blue}{\( y \)} une \textcolor{blue}{fonction de \((\mathbb{R})\)} et \textcolor{red}{\( g(x) = y(x)e^{ax} \)} \\
Si \( g(x) \) est dérivable sur \( \mathbb{R} \), alors :

\[
g(x) = y(x) e^{ax}
\]

On a :
\[
g'(x) = y'(x)e^{ax} + ay(x)e^{ax} = e^{ax} \left( y'(x) + ay(x) \right)
\]

Comme \( y'(x) + ay(x) = 0 \), alors :
\[
g'(x) = e^{ax} \cdot 0 = 0
\Rightarrow g'(x) = 0
\]

Donc \( g \) est une constante :
\[
\text{Il existe } k \in \mathbb{R} \text{ tel que } g(x) = k
\Rightarrow y(x)e^{ax} = k \Rightarrow y(x) = ke^{-ax}
\]

\vspace{0.5cm}

\textbf{Conclusion :}

\begin{tcolorbox}[colback=white, colframe=red, sharp corners=southwest, boxrule=1pt]
\[
\boxed{(E):\quad y' + ay = 0 \quad \Rightarrow \quad y(x) = ke^{-ax}}
\]
\end{tcolorbox}
\end{document}
