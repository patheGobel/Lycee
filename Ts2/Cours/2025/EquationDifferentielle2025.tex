\documentclass[12pt,a4paper]{article}
\usepackage[utf8]{inputenc} % inutile avec XeLaTeX/LuaLaTeX
\usepackage[T1]{fontenc}
\usepackage{amsmath,amssymb,mathrsfs,tikz,times,pifont}
\usepackage{enumitem}
\usepackage{multicol}
\usepackage{lmodern}
\newcommand\circitem[1]{%
\tikz[baseline=(char.base)]{
\node[circle,draw=gray, fill=red!55,
minimum size=1.2em,inner sep=0] (char) {#1};}}
\newcommand\boxitem[1]{%
\tikz[baseline=(char.base)]{
\node[fill=cyan,
minimum size=1.2em,inner sep=0] (char) {#1};}}
\setlist[enumerate,1]{label=\protect\circitem{\arabic*}}
\setlist[enumerate,2]{label=\protect\boxitem{\alph*}}
\everymath{\displaystyle}
\usepackage[left=1cm,right=1cm,top=1cm,bottom=1.7cm]{geometry}
\usepackage[colorlinks=true, linkcolor=blue, urlcolor=blue, citecolor=blue]{hyperref}
\usepackage{array,multirow}
\usepackage[most]{tcolorbox}
\usepackage{varwidth}
\usepackage{float}
\tcbuselibrary{skins,hooks}
\usetikzlibrary{patterns}

\newtcolorbox{exa}[2][]{enhanced,breakable,before skip=2mm,after skip=5mm,
colback=yellow!20!white,colframe=black!20!blue,boxrule=0.5mm,
attach boxed title to top left ={xshift=0.6cm,yshift*=1mm-\tcboxedtitleheight},
fonttitle=\bfseries,
title={#2},#1,
boxed title style={frame code={
\path[fill=tcbcolback!30!black]
([yshift=-1mm,xshift=-1mm]frame.north west)
arc[start angle=0,end angle=180,radius=1mm]
([yshift=-1mm,xshift=1mm]frame.north east)
arc[start angle=180,end angle=0,radius=1mm];
\path[left color=tcbcolback!60!black,right color = tcbcolback!60!black,
middle color = tcbcolback!80!black]
([xshift=-2mm]frame.north west) -- ([xshift=2mm]frame.north east)
[rounded corners=1mm]-- ([xshift=1mm,yshift=-1mm]frame.north east)
-- (frame.south east) -- (frame.south west)
-- ([xshift=-1mm,yshift=-1mm]frame.north west)
[sharp corners]-- cycle;
},interior engine=empty,
},interior style={top color=yellow!5}}

\usepackage{fancyhdr}
\usepackage{eso-pic}
\usepackage{tkz-tab}
\AddToShipoutPicture{
    \AtTextCenter{%
        \makebox[0pt]{\rotatebox{80}{\textcolor[gray]{0.7}{\fontsize{5cm}{5cm}\selectfont PGB}}}
    }
}

\usepackage{verbatim}

\usepackage{color,soul}

\usepackage{amsmath}
\usepackage{amsfonts}
\usepackage{amssymb}
\usepackage{systeme}
\usepackage{tkz-tab}
\usepackage{tikz}
\usetikzlibrary{arrows}
\newcounter{exemple} % Compteur pour les questions

% Définir la commande pour afficher une question numérotée
\newcommand{\exemple}{%
  \refstepcounter{exemple}%
  \textbf{\textcolor{green}{Exemple \theexemple :}} \ignorespaces
}
%---------------------------------------
\definecolor{myorange}{rgb}{1.0, 0.8, 0.0}

% Définir un compteur pour les exercices d'application
\newcounter{exerciceapp}

% Définir la commande pour afficher un exercice d'application numéroté
\newcommand{\exerciceapp}{%
  \refstepcounter{exerciceapp}%
  \textbf{\textcolor{myorange}{Exercice d'application \theexerciceapp :}} \ignorespaces
}
%--------------------------------------
% Définir un compteur pour les exercices d'application
\newcounter{correction}

% Définir la commande pour afficher un correction exercice d'application numéroté
\newcommand{\correction}{%
  \refstepcounter{correction}%
  \textbf{\textcolor{myorange}{Correction \thecorrection :}} \ignorespaces
}
%--------------------------------------
% Commande pour ajouter du texte en arrière-plan
\usepackage{fancyhdr}
\usepackage{eso-pic}
%\usepackage{tkz-tab}
\AddToShipoutPicture{
    \AtTextCenter{%
        \makebox[0pt]{\rotatebox{80}{\textcolor[gray]{0.7}{\fontsize{5cm}{5cm}\selectfont PGB}}}
    }
}
%This command takes a colour as an optional argument; the default colour is black.
\usetikzlibrary{shapes.geometric,fit}
\newcommand{\myul}[2][black]{\setulcolor{#1}\ul{#2}\setulcolor{black}}
\newcommand\tab[1][1cm]{\hspace*{#1}}

\begin{document}
% En-tête personnalisée
\begin{center}
    \Large\textbf{\underline{\textcolor{red}{Équation Différentielle (\(\neq\))}}}\\[-0.1cm]
    \normalsize\textbf{Prof : M. BA} \hfill \textbf{Classe : TS2}\\[-0.1cm]
    \textbf{Année scolaire : 2024 -- 2025}
\end{center}

%\usepackage{xcolor}
%\usepackage{tcolorbox} % Pour les encadrés
%\usepackage{mathrsfs}  % Pour une jolie écriture mathématique si besoin
%\usepackage{lmodern}   % Police plus moderne


%\section*{\textcolor{blue}{Chapitre III} \hfill \textcolor{red}{Équation Différentielle (\(\neq\))}}

\subsection*{\underline{\textcolor{red}{I°) Équation du type :}} \quad \( y' + ay = 0 \quad (a \in \mathbb{R}) \)}

\subsubsection*{\textcolor{red}{1/ Définition :}}

Toute équation \((E):\quad y' + ay = 0 \quad (a \in \mathbb{R})\) est appelée \\
\textcolor{red}{équation différentielle linéaire du premier ordre sans second membre.}

\vspace{0.5cm}

\textbf{\textcolor{red}{Exemple :}} \\
\((E_1):\quad y' + 2y = 0\) est une équation différentielle linéaire du premier ordre sans second membre.

\subsection*{\underline{\textcolor{red}{I-2) Résolution}}}

On considère l'équation différentielle \((E) :\quad y' + ay = 0 \quad (a \in \mathbb{R})\)

\vspace{0.3cm}

L'\textbf{ensemble des solutions} de \((E)\) est l'\textbf{ensemble des fonctions dérivables sur \(\mathbb{R}\)} vérifiant :
\begin{center}
    \begin{tcolorbox}[colback=white, colframe=red, sharp corners=southwest, boxrule=0.7pt]
        \( f'(x) + af(x) = 0 \)
    \end{tcolorbox}
\end{center}

\vspace{0.3cm}

Soit \((E):\quad y' + ay = 0 \quad (a \in \mathbb{R})\)

\begin{itemize}
    \item La fonction nulle est solution de \((E)\)
    \item \( y \neq 0 \) avec \( y \) solution de \((E)\) sur \(\mathbb{R}\)
\end{itemize}

\[
y' + ay = 0 \quad \Leftrightarrow \quad y' = -ay
\]

\[
\frac{y'}{y} = -a
\]

\vspace{0.3cm}

On intègre :

\begin{align*}
\int \frac{y'}{y} \, dx &= \int -a \, dx \\
\ln |y| &= -ax + c \quad \text{(avec \( c \in \mathbb{R} \))} \\
|y| &= e^{-ax + c} = e^{c} \cdot e^{-ax}
\end{align*}

D’où :

\[
y_1 = ke^{-ax} \quad \text{avec } k = e^c \in \mathbb{R}^{*}
\]

\textbf{Conclusion :} L'ensemble des solutions est :
\[
\boxed{y(x) = ke^{-ax},\quad k \in \mathbb{R}}
\]

\subsection*{\underline{\textcolor{red}{Réciproquement :}}}

Soit \textcolor{blue}{\( y \)} une \textcolor{blue}{fonction de \((\mathbb{R})\)} et \textcolor{red}{\( g(x) = y(x)e^{ax} \)} \\
Si \( g(x) \) est dérivable sur \( \mathbb{R} \), alors :

\[
g(x) = y(x) e^{ax}
\]

On a :
\[
g'(x) = y'(x)e^{ax} + ay(x)e^{ax} = e^{ax} \left( y'(x) + ay(x) \right)
\]

Comme \( y'(x) + ay(x) = 0 \), alors :
\[
g'(x) = e^{ax} \cdot 0 = 0
\Rightarrow g'(x) = 0
\]

Donc \( g \) est une constante :
\[
\text{Il existe } k \in \mathbb{R} \text{ tel que } g(x) = k
\Rightarrow y(x)e^{ax} = k \Rightarrow y(x) = ke^{-ax}
\]

\vspace{0.5cm}

\textbf{Conclusion :}

\begin{tcolorbox}[colback=white, colframe=red, sharp corners=southwest, boxrule=1pt]
\[
\boxed{(E):\quad y' + ay = 0 \quad \Rightarrow \quad y(x) = ke^{-ax}}
\]
\end{tcolorbox}
\end{document}
