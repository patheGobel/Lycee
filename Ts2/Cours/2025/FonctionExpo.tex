\documentclass[12pt,a4paper]{article}
\usepackage[utf8]{inputenc} % inutile avec XeLaTeX/LuaLaTeX
\usepackage[T1]{fontenc}
\usepackage{amsmath,amssymb,mathrsfs,tikz,times,pifont}
\usepackage{enumitem}
\usepackage{multicol}
\usepackage{lmodern}
\newcommand\circitem[1]{%
\tikz[baseline=(char.base)]{
\node[circle,draw=gray, fill=red!55,
minimum size=1.2em,inner sep=0] (char) {#1};}}
\newcommand\boxitem[1]{%
\tikz[baseline=(char.base)]{
\node[fill=cyan,
minimum size=1.2em,inner sep=0] (char) {#1};}}
\setlist[enumerate,1]{label=\protect\circitem{\arabic*}}
\setlist[enumerate,2]{label=\protect\boxitem{\alph*}}
\everymath{\displaystyle}
\usepackage[left=1cm,right=1cm,top=1cm,bottom=1.7cm]{geometry}
\usepackage[colorlinks=true, linkcolor=blue, urlcolor=blue, citecolor=blue]{hyperref}
\usepackage{array,multirow}
\usepackage[most]{tcolorbox}
\usepackage{varwidth}
\usepackage{float}
\tcbuselibrary{skins,hooks}
\usetikzlibrary{patterns}

\newtcolorbox{exa}[2][]{enhanced,breakable,before skip=2mm,after skip=5mm,
colback=yellow!20!white,colframe=black!20!blue,boxrule=0.5mm,
attach boxed title to top left ={xshift=0.6cm,yshift*=1mm-\tcboxedtitleheight},
fonttitle=\bfseries,
title={#2},#1,
boxed title style={frame code={
\path[fill=tcbcolback!30!black]
([yshift=-1mm,xshift=-1mm]frame.north west)
arc[start angle=0,end angle=180,radius=1mm]
([yshift=-1mm,xshift=1mm]frame.north east)
arc[start angle=180,end angle=0,radius=1mm];
\path[left color=tcbcolback!60!black,right color = tcbcolback!60!black,
middle color = tcbcolback!80!black]
([xshift=-2mm]frame.north west) -- ([xshift=2mm]frame.north east)
[rounded corners=1mm]-- ([xshift=1mm,yshift=-1mm]frame.north east)
-- (frame.south east) -- (frame.south west)
-- ([xshift=-1mm,yshift=-1mm]frame.north west)
[sharp corners]-- cycle;
},interior engine=empty,
},interior style={top color=yellow!5}}

\usepackage{fancyhdr}
\usepackage{eso-pic}
\usepackage{tkz-tab}
\AddToShipoutPicture{
    \AtTextCenter{%
        \makebox[0pt]{\rotatebox{80}{\textcolor[gray]{0.7}{\fontsize{5cm}{5cm}\selectfont PGB}}}
    }
}

\usepackage{verbatim}

\usepackage{color,soul}

\usepackage{amsmath}
\usepackage{amsfonts}
\usepackage{amssymb}
\usepackage{systeme}
\usepackage{tkz-tab}
\usepackage{tikz}
\usetikzlibrary{arrows}
\newcounter{exemple} % Compteur pour les questions

% Définir la commande pour afficher une question numérotée
\newcommand{\exemple}{%
  \refstepcounter{exemple}%
  \textbf{\textcolor{green}{Exemple \theexemple :}} \ignorespaces
}
%---------------------------------------
\definecolor{myorange}{rgb}{1.0, 0.8, 0.0}

% Définir un compteur pour les exercices d'application
\newcounter{exerciceapp}

% Définir la commande pour afficher un exercice d'application numéroté
\newcommand{\exerciceapp}{%
  \refstepcounter{exerciceapp}%
  \textbf{\textcolor{myorange}{Exercice d'application \theexerciceapp :}} \ignorespaces
}
%--------------------------------------
% Définir un compteur pour les exercices d'application
\newcounter{correction}

% Définir la commande pour afficher un correction exercice d'application numéroté
\newcommand{\correction}{%
  \refstepcounter{correction}%
  \textbf{\textcolor{myorange}{Correction \thecorrection :}} \ignorespaces
}
%--------------------------------------
% Commande pour ajouter du texte en arrière-plan
\usepackage{fancyhdr}
\usepackage{eso-pic}
%\usepackage{tkz-tab}
\AddToShipoutPicture{
    \AtTextCenter{%
        \makebox[0pt]{\rotatebox{80}{\textcolor[gray]{0.7}{\fontsize{5cm}{5cm}\selectfont PGB}}}
    }
}
%This command takes a colour as an optional argument; the default colour is black.
\usetikzlibrary{shapes.geometric,fit}
\newcommand{\myul}[2][black]{\setulcolor{#1}\ul{#2}\setulcolor{black}}
\newcommand\tab[1][1cm]{\hspace*{#1}}

\begin{document}
% En-tête personnalisée
\begin{center}
    \Large\textbf{\underline{\textcolor{red}{Fonction Exponentielle}}}\\[-0.1cm]
    \normalsize\textbf{Prof : M. BA} \hfill \textbf{Classe : TS2}\\[-0.1cm]
    \textbf{Année scolaire : 2024 -- 2025}
\end{center}

\subsection*{\underline{\textbf{\textcolor{red}{1.Définition :}}}}
La fonction \( f(x) = \ln(x) \) est continue et strictement croissante sur \( ]0; +\infty[ \), donc c'est une bijection de \( ]0; +\infty[ \) sur \( \mathbb{R} \). Ainsi, \( f \) admet une bijection réciproque \( f^{-1} \), qui est continue et strictement croissante de \( \mathbb{R} \) vers \( ]0; +\infty[ \). 

Cette fonction réciproque est appelée **fonction exponentielle** et est notée :
\[
\forall x \in \mathbb{R}, \quad \exp(x) = e^x
\]
Elle est caractérisée par la relation :
\[
e^x = y \quad \Leftrightarrow \quad x = \ln(y).
\]
\subsection*{\underline{\textbf{\textcolor{red}{2.Conséquences de la définition}}}}

\begin{enumerate}[label=(\alph*)]
    \item \textbf{Image et ensemble de définition :}
    \begin{itemize}
        \item \( e^x \) est définie sur \( \mathbb{R} \) et prend ses valeurs dans \( ]0; +\infty[ \).
        \item \( e^x \) est toujours strictement positive : \( e^x > 0 \) pour tout \( x \in \mathbb{R} \).
    \end{itemize}

    \item \textbf{Lien avec le logarithme :}
    \begin{itemize}
        \item \( e^{\ln(x)} = x \) pour tout \( x > 0 \).
        \item \( \ln(e^x) = x \) pour tout \( x \in \mathbb{R} \).
    \end{itemize}

    \item \textbf{Comportement aux valeurs remarquables :}
    \begin{itemize}
        \item \( e^0 = 1 \).
        \item \( e^1 = e \approx 2.718 \).
    \end{itemize}

    \item \textbf{Monotonie de \( e^x \) :}
    \begin{itemize}
        \item La fonction exponentielle est strictement \textbf{croissante} sur \( \mathbb{R} \), car sa dérivée est toujours positive.
    \end{itemize}
\end{enumerate}

\subsection*{\underline{\textbf{\textcolor{red}{3.Propriétés fondamentales de la fonction exponentielle}}}}
\textbf{\textcolor{red}{Propriété fondamentale}}\\

Pour tout réel a et b, on a: $e^{a+b}=e^{a} \times e^{b}$.

\textbf{\textcolor{red}{ Autres propriétés :}}\\
\begin{itemize}
    \item \textbf{} \(e^{-a}=\frac{1}{e^{a}}\)
    \item \textbf{} \(e^{a-b}=\frac{e^{a}}{e^{b}}\)
    \item \textbf{} \(e^{ra}=(e^{a})^{r}\)
    \item \textbf{} \(e^{a}=e^{b} \Leftrightarrow a=b\)
    \item \textbf{} \(e^{a}<e^{b} \Leftrightarrow a<b\)
\end{itemize}

\subsection*{\textbf{\textcolor{red}{4. Limites }}}

\begin{enumerate}[label=(\alph*)]
    \item \textbf{Les limites aux bornes de l'ensemble de définition de \( e^x \):}
    \[
    \lim_{x \to -\infty} e^x = 0
    \]
    \[
    \lim_{x \to +\infty} e^x = +\infty
    \]

    \item \textbf{Limites Usuelles :}

\end{enumerate}

\(
\lim\limits_{x \to +\infty} \frac{e^x}{x} = +\infty
\)

$\lim\limits_{x \to -\infty}xe^{x}=0$

$\lim\limits_{x \to 0}\frac{e^{x}-1}{x}=1$

\underline{\textbf{\textcolor{red}{Preuve de quelques limites}}}\\

\underline{\textbf{\textcolor{red}{Exercice d'application}}}\\
Déterminer les limites suivantes:\\
Calculer les limites suivantes.\\
a)\( \lim\limits_{x \to +\infty}\frac{3e^{x}-2}{5e^{x}+3}\) ; b)\(\lim\limits_{x \to -\infty}\frac{\ln(1+e^{x})}{e^{x}}\)
c)\( \lim\limits_{x \to +\infty}(x-e^{x})\) ; d)\( \lim\limits_{x \to +\infty}\frac{\sin2x}{1-e^{x}}\)
\subsection*{\underline{\textbf{\textcolor{red}{6.Limites des composées avec exp}}}}
\underline{\textbf{\textcolor{red}{Propriété}}}\\
Soit U une fonction dérivablesur un intervalle I de $\mathbb{R}$.\\
La fonction $\exp\circ u $ est dérivable sur I  et on a: $(\exp\circ u )'=u' \times \exp\circ u$\\
\underline{\textbf{\textcolor{red}{NB}}}\\
La fonction $\exp\circ u $ est généralement notée $e^{u}$ ; sa dérivée est alors $u'e^{u}$.\\
\underline{\textbf{\textcolor{red}{Exemple}}}\\
Calcule la limite suivante\\

\underline{\textbf{\textcolor{red}{Solution}}}\\
\subsection*{\underline{\textbf{\textcolor{red}{7.Dérivée}}}}
Soit u et v deux fonctions strictement positives \\

\underline{\textbf{\textcolor{red}{Exemple}}}\\
Déterminer les limites suivantes:\\
$\bullet$ La fonction $x \longmapsto e^{-x^{2}+x}$ est dérivable sur $\mathbb{R}$ et sa dérivée est la fonction\\
$\bullet$ La fonction $x \longmapsto e^{\cos x}$ est dérivable sur $\mathbb{R}$ et sa dérivée est  la fonction\\
$\bullet$ La fonction $x \longmapsto e^{\frac{1}{x}}$ est dérivable sur $\mathbb{R}^{*}$ et sa dérivée est  la fonction\\
\subsection*{\underline{\textbf{\textcolor{red}{8.Croissance Comparée de $\ln x$ $e^{x}$ $x^{\alpha}$}}}}
$\lim_{x \to +\infty}\frac{e^{x}}{x^{\alpha}}=+\infty$\\
$\lim_{x \to +\infty}x^{\alpha}e^{-x}=0$\\
\underline{\textbf{\textcolor{red}{Remarque}}}\\
\underline{\textbf{\textcolor{red}{Exemple}}}\\
Détermine: $\lim_{x \to +\infty}\frac{e^{x}}{\ln(x^{2}+1)}$
\subsection*{\underline{\textbf{\textcolor{red}{9.Equation système et Inequation avec exp}}}}
\underline{\textbf{\textcolor{red}{a°)Equation}}}\\
\underline{\textbf{\textcolor{red}{Exemple}}}\\
Résoudre dans $\mathbb{R}$ les équations suivantes\\
a)$e^{x}=-1$;\\  b)$e^{x+1}=3$;\\ c)$e^{x^{2}}=e^{x+2}$;\\ d)$(e^{x}-2)(e^{-x}+1)$\\
\underline{\textbf{\textcolor{red}{b°)Système d'inéquations avec $\exp$:}}}\\
\[
\begin{cases}
4e^{x}-3e^{y} = 9 \\
2e^{x}+e^{y} = 7
\end{cases}
\]

\[
\begin{cases}
e^{x}e^{y} = 10 \\
e^{x-y} = \frac{2}{5}
\end{cases}
\]

\[
\begin{cases}
e^{2x}-7e^{y+1} = -10 \\
x-y = 1
\end{cases}
\]
\underline{\textbf{\textcolor{red}{c°)Inéquations avec $\exp$:}}}\\
a)$e^{-x}\geq2$\\
b)$e^{x^{2}-3}\leq e^{2x}$\\
c)$2e^{2x}-5e^{x}+2>0$
\section*{\underline{\textbf{\textcolor{red}{10.Etude le fonction exp}}}}
Soit $f(x)=exp(x)$ le domaine \\
Le Domaine $D_{f}$ \\
$D_{f}=\mathbb{R}$\\
$\bigotimes$ Limites aux bornes de $D_{f}$\\
\underline{En $-\infty $}\\
$\lim_{x \to -\infty} e^{x}=0$\\
\underline{En $+\infty$}\\
$\lim_{{x \to +\infty}} e^{x}=+\infty$\\
$\bigotimes$ La dérivée de f\\
$f'(x)=e^{x}$\\
$\forall x \in \mathbb{R}$, $f'(x)>0$, donc f est croissante sur $]0;+\infty[$\\
$\bigotimes$ Tableau de variation\\
%Tableau de Variation
\definecolor{cqcqcq}{rgb}{0.7529411764705882,0.7529411764705882,0.7529411764705882}
\begin{tikzpicture}[line cap=round,line join=round,>=triangle 45,x=1cm,y=1cm]
\draw [color=cqcqcq,, xstep=1cm,ystep=1cm] (-7,-10) grid (-22,17);
\clip(-22,-5) rectangle (12,10);
\draw [line width=2pt] (-23,8)-- (-7,8); %première ligne A(-22,8)---B(-7,8)
\draw [line width=2pt] (-22,6)-- (-7,6); %deuxième ligne
\draw [line width=2pt] (-22,4)-- (-7,4); %troisième ligne
\draw [line width=2pt] (-22,-2)-- (-7,-2);%dernière ligne
\draw [line width=2pt] (-22,-2)-- (-22,8); %première colonne
\draw [line width=2pt] (-19,8)-- (-19,-2); %deuxième colone
\draw [line width=2pt] (-7,8)-- (-7,-2); %troisième colonne
\draw (-21,1.5) node[anchor=north west] {$f(x)$};
\draw (-21,5.5) node[anchor=north west] {$f'(x)$};
\draw (-21,7) node[anchor=north west] {$x$};
\draw (-19,7) node[anchor=north west] {$-\infty$};
\draw (-8,7) node[anchor=north west] {$+\infty$};
%Asymptote verticale
%\draw [line width=2pt] (-18.7,6)-- (-18.7,-2);
%¨\draw [line width=2pt] (-18.79,6)-- (-18.79,-2);
%signe de la dérivé
\draw (-13.5,5.3) node[anchor=north west] {$+$};
%zéro de exp
\draw [line width=2pt] (-13,6)-- (-13,1);
\draw (-13.2,6.5) node[anchor=north west] {$0$};
\draw (-13.2,1.2) node[anchor=north west] {$1$};
\draw [->,line width=2pt] (-18,-1) -- (-8,3);
\draw (-18.5,-1) node[anchor=north west] {$0$};
\draw (-8,3.5) node[anchor=north west] {$+\infty$};
\end{tikzpicture}

\begin{tikzpicture}[scale=1.5]
  \draw[->] (-5,0) -- (2,0) node[right] {$x$};
  \draw[->] (0,-0.5) -- (0,2) node[above] {$y$};
  \draw[scale=1,domain=-5:1.5,smooth,variable=\x,blue] plot ({\x},{exp(\x)});
  %\node[right,blue] at (1.5,1.5) {$f(x) = e^x$};
  \draw[scale=1,domain=-1.5:1.5,smooth,variable=\x,blue] plot ({\x},{\x+1});
  
%  \draw[scale=1,domain=-5:1.5,smooth,variable=\x,blue] plot ({\x},{\ln(\x)});
  %\node[right,blue] at (1.5,1.5) {$f(x) = e^x$};
%  \draw[scale=1,domain=-1.5:1.5,smooth,variable=\x,blue] plot ({\x},{\x-1});
\end{tikzpicture}

$C_{f}$ est au-dessous de sa tangente en J; donc $\forall x \in \mathbb{R}, e^{x}>x+1$
\section*{\underline{\textbf{\textcolor{red}{11.Branche infinie de ln}}}}
On a  $\lim_{{x \to +\infty}} e^{x}=+\infty$ et $\lim_{{x \to +\infty}} \frac{e^{x}}{x}=+\infty$\\
Nous avons ainsi une branche parabolique de direction (Oy) au voisinage de +$\infty$.\\
Car $\lim_{{x \to +\infty}} \frac{ln(x)}{x}=0$\\
\section*{\underline{\textbf{\textcolor{red}{12.Application}}}}
\end{document}
