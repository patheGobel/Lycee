\documentclass[12pt,a4paper]{article}
\usepackage{amsmath,amssymb,mathrsfs,tikz,times,pifont}
\usepackage{enumitem}
\newcommand\circitem[1]{%
\tikz[baseline=(char.base)]{
\node[circle,draw=gray, fill=red!55,
minimum size=1.2em,inner sep=0] (char) {#1};}}
\newcommand\boxitem[1]{%
\tikz[baseline=(char.base)]{
\node[fill=cyan,
minimum size=1.2em,inner sep=0] (char) {#1};}}
\setlist[enumerate,1]{label=\protect\circitem{\arabic*}}
\setlist[enumerate,2]{label=\protect\boxitem{\alph*}}
%%%::::::by chnini ameur :::::::%%%
\everymath{\displaystyle}
\usepackage[left=1cm,right=1cm,top=1cm,bottom=1.7cm]{geometry}
\usepackage[colorlinks=true, linkcolor=blue, urlcolor=blue, citecolor=blue]{hyperref}
\usepackage{array,multirow}
\usepackage[most]{tcolorbox}
\usepackage{varwidth}
\usepackage{float} %pour utiliser l'option [H] qui force l'image à apparaître exactement à l'endroit où elle est placée dans le code.
\tcbuselibrary{skins,hooks}
\usetikzlibrary{patterns}
%%%::::::by chnini ameur :::::::%%%
\newtcolorbox{exa}[2][]{enhanced,breakable,before skip=2mm,after skip=5mm,
colback=yellow!20!white,colframe=black!20!blue,boxrule=0.5mm,
attach boxed title to top left ={xshift=0.6cm,yshift*=1mm-\tcboxedtitleheight},
fonttitle=\bfseries,
title={#2},#1,
% varwidth boxed title*=-3cm,
boxed title style={frame code={
\path[fill=tcbcolback!30!black]
([yshift=-1mm,xshift=-1mm]frame.north west)
arc[start angle=0,end angle=180,radius=1mm]
([yshift=-1mm,xshift=1mm]frame.north east)
arc[start angle=180,end angle=0,radius=1mm];
\path[left color=tcbcolback!60!black,right color = tcbcolback!60!black,
middle color = tcbcolback!80!black]
([xshift=-2mm]frame.north west) -- ([xshift=2mm]frame.north east)
[rounded corners=1mm]-- ([xshift=1mm,yshift=-1mm]frame.north east)
-- (frame.south east) -- (frame.south west)
-- ([xshift=-1mm,yshift=-1mm]frame.north west)
[sharp corners]-- cycle;
},interior engine=empty,
},interior style={top color=yellow!5}}
%%%%%%%%%%%%%%%%%%%%%%%

\usepackage{fancyhdr}
\usepackage{eso-pic}         % Pour ajouter des éléments en arrière-plan
% Commande pour ajouter du texte en arrière-plan
\usepackage{tkz-tab}
\AddToShipoutPicture{
    \AtTextCenter{%
        \makebox[0pt]{\rotatebox{80}{\textcolor[gray]{0.7}{\fontsize{5cm}{5cm}\selectfont PGB}}}
    }
}
\usepackage{lastpage}
\fancyhf{}
\pagestyle{fancy}
\renewcommand{\footrulewidth}{1pt}
\renewcommand{\headrulewidth}{0pt}
\renewcommand{\footruleskip}{10pt}
\fancyfoot[R]{
\color{blue}\ding{45}\ \textbf{2025}
}
\fancyfoot[L]{
\color{blue}\ding{45}\ \textbf{Prof:M. BA}
}
\cfoot{\bf
\thepage /
\pageref{LastPage}}
\begin{document}
\renewcommand{\arraystretch}{1.5}
\renewcommand{\arrayrulewidth}{1.2pt}
\begin{tikzpicture}[overlay,remember picture]
    \node[draw=blue,line width=1.2pt,fill=purple,text=blue,inner sep=3mm,rounded corners,pattern=dots]at ([yshift=-2.5cm]current page.north) {\begingroup\setlength{\fboxsep}{0pt}\colorbox{white}{\begin{tabular}{|*1{>{\centering \arraybackslash}p{0.28\textwidth}} |*2{>{\centering \arraybackslash}p{0.2\textwidth}|} *1{>{\centering \arraybackslash}p{0.19\textwidth}|} }
                \hline
                \multicolumn{3}{|c|}{$\diamond$$\diamond$$\diamond$\ \textbf{Lycée de Dindéfélo}\ $\diamond$$\diamond$$\diamond$ } & \textbf{A.S. : 2024/2025}                                              \\ \hline
                \textbf{Matière: Mathématiques}                                                                                    & \textbf{Niveau : T}\textbf{S2} & \textbf{Date: 21/02/2025} & \textbf{} \\ \hline
                \multicolumn{4}{|c|}{\parbox[c]{10cm}{\begin{center}
                                                                  \textbf{{\Large\sffamily Td Ln Expo}}
                                                              \end{center}}}                                                                                                        \\ \hline
            \end{tabular}}\endgroup};
\end{tikzpicture}
\vspace{3cm}

\section*{\fbox{\textbf{Exercice 4}}}

\subsection*{\underline{\textbf{Partie A}}}

Soit \( g(x) = 2x \ln(-x) + x + 1 \).

\begin{enumerate}
    \item Déterminer l’ensemble de définition \( D_g \) de \( g \).
    \item Calculer les limites aux bornes de \( D_g \).
    \item Étudier les variations de \( g \).
    \item Calculer \( g(-1) \) puis en déduire le signe de \( g(x) \).
\end{enumerate}

\subsection*{\underline{\textbf{Partie B}}}

On considère la fonction \( f \) définie par :

\[
    f(x) =
    \begin{cases}
        x^2 \ln(-x) + x + 1 & \text{si } x < 0 \\
        x \ln(x)^2 + x + 1  & \text{si } x > 0 \\
        1                   & \text{si } x = 0
    \end{cases}
\]

On note \( (C_f) \) sa courbe représentative dans un repère orthonormé.

\begin{enumerate}
    \item Justifier que \( f \) est définie sur \( \mathbb{R} \).
    \item Étudier la continuité et la dérivabilité de \( f \) en 0. Interprétez graphiquement les résultats.
    \item Donner le domaine de dérivabilité de \( f \) puis montrer que
          \[
              f'(x) =
              \begin{cases}
                  g(x)          & \text{si } x < 0 \\
                  (1 + \ln x)^2 & \text{si } x > 0
              \end{cases}
          \]
    \item Calculer les limites de \( f \) aux bornes de son domaine de définition.
    \item Étudier les branches infinies de \( (C_f) \).
    \item Dresser le tableau de variations de \( f \).
    \item Montrer que dans \( ]-\infty; -1[ \), l’équation \( f(x) = 1 \)

          admet une unique solution \( \alpha \) puis vérifier que \(-1,8 < \alpha < -1,7\).
    \item Construire \( (C_f) \) (unité 2 cm) (on précisera la tangente au point d’abscisse \( e^{-1} \) et on placera le point d’abscisse 1).
\end{enumerate}

\subsection*{\underline{\textbf{Partie C}}}

Soit \( h \) la restriction de \( f \) à \( I =]0; +\infty[ \).

    \begin{enumerate}
        \item Montrer que \( h \) admet une bijection réciproque \( h^{-1} \) définie sur un intervalle \( J \) à préciser.
        \item Étudier la dérivabilité de \( h^{-1} \) sur \( J \).
        \item
              \begin{enumerate}
                  \item Calculer \( h(1) \).
                  \item Calculer \( (h^{-1})'(2) \).
              \end{enumerate}
        \item Construire la courbe de \( h^{-1} \).
    \end{enumerate}

    \section*{\fbox{\textbf{Exercice 5}}}

    \subsection*{\underline{\textbf{Partie A}}}

    Soit \( g \) la fonction définie sur \( ]0; +\infty[ \) par
    \( g(x) = -x + 1 - 2 \ln x \)

    \begin{enumerate}
        \item Étudier les variations de \( g \) puis dresser son tableau de variations.
        \item Calculer \( g(1) \). En déduire le signe de \( g(x) \) sur \( ]0; +\infty[ \).
    \end{enumerate}

    \subsection*{\underline{\textbf{Partie B}}}

    Soit \( f \) la fonction définie par \( f(x) = \frac{x + \ln x}{x^2} \)
    et \( (C) \) sa courbe représentative dans un repère \( (O, \vec{i}, \vec{j}) \).

    \begin{enumerate}
        \item Déterminer le domaine de définition de \( f \).
        \item
              \begin{enumerate}
                  \item Calculer \( \lim\limits_{x \to 0^+} f(x) \). Interpréter graphiquement le résultat.
                  \item Calculer \( \lim\limits_{x \to +\infty} f(x) \). Interpréter graphiquement le résultat.
              \end{enumerate}
        \item
              \begin{enumerate}
                  \item Montrer que pour tout \( x \in ]0; +\infty[ \), \( f'(x) = \frac{g(x)}{x^2} \)
                  \item Dresser le tableau de variations de \( f \).
                  \item Montrer que l’équation \( f(x) = 0 \) admet dans \( ]0; +\infty[ \) une unique solution \( \beta \) et que  \( \beta \in ]0,56; 0,57[. \)
              \end{enumerate}
        \item Construire \( (C) \).
    \end{enumerate}
    \subsection*{\underline{\textbf{Partie C}}}
    Soit \( h \) la fonction définie sur \( ]0; +\infty[ \) par \( h(x) = \frac{1}{x} \) et \( (\Gamma) \) sa courbe.

    \begin{enumerate}
        \item Étudier les positions relatives de \( (C) \) et \( (\Gamma) \).
        \item Construire dans le même repère \( (\Gamma) \).
        \item Soit \( I_{\lambda} \) l’aire en unité d’aires de la partie du plan délimitée par les courbes \( (C) \), \( (\Gamma) \) et les droites d’équations \( x = 1 \) et \( x = \lambda \) où \( \lambda \) est un nombre réel strictement supérieur à 1.
              \begin{enumerate}
                  \item Montrer que \( I_{\lambda} = 1 - \frac{1}{\lambda} (1 + \ln \lambda). \)
                  \item Déterminer \( \lim\limits_{\lambda \to +\infty} I_{\lambda}. \)
              \end{enumerate}
    \end{enumerate}

    \section*{\fbox{\textbf{Exercice 6}}}
    Soit \( f(x) = \frac{\ln x}{1 + x^2} \)

    \subsection*{\underline{\textbf{Partie A}}}

    Soit \( g(x) = 1 + x^2 - 2x^2 \ln x. \)

    \begin{enumerate}
        \item Déterminer \( D_g \) et montrer que si \( 0 < x < 1 \) alors \( g(x) \geq 1 \).
        \item Montrer que \( g \) est strictement décroissante sur \( [1; +\infty[ \).
        \item Calculer \( g(1) \) et \( g(2) \). Montrer qu’il existe un unique réel \( \alpha \) strictement positif tel que \( g(\alpha) = 0 \). Donner un encadrement de \( \alpha \) à \( 10^{-1} \) près.
        \item Donner le signe de \( g(x) \) sur \( ]0; +\infty[ \).
    \end{enumerate}

    \subsection*{\underline{\textbf{Partie B}}}

    \begin{enumerate}
        \item
              \begin{enumerate}
                  \item Calculer \( f'(x) \) et étudier les variations de \( f \).
                  \item Montrer que \( f(\alpha) = \frac{1}{2\alpha^2} \).
                  \item Calculer \( \lim\limits_{x \to 0^+} f(x) \) et \( \lim\limits_{x \to +\infty} f(x) \).
              \end{enumerate}
        \item Donner une équation de la tangente \( (T) \) à \( (C_f) \) au point d’abscisse \( 1 \).
        \item Donner le tableau de variations de \( f \) puis tracer \( (C_f) \).
    \end{enumerate}

    \section*{\fbox{\textbf{Exercice 7}}}

    \subsection*{\underline{\textbf{Partie A}}}

    On considère dans \( ]0; +\infty[ \) la fonction \( g \) donnée par :  \( g(x) = x \ln(x) - 1. \)

    \begin{enumerate}
        \item Dresser le tableau des variations de \( g \).
        \item
              \begin{enumerate}
                  \item Montrer que l’équation \( g(x) = 0 \) admet une solution unique \( \alpha \) sur \( ]0; +\infty[ \).
                  \item Montrer que \( 1,76 < \alpha < 1,77 \).
                  \item En déduire le signe de \( g(x) \) suivant les valeurs de \( x \).
              \end{enumerate}
    \end{enumerate}

    \subsection*{\underline{\textbf{Partie B}}}

    On considère la fonction \( f \) de \( \mathbb{R} \) vers \( \mathbb{R} \) définie par :
    \( f(x) = \frac{1 + x}{1 + \ln(x)}. \)

    \begin{enumerate}
        \item Justifie que \( D_f = ]0; +\infty[ \setminus \left\{ \frac{1}{e} \right\} \).
        \item Calcule les limites de \( f \) aux bornes de \( D_f \).
        \item Étudie les branches infinies à \( (C_f) \).
        \item
              \begin{enumerate}
                  \item Démontre que \( \forall x \in D_f \), \(  f'(x) = \frac{g(x)}{x (1 + \ln(x))^2}. \)
                  \item Donne le signe de \( f'(x) \) suivant les valeurs de \( x \).
                  \item En déduire le sens de variation de \( f \) et dresser son tableau de variation.
              \end{enumerate}
        \item Démontre que \( f(\alpha) = \alpha \).
        \item Trace la courbe \( (C_f) \) et son asymptote. (On prendra \( \alpha = 1,76 \))
    \end{enumerate}

    \section*{\fbox{\textbf{Exercice 8}}}

    \subsection*{\underline{\textbf{Partie A}}}

    Soit  \( g(x) = 2x - (x+1) \ln(x+1). \)

    \begin{enumerate}
        \item Déterminer le domaine de définition \( D_g \) de \( g \).
        \item Calculer les limites aux bornes de \( D_g \).
        \item Dresser le tableau de variations de \( g \).
        \item Montrer que l’équation \( g(x) = 0 \) admet deux solutions \( 0 \) et \( \alpha \) avec
              \(     \alpha \in ]3,9; 4[. \)
        \item Donner le signe de \( g(x) \) suivant les valeurs de \( x \).
    \end{enumerate}

    \subsection*{\underline{\textbf{Partie B}}}

    Soit \( f \) la fonction définie par :
    \( f(x) =
    \begin{cases}
        x + \ln(1 + x^2)          & \text{si } x \leq 0 \\
        \frac{\ln(x+1)}{\sqrt{x}} & \text{si } x > 0
    \end{cases} \)

    \begin{enumerate}
        \item Déterminer le domaine de définition \( D_f \) de \( f \).
        \item Calculer les limites aux bornes de \( D_f \).
        \item Étudier les branches infinies de \( (C_f) \).
        \item Étudier la continuité puis la dérivabilité de \( f \) en 0. Interpréter graphiquement les résultats.
        \item Montrer que \(     f(\alpha) = \frac{2\sqrt{\alpha}}{\alpha + 1}, \)
              puis encadrer \( f(\alpha) \).
        \item Montrer que \( f \) est dérivable sur \( ]-\infty; 0[ \) et  \(     f'(x) = \frac{(x+1)^2}{x^2 + 1}. \)

        \item Montrer que \( f \) est dérivable sur \( ]0; +\infty[ \) et montrer que pour tout \( x > 0 \),  \( f'(x) = \frac{g(x)}{2x(x+1)\sqrt{x}}. \)

        \item Étudier les variations de \( f \).
        \item Dresser le tableau de variations de \( f \).
        \item Calculer \( f(-2) \).
        \item Tracer \( (C_f) \).
    \end{enumerate}

    \subsection*{\underline{\textbf{Partie C}}}

    Soit \( h \) la restriction de \( f \) sur \( ]-\infty; 0] \).

\begin{enumerate}
    \item Montrer que \( h \) réalise une bijection de \( ]-\infty; 0] \) vers un intervalle \( J \) à préciser.
    \item Étudier la dérivabilité de \( h^{-1} \).
    \item Déterminer \( (h^{-1})'(\ln 5 - 2) \).
    \item Tracer \( (C_{h^{-1}}) \).
\end{enumerate}
\section*{\fbox{\textbf{Exercice 9}}}

\subsection*{\underline{\textbf{Partie A}}}

On considère la fonction :

\(
\begin{aligned}
    u : [0; +\infty[ \quad & \longrightarrow \quad \mathbb{R}                                          \\
    x \quad                & \longmapsto \quad \ln \left| \frac{x+1}{x-1} \right| - \frac{2x}{x^2 - 1}
\end{aligned}
\)

\begin{enumerate}
    \item Déterminer l’ensemble de définition de \( u \), calculer \( u(0) \) et \( \lim\limits_{x \to +\infty} u(x) \).
    \item Étudier les variations de \( u \), dresser son tableau de variations. (il n’est pas nécessaire de calculer la limite de \( u \) en 1).
    \item Déduire des résultats précédents que :
          \begin{enumerate}
              \item \( \forall x \in [0;1[ \), \( u(x) \geq 0 \).
              \item \( \forall x \in ]1; +\infty[ \), \( u(x) < 0 \).
          \end{enumerate}
\end{enumerate}

\subsection*{\underline{\textbf{Partie B}}}

Soit \( g \) la fonction définie par :

\(
\begin{aligned}
    g : [0; +\infty[ \quad & \longrightarrow \quad \mathbb{R}                           \\
    x \quad                & \longmapsto \quad x \ln \left| \frac{x+1}{x-1} \right| - 1
\end{aligned}
\)

\begin{enumerate}
    \item Déterminer \( D_g \) le domaine de définition de \( g \), puis étudier la limite de \( g \) en 1.
    \item
          \begin{enumerate}
              \item Vérifier que \( \frac{x+1}{x-1} = 1 + \frac{2}{x-1}. \)

              \item Montrer que \( \lim\limits_{x \to +\infty} \frac{x-1}{2} \ln \left( 1 + \frac{2}{x-1} \right) = 1. \)
              \item En déduire que \(         \lim\limits_{x \to +\infty} g(x) = 1. \)
                    Interpréter graphiquement ce résultat.
              \item Dresser le tableau de variations de \( g \).
              \item Montrer qu’il existe un réel unique \( \alpha \in ]0;1[ \) tel que \( g(\alpha) = 0 \). Donner un encadrement de d’ordre 1 de \( \alpha \).
          \end{enumerate}
    \item Tracer la courbe \( (C_g) \) de \( g \) dans le plan rapporté à un repère orthonormé (unité : \( 2cm \)).
\end{enumerate}

\subsection*{\underline{\textbf{Partie C}}}

Soit la fonction définie par : \( f(x) = (x^2 - 1) \ln \left( \sqrt{\frac{x+1}{1-x}} \right) \)

\begin{enumerate}
    \item Montrer que \( f \) est dérivable sur \( [0;1[ \) et que  \(     f'(x) = g(x) \quad \text{pour tout } x \in [0;1[. \)
    \item Déterminer l’aire du domaine plan limité par la courbe \( (C_g) \) ; l’axe des abscisses ; l’axe des ordonnées et la droite d’équation \( x = \alpha \).
\end{enumerate}
\section*{\fbox{\textbf{Exercice 10}}}

\subsection*{\underline{\textbf{Partie A}}}

Soit la fonction \( g \) définie par  \( g(x) = \frac{1}{x} - 2 - \ln x. \)

\begin{enumerate}
    \item Étudier les variations de \( g \).
    \item Montrer que l’équation \( g(x) = 0 \) admet une unique solution \( \alpha \in ]0; +\infty[ \). Vérifier que
          \[
              0,6 < \alpha < 0,7.
          \]
    \item En déduire le signe de \( g(x) \).
\end{enumerate}

\subsection*{\underline{\textbf{Partie B}}}

Soit \( f \) la fonction définie par :
\(
f(x) =
\begin{cases}
    (1 - x)(1 + \ln x) & \text{si } x \geq 1 \\
    (1 - x) \ln(1 - x) & \text{si } x < 1
\end{cases}
\)

\begin{enumerate}
    \item Déterminer l’ensemble de définition \( D_f \) de \( f \).
    \item Étudier la continuité de \( f \) en 1.
    \item Étudier la dérivabilité de \( f \) en 1.
    \item Déterminer les limites de \( f \) aux bornes de \( D_f \).
    \item Étudier les branches infinies de \( f \).
    \item Dresser le tableau de variations de \( f \).
    \item Tracer \( C_f \) dans un repère orthonormé, unité : \( 2cm \).
    \item Soit \( h \) la restriction de \( f \) sur \( [1; +\infty[ \).
          \begin{enumerate}
              \item Montrer que \( h \) admet une bijection réciproque \( h^{-1} \) dont on précisera l’ensemble de définition et les variations.
              \item Résoudre  \(  h^{-1}(x) = e \)

                    puis calculer  \(  (h^{-1})'(2 - 2e). \)

              \item Tracer la courbe de \( h^{-1} \) dans le même repère.
          \end{enumerate}
\end{enumerate}
\section*{\fbox{\textbf{Exercice 12}}}

On considère la fonction définie sur \( \mathbb{R} \) par  \( f(x) = x \left(1 + e^{2-x} \right). \)

On note \( (C_f) \) sa courbe représentative dans un repère orthonormé \( (O; \vec{i}, \vec{j}) \), unité \( 2cm \).

\begin{enumerate}
    \item Soit \( h \) la fonction définie sur \( \mathbb{R} \) par  \( h(x) = 1 + (1 - x)e^{2-x} \)
          \begin{enumerate}
              \item Étudier les variations de \( h \) (on ne déterminera pas les limites aux bornes de \( D_h \)).
              \item En déduire le signe de \( h(x) \) sur \( \mathbb{R} \).
          \end{enumerate}
    \item
          \begin{enumerate}
              \item Étudier les limites de \( f \) en \( +\infty \) et en \( -\infty \).
              \item Préciser la nature de la branche infinie de \( f \) en \( -\infty \).
              \item Calculer  \( \lim\limits_{x \to +\infty} \left[ f(x) - x \right], \) puis interpréter le résultat obtenu.
              \item Préciser la position de \( (C_f) \) par rapport à la droite d’équation \( (\Delta) : y = x \).
          \end{enumerate}
    \item
          \begin{enumerate}
              \item Dresser le tableau de variations de \( f \).
              \item Montrer que \( f \) admet une bijection réciproque notée \( f^{-1} \) définie sur \( \mathbb{R} \).
              \item \( f^{-1} \) est-elle dérivable en 4 ?
              \item Étudier la position \( (C_f) \) par rapport à sa tangente au point d’abscisse 2.
              \item Construire \( (C_f) \) (on tracera la tangente à \( (C_f) \) au point d’abscisse 2).
              \item Construire \( (C_{f^{-1}}) \) dans le repère précédent.
          \end{enumerate}
\end{enumerate}

\section*{\fbox{\textbf{Exercice 13}}}

Soit \( f \) la fonction définie par
\( f(x) =
\begin{cases}
    (2x - 1) e^{\frac{1}{x}}, & \text{si } x < 0    \\
    x \ln(x+1),               & \text{si } x \geq 0
\end{cases} \)

\subsection*{\underline{\textbf{Partie A}}}

\begin{enumerate}
    \item Déterminer le domaine de définition \( D_f \) de \( f \).
    \item Calculer les limites de \( f \) aux bornes de \( D_f \).
    \item Montrer que la droite \( (D) \) d’équation \( y = 2x + 1 \) est asymptote à \( (C_f) \) en \( -\infty \).
    \item Étudier la branche infinie de \( (C_f) \) en \( +\infty \). La continuité et la dérivabilité de \( f \) en 0. Interpréter graphiquement ce résultat.
\end{enumerate}

\subsection*{\underline{\textbf{Partie B}}}

\begin{enumerate}
    \item Montrer que \( f \) est deux fois dérivable sur \( ]0; +\infty[ \).
    \item Calculer \( f'(x) \) et \( f''(x) \) sur \( ]0; +\infty[ \).
    \item Étudier les variations de \( f'(x) \) puis en déduire le signe de \( f'(x) \) sur \( ]0; +\infty[ \).
    \item Calculer \( f'(x) \) pour tout \( x < 0 \). En déduire le signe de \( f'(x) \).
    \item Dresser le tableau de variations de \( f \).
\end{enumerate}

\subsection*{\underline{\textbf{Partie C}}}

Soit \( g \) la restriction de \( f \) sur \( I = ]-\infty; 0[ \).

\begin{enumerate}
    \item Montrer que \( g \) réalise une bijection de \( I \) vers un intervalle \( J \) à préciser.
    \item Soit \( g^{-1} \) la bijection réciproque de \( g \). Calculer \( g(-1) \) puis \( (g^{-1})'\left(-\frac{3}{e}\right) \).
    \item Construire \( (C_f) \) et \( (C_g) \) dans un même repère.
\end{enumerate}


\section*{\fbox{\textbf{Exercice 14}}}

Soit \( f \) la fonction définie par :

\( f(x) =
\begin{cases}
    e^{-x} (1 + \ln x) - e^{-1},  & \text{si } x \geq 1 \\
    (1 - x) e^{-\frac{1}{1 - x}}, & \text{si } x < 1
\end{cases} \)

\begin{enumerate}
    \item Vérifier que \( f \) est définie sur \( \mathbb{R} \) puis calculer les limites de \( f \) aux bornes de \( D_f \).
    \item Étudier la continuité de \( f \) en 1.
    \item
          \begin{enumerate}
              \item Montrer que pour tout \( x \in ]1;+\infty[ \),
                    \( \frac{f(x) - f(1)}{x - 1} = e^{-x} \left( \frac{1 - e^{x-1}}{x - 1} + \frac{\ln x}{x - 1} \right). \)
              \item Étudier la dérivabilité de \( f \) en 1. Interpréter graphiquement le résultat.
          \end{enumerate}
    \item Soit la fonction \( g \) définie par : \( g(x) = - \ln x + \frac{1}{x} - 1 \)
          \begin{enumerate}
              \item Dresser le tableau de variations de \( g \).
              \item Calculer \( g(1) \) et préciser le signe de \( g(x) \).
              \item Calculer la dérivée de \( f \) sur chaque intervalle où elle est dérivable.
              \item Étudier le signe de \( f'(x) \) et donner le tableau de variations de \( f \).
          \end{enumerate}
    \item
          \begin{enumerate}
              \item Donner la nature de la branche infinie de \( f \) en \( +\infty \).
              \item Montrer que la droite \( (D) : y = -x \) est une asymptote oblique à \( (C_f) \) en \( -\infty \).
              \item Étudier la position de \( (C_f) \) par rapport à \( (D) \) sur \( ]-\infty;1[ \).
          \end{enumerate}
    \item Construire \( (C_f) \).
    \item Soit \( h \) la restriction de \( f \) sur \( ]-\infty;1[ \).
          \begin{enumerate}
              \item Montrer que \( h \) réalise une bijection de \( ]-\infty;1[ \) vers un intervalle \( J \) à préciser.
              \item Calculer \( h(0) \).
                    \( h^{-1} \) est-elle dérivable en \( e^{-1} \) ? Si oui, calculer  \( (h^{-1})'(e^{-1}). \)
              \item Tracer \( (C_{h^{-1}}) \) dans le même repère.
          \end{enumerate}
\end{enumerate}

\section*{\fbox{\textbf{Exercice 15}}} 

\subsection*{\underline{\textbf{Partie A}}}

On considère la fonction \( g \) définie par  \(g(x) = (x+1)^2 e^{-x} - 2.\)

\begin{enumerate}
    \item Dresser le tableau de variations de \( g \).
    \item 
    \begin{enumerate}
        \item Montrer que l’équation \( g(x) = 0 \) admet une solution unique \( \alpha \) dans \( \mathbb{R} \), puis vérifier que \( \alpha \in ]-2; -1[ \).
        \item Donner le signe de \( g(x) \) dans \( \mathbb{R} \).
    \end{enumerate}
    \item Montrer que \( \alpha = -1 - \sqrt{2} e^{\frac{\alpha}{2}} \)
    \item Tracer la courbe \( (C_g) \) de \( g \) dans un repère orthonormé \( (O; \vec{i}, \vec{j}) \), d’unité \( 2cm \).
\end{enumerate}

\subsection*{\underline{\textbf{Partie B}}}

Soit la fonction \( f \) définie par \( f(x) = -1 - \sqrt{2} e^{\frac{x}{2}} \)
si \( x \in I = [-2; -1] \).

\begin{enumerate}
    \item Étudier les variations de \( f \) sur \( I \).
    \item En déduire que si \( x \in I \), alors \( f(x) \leq -1 \).
    \item Montrer que si \( x \in I \), alors \( |f(x) - \alpha| \leq \frac{1}{2} |x - \alpha| \)
    \item Montrer que \( |f(x) - \alpha| \leq \frac{1}{2} |x - \alpha| \)
    pour tout \( x \in I \).
    \item Soit la suite \( (u_n) \) définie par \( u_0 = -2, \quad u_{n+1} = f(u_n) \)
    \begin{enumerate}
        \item Montrer que pour tout \( n \in \mathbb{N} \), \( u_n \in I \).
        \item Montrer que pour tout \( n \in \mathbb{N} \), \( |u_{n+1} - \alpha| \leq \frac{1}{2} |u_n - \alpha|\)
        \item Montrer que pour tout \( n \in \mathbb{N} \), \( |u_n - \alpha| \leq \left( \frac{1}{2} \right)^n |u_0 - \alpha| \)
        \item En déduire que la suite \( (u_n) \) est convergente et préciser sa limite.
        \item Donner une valeur approchée de \( \alpha \) à \( 10^{-3} \) près.
    \end{enumerate}
\end{enumerate}

\section*{\fbox{\textbf{Exercice 16}}} 

On considère la fonction \( f \) définie par :  

\( f(x) =  
\begin{cases}  
x + 1 + \frac{3e^x}{e^x + 2 \ln(x+1)} \quad \text{si } x \leq 0 \\  
x + 2 + \frac{x}{x+1} \quad \text{si } x > 0  
\end{cases} \)  

et \( (C_f) \) sa courbe représentative dans un repère orthonormé \( (O; \vec{i}, \vec{j}) \) d’unité graphique \( 1cm \).

\begin{enumerate}
    \item Etablir que \( f \) est définie sur \( \mathbb{R} \).
    \item 
    \begin{enumerate}
        \item Etudier la continuité de \( f \) en \( 0 \).
        \item Pour \( x < 0 \), montrer que  
        \( \frac{f(x) - 2}{x - 0} = 1 + \frac{2(e^x - 1)}{x} \times \frac{1}{e^x} \) 
         
        En déduire que  
        \( \lim_{x \to 0^-} \frac{f(x) - f(0)}{x - 0} \).
        \item Conclure sur la dérivabilité de \( f \) en \( 0 \) et interpréter les résultats.
    \end{enumerate}
    \item
    \begin{enumerate}
        \item En utilisant les variations de la \( h \) définie par \( h(x) = -x \), montrer que \( x < (x+1)^2 \) pour \( x > 0 \). En déduire que  
        \( \ln(x+1) < (x+1)^2 \) pour \( x > 0 \).
        \item Calculer \( f'(x) \) pour \( x > 0 \) et utiliser \( 3.a \) pour déterminer son signe.
        \item Calculer \( f'(x) \) pour \( x < 0 \) et donner son signe.
    \end{enumerate}
    \item
    \begin{enumerate}
        \item Calculer les limites de \( f \) aux bornes de son domaine de définition \( D_f \).
        \item Calculer  
        \( \lim_{x \to -\infty} [f(x) - (x+1)] \)  
        et interpréter graphiquement le résultat.
        \item Calculer  
        \( \lim_{x \to +\infty} [f(x) - (x+2)] \)  
        et interpréter graphiquement le résultat.
        \item Etudier le signe de \( f(x) - (x+1) \) pour \( x < 0 \), montrer que  
        \( f(x) - (x+2) > 0 \)  
        pour \( x > 0 \) et interpréter graphiquement les résultats.
    \end{enumerate}
    \item Déterminer les coordonnées du point \( A \) de la courbe où la tangente est parallèle à l’asymptote pour \( x > 0 \).
    \item Etablir que \( f \) est une bijection de \( \mathbb{R} \) sur un intervalle \( J \) à préciser.
    \item Représenter graphiquement les courbes de \( f \) et de \( f^{-1} \) dans un même repère.
    \item Calculer \( \int_{-\ln 3}^{0} (f(x) - (x+1))dx \).
    \item Interpréter graphiquement le résultat précédent en terme d’aire.
\end{enumerate}
\section*{\fbox{\textbf{Exercice 17}} }

\subsection*{\underline{\textbf{Partie A}}}

On considère la fonction \( g \) définie par  
\( g(x) = \ln x + 1 - e^{-x} \).

\begin{enumerate}
    \item Etudier les variations de \( g \).
    \item Montrer que l’équation \( g(x) = 0 \) admet une solution unique \( \alpha \) telle que \( 0,6 < \alpha < 0,7 \).
    \item En déduire le signe de \( g(x) \).
\end{enumerate}

\subsection*{\underline{\textbf{Partie B}}}

On considère la fonction \( f \) définie par  

\(
f(x) =
\begin{cases}
    x \ln x + e^{-x} & \text{si } x > 0 \\
    \frac{e^{\frac{1}{2} x}}{|x^2 - 1|} & \text{si } x \leq 0
\end{cases}
\)

\begin{enumerate}
    \item Justifier que \( D_f = \mathbb{R} \setminus \{-1\} \).
    \item Ecrire \( f(x) \) sans valeur absolue.
    \item Etudier la continuité et la dérivabilité de \( f \) en \( 0 \).  
    On pourra montrer que si \( x \in ]-1; 0[ \),

    \(
    \frac{f(x) - f(0)}{x - 0} = \frac{-\frac{1}{2} \left( \frac{e^{\frac{1}{2} x} - 1}{\frac{1}{2} x} \right) - x}{x^2 - 1}\)
    
    
    \item Déterminer les limites de \( f \) aux bornes de \( D_f \) puis étudier les branches infinies.
    \item Dresser le tableau de variations de \( f \).
    \item Montrer que \( f(\alpha) = (\alpha + 1) \ln \alpha + 1 \).
    \item Soit \( h \) la restriction de \( f \) à \( ]\alpha; +\infty[ \)
    \begin{enumerate}
        \item Montrer que \( h \) admet une bijection réciproque \( h^{-1} \) définie sur un intervalle \( J \) à préciser.
        \item \( h^{-1} \) est-elle dérivable sur \( J \) ? Justifier.
        \item Calculer \( h(1) \) puis \( (h^{-1})' \left( \frac{1}{e} \right) \).
    \end{enumerate}
    \item Tracer \( C_f \) et \( C_{h^{-1}} \) dans le même repère d’unité \( 2cm \).
    \item Soit \( \lambda > 1 \) et \( \mathcal{A}(\lambda) \) l’aire en \( cm^2 \) du domaine limité par \( C_h \), l’axe des abscisses et les droites d’équations \( x = 1 \) et \( x = \lambda \).  
    Calculer \( \mathcal{A}(\lambda) \) et sa limite en \( +\infty \).
\end{enumerate}

\section*{\fbox{\textbf{Exercice 18}}}

\subsection*{\underline{\textbf{Partie A}}}

Soit \( f \) la fonction définie sur \( \mathbb{R} \) par \( f(x) = x - e^{2x-2} \).

\begin{enumerate}
    \item Déterminer la limite de \( f \) en \( -\infty \).
    \item Vérifier que pour tout réel \( x \) non nul,  
    \(
    f(x) = x \left[ 1 - 2e^{-2} \left( \frac{e^{2x}}{2x} \right) \right].
    \)
    En déduire la limite de \( f \) en \( +\infty \).
    \item Dresser le tableau de variations de \( f \).
    \item Montrer que la droite \( (D) : y = x \) est asymptote à \( C_f \) en \( -\infty \).  
    Étudier la position relative de \( C_f \) par rapport à \( (D) \).
    \item Étudier la branche infinie de \( C_f \) en \( +\infty \).
    \item Déterminer une équation de la tangente \( (T) \) au point d’abscisse \( 1 \).
    \item Montrer que l’équation \( f(x) = 0 \) admet une solution unique \( \alpha \in I = ]0; \frac{1}{2}[ \).  
    Donner une valeur approchée de \( \alpha \) à \( 10^{-1} \) près.
    \item Tracer \( C_f \), \( (D) \) et \( (T) \).
\end{enumerate}

\subsection*{\underline{\textbf{Partie B}}}

On définit la suite \( (u_n) \) définie par :\(
\begin{cases}
u_0 = 0\\ 
u_{n+1} = e^{2u_n - 2}
\end{cases}
\)

\begin{enumerate}
    \item Soit \( g \) la fonction définie sur \( \mathbb{R} \) par \( g(x) = e^{2x-2} \).  
    Démontrer que l’équation \( f(x) = 0 \) est équivalente à \( g(x) = x \).  
    En déduire \( g(\alpha) \).
    \item Démontrer que pour tout \( x \in I \), \( |g'(x)| \leq \frac{2}{e} \).
    \item Démontrer que pour tout \( x \in I \), \( g(x) \in I \).
    \item 
    \begin{enumerate}
        \item En utilisant l’inégalité des accroissements finis, démontrer que pour tout \( n \in \mathbb{N} \),\( |u_{n+1} - \alpha| \leq \frac{2}{e} |u_n - \alpha| \)
        \item En déduire que 
\( |u_n - \alpha| \leq \left( \frac{2}{e} \right)^n. \)
        \item Montrer que \( (u_n) \) converge vers un réel à déterminer.
        \item Déterminer le plus petit entier naturel \( n \) tel que  
\( |u_n - \alpha| < 10^{-5}. \)
    \end{enumerate}
\end{enumerate}

\section*{\fbox{\textbf{Exercice 19}}}

\subsection*{\underline{\textbf{Partie B}}}

Soit la fonction \( h \) définie par \( h(x) = 1 - (x^2 - 2x + 2)e^{-x} \).

\begin{enumerate}
    \item Dresser le tableau de variations de \( h \).
    \item 
    \begin{enumerate}
        \item Démontrer que l’équation \( h(x) = 0 \) admet une unique solution \( \alpha \in ]0,1[ \).
    \end{enumerate}
\end{enumerate}

\begin{enumerate}
    \setcounter{enumi}{1}
    \item Donner une valeur approchée de \( \alpha \) à \( 10^{-1} \) près.
    \item En déduire le signe de \( h(x) \).
\end{enumerate}

\subsection*{\underline{\textbf{Partie B}}}

Soit la fonction \( f \) définie par  
\(
f(x) =
\begin{cases}
    1 + \frac{1}{x}e^{\frac{1}{x}}  \quad \text{si } x < 0 \\
    x - 1 + (x^2 + 2)e^{-x} \quad \text{si } x \geq 0
\end{cases}
\)

\begin{enumerate}
    \item Justifier que \( f \) est définie sur \( \mathbb{R} \).
    \item Etudier la continuité et la dérivabilité de \( f \) en \( 0 \).  
    Interpréter les résultats.
    \item Etudier les branches infinies de \( C_f \).
    \item Dresser le tableau de variations de \( f \).
    \item Montrer que \( f(\alpha) = \alpha (1 + 2e^{-\alpha}) \).
    \item Tracer \( C_f \) dans un repère orthonormé d’unité \( 2cm \).
\end{enumerate}

\subsection*{\underline{\textbf{Partie C}}}

Soit \( g \) la restriction de \( f \) à \( ]-\infty, -1[ \).

\begin{enumerate}
    \item Prouver l’existence de \( g^{-1} \). Etudier la dérivabilité de \( g^{-1} \).
    \item Résoudre \( g^{-1}(x) = 2 \), puis calculer  
    \[
    \left( g^{-1}\right)' \left( \frac{2-e^{-\frac{1}{2}}}{2} \right).
    \]
    \item Tracer la courbe de \( g^{-1} \) dans le même repère.
\end{enumerate}

\end{document}