\documentclass[12pt]{article}
\usepackage[utf8]{inputenc}
\usepackage[T1]{fontenc}
\usepackage[french]{babel}
\usepackage{amsmath}
\usepackage{amssymb}
\usepackage{enumitem}
\usepackage{xcolor}
\usepackage{geometry}
\geometry{a4paper, margin=2cm}

\begin{document}

\textbf{\textcolor{red}{Exercice 1 :}}

On dispose d’un dé pipé de 6 faces numérotées de 1 à 6. On jette ce dé et on note le chiffre obtenu sur la face supérieure du dé. La probabilité \( P_n \) d’obtenir le chiffre \( n \) est donnée par l’hypothèse suivante :

\[
P_1 = \frac{1}{4} \quad \text{et} \quad (P_n) \text{ est une suite arithmétique de raison } r.
\]

\begin{enumerate}
    \item Déterminer \( r \).
    \item En déduire la probabilité de chaque numéro du dé.
    \item Soit \( A \) : « le numéro est un multiple de 3 »,\\
          \hspace*{0.7cm} \( B \) : « le numéro est pair »
    \begin{enumerate}
        \item Calculer \( P(A) \) ; \( P(B) \)
        \item Calculer \( P(A \cap B) \)
        \item En déduire \( P(A \cup B) \)
    \end{enumerate}
\end{enumerate}

\vspace{1cm}

\textbf{\textcolor{red}{Exercice 2 :}}

On lance un dé parfait de 6 faces numérotées de 1 à 6.\\
Calculer la probabilité des éléments suivants :

\begin{itemize}
    \item[$A$ :] « le numéro apparu est pair »
    \item[$B$ :] « le numéro apparu est supérieur à 3 »
    \item[$C$ :] « le numéro apparu est un multiple de 3 »
\end{itemize}

\[
\mathcal{U} = \{1 ; 2 ; 3 ; 4 ; 5 ; 6\}
\quad
A = \{2 ; 4 ; 6\}
\quad
B = \{4 ; 5 ; 6\}
\quad
C = \{3 ; 6\}
\]

\textbf{\textcolor{red}{Exercice 3 :}}

Une urne contient 3 boules blanches numérotées de 1 à 3, 4 boules rouges numérotées de 1 à 4, 2 boules noires numérotées de 1 à 2. Les boules sont indiscernables au toucher.\\
On tire successivement sans remise trois boules de l’urne.

\begin{itemize}
    \item Calculer le nombre de tirages possibles.
    \item Calculer la probabilité des éléments suivants :
\end{itemize}

\begin{itemize}
    \item[$A$ :] « le tirage est unicolore »
    \item[$B$ :] « le tirage contient 2 blanches exactement »
    \item[$C$ :] « le tirage contient au moins 1 boule de numéro impair »
     \item[$D$ :] « la première boule tirée porte un numéro pair »
    \item[$E$ :] « la première boule est noire et la deuxième porte le numéro 2 »
    \item[$F$ :] « le tirage contient au plus 2 boules rouges »
\end{itemize}

\end{document}
