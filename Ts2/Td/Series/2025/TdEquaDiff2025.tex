\documentclass[12pt,a4paper]{article}
\usepackage{amsmath,amssymb,mathrsfs,tikz,times,pifont}
\usepackage{enumitem}
\usepackage{multicol}
\usepackage{lmodern}
\newcommand\circitem[1]{%
\tikz[baseline=(char.base)]{
\node[circle,draw=gray, fill=red!55,
minimum size=1.2em,inner sep=0] (char) {#1};}}
\newcommand\boxitem[1]{%
\tikz[baseline=(char.base)]{
\node[fill=cyan,
minimum size=1.2em,inner sep=0] (char) {#1};}}
\setlist[enumerate,1]{label=\protect\circitem{\arabic*}}
\setlist[enumerate,2]{label=\protect\boxitem{\alph*}}
\everymath{\displaystyle}
\usepackage[left=1cm,right=1cm,top=1cm,bottom=1.7cm]{geometry}
\usepackage[colorlinks=true, linkcolor=blue, urlcolor=blue, citecolor=blue]{hyperref}
\usepackage{array,multirow}
\usepackage[most]{tcolorbox}
\usepackage{varwidth}
\usepackage{float}
\tcbuselibrary{skins,hooks}
\usetikzlibrary{patterns}

\newtcolorbox{exa}[2][]{enhanced,breakable,before skip=2mm,after skip=5mm,
colback=yellow!20!white,colframe=black!20!blue,boxrule=0.5mm,
attach boxed title to top left ={xshift=0.6cm,yshift*=1mm-\tcboxedtitleheight},
fonttitle=\bfseries,
title={#2},#1,
boxed title style={frame code={
\path[fill=tcbcolback!30!black]
([yshift=-1mm,xshift=-1mm]frame.north west)
arc[start angle=0,end angle=180,radius=1mm]
([yshift=-1mm,xshift=1mm]frame.north east)
arc[start angle=180,end angle=0,radius=1mm];
\path[left color=tcbcolback!60!black,right color = tcbcolback!60!black,
middle color = tcbcolback!80!black]
([xshift=-2mm]frame.north west) -- ([xshift=2mm]frame.north east)
[rounded corners=1mm]-- ([xshift=1mm,yshift=-1mm]frame.north east)
-- (frame.south east) -- (frame.south west)
-- ([xshift=-1mm,yshift=-1mm]frame.north west)
[sharp corners]-- cycle;
},interior engine=empty,
},interior style={top color=yellow!5}}

\usepackage{fancyhdr}
\usepackage{eso-pic}
\usepackage{tkz-tab}
\AddToShipoutPicture{
    \AtTextCenter{%
        \makebox[0pt]{\rotatebox{80}{\textcolor[gray]{0.7}{\fontsize{5cm}{5cm}\selectfont PGB}}}
    }
}
\usepackage{lastpage}
\fancyhf{}
\pagestyle{fancy}
\renewcommand{\footrulewidth}{1pt}
\renewcommand{\headrulewidth}{0pt}
\renewcommand{\footruleskip}{10pt}
\fancyfoot[R]{\color{blue}\ding{45}\ \textbf{2025}}
\fancyfoot[L]{\color{blue}\ding{45}\ \textbf{Prof : M. BA}}
\cfoot{\bf \thepage / \pageref{LastPage}}

\newcommand{\exo}[1]{%
        \textbf{\underline{Exercice #1}}
}

\begin{document}
\renewcommand{\arraystretch}{1.5}
\renewcommand{\arrayrulewidth}{1.2pt}
\begin{tikzpicture}[overlay,remember picture]
    \node[draw=blue,line width=1.2pt,fill=purple,text=blue,inner sep=3mm,rounded corners,pattern=dots]at ([yshift=-2.5cm]current page.north) {\begingroup\setlength{\fboxsep}{0pt}\colorbox{white}{\begin{tabular}{|*1{>{\centering \arraybackslash}p{0.28\textwidth}} |*2{>{\centering \arraybackslash}p{0.2\textwidth}|} *1{>{\centering \arraybackslash}p{0.19\textwidth}|} }
                \hline
                \multicolumn{3}{|c|}{$\diamond$$\diamond$$\diamond$\ \textbf{Lycée de Dindéfélo}\ $\diamond$$\diamond$$\diamond$ } & \textbf{A.S. : 2024/2025} \\ \hline
                \textbf{Matière : Mathématiques} & \textbf{Niveau : T S2} & \textbf{Date : 27/05/2025} & \textbf{} \\ \hline
                \multicolumn{4}{|c|}{\parbox[c]{10cm}{\begin{center}
                  \textbf{{\Large\sffamily TD : Équations différentielles}}
                \end{center}}} \\ \hline
            \end{tabular}}\endgroup};
\end{tikzpicture}
\vspace{3cm}

\begin{multicols}{2}
\setlength{\columnseprule}{0.1mm}

\exo{1} Résoudre les équations différentielles suivantes :
\begin{multicols}{2}
\begin{enumerate}[align=left]
    \item \( y' - y = 0 \)
    \item \( 2y' + y = 0 \)
    \item \( 3y' + \sqrt{2}y = 0 \)
    \item \( y^{\prime 2} - yy' - 2y^2 = 0 \)
    \item \( y'' - 4y' + 3y = 0 \)
    \item \( 9y'' + 6y' + y = 0 \)
    \item \( y'' - 4y' + 13y = 0 \)
    \item \( 8y'' + y' - 9y = 0 \)
    \item \( 2y'' - 2\sqrt{2}y' + y = 0 \)
    \item \( \tfrac{1}{3}y'' - 2y' + 9y = 0 \)
    \item \( y'' - 4y' + 8y = 0 \)
    \item \( y'' + 4y = 0 \)
\end{enumerate}
\end{multicols}

\exo{2} Résoudre dans chacun des cas suivants l’équation différentielle proposée, puis déterminer la solution particulière satisfaisant aux conditions initiales données.
\begin{enumerate}
    \item \( y' - 3y = 0 \ ;\ y(0) = -3 \)
    \item \( 3y' + 5y = 0 \ ;\ y(3) = 5 \)
    \item \( y'' - 2y' + y = 0 \ ;\ y(0) = y'(0) = 2 \)
    \item \( y'' - 4y' + 3y = 0 \ ;\ y(0) = 6 ;\ y'(0) = 10 \)
    \item \( 4y'' + 9y = 0 \ ;\ y(\pi) = 1 ;\ y'(\pi) = 0 \)
    \item \( y'' + 6y' + 9y = 0 \ ;\ y(0) = y'(0) = 1 \)
    \item \( y'' + y = 0 \ ;\ y(0) = 1 ;\ y'(0) = 0 \)
    \item \( y'' - (\ln x)^2 y = 0 \ ;\ y(0) = 1 ;\ y'(2) = 1 \)
\end{enumerate}

\exo{3} Donner les solutions générales des équations différentielles suivantes après avoir trouvé leurs solutions particulières sous la forme indiquée :
\begin{enumerate}
    \item \( y'' + 2y' - 3y = \cos x \quad ;\quad f(x) = a \cos x + b \sin x \)
    \item \( y'' - 2y' + 3y = e^{-x} \cos x \quad ;\quad f(x) = e^{-x}(a \cos x + b \sin x) \)
    \item \( y'' + y' - 2y = \sin x \quad ;\quad f(x) = a \cos x + b \sin x \)
    \item \( y'' + y' + y = xe^{-x} \quad ;\quad f(x) = (ax^2 + bx + c)e^{-x} \)
\end{enumerate}

\vspace{1em}

\exo{4}
\begin{enumerate}
    \item Résoudre dans \( \mathbb{R} \) l’équation différentielle \( (E_0) : y'' - 2y' + y = 0 \).
    \item Soit l’équation différentielle \( (E) : y'' - 2y' + y = x^2 - 4x + 2 \). Vérifier que le polynôme \( h(x) = x^2 \) est une solution particulière de \( (E_0) \).
    \item
    \begin{enumerate}
        \item Montrer que si \( f \) est une solution, alors \( g \) telle que \( g = f - h \) est une solution de \( (E_0) \).
    
        \item Réciproquement, montrer que si \( g \) est une solution de \( (E_0) \), alors la fonction \( f \) telle que \( f = g + h \) est solution de \( (E) \).
    
        \item En déduire la générale des solutions de \( (E) \).
    \end{enumerate}
    \item En déduire une solution de \( (E) \) satisfaisant à la condition \( f(1) = 1 \) et \( f'(1) = 0 \).
\end{enumerate}


\vspace{1em}

\exo{5}

Soit l’équation différentielle : \(y'' + 4y' + 8y = 20 \sin 2x + 10 \cos 2x.\)

\begin{enumerate}
    \item Résoudre l’équation différentielle  \( y'' + 4y' + 8y = 0 \quad (E').\)
    
    \item Déterminer les réels \( a \) et \( b \) tels que la fonction \( g \) définie par : \(    g(x) = a \cos 2x + b \sin 2x\)
    est solution de \( (E) \).
    
    \item Démontrer que \( f \) est solution de \( (E) \) si et seulement si \( f - g \) est solution de \( (E') \).
    \item Donner la solution générale de \( (E) \).
\end{enumerate}
\exo{6}
Soit \( f \) la solution de l’équation différentielle \( y' - 2y = 0 \), vérifiant \( f(0) = 1 \).

\begin{enumerate}
    \item Déterminer \( f(x) \) pour tout réel \( x \).
    
    \item Déterminer les réels \( a \) et \( b \) tels que la fonction \( g(x) = (ax + b)f(x) \) soit solution de l’équation différentielle \( y' - 2y = e^{2x} \) et vérifiant \( g(0) = 1 \).
    
    \item Déduisez-en sans intégration par parties la valeur des intégrales :  
    \( I = \int_{0}^{1} (x + 1)e^{2x} \, dx \quad \) et \( \quad J = \int_{0}^{1} (2x + 1)e^{2x} \, dx \)
\end{enumerate}

\vspace{0.5cm}

\exo{7} \textbf{BAC 2009}

\begin{enumerate}
    \item Résoudre l’équation différentielle\\ \( (E) :\quad y'' + 2y' + y = 0 \)
    
    \item Soit \( (E') \) l’équation différentielle\\ \( y'' + 2y' + y = x + 3 \).\\
    Déterminer les réels \( a \) et \( b \) tels que la fonction \( h(x) = ax + b \) soit solution de \( (E') \).
    \begin{enumerate}
        \item Démontrer que \( g \) est solution de \( (E') \) si et seulement si \( g - h \) est solution de \( (E) \).
        \item Résoudre alors \( (E') \).
    \end{enumerate}
    
    \item Soit la fonction \( k \) définie par :\\ \( k(x) = (x + 2)e^{-x} \)
    \begin{enumerate}
        \item Étudier les variations de \( k \).
        \item Déterminer l’équation de la tangente \( (T) \) à la courbe \( (C_k) \) de \( k \) au point d’abscisse 0.
        \item Démontrer que le point \( I(0;2) \) est un point d’inflexion de la courbe \( (C_k) \).
        \item Tracer \( (C_k) \) et \( (T) \) dans le même repère orthonormé \( (0,\vec{i},\vec{j}) \).
    \end{enumerate}
\end{enumerate}

\exo{8} \textbf{BAC 2008}

\begin{enumerate}
    \item Soient les équations différentielles \\ 
    \( (E_0) : y' + y = 0 \quad \text{et} \quad (E) : y' + y = e^{-x} \cos x \).
    \begin{enumerate}
        \item Déterminer les réels \( a \) et \( b \) tels que la fonction \( h \) définie par : \\ 
        \( h(x) = e^{-x}(a \cos x + b \sin x) \) soit solution de \( (E) \).
        
        \item Démontrer que \( f \) est solution de \( (E) \) si et seulement si \( f - h \) est solution de \( (E_0) \).
        
        \item Résoudre \( (E_0) \).
        
        \item Déduire des questions précédentes la solution générale de \( (E) \).
        
        \item Déterminer la solution \( g \) de \( (E) \) telle que \( g(0) = 0 \).
    \end{enumerate}
    
    \item Soit \( I(x) = e^{-x} \sin x \).
    \begin{enumerate}
        \item Exprimer \( \cos\left(x + \dfrac{\pi}{4}\right) \) en fonction de \( \cos x \) et \( \sin x \).
        
        \item Étudier les variations de \( I \) sur \( [0\,;\,2\pi[ \).
        
        \item Calculer \( \int_0^{\frac{\pi}{4}} I(x) \, dx \).
        
        \item En déduire la valeur de \( \int_0^{\frac{\pi}{4}} e^{-x} \cos x \, dx \)  
        sans utiliser la technique d’intégration par parties.
    \end{enumerate}
\end{enumerate}

\exo{9} \\

\textbf{Partie A :}\\
Soit \( (E) : y'' + 3y' + 2y = \dfrac{x - 1}{x^2} e^{-x} \), \( x \ne 0 \).

\begin{enumerate}
    \item Vérifier que \( h(x) = e^{-x} \ln x \) est solution de \( (E) \).
    
    \item Montrer que \( f \) est solution de \( (E) \) si et seulement si \( f - h \) est solution de  \\
    \( (E') : y'' + 3y' + 2y = 0 \).
    
    \item Résoudre \( (E') \) puis \( (E) \).
    
    \item Trouver la solution de \( (E) \) dont la courbe passe par \( A(1 ; e^{-1}) \) et admet en ce point une tangente parallèle à l’axe \( (Ox) \).
\end{enumerate}

\vspace{0.3cm}

\textbf{Partie B :} Soit la fonction \( f \) définie par :

\( f(x) =
\begin{cases}
e^{-x}(1 + \ln x) - e^{-1} & \text{si } x \geq 1 \\
(1 - x)e^{-\frac{1}{1 - x}} & \text{si } x < 1
\end{cases} \)

\begin{enumerate}
    \item Vérifier que \( f \) est définie sur \( \mathbb{R} \) et calculer les limites aux bornes.
    
    \item Étudier la continuité de \( f \) en 1.
    
    \item 
    \begin{enumerate}
        \item Montrer que \( \forall x \in ]1 ; +\infty[ \),\\  
        \( \dfrac{f(x) - f(1)}{x - 1} = e^{-x} \left( \dfrac{1 - e^{x-1}}{x - 1} + \dfrac{\ln x}{x - 1} \right) \).
        
        \item Étudier la dérivabilité de \( f \) en 1. Interpréter.
    \end{enumerate}
\end{enumerate}

\vspace{0.3cm}

\textbf{Partie C :} Soit la fonction \( g \) définie par\\  
\( g(x) = -\ln x + \dfrac{2}{x} - 1 \).

\begin{enumerate}
    \item Dresser le tableau de variation de \( g \).
    
    \item Calculer \( g(1) \) et préciser le signe de \( g(x) \).
    
    \item
    \begin{enumerate}
        \item Calculer \( f'(x) \) sur les intervalles où \( f \) est dérivable.
        
        \item Étudier le signe de \( f'(x) \) et dresser le tableau de variations de \( f \).
    \end{enumerate}
    
    \item
    \begin{enumerate}
        \item Donner la nature de la branche infinie en \( +\infty \).
        
        \item Montrer que \( (\Delta) : y = 1 - x \) est une asymptote à \( (C_f) \) en \( -\infty \).
        
        \item Étudier la position relative de \( (C_f) \) par rapport à \( (\Delta) \) sur \( ] -\infty\,;\,1[ \).
    \end{enumerate}
    
    \item Tracer \( (C_f) \) dans un repère orthonormé \( (O,\vec{i},\vec{j}) \) unité \( 1\,\text{cm} \).
\end{enumerate}
\end{multicols}

\end{document}
