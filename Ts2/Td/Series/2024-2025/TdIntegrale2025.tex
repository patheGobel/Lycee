\documentclass[12pt,a4paper]{article}
\usepackage[T1]{fontenc}
\usepackage{amsmath,amssymb,mathrsfs,tikz,times,pifont}
\usepackage{enumitem}
\usepackage{multicol}
\usepackage{lmodern}
\usetikzlibrary{trees}
\newcommand\circitem[1]{%
\tikz[baseline=(char.base)]{
\node[circle,draw=gray, fill=red!55,
minimum size=1.2em,inner sep=0] (char) {#1};}}
\newcommand\boxitem[1]{%
\tikz[baseline=(char.base)]{
\node[fill=cyan,
minimum size=1.2em,inner sep=0] (char) {#1};}}
\setlist[enumerate,1]{label=\protect\circitem{\arabic*}}
\setlist[enumerate,2]{label=\protect\boxitem{\alph*}}
%%%::::::by chnini ameur :::::::%%%
\everymath{\displaystyle}
\usepackage[left=1cm,right=1cm,top=1cm,bottom=1.7cm]{geometry}
\usepackage[colorlinks=true, linkcolor=blue, urlcolor=blue, citecolor=blue]{hyperref}
\usepackage{array,multirow}
\usepackage[most]{tcolorbox}
\usepackage{varwidth}
\usepackage{float} %pour utiliser l'option [H] qui force l'image à apparaître exactement à l'endroit où elle est placée dans le code.
\tcbuselibrary{skins,hooks}
\usetikzlibrary{patterns}
%%%::::::by chnini ameur :::::::%%%
\newtcolorbox{exa}[2][]{enhanced,breakable,before skip=2mm,after skip=5mm,
colback=yellow!20!white,colframe=black!20!blue,boxrule=0.5mm,
attach boxed title to top left ={xshift=0.6cm,yshift*=1mm-\tcboxedtitleheight},
fonttitle=\bfseries,
title={#2},#1,
% varwidth boxed title*=-3cm,
boxed title style={frame code={
\path[fill=tcbcolback!30!black]
([yshift=-1mm,xshift=-1mm]frame.north west)
arc[start angle=0,end angle=180,radius=1mm]
([yshift=-1mm,xshift=1mm]frame.north east)
arc[start angle=180,end angle=0,radius=1mm];
\path[left color=tcbcolback!60!black,right color = tcbcolback!60!black,
middle color = tcbcolback!80!black]
([xshift=-2mm]frame.north west) -- ([xshift=2mm]frame.north east)
[rounded corners=1mm]-- ([xshift=1mm,yshift=-1mm]frame.north east)
-- (frame.south east) -- (frame.south west)
-- ([xshift=-1mm,yshift=-1mm]frame.north west)
[sharp corners]-- cycle;
},interior engine=empty,
},interior style={top color=yellow!5}}
%%%%%%%%%%%%%%%%%%%%%%%
\usepackage{fancyhdr}
\usepackage{eso-pic}         % Pour ajouter des éléments en arrière-plan
% Commande pour ajouter du texte en arrière-plan
\usepackage{tkz-tab}
\AddToShipoutPicture{
    \AtTextCenter{%
        \makebox[0pt]{\rotatebox{80}{\textcolor[gray]{0.7}{\fontsize{5cm}{5cm}\selectfont PGB}}}
    }
}
\usepackage{lastpage}
\fancyhf{}
\pagestyle{fancy}
\renewcommand{\footrulewidth}{1pt}
\renewcommand{\headrulewidth}{0pt}
\renewcommand{\footruleskip}{10pt}
\fancyfoot[R]{
\color{blue}\ding{45}\ \textbf{2025}
}
\fancyfoot[L]{
\color{blue}\ding{45}\ \textbf{Prof:M. BA}
}
\cfoot{\bf
\thepage /
\pageref{LastPage}}
% Création du compteur pour les exercices
\newcounter{exercice}
\renewcommand{\theexercice}{\arabic{exercice}}  % Définit l'affichage du compteur en chiffres arabes

% Définir la commande \exo
\newcommand{\exo}{\refstepcounter{exercice}\textbf{Exercice \theexercice} }
\begin{document}
\renewcommand{\arraystretch}{1.5}
\renewcommand{\arrayrulewidth}{1.2pt}
\begin{tikzpicture}[overlay,remember picture]
    \node[draw=blue,line width=1.2pt,fill=purple,text=blue,inner sep=3mm,rounded corners,pattern=dots]at ([yshift=-2.5cm]current page.north) {\begingroup\setlength{\fboxsep}{0pt}\colorbox{white}{\begin{tabular}{|*1{>{\centering \arraybackslash}p{0.28\textwidth}} |*2{>{\centering \arraybackslash}p{0.2\textwidth}|} *1{>{\centering \arraybackslash}p{0.19\textwidth}|} }
                \hline
                \multicolumn{3}{|c|}{$\diamond$$\diamond$$\diamond$\ \textbf{Lycée de Dindéfélo}\ $\diamond$$\diamond$$\diamond$ } & \textbf{A.S. : 2024/2025}                                              \\ \hline
                \textbf{Matière: Mathématiques}                                                                                    & \textbf{Niveau : T}\textbf{S2} & \textbf{Date: 17/04/2025} & \textbf{} \\ \hline
                \multicolumn{4}{|c|}{\parbox[c]{10cm}{\begin{center}
                                                                  \textbf{{\Large\sffamily Td Probabilité}}
                                                              \end{center}}}                                                                                                        \\ \hline
            \end{tabular}}\endgroup};
\end{tikzpicture}
\vspace{3cm}
\begin{multicols}{2}
\setlength{\columnseprule}{0.1mm} % La largeur de la ligne verticale entre les colonnes
\fbox{\textbf{\exo}} Calculer les intégrales :
\begin{multicols}{2}
\setlength{\columnseprule}{0.1mm} % La largeur de la ligne verticale entre les colonnes
\begin{enumerate}
    \item $\int_1^2 2t^2 \, dt$
    \item $\int_{-1}^3 -t^4 \, dt$
    \item $\int_1^2 \frac{1}{s^4} \, ds$
    \item $\int_0^2 \frac{1}{(x+2)^2} \, dx$
    \item $\int_0^2 \frac{x}{(x^2 + 2)^2} \, dx$
    \item $\int_0^\frac{\pi}{2} (2 - 3\sin t) \, dt$
    \item $\int_0^\frac{\pi}{2} \sin \left( x + \frac{\pi}{4} \right) \, dx$
    \item $\int_0^2 \frac{e^{x}}{e^{x}+3} \, dx$
    \item $\int_{-1}^5 \mid x^2 - 9 \mid \, dx$
    \item $\int_0^2 e^{x}(e^x-3) \, dx$
    \item $\int_e^{e^3} \frac{dx}{x\ln x}$
    \item $\int_0^\frac{\pi}{4} \tan^2 x \, dx$
    \item $\int_{-3}^{-2} \frac{x}{\sqrt{x^2 - 1}} \, dx$
    \item $\int_0^\pi \frac{ \sin x}{(2 + \cos x)^2} \, dx$
    \item $\int_0^\pi \frac{dx}{\cos^2 x}$
    \item $\int_{\frac{\pi}{4}}^{\frac{\pi}{4}} \frac{1 + \tan^2 x}{\tan x} \, dx$
    \item $\int_0^1 \frac{e^x + 1}{e^x + x} \, dx$
    \item $\int_e^{e^2} \frac{\ln u}{u} \, du$
    \item $\int_0^{\frac{\pi}{4}} \cos 2u \, du$
    \item $\int_{-2}^1 |2x - 1| \, dx$
    \item $\int_0^{\frac{\pi}{4}} \tan^2 x \, dx$
\end{enumerate}
\end{multicols}
$\int_0^1 \left( \sqrt{2t + 1} + \frac{1}{\sqrt{t + 5}} \right) dt$
,$\int_0^2 (2t + 3)\sqrt{t^2 + 3t} \, dt$

\fbox{\textbf{\exo}}
\begin{enumerate}
    \item Sans chercher de primitives, calculer les intégrales :
    \(
\int_{-\frac{\pi}{4}}^{\frac{\pi}{4}} \sin 4t \, dt \quad ; \quad \int_{-\frac{\pi}{3}}^{\frac{\pi}{3}} \cos 3x \, dx
\)

\(
\int_{-1}^{1} \ln \left( \frac{2 - x}{2 + x} \right) \, dx \quad ; \quad \int_{\ln 2}^{\ln 2} \frac{1 - e^x}{1 + e^x} \, dx
\)
    \item Calculer les intégrales suivantes :

\( \int_0^{\frac{\pi}{2}} \sin x \cos^2 x \, dx \quad ; \quad \int_0^{\frac{\pi}{3}} \sin^3 x \, dx \)

\( \int_0^{\frac{\pi}{4}} \cos^4 u \, du \quad ; \quad \int_0^{\frac{\pi}{3}} \sin^2 x \cos^3 x \, dx \)
\end{enumerate}

\fbox{\textbf{\exo}}
\begin{enumerate}
    \item \begin{enumerate}
    \item Vérifier que : 
\(     \forall x \in \mathbb{R} \setminus \left\{-1 ; -\frac{1}{2}\right\}, \quad \frac{8x + 5}{2x^2 + 3x + 1} = \frac{3}{x+1} + \frac{2}{2x+1}. \)
    
    \item En déduire la valeur de :
\( \int_0^2 \frac{8x + 5}{2x^2 + 3x + 1} \, dx. \)
\end{enumerate}
\item Soit la fonction $f$ définie par :\\
\(f(x) = \frac{3x^3 - 5x^2 + 2x - 1}{(x-2)^2}.\)

\begin{enumerate}
    \item Déterminer les réels $a$, $b$, $c$ et $d$ tels que :
    \(f(x) = ax + b + \frac{c}{x-2} + \frac{d}{(x-2)^2}.\)
    \item En déduire une primitive de $f$ sur $[-1 ; 1]$ puis calculer :
    \(\int_{-1}^1 f(x) \, dx.\)
\end{enumerate}
\end{enumerate}
\fbox{\textbf{\exo}} : Calcul au moins d'une intégration par parties
\begin{multicols}{2}
\begin{enumerate}
    \item $\int_0^{\frac{\pi}{2}} x \sin x \, dx$
    \item $\int_1^e x \ln x \, dx$
    \item $\int_0^{\frac{\pi}{2}} x \cos x \, dx$
    \item $\int_1^e \ln x \, dx$
    \item $\int_1^e (2 - t) e^t \, dt$
    \item $\int_0^{\frac{\pi}{2}} (x - 1) \sin x \, dx$
    \item $\int_1^3 (2x + 1) \ln x \, dx$
    \item $\int_0^{\frac{\pi}{2}} (u - 1)^2 \sin u \, du$
    \item $\int_0^{\frac{\pi}{2}} e^x \sin x \, dx$
    \item $\int_0^1 (x^2 - 1) e^{2x} \, dx$
    \item $\int_0^{\pi} e^{-x} \cos x \, dx$
    \item $\int_1^2 x \sqrt{3 - x} \, dx$
    \item $\int_1^2 (3x^2 + 1)^2 e^x \, dx$
    \item $\int_1^2 u (\ln u)^2 \, du$
    \item $\int_0^1 (x + 1)\sqrt{x + 1} \, dx$
\end{enumerate}
\end{multicols}
\fbox{\textbf{\exo}}
Soit $f$ la fonction numérique de la variable réelle $x$ définie par :\\
\(f(x) = \frac{x+1}{x} + \ln x - \ln(x + 1).\)

\begin{enumerate}
    \item Étudier les variations de $f$ et tracer sa courbe.
    \item Soit $\lambda$ un réel supérieur à 1 et\\
    \(D_{\lambda} = \{M(x,y),\ 1 \leq x \leq \lambda \text{ et } 1 \leq y \leq f(x)\}.\)
    Calculer l'aire $A(D_{\lambda})$ de $D_{\lambda}$. Étudier la limite de $A(D_{\lambda})$ quand $\lambda$ tend vers $+\infty$.
\end{enumerate}

\fbox{\textbf{\exo}}

On pose 
\(
I = \int_0^{\frac{\pi}{4}} \left( \int_0^x \sin^5 t \cos t \, dt \right) dx.
\)

\begin{enumerate}
    \item Liénéariser \( \sin^6 x \).
    \item Démontrer que 
    \(
    I = \frac{15\pi - 44}{1152}.
    \)
\end{enumerate}

\fbox{\textbf{\exo}}

Soit \( f \) la fonction définie sur \( ]0 ; +\infty[ \) par :\\\( f(x) = \frac{2 \ln x}{x^2 + x} \). On se propose de trouver un encadrement de l'aire \( A \) de l'ensemble des points \( M(x; y) \) tels que \( 1 \leq x \leq \frac{3}{2} \) et \( 0 \leq y \leq f(x) \).

\begin{enumerate}
    \item Étudier et tracer \( (C_f) \).
    \item Montrer que pour tout \( x \geq 1 \),\\ \( \frac{\ln x}{x^2} \leq f(x) \leq \frac{\ln x}{x} \).
    \item Calculer \( I = \int_1^{\frac{3}{2}} \frac{\ln x}{x} \, dx \) et \( \int_1^{\frac{3}{2}} \frac{\ln x}{x^2} \, dx \).
    \item En déduire un encadrement de \( \int_1^{\frac{3}{2}} f(x) \, dx \) puis un encadrement de \( A \).
\end{enumerate}
\fbox{\textbf{\exo}}
On se propose de calculer l'intégrale 
\( J = \int_0^1 \frac{x e^x}{(1 + e^x)^3} \, dx \).

\begin{enumerate}
    \item Calculer les deux intégrales suivantes :
    \( A = \int_0^1 \frac{e^x}{e^x + 1} \, dx \) et \( \int_0^1 \frac{e^x}{(e^x + 1)^2} \, dx \).
    \item Déterminer les réels \(a\), \(b\) et \(c\) tels que pour \( x \geq 0 \),
\begin{equation}\tag{\textbf{1}}\frac{1}{(1 + t)^2} = a + \frac{b t}{t+1} + \frac{c t}{(1+t)^2}.\end{equation}
    \item En posant \( t = e^x \) dans l'égalité (1), calculer l'intégrale 
    \( I = \int_0^1 \frac{dx}{(1 + e^x)^2} \).
    \item Établir une relation entre \(I\) et \(J\) et en déduire \(J\).
\end{enumerate}
\fbox{\textbf{\exo}}
On pose, pour tout nombre entier naturel non nul :
\( I_n = \int_1^0 x^n (\ln x)^n \, dx \),
où \( \ln \) désigne la fonction logarithme népérien, et \( I_0 = \int_1^e x^2 \, dx \).

\begin{enumerate}
    \item Calculer \( I_0 \) et \( I_1 \).
    \item En utilisant une intégration par parties, démontrer que pour tout entier naturel \( n \) non nul :
    \begin{equation}\tag{\textbf{1}} 3 I_{n+1} + (n+1) I_n = e^3 .\end{equation} En déduire \( I_2 \).
    \item 
    \begin{enumerate}
        \item Démontrer que pour tout entier naturel \( n \), \( I_n \) est positif.
        \item Déduire de l'égalité \textbf{(1)} pour tout entier naturel non nul, \( I_n \leq \frac{e^{3}}{n+1} \).
    \end{enumerate}
\end{enumerate}
\fbox{\textbf{\exo}}
Soient \( f, g, h \) et \( k \) les fonctions de \( \mathbb{R} \) vers \( \mathbb{R} \) telles que : 
\[
f(x) = \sqrt{x^2 + 1}, \quad g(x) = x + \sqrt{x^2 + 1}, \quad h(x) = \ln(x + \sqrt{x^2 + 1}) \quad \text{et} \quad k(x) = x \sqrt{x^2 + 1}.
\]
Soit \( \alpha \) un réel, on pose \( \forall n \in \mathbb{N}, I_n = \int_0^\alpha \frac{x^n}{\sqrt{x^2 + 1}} \, dx \) et 
\[
F = \int_0^\alpha \sqrt{x^2 + 1} \, dx, \quad G = \int_0^\alpha (x + \sqrt{x^2 + 1}) \, dx, \quad H = \int_0^\alpha \ln(x + \sqrt{x^2 + 1}) \, dx.
\]
\begin{enumerate}
    \item Démontrer que les fonctions \( f \), \( g \), \( h \) et \( k \) sont dérivables sur \( \mathbb{R} \), et déterminer leurs applications dérivées \( f' \), \( g' \), \( h' \) et \( k' \).
    \item Justifier l'existence de \( I_n \) pour tout entier naturel \( n \). Calculer \( I_0 \) et \( I_1 \).
    \item Justifier l'existence de \( F \). Calculer \( F + I_2 \) et \( F - I_2 \). En déduire \( I_2 \). Justifier l'existence de \( G \), calculer \( G \).
    \item Justifier l'existence de \( H \). Par une intégration par parties, calculer \( H \).
\end{enumerate}
\fbox{\textbf{\exo}}
Pour tout entier naturel \(n\), on considère les intégrales :
\[
I_n = \int_0^{\frac{\pi}{2}} e^{-nx} \sin x \, dx \quad \text{et} \quad J_n = \int_0^{\frac{\pi}{2}} e^{-nx} \cos x \, dx.
\]

\begin{enumerate}
    \item Calculer \( I_0 \) et \( J_0 \).
    \item Soit \( n \) un entier naturel non nul.
    \begin{enumerate}
        \item En intégrant par parties \( I_n \) puis \( J_n \), montrer que :
        \[
        \begin{cases}
        I_n + nJ_n = 1 \\
        -nI_n + J_n = e^{-n\frac{\pi}{2}}.
        \end{cases}
        \]
        \item En déduire les expressions de \( I_n \) et \( J_n \) en fonction de \( n \).
    \end{enumerate}
    \item Déterminer \( \lim\limits_{n \to +\infty} I_n \) et \( \lim\limits_{n \to +\infty} J_n \).
\end{enumerate}
\fbox{\textbf{\exo}}
Dans cet exercice \(n\) est un entier naturel non nul. On considère la suite \( (U_n) \) par :
\(
U_n = \int_0^2 \left( \frac{2t + 3}{t + 2}\right) e^{\frac{7}{n}} \, dt.
\)

\begin{enumerate}
    \item Soit f la fonction définie sur [0; 2] par :\( f(t) = \frac{2t + 3}{t + 2} \). 
    
    Étudier les variations de \( f \) sur \([0; 2]\).
    \item En déduire que pour tout réel \( t \) de \([0; 2]\) :
\( \frac{3}{2} \leq f(t) \leq \frac{7}{4}.\)
    \item Montrer que pour tout réel \( t \) de \([0; 2]\) :
\(\frac{3}{2}e^{\frac{1}{n}} \leq f(t) e^{\frac{1}{n}} \leq \frac{7}{4} e^{\frac{1}{n}}.\)
    \item Par une intégration par parties, déduire que :
    
\(\frac{3}{2}n \left( e^{\frac{2}{n}} - 1 \right) \leq U_n \leq \frac{7}{4}n \left( e^{\frac{2}{n}} - 1 \right).\)
    On rappelle que \(\lim\limits_{h \to 0} \frac{e^h - 1}{h} = 1\). Montrer que si \( (U_n) \) possède une limite \( L \), alors \( 3 \leq L \leq \frac{7}{2} \).
    \item Vérifier que, pour tout réel \( t \) dans \([0; 2]\) on a :
\(\frac{2t + 3}{t + 2} = 2 - \frac{1}{t + 2}.\)
\begin{enumerate}
    \item En déduire l'intégrale 
    \(I = \int_0^2 \frac{2t + 3}{t + 2} \, dt.\)
    \item Montrer que pour tout \( t \) de \( [0; 2] \) on a :
    \(1 \leq e^{\frac{t}{n}} \leq e^{\frac{2}{n}}.\)
    \item En déduire que :
    \(I \leq U_n \leq e^{\frac{2}{n}} \times I.\)
    \item Montrer que \( (U_n) \) est convergente et déterminer sa limite \( L \).
\end{enumerate}
\end{enumerate}
\fbox{\textbf{\exo}}

\textbf{PARTIE A}

\begin{enumerate}
    \item Étudier sur \( \mathbb{R} \) le signe de \( 4e^{2x} - 5e^x + 1 \).
    \item Soit \( \varphi \) la fonction définie par : 
    \[
    \varphi(x) = \ln x - 2\sqrt{x} + 2.
    \]
    \begin{enumerate}
        \item Déterminer son domaine de définition \( D_\varphi \) et calculer ses limites aux bornes de \( D_\varphi \).
        \item Étudier ses variations et dresser son tableau de variations.
        \item En déduire son signe.
    \end{enumerate}
\end{enumerate}

\textbf{PARTIE B}

Soit \( f \) la fonction définie par :
\[
f(x) = \begin{cases}
x + \frac{e^x}{2e^x - 1} & \text{si } x \leq 0, \\
1 - x + \sqrt{x} \ln x & \text{si } x > 0.
\end{cases}
\]
On désigne par \( (\tau) \) la courbe représentative de \( f \) dans un repère orthonormé d’unité 2 cm.
\begin{enumerate}
    \item 
    \begin{enumerate}
    \item Déterminer \( D_f \) le domaine de définition de \( f \).
    \item Calculer les limites de \( f \) aux bornes de \( D_f \) et étudier les branches infinies de \( (\tau) \).
    \item Étudier la position de \( (\tau) \) par rapport à l’asymptote non parallèle aux axes dans \( ]-\infty; 0] \).
\end{enumerate}
\item  
\begin{enumerate}
    \item Étudier la continuité de \( f \) en \( 0 \).
    \item Étudier la dérivabilité de \( f \) en \( 0 \) et interpréter graphiquement les résultats.
\end{enumerate}

\item  Déterminer la dérivée de \( f \) et dresser le tableau de variations de \( f \).

\item Construire \( (\tau) \), les asymptotes et les demi-tangentes. On remarquera que \( f(1) = 0 \) et \( f'(1) = 0 \).

\item Calculer en cm\(^2\) l’aire du domaine limité par \( (\tau) \), la droite d’équation \( y = x \) et les droites d’équations \( x = -\ln 8 \) et \( x = -\ln 4 \).
\end{enumerate}
\fbox{\textbf{\exo}}

\textbf{Partie A :BAC 1999 Remplacement} 

Soit \( f \) la fonction définie sur \( \mathbb{R} \) par :
\[
\begin{cases}
    f(x) = e^{-\frac{1}{x^2}}, \quad x \in ]-\infty; 0[\\
    f(x) = \ln\left|\frac{x-1}{x+1}\right|, \quad x \in [0; 1] \cup ]1; +\infty[
\end{cases}
\]
et \( (C_f) \) sa courbe représentative dans un repère orthonormé \( (O; \overrightarrow{i}; \overrightarrow{j}) \) d’unité 2 cm.

\begin{enumerate}
    \item Étudier la continuité de \( f \) en \( 0 \).
    \item 
    \begin{enumerate}
        \item Montrer que pour tout \( x \in [0; 1],\\ f(x) = \frac{\ln(1 - x)}{x} - \frac{\ln(1 + x)}{x} \).
        \item Étudier la dérivabilité de \( f \) en \( 0 \).
        \item En déduire que \( (C_f) \) admet au point d’abscisse \( 0 \) deux demi-tangentes dont on donnera les équations.
    \end{enumerate}
    \item Étudier les variations de \( f \).
    \item Tracer \( (C_f) \).
\end{enumerate}
    \textbf{Partie B :} Soit \( g \) la restriction de \( f \) à \( ]1; +\infty[ \).
\begin{enumerate}
    \item Montrer que \( g \) est une bijection de \( ]1; +\infty[ \) vers un intervalle \( J \) à préciser. On notera \( g^{-1} \) la bijection réciproque de \( g \).
    \item Montrer que l’équation \( g(x) = -e \) admet une unique solution \( \alpha \) sur l’intervalle \( ]1; +\infty[ \). \\
    \textit{(On ne demande pas de calculer \( \alpha \)).}
    \item Montrer que \( \forall x \in J, \, g^{-1}(x) = 1 - \frac{e^x}{e^x - 1}. \)
    \item Construire \( (C_{g^{-1}}) \). \textit{(On indiquera la nature et l’équation de chacune des asymptotes à \( (C_g) \) et \( (C_{g^{-1}}) \).}
    \item Calculer en \( \text{cm}^2 \) l’aire \( A \) de l’ensemble des points \( M(x; y) \) défini par :
\[
\left\{
\begin{array}{ll}
    -\ln 7 \leq x \leq -1, \\
    0 \leq y \leq g^{-1}(x).
\end{array}
\right.
\]
\end{enumerate}
\end{multicols}
\end{document}