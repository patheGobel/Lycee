\documentclass[a4paper,12pt]{article}
\usepackage{graphicx}
\usepackage[a4paper, top=2cm, bottom=2cm, left=2cm, right=2cm]{geometry} % Ajuste les marges
\usepackage{xcolor} % Pour ajouter des couleurs
\usepackage{hyperref} % Pour avoir des références colorées si nécessaire
\usepackage{eso-pic} % Pour ajouter des éléments en arrière-plan

\usepackage[french]{babel}
\usepackage[T1]{fontenc}
\usepackage{mathrsfs}
\usepackage{amsmath}
\usepackage{amsfonts}
\usepackage{amssymb}
\usepackage{tkz-tab}

\usepackage{tikz}
\usetikzlibrary{arrows, shapes.geometric, fit}
\newcounter{correction} % Compteur pour les questions

% Définir la commande pour afficher une question numérotée
\newcommand{\question}{%
  \refstepcounter{correction}%
  \textbf{\textcolor{black}{Question \thecorrection (1 point) :}} \ignorespaces
}
% Commande pour ajouter du texte en arrière-plan
\AddToShipoutPicture{
    \AtTextCenter{%
        \makebox[0pt]{\rotatebox{80}{\textcolor[gray]{0.9}{\fontsize{10cm}{10cm}\selectfont PGB}}}
    }
}

\begin{document}
\hrule % Barre horizontale
% En-tête
\begin{center}
    \begin{tabular}{@{} p{5cm} p{5cm} p{5cm} @{}} % 3 colonnes avec largeurs fixées
        Lycée Dindéfélo & \quad\quad Test 9 & 29 Novembre 2024 \\
    \end{tabular}
    \\[-0.01cm] % Ajuster l'espace vertical entre le tableau et la barre
    \hrule % Barre horizontale
\end{center}
\begin{center}
    \textbf{\Large Fonctions-Suites} \\[0.2cm]
    \textbf{\large Professeur : M. BA} \\[0.2cm]
    \textbf{Classe : Terminale S2} \\[0.2cm]
    \textbf{\small Durée : 10 minutes} \\[0.2cm]
    \textbf{\small Note :\quad\quad /5}
\end{center}

% Nom de l'élève
\textbf{\small Nom de l'élève :} \underline{\hspace{8cm}} \\[0.5cm]

% Introduction aux questions
Complétez les exercices suivants en utilisant le cours et vos connaissances. \\[0.3cm]

\question\\
Énoncer le théorème de convergence des suites.\\[0.1cm]
\underline{\hspace{20cm}}\\[0.3cm]
\underline{\hspace{20cm}}

\question\\
Une suite $(u_n)$ est dite \textbf{majorée} si \underline{\hspace{20cm}}\\[0.3cm]
Une suite $(u_n)$ est dite \textbf{bornée} si \underline{\hspace{20cm}}\\[0.3cm]

\question\\
Le point \( I(a ; b) \) est un centre de symétrie de la courbe \( (C_{f}) \) si les deux conditions suivantes sont réalisées :  
\vspace{0.2cm}\\
$\bullet$ \underline{\hspace{20cm}}\\
\vspace{0.2cm}\\
$\bullet$ \underline{\hspace{20cm}}\\
\vspace{0.1cm}\\
Soit \( f(x)=\frac{3x+1}{2x-6} \). Montrer que \( I(3 ; \frac{3}{2}) \) est un centre de symétrie pour la courbe \( (C_{f}) \).\\[0.2cm]
\underline{\hspace{15cm}}\\[0.3cm]
\underline{\hspace{20cm}}

\question\\
\[\text{Si}\lim_{x \to +\infty} f(x) =-\infty \text{ et }\lim_{x \to +\infty}\frac{f(x)}{x}= \underline{\hspace{1cm}}\text{ alors } (C_{f}) \underline{\hspace{10cm}}\]\\
\underline{\hspace{20cm}}\\[0.3cm]
\[\text{Si}\lim_{x \to +\infty} f(x) =+\infty \text{ et }\lim_{x \to +\infty}\frac{f(x)}{x}=\beta \in\mathbb{R}^{*}\text{ et }\lim_{x \to +\infty}[f(x)-\beta x]=+\infty\text{ alors } \underline{\hspace{10cm}}\]\\
\underline{\hspace{20cm}}\\[0.3cm]

\question Énoncer le théorème d’existence et d’unicité d’une solution.\\
\underline{\hspace{20cm}}\\
\underline{\hspace{20cm}}\\
\underline{\hspace{20cm}}\\

\end{document}
