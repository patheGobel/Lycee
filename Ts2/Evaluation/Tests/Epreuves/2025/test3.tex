\documentclass[a4paper,12pt]{article}
\usepackage{graphicx}
\usepackage[a4paper, top=0cm, bottom=2cm, left=2cm, right=2cm]{geometry} % Ajuste les marges
\usepackage{xcolor} % Pour ajouter des couleurs
\usepackage{hyperref} % Pour avoir des références colorées si nécessaire
\usepackage{eso-pic}         % Pour ajouter des éléments en arrière-plan

\usepackage[french]{babel}
\usepackage[T1]{fontenc}
\usepackage{mathrsfs}
\usepackage{amsmath}
\usepackage{amsfonts}
\usepackage{amssymb}
\usepackage{tkz-tab}

\usepackage{tikz}
\usetikzlibrary{arrows, shapes.geometric, fit}
\newcounter{correction} % Compteur pour les questions

% Définir la commande pour afficher une question numérotée
\newcommand{\question}{%
  \refstepcounter{correction}%
  \textbf{\textcolor{black}{Question \thecorrection (1 point) :}} \ignorespaces
}
% Commande pour ajouter du texte en arrière-plan
\AddToShipoutPicture{
    \AtTextCenter{%
        \makebox[0pt]{\rotatebox{80}{\textcolor[gray]{0.9}{\fontsize{10cm}{10cm}\selectfont Pathé Gobel BA}}}
    }
}
\begin{document}
\hrule % Barre horizontale
% En-tête
\begin{center}
    \begin{tabular}{@{} p{5cm} p{5cm} p{5cm} @{}} % 3 colonnes avec largeurs fixées
        Lycée Dindéfélo & Test 3 & 28 Octobre 2024 \\
    \end{tabular}
    \\[-0.01cm] % Ajuster l'espace vertical entre le tableau et la barre
    \hrule % Barre horizontale
\end{center}
\begin{center}
    \textbf{\Large Limites et Continuité} \\[0.2cm]
    \textbf{\large Professeur : M. BA} \\[0.2cm]
    \textbf{Classe : Terminale S2} \\[0.2cm]
    \textbf{\small Durée : 10 minutes} \\[0.2cm]
    \textbf{\small Note :\quad\quad /5}
\end{center}

% Nom de l'élève
\textbf{\small Nom de l'élève :} \underline{\hspace{8cm}} \\[0.5cm]

% Introduction aux questions
Complétez les exercices suivants en utilisant le cours et vos connaissances sur la continuité des fonctions. \\[0.3cm]

\question
\[\text{ Pour calculer la limite \( \lim_{x \to a} g(f(x)) \), il faut d'abord déterminer \( \lim_{x \to a} f(x) = \dots \) puis, en utilisant cette}\]

\[\text{  valeur, calculer \( \lim_{y \to \dots} g(y) = \dots \).}\]

\question\
Complétez la phrase suivante : Une fonction est dite continue sur un intervalle \( I \) si elle est continue \underline{\hspace{15cm}}. \\[0.3cm]
\underline{\hspace{20cm}}

\question\\
\[\text{Si}\lim_{x \to -\infty} h(x) =-\infty \text{ et }\lim_{x \to -\infty}\frac{h(x)}{x}= +\infty\text{ alors } (C_{h}) \underline{\hspace{10cm}}\]\\
\underline{\hspace{20cm}}\\[0.3cm]
\[\text{Si}\lim_{x \to +\infty} h(x) =+\infty \text{ et }\lim_{x \to +\infty}\frac{h(x)}{x}=\gamma \in\mathbb{R}^{*}\text{ et }\lim_{x \to +\infty}[h(x)-\gamma x]=+\infty\text{ alors } \underline{\hspace{10cm}}\]\\
\underline{\hspace{20cm}}\\[0.3cm]
\question\\
Soit \( g(x) = \left\{
\begin{array}{ll}
    x^2 + 1 & \text{si } x \leq 1 \\
    3x - 1 & \text{si } x > 1
\end{array}
\right. \). \\
\[
\lim_{x \to 1^-} g(x) = \underline{\hspace{2cm}}, \quad \lim_{x \to 1^+} g(x) = \underline{\hspace{2cm}}, \quad g(1) = \underline{\hspace{2cm}}
\]
\underline{\hspace{20cm}}\\[0.3cm]
\underline{\hspace{20cm}}

\question\\
Soit \( h \) une fonction continue et croissante sur \( [a;b] \) alors \( h\left( [a;b] \right)=\underline{\hspace{5cm}}\\[0.3cm]  \)\\
Soit \( g \) une fonction continue et décroissante sur \( [a;b[ \) alors \( h\left( [a;b[ \right)=\underline{\hspace{5cm}}\\[0.3cm]  \)\\
\end{document}  