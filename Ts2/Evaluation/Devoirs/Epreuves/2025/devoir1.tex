\documentclass[12pt,a4paper]{article}
\usepackage{amsmath,amssymb,mathrsfs,tikz,times,pifont}
\usepackage{enumitem}
\newcommand\circitem[1]{%
\tikz[baseline=(char.base)]{
\node[circle,draw=gray, fill=red!55,
minimum size=1.2em,inner sep=0] (char) {#1};}}
\newcommand\boxitem[1]{%
\tikz[baseline=(char.base)]{
\node[fill=cyan,
minimum size=1.2em,inner sep=0] (char) {#1};}}
\setlist[enumerate,1]{label=\protect\circitem{\arabic*}}
\setlist[enumerate,2]{label=\protect\boxitem{\alph*}}
%%%::::::by chnini ameur :::::::%%%
\everymath{\displaystyle}
\usepackage[left=1cm,right=1cm,top=1cm,bottom=1.7cm]{geometry}
\usepackage{array,multirow}
\usepackage[most]{tcolorbox}
\usepackage{varwidth}
\tcbuselibrary{skins,hooks}
\usetikzlibrary{patterns}
%%%::::::by chnini ameur :::::::%%%
\newtcolorbox{exa}[2][]{enhanced,breakable,before skip=2mm,after skip=5mm,
colback=yellow!20!white,colframe=black!20!blue,boxrule=0.5mm,
attach boxed title to top left ={xshift=0.6cm,yshift*=1mm-\tcboxedtitleheight},
fonttitle=\bfseries,
title={#2},#1,
% varwidth boxed title*=-3cm,
boxed title style={frame code={
\path[fill=tcbcolback!30!black]
([yshift=-1mm,xshift=-1mm]frame.north west)
arc[start angle=0,end angle=180,radius=1mm]
([yshift=-1mm,xshift=1mm]frame.north east)
arc[start angle=180,end angle=0,radius=1mm];
\path[left color=tcbcolback!60!black,right color = tcbcolback!60!black,
middle color = tcbcolback!80!black]
([xshift=-2mm]frame.north west) -- ([xshift=2mm]frame.north east)
[rounded corners=1mm]-- ([xshift=1mm,yshift=-1mm]frame.north east)
-- (frame.south east) -- (frame.south west)
-- ([xshift=-1mm,yshift=-1mm]frame.north west)
[sharp corners]-- cycle;
},interior engine=empty,
},interior style={top color=yellow!5}}
%%%%%%%%%%%%%%%%%%%%%%%

\usepackage{fancyhdr}
\usepackage{eso-pic}         % Pour ajouter des éléments en arrière-plan
% Commande pour ajouter du texte en arrière-plan
\AddToShipoutPicture{
    \AtTextCenter{%
        \makebox[0pt]{\rotatebox{80}{\textcolor[gray]{0.7}{\fontsize{5cm}{5cm}\selectfont PGB}}}
    }
}
\usepackage{lastpage}
\fancyhf{}
\pagestyle{fancy}
\renewcommand{\footrulewidth}{1pt}
\renewcommand{\headrulewidth}{0pt}
\renewcommand{\footruleskip}{10pt}
\fancyfoot[R]{
\color{blue}\ding{45}\ \textbf{2024}
}
\fancyfoot[L]{
\color{blue}\ding{45}\ \textbf{Prof:M. BA}
}
\cfoot{\bf
\thepage /
\pageref{LastPage}}
\begin{document}
\renewcommand{\arraystretch}{1.5}
\renewcommand{\arrayrulewidth}{1.2pt}
\begin{tikzpicture}[overlay,remember picture]
\node[draw=blue,line width=1.2pt,fill=purple,text=blue,inner sep=3mm,rounded corners,pattern=dots]at ([yshift=-2.5cm]current page.north) {\begingroup\setlength{\fboxsep}{0pt}\colorbox{white}{\begin{tabular}{|*1{>{\centering \arraybackslash}p{0.28\textwidth}} |*2{>{\centering \arraybackslash}p{0.2\textwidth}|} *1{>{\centering \arraybackslash}p{0.19\textwidth}|} }
\hline
\multicolumn{3}{|c|}{$\diamond$$\diamond$$\diamond$\ \textbf{Lycée de Dindéfélo}\ $\diamond$$\diamond$$\diamond$ }& \textbf{A.S. : 2024/2025} \\ \hline
\textbf{Matière: Mathématiques}& \textbf{Niveau : T}\textbf{S2} &\textbf{Date: 09/12/2024} & \textbf{Durée : 4 heures} \\ \hline
\multicolumn{4}{|c|}{\parbox[c]{10cm}{\begin{center}
\textbf{{\Large\sffamily Devoir n$ ^{\circ} $ 1 Du 1$ ^\text{\bf er} $ Semestre}}
\end{center}}} \\ \hline
\end{tabular}}\endgroup};
\end{tikzpicture}
\vspace{3cm}

\section*{\underline{Exercice 1 :} $0,5 \times 8 = 4$ points}
\begin{enumerate}
\item Énoncer le théorème des valeurs intermédiaires.
\item Énoncer le théorème d’existence et d’unicité d’une solution.
\item Énoncer le théorème de l’inégalité des accroissements finis (IAF).
\item \(\text{Si }\lim_{x \to x_0} \frac{f(x) - f(x_0)}{x - x_0} = a \ (a \neq 0) \text{ alors ... }\)
\item \(\text{Si }\lim_{x \to x_0^-} \frac{f(x) - f(x_0)}{x - x_0} = +\infty \text{ alors ... }\)
\item \(\text{Si }\lim_{x \to x_0} \frac{f(x) - f(x_0)}{x - x_0} = 0 \text{ alors ... }\)
\item \(\text{Si }\lim_{x \to +\infty} f(x) =+\infty \text{ et }\lim_{x \to +\infty}\frac{f(x)}{x}=\beta \in\mathbb{R}^{*}\text{ et }\lim_{x \to +\infty}[f(x)-\beta x]=+\infty\text{ alors ...}\)
\item Si \(f\) est continue et strictement décroissante sur \( ]-\infty; b] \), alors \( f(]-\infty; b]) = ... \)
\end{enumerate}

\section*{\underline{Exercice 2 :} 4 points}

\begin{enumerate}
    \item Calculer les limites suivantes : \textbf{(3 × 1 pt)}
    \[
    \lim_{x \to 0} \frac{\sqrt{1+\sin x} - 1}{\sin 2x} \; ; \quad
    \lim_{x \to 0} \frac{\cos x - 1}{x^3 + x^2} \; ; \quad
    \lim_{x \to 1} \frac{\sqrt{x + 3} - \sqrt{5 - x}}{\sqrt{2x + 7} - \sqrt{10 - x}}.
    \]
    \item Donner les primitives des fonctions \(f\) et \(g\) respectivement sur \(\mathbb{R}\) et \(\mathbb{R} \setminus \{1; 2\}\). \textbf{(2 × 0,5 pt)}
    \[
    f(x) = (3x-1)(3x^2-2x+3)^3 \; ; \quad
    g(x) = \frac{1-x^2}{(x^3-3x+2)^3}.
    \]
\end{enumerate}

\section*{\underline{Problème :} 12 points}

\underline{\textbf{Partie A :}}

Soit \( f \) la fonction définie par :
\[
f(x) = x - 2 - \sqrt{x^2 - 2x}.
\]

\begin{enumerate}
\item  Déterminer \( D_f \). \hspace{1cm} \textbf{(0,5 pt)}

\begin{enumerate}
    \item Calculer \( \lim_{x \to +\infty} f(x) \) et \( \lim_{x \to -\infty} f(x) \). \hspace{1cm} (0,25 pt), \textbf{(0,5 pt)}
    \item Étudier la branche infinie de la courbe \( (C_f) \) au voisinage de \( -\infty \). \hspace{1cm} \textbf{(0,75 pt)}
    \item Interpréter la limite de \( f \) au voisinage de \( +\infty \). \hspace{1cm} \textbf{(0,75 pt)}
\end{enumerate}

\item Étudier la dérivabilité de la fonction \( f \) à droite de 2 et à gauche de 0, puis interpréter géométriquement les résultats obtenus. \hspace{1cm} \textbf{(2 pt)}

\begin{enumerate}
    \item Justifier la dérivabilité de la fonction sur \( ]-\infty, 0[ \cup ]2, +\infty[ \), puis\\ montrer que pour tout \( x \in ]-\infty, 0[ \cup ]2, +\infty[ \) :
    \(
    f'(x) = \frac{\sqrt{x^2 - 2x} - (x - 1)}{\sqrt{x^2 - 2x}}.
    \)    \hspace{1cm} \textbf{(1,5 pt)}

    
    \item Montrer que : \( \forall x \in ]-\infty, 0], f'(x) > 0 \) et \( \forall x \in ]2, +\infty[, f'(x) < 0 \). \hspace{1cm} \textbf{(1 pt)}
    
    \item Dresser le tableau de variations de la fonction \( f \). \hspace{1cm} \textbf{(1,25 pt)}
\end{enumerate}

\item Tracer la courbe \( (C_f) \) dans un repère orthonormé \( (O, \vec{i}, \vec{j}) \). \hspace{1cm} \textbf{(1,25 pt)}

\underline{\textbf{Partie B :}}

On considère la fonction \( g \) la restriction de la fonction \( f \) sur \( [2, +\infty[ \) :

\begin{enumerate}
    \item Montrer que \( g \) admet une fonction réciproque \( g^{-1} \) définie sur un intervalle \( J \)\\ que l’on déterminera . \hspace{1cm} \textbf{(0,5 pt)}
    
    \item Calculer \( g^{-1}(2 - 2\sqrt{2}) \). (On donne : \( g(4) = (2 - 2\sqrt{2} \)). \hspace{1cm} \textbf{(0,75 pt)}
    
    \item Déterminer \( g^{-1}(x) \) pour tout \( x \in J \). \hspace{3cm} \textbf{(0,5 pt)}
    
    \item Tracer la courbe \( (C_{g^{-1}}) \) dans le même repère orthonormé \( (O, \vec{i}, \vec{j}) \). \hspace{2cm} \textbf{(0,5 pt)}
\end{enumerate}

\end{enumerate}

\end{document}
