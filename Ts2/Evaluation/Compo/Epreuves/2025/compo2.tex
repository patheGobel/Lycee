\documentclass[12pt,a4paper]{article}
\usepackage{amsmath,amssymb,mathrsfs,tikz,times,pifont}
\usepackage[T1]{fontenc}
\usepackage{enumitem}
\newcommand\circitem[1]{%
\tikz[baseline=(char.base)]{
\node[circle,draw=gray, fill=red!55,
minimum size=1.2em,inner sep=0] (char) {#1};}}
\newcommand\boxitem[1]{%
\tikz[baseline=(char.base)]{
\node[fill=cyan,
minimum size=1.2em,inner sep=0] (char) {#1};}}
\setlist[enumerate,1]{label=\protect\circitem{\arabic*}}
\setlist[enumerate,2]{label=\protect\boxitem{\alph*}}
%%%::::::by chnini ameur :::::::%%%
\everymath{\displaystyle}
\usepackage[left=1cm,right=1cm,top=1cm,bottom=1.7cm]{geometry}
\usepackage[colorlinks=true, linkcolor=blue, urlcolor=blue, citecolor=blue]{hyperref}
\usepackage{array,multirow}
\usepackage[most]{tcolorbox}
\usepackage{varwidth}
\usepackage{float} %pour utiliser l'option [H] qui force l'image à apparaître exactement à l'endroit où elle est placée dans le code.
\tcbuselibrary{skins,hooks}
\usetikzlibrary{patterns}
%%%::::::by chnini ameur :::::::%%%
\newtcolorbox{exa}[2][]{enhanced,breakable,before skip=2mm,after skip=5mm,
colback=yellow!20!white,colframe=black!20!blue,boxrule=0.5mm,
attach boxed title to top left ={xshift=0.6cm,yshift*=1mm-\tcboxedtitleheight},
fonttitle=\bfseries,
title={#2},#1,
% varwidth boxed title*=-3cm,
boxed title style={frame code={
\path[fill=tcbcolback!30!black]
([yshift=-1mm,xshift=-1mm]frame.north west)
arc[start angle=0,end angle=180,radius=1mm]
([yshift=-1mm,xshift=1mm]frame.north east)
arc[start angle=180,end angle=0,radius=1mm];
\path[left color=tcbcolback!60!black,right color = tcbcolback!60!black,
middle color = tcbcolback!80!black]
([xshift=-2mm]frame.north west) -- ([xshift=2mm]frame.north east)
[rounded corners=1mm]-- ([xshift=1mm,yshift=-1mm]frame.north east)
-- (frame.south east) -- (frame.south west)
-- ([xshift=-1mm,yshift=-1mm]frame.north west)
[sharp corners]-- cycle;
},interior engine=empty,
},interior style={top color=yellow!5}}
%%%%%%%%%%%%%%%%%%%%%%%

\usepackage{fancyhdr}
\usepackage{eso-pic}         % Pour ajouter des éléments en arrière-plan
% Commande pour ajouter du texte en arrière-plan
\usepackage{tkz-tab}
\AddToShipoutPicture{
    \AtTextCenter{%
        \makebox[0pt]{\rotatebox{80}{\textcolor[gray]{0.7}{\fontsize{5cm}{5cm}\selectfont PGB}}}
    }
}
\usepackage{lastpage}
\fancyhf{}
\pagestyle{fancy}
\renewcommand{\footrulewidth}{1pt}
\renewcommand{\headrulewidth}{0pt}
\renewcommand{\footruleskip}{10pt}
\fancyfoot[R]{
\color{blue}\ding{45}\ \textbf{2025}
}
\fancyfoot[L]{
\color{blue}\ding{45}\ \textbf{Prof:M. BA}
}
\cfoot{\bf
\thepage /
\pageref{LastPage}}
\begin{document}
\renewcommand{\arraystretch}{1.5}
\renewcommand{\arrayrulewidth}{1.2pt}
\begin{tikzpicture}[overlay,remember picture]
\node[draw=blue,line width=1.2pt,fill=purple,text=blue,inner sep=3mm,rounded corners,pattern=dots]at ([yshift=-2.5cm]current page.north) {\begingroup\setlength{\fboxsep}{0pt}\colorbox{white}{\begin{tabular}{|*1{>{\centering \arraybackslash}p{0.28\textwidth}} |*2{>{\centering \arraybackslash}p{0.2\textwidth}|} *1{>{\centering \arraybackslash}p{0.19\textwidth}|} }
\hline
\multicolumn{3}{|c|}{$\diamond$$\diamond$$\diamond$\ \textbf{Lycée de Dindéfélo}\ $\diamond$$\diamond$$\diamond$ }& \textbf{A.S. : 2024/2025} \\ \hline
\textbf{Matière: Mathématiques}& \textbf{Niveau : T}\textbf{S2} &\textbf{Date: 22/05/2025} & \textbf{Durée : 4 heures} \\ \hline
\multicolumn{4}{|c|}{\parbox[c]{10cm}{\begin{center}
\textbf{{\Large\sffamily Composition Du 2$ ^\text{\bf nd} $ Semestre}}
\end{center}}} \\ \hline
\end{tabular}}\endgroup};
\end{tikzpicture}
\vspace{3cm}
\section*{\underline{Exercice 1 :($3$ pts)} Restitution de Connaissances}
\begin{enumerate}
    \item Soit \( \Omega \) l’univers associé à une expérience aléatoire \( E \) et \( p \) une probabilité définie sur \( \Omega \).\\
    Recopie et complète les relations ci-dessous :
    \begin{enumerate}
        \item \( \mathbb{P}(\Omega) = \ldots \) \hfill \textbf{(0,25 pt)}
        \item \( \mathbb{P}(\varnothing) = \ldots \) \hfill \textbf{(0,25 pt)}
        \item Si \( A \) et \( B \) sont deux événements incompatibles de \( \Omega \), alors \( \mathbb{P}(A \cup B) = \ldots \)\hfill \textbf{(0,25 pt)}
        \item Soit \( D \) un événement quelconque de \( \Omega \). \( \mathbb{P}(D) = 1{,}5 \) est-il possible ?\\
        Si non, justifier votre réponse. \hfill \textbf{(0,25 pt)}
    \end{enumerate}
    
    \item Soit \( f \) une fonction continue sur un intervalle \( I \) et \( (u_n) \) une suite convergente vers un nombre réel \( L \in I \), définie par \( u_{n+1} = f(u_n) \).\\
    Répondre par vrai ou faux à l’affirmation : \textbf{\( L \) est solution de l’équation \( f(L) = L \)}. \hfill \textbf{(0,5 pt)}
    
    \item Soit \( (u_n) \) une suite géométrique de raison \( q = \frac{1}{2} \) et de premier terme \( u_2 = -3 \).\\
    Choisir la bonne réponse dans chaque cas : \hfill (\( \textbf{3} \times \textbf{0,25 pt} \))

    \begin{center}
    \renewcommand{\arraystretch}{1.5}
    \begin{tabular}{|>{\centering\arraybackslash}m{5cm}|>{\centering\arraybackslash}m{3cm}|>{\centering\arraybackslash}m{3cm}|>{\centering\arraybackslash}m{3cm}|}
        \hline
        \textbf{Réponses} & \textbf{A} & \textbf{B} & \textbf{C} \\
        \hline
        \( \lim u_n \) est : & \( -\infty \) & \( +\infty \) & \( 0 \) \\
        \hline
        L’expression de \( u_n \) est : & 
        \( -3\left( \frac{1}{2} \right)^n \) & 
        \( -3\left( \frac{1}{2} \right)^{n-3} \) & 
        \( -3\left( \frac{1}{2} \right)^{n-2} \) \\
        \hline
        L’expression de \( S_n = u_2 + u_3 + \cdots + u_n \) est : & 
        \( u_0 \times \dfrac{1 - 0{,}5^{n-1}}{0{,}5} \) & 
        \( u_2 \times \dfrac{1 - 0{,}5^{n-2}}{0{,}5} \) & 
        \( u_2 \times \dfrac{1 - 0{,}5^{n-2}}{0{,}5} \) \\
        \hline
    \end{tabular}
    \end{center}

\end{enumerate}

\section*{\underline{Exercice 2 :($3$ pts)} }
Le plan complexe est muni d’un repère orthonormé direct \( (O ; \vec{u}, \vec{v}) \). 

\begin{enumerate}
    \item On considère la transformation \( S \) du plan d’écriture complexe \( z' = (1 - i\sqrt{3})z + 2 \).\\
    Déterminer la nature de \( S \). \hfill \textbf{(0,5 pt)}
    
    \item Déterminer le rapport et l’angle de \( S \). \hfill \textbf{(0,5 pt)}
    
    \item Déterminer l’affixe du point \( C \) image par \( S \) du point \( A(2 - i\sqrt{3}) \). \hfill \textbf{(1 pt)}
    
    \item Quelle est l’affixe du point image par \( S \) du point \( D\left(-\dfrac{2\sqrt{3}}{3}i\right) \) ?\\
    Que représente \( D \) pour la transformation \( S \) ? \hfill \textbf{(0,75 +0,25 pt)}
\end{enumerate}
\section*{\underline{Exercice 3 :($4$ pts)} }

On dispose de deux urnes identiques \( u_1 \) et \( u_2 \) contenant des boules indiscernables au toucher :\\
\( u_1 \) contient 3 boules blanches et 6 boules noires.\\
\( u_2 \) contient deux boules noires, une blanche et une rouge.\\

Une épreuve consiste à tirer au hasard une boule dans \( u_1 \), la mettre dans \( u_2 \), et tirer ensuite au hasard une boule dans \( u_2 \).\\
On note \( B_k \) l'événement : « Tirer une boule blanche dans \( u_k \) », \( N_k \) l'événement : « Tirer une boule noire dans \( u_k \) », avec \( k\in \{ 1,2 \} \).\\
\( R \) l'événement : « Tirer une boule rouge dans \( u_2 \) ».
\begin{enumerate}
    \item Construire un arbre pondéré correspondant à cette épreuve. \hfill \textbf{(0,5 pt)}
    \item 
    \begin{enumerate}
        \item Montrer que \( \mathbb{P}(N_2) = \dfrac{8}{15} \). \hfill \textbf{(0,5 pt)}
        \item Déterminer la probabilité de l’événement \( B_2 \). \hfill \textbf{(0,5 pt)}
        \item Déterminer la probabilité de tirer une boule blanche de \( u_1 \), sachant que la boule tirée dans \( u_2 \) est noire. \hfill \textbf{(0,5 pt)}
    \end{enumerate}

    \item Un joueur mise 500F et effectue une épreuve. Si à la fin de l’épreuve le joueur tire une boule blanche dans \( u_2 \), il reçoit 3000F ; si la boule tirée dans \( u_2 \) est noire, le joueur ne reçoit rien et si elle est rouge, il reçoit 500F.\\
    On désigne \( X \) le gain du joueur (gain = différence entre ce qu’il reçoit et sa mise).

    \begin{enumerate}
        \item Donner la loi de probabilité de \( X \). \hfill \textbf{(0,5 pt)}
        \item Calculer l’espérance mathématique, la variance et l’écart-type de \( X \). \hfill \textbf{(0,75 pt)}
        \item Déterminer la fonction de répartition de \( X \) et la représenter. \hfill \textbf{(0,75 pt)}
    \end{enumerate}

    \item Un joueur participe à plusieurs parties du jeu et on suppose que les épreuves sont indépendantes.\\
    Quelle est le nombre minimal de parties pour que la probabilité de réaliser au moins une fois l’événement \( X = 2500 \) soit supérieure à 0{,}97 ? \hfill \textbf{(0,25 pt)}
\end{enumerate}

\section*{\underline{Problème :}\quad\textbf{10 pts}} 
Le plan est muni d’un repère orthonormé \( (O ; I ; J) \) et \( \mathcal{C}_f \) la courbe représentative de \( f \) dans ce plan.

\vspace{0.4em}
\subsection*{\underline{Partie A:}\quad\textbf{2,5 pts}}
On considère la fonction \( g \) définie sur \( ]0 ; +\infty[ \) par \( g(x) = \ln\left( \frac{x+1}{x} \right) - \frac{1}{x+1} \).

\begin{enumerate}
    \item Calculer les limites \( \lim\limits_{x \to 0^+} g(x) \) et \( \lim\limits_{x \to +\infty} g(x) \). \hfill \textbf{(0,5 pt)}
    \item Étudier les variations de \( g \) et dresser le tableau de variations de \( g \). \hfill \textbf{(1,5 pt)}
    \item Déduire du tableau de variations le signe de \( g(x) \) pour tout \( x > 0 \). \hfill \textbf{(0,5 pt)}
\end{enumerate}

\vspace{0.4em}
\subsection*{\underline{Partie B:}\quad\textbf{5 pts}}
On considère la fonction \( f \) donnée par :
\(
f(x) =
\begin{cases}
x \ln\left( \frac{x+1}{x} \right) + 1 & \text{si } x > 0 \\
(x^2 - 3x + 1)e^x & \text{si } x \leq 0
\end{cases}
\)

\begin{enumerate}
    \item 
    \begin{enumerate}
        \item Déterminer le domaine de définition de \( f \). \hfill \textbf{(0,5 pt)}
        \item Calculer les limites aux bornes de \( D_f \). \hfill \textbf{(0,5 pt)}
        \item En déduire l’existence d’asymptotes dont on précisera la nature et l’équation. \hfill \textbf{(0,5 pt)}
    \end{enumerate}

    \item 
    \begin{enumerate}
        \item Étudier la continuité de \( f \) en 0. \hfill \textbf{(0,75 pt)}
        \item Étudier la dérivabilité de \( f \) en 0 et interpréter les résultats. \hfill \textbf{(0,75 + 0,25 pt)}
    \end{enumerate}

    \item Montrer que 
    \(
    f'(x) = 
    \begin{cases}
    g(x) & \text{si } x > 0 \\
    (x^2 - x - 2)e^x & \text{si } x < 0
    \end{cases}
    \) \hfill \textbf{(1 pt)}

    \item Dresser le tableau de variations de \( f \). \hfill \textbf{(0,75 pt)}
\end{enumerate}

\vspace{0.4em}
\subsection*{\underline{Partie C:}\quad\textbf{2,5 pts}}
Soit \( h \) la restriction de \( f \) sur l’intervalle \( ]0 ; +\infty[ \).
\begin{enumerate}
    \item Montrer que \( h \) est bijective de \( ]0 ; +\infty[ \) vers un intervalle \( J \) à préciser. \hfill \textbf{(0,5 pt)}
    \item Étudier la dérivabilité de \( h^{-1} \) sur \( J \). \hfill \textbf{(0,5 pt)}
    \item Tracer sur le même graphe les asymptotes, \( \mathcal{C}_f \) et \( \mathcal{C}_{h^{-1}} \). \hfill \textbf{(1,5 pt)}
\end{enumerate}
\end{document}