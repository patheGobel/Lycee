\documentclass[a4paper,12pt]{article}
\usepackage{graphicx}
\usepackage[a4paper, top=0cm, bottom=2cm, left=2cm, right=2cm]{geometry} % Ajuste les marges
\usepackage{xcolor} % Pour ajouter des couleurs
\usepackage{hyperref} % Pour avoir des références colorées si nécessaire
\usepackage{eso-pic}         % Pour ajouter des éléments en arrière-plan

\usepackage[french]{babel}
\usepackage[T1]{fontenc}
\usepackage{mathrsfs}
\usepackage{amsmath}
\usepackage{amsfonts}
\usepackage{amssymb}
\usepackage{tkz-tab}

\usepackage{tikz}
\usetikzlibrary{arrows, shapes.geometric, fit}
\newcounter{correction} % Compteur pour les questions

% Définir la commande pour afficher une question numérotée

% Commande pour ajouter du texte en arrière-plan
\AddToShipoutPicture{
    \AtTextCenter{%
        \makebox[0pt]{\rotatebox{80}{\textcolor[gray]{0.6}{\fontsize{10cm}{10cm}\selectfont PGB}}}
    }
}
\begin{document}
\hrule % Barre horizontale
% En-tête
\begin{center}
    \begin{tabular}{@{} p{5cm} p{5cm} p{5cm} @{}} % 3 colonnes avec largeurs fixées
        Aïssatou Diao & \quad\quad\quad 18H-20H & 10 Juin 2025 \\
    \end{tabular}
    \\[-0.01cm] % Ajuster l'espace vertical entre le tableau et la barre
    \hrule % Barre horizontale
\end{center}

\textbf{\underline{Problème} : \textcolor{red}{(8 points)} }

\textbf{Partie A :} On considère la fonction $g$ définie sur $\mathbb{R}$ par : $g(x) = x^3 + 3x + 2$

\begin{enumerate}
    \item Étudier les limites de $g$, aux bornes de $D_g$.
    \item Calculer l'expression de sa dérivée $g'(x)$, puis dresser le tableau de variations de la fonction $g$.
    \item Montrer que l'équation $g(x) = 0$ admet une unique solution $\alpha$ dans l'intervalle $[-1;0]$
    \item Donner une valeur approchée de $\alpha$ à $10^{-1}$ près, en utilisant la méthode de dichotomie.
    \item Déduire des questions précédentes le signe de la fonction $g$ suivant les valeurs de $x$.
\end{enumerate}

\vspace{0.5cm}

\textbf{Partie B :} On considère la fonction $f$ définie sur $\mathbb{R}$ par : 
\[
f(x) = \frac{x^3 - 1}{x^2 + 1}
\]

On note aussi $(\mathcal{C}_f)$ sa représentation graphique dans le plan muni d’un repère orthogonal $(O, \vec{i}, \vec{j})$

\begin{enumerate}
    \item[1./] Étudier les limites de $f$, aux bornes de $D_f$.
    \item[2./] Montrer que, pour tout $x \in \mathbb{R}$, 
    \[
    f'(x) = \frac{xg(x)}{(x^2+1)^2}
    \]
    \item[3./] En déduire le tableau de variations de la fonction $f$. (On précisera la valeur de $f(\alpha)$)
    \item[4./] Déterminer l’équation de la tangente $(T)$ à $(\mathcal{C}_f)$ au point d’abscisse $x_0 = 0$
    \item[5.a)] Montrer que $(\mathcal{C}_f)$ admet une asymptote oblique $(\mathcal{A.O})$ (on précisera son équation) en $\pm\infty$.
    
    \item[b)] Tracer $(T)$, $(\mathcal{A.O})$ et $(\mathcal{C}_f)$. (unités graphiques conseillées : $1\| \vec{i} \| = 2~\text{cm}$, $1\| \vec{j} \| = 4~\text{cm}$)
    
    \item[6./] Construire dans le même repère la courbe de $h$ définie par : 
    \[
    h(x) = |f(x)|
    \]
\end{enumerate}

\end{document}
