\documentclass[a4paper,12pt]{article}

\usepackage[utf8]{inputenc}
\usepackage[T1]{fontenc}
\usepackage[french]{babel}
\usepackage{amsmath, amssymb, amsfonts}
\usepackage{graphicx}
\usepackage{tikz}
\usepackage{tkz-tab}
\usepackage{xcolor}
\usepackage{hyperref}
\usepackage{eso-pic}
\usepackage[a4paper, top=0cm, bottom=2cm, left=2cm, right=2cm]{geometry}

\usetikzlibrary{arrows, shapes.geometric, fit}
\newcounter{correction}

% Filigrane en fond
\AddToShipoutPicture{
    \AtTextCenter{%
        \makebox[0pt]{\rotatebox{80}{\textcolor[gray]{0.6}{\fontsize{10cm}{10cm}\selectfont PGB}}}
    }
}

\begin{document}

\hrule
\begin{center}
    \begin{tabular}{@{} p{5cm} p{5cm} p{5cm} @{}} 
        Aïssatou Diao & \quad\quad\quad 18h–20h & 10 juin 2025 \\
    \end{tabular}
    \\[-0.01cm]
    \hrule
\end{center}

\vspace{0.3cm}
\textbf{\underline{Problème} : \textcolor{red}{(8 points)}}

\section*{Partie A}

\begin{enumerate}
    \item \textbf{Étudier les limites de \( g(x) = x^3 + 3x + 2 \) aux bornes de \( \mathbb{R} \) :}

    \begin{itemize}
        \item Limite quand \( x \to +\infty \) :
        \[
        \lim_{x \to +\infty} g(x) = \lim_{x \to +\infty} (x^3 + 3x + 2) = +\infty
        \]
        \item Limite quand \( x \to -\infty \) :
        \[
        \lim_{x \to -\infty} g(x) = \lim_{x \to -\infty} (x^3 + 3x + 2) = -\infty
        \]
    \end{itemize}

    \item \textbf{Calculer la dérivée \( g'(x) \) et dresser le tableau de variations :}

    \[
    g'(x) = 3x^2 + 3 = 3(x^2 + 1)
    \]
    Comme \( x^2 + 1 > 0 \) pour tout \( x \in \mathbb{R} \), on a :

    \[
    g'(x) > 0 \quad \text{pour tout } x \in \mathbb{R}
    \]
    Donc, \( g \) est strictement croissante sur \( \mathbb{R} \).

    \begin{center}
    \begin{tikzpicture}
        \tkzTabInit[lgt=1,espcl=3]{$x$/1, $g'(x)$/1, $g(x)$/1}{$-\infty$, $+\infty$}
        \tkzTabLine{,+,}
        \tkzTabVar{-/$-\infty$, +/$+\infty$}
    \end{tikzpicture}
    \end{center}

    \item \textbf{Montrer que l’équation \( g(x) = 0 \) admet une unique solution \( \alpha \) dans \( [-1, 0] \) :}

    Calcul aux bornes :
    \[
    g(-1) = (-1)^3 + 3(-1) + 2 = -1 - 3 + 2 = -2 < 0
    \]
    \[
    g(0) = 0 + 0 + 2 = 2 > 0
    \]

    Comme \( g \) est continue et strictement croissante sur \( [-1, 0] \), il existe une unique solution \( \alpha \in [-1, 0] \) telle que \( g(\alpha) = 0 \), d’après le théorème des valeurs intermédiaires.

    \item \textbf{Donner une valeur approchée de \( \alpha \) à \( 10^{-1} \) près par dichotomie :}

    \begin{itemize}
        \item \( x = -0{,}5 \Rightarrow g(-0{,}5) = (-0{,}5)^3 + 3(-0{,}5) + 2 = -0{,}125 - 1{,}5 + 2 = 0{,}375 > 0 \Rightarrow \alpha \in [-1, -0{,}5] \)
        \item \( x = -0{,}75 \Rightarrow g(-0{,}75) = -0{,}422 - 2{,}25 + 2 = -0{,}672 < 0 \Rightarrow \alpha \in [-0{,}75, -0{,}5] \)
        \item \( x = -0{,}6 \Rightarrow g(-0{,}6) = -0{,}216 - 1{,}8 + 2 = -0{,}016 < 0 \Rightarrow \alpha \in [-0{,}6, -0{,}5] \)
        \item \( x = -0{,}55 \Rightarrow g(-0{,}55) = -0{,}166 - 1{,}65 + 2 = 0{,}184 > 0 \Rightarrow \alpha \in [-0{,}6, -0{,}55] \)
    \end{itemize}

    Donc une valeur approchée de \( \alpha \) à \( 10^{-1} \) près est :
    \[
    \boxed{-0{,}6}
    \]

\end{enumerate}
\section*{Partie B}

\begin{enumerate}
\setcounter{enumi}{4}

\item \textbf{Montrer que \( f'(x) = \dfrac{x g(x)}{(x^2 + 1)^2} \)}

Posons \( f(x) = \dfrac{x^3 - 1}{x^2 + 1} \), donc :
\[
u(x) = x^3 - 1, \quad v(x) = x^2 + 1
\]
\[
u'(x) = 3x^2, \quad v'(x) = 2x
\]
\[
f'(x) = \frac{u'(x)v(x) - u(x)v'(x)}{v(x)^2} = \frac{3x^2(x^2 + 1) - (x^3 - 1)(2x)}{(x^2 + 1)^2}
\]

Développons le numérateur :
\[
3x^2(x^2 + 1) = 3x^4 + 3x^2, \quad (x^3 - 1)(2x) = 2x^4 - 2x
\]
\[
f'(x) = \frac{3x^4 + 3x^2 - 2x^4 + 2x}{(x^2 + 1)^2} = \frac{x^4 + 3x^2 + 2x}{(x^2 + 1)^2}
\]

Factorisons :
\[
x^4 + 3x^2 + 2x = x(x^3 + 3x + 2) = x g(x)
\]

Donc :
\[
\boxed{f'(x) = \frac{x g(x)}{(x^2 + 1)^2}}
\]

\vspace{0.4cm}

\item \textbf{Étudier les variations de \( f \)}

Les points critiques sont donnés par \( f'(x) = 0 \), soit :
\[
x g(x) = 0 \Rightarrow x = 0 \quad \text{ou} \quad g(x) = 0 \Rightarrow x = \alpha \approx -0{,}6
\]

On étudie le signe de \( f'(x) \) selon les intervalles :

\begin{itemize}
    \item Pour \( x < \alpha \), on a \( x < 0 \) et \( g(x) < 0 \) donc \( x g(x) > 0 \)
    \item Pour \( \alpha < x < 0 \), on a \( x < 0 \), \( g(x) > 0 \) donc \( x g(x) < 0 \)
    \item Pour \( x > 0 \), on a \( x > 0 \), \( g(x) > 0 \) donc \( x g(x) > 0 \)
\end{itemize}

Donc :
\[
f'(x)
\begin{cases}
> 0 & \text{si } x \in (-\infty, \alpha) \\
< 0 & \text{si } x \in (\alpha, 0) \\
> 0 & \text{si } x \in (0, +\infty)
\end{cases}
\]

\textbf{Valeur particulière :} \( f(\alpha) = \dfrac{\alpha^3 - 1}{\alpha^2 + 1} \)

\textbf{Tableau de variations (schématique)} :

\begin{center}
\begin{tikzpicture}
\tkzTabInit[lgt=1.2, espcl=2.2]{$x$/1, $f'(x)$/1, $f(x)$/2}{$-\infty$, $\alpha$, $0$, $+\infty$}
\tkzTabLine{,+, -, +,}
\tkzTabVar{-/ , +/$f(\alpha)$, -/$f(0) = -1$, +/}
\end{tikzpicture}
\end{center}

\vspace{0.4cm}

\item \textbf{Déterminer l’équation de la tangente \( T \) à la courbe \( (C_f) \) au point d’abscisse \( x_0 = 0 \)}

\[
f(0) = \frac{0^3 - 1}{0^2 + 1} = -1, \quad
f'(0) = \frac{0 \cdot g(0)}{(0^2 + 1)^2} = 0
\]

Donc la tangente \( T \) est horizontale et passe par le point \( (0, -1) \). Son équation est donc :

\[
\boxed{y = -1}
\]

\end{enumerate}

\end{document}
