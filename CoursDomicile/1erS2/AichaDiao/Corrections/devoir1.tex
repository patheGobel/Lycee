\documentclass[a4paper,12pt]{article}
\usepackage[utf8]{inputenc}
\usepackage[french]{babel}
\usepackage{amsmath, amssymb}
\usepackage{geometry}
\geometry{top=2cm, bottom=2cm, left=2cm, right=2cm}

\begin{document}

\begin{center}
    \textbf{Inspection d'académie de Tamba} \\
    \textbf{COLLÈGE JEAN 23} \\
    \vspace{0.4cm}
    Prof : M. TRAORÉ \\
    \textbf{DEVOIR DU SECOND SEMESTRE} \\
    \textbf{Épreuve : MATHÉMATIQUES} \\
    Année scolaire : 2023/2024 \\
    Classe : 1\textsuperscript{S}2 \\
    Durée : 2h
\end{center}

\vspace{0.5cm}

\section*{Exercice 1 \hfill (15,5 points)}

\begin{enumerate}
    \item Déterminer le domaine de définition de chacune des fonctions suivantes : \hfill (0,5 + 1 + 1,5 + 1 pts)

    \item Étudier la parité des fonctions suivantes : \hfill (1 + 1 pt)

    \item On considère les fonctions \( f \) et \( g \) :
    \begin{itemize}
        \item[a)] Déterminer \( D_{f \circ g} \) et \( D_{g \circ f} \) \hfill (1 + 1 pt)
        \item[b)] Expliciter \( f \circ g(x) \) \hfill (1,5 pt)
    \end{itemize}

    \item Soit \( f(x) = 3x^2 - 4x + 1 \). \\
    Montrer que la droite \( \Delta : x = \frac{2}{3} \) est un axe de symétrie pour la courbe \( \mathcal{C}_f \). \hfill (1,5 pt)

    \item Soit \( g(x) = \dfrac{x^2}{x - 1} \). \\
    Montrer que le point \( I(1; 2) \) est un centre de symétrie pour la courbe \( \mathcal{C}_f \). \hfill (1,5 pt)

    \item Soit \( h(x) = 3x + 2 + |x - 2| \) :
    \begin{itemize}
        \item[a)] Écrire \( h(x) \) sans le symbole de la valeur absolue \hfill (1,5 pt)
        \item[b)] Donner la restriction de \( h \) sur \( [-\frac{2}{3} ; 2] \) \hfill (1,5 pt)
    \end{itemize}
\end{enumerate}

\vspace{0.5cm}

\section*{Exercice 2 \hfill (4,5 points)}

On donne les applications suivantes :
\[
\begin{aligned}
g &: [0 ; +\infty[ \to \mathbb{R},\quad x \mapsto g(x) = x^2 + 1 \\
h &: \mathbb{R} \to [-1 ; +\infty[, \quad x \mapsto h(x) = x^2 - 1 \\
j &: \mathbb{R} \to \mathbb{R},\quad x \mapsto j(x) = 3x + 2
\end{aligned}
\]

Montrer que :
\begin{itemize}
    \item[a)] \( g \) est injective \hfill (1,5 pt)
    \item[b)] \( h \) est surjective \hfill (1,5 pt)
    \item[c)] \( j \) est bijective \hfill (1,5 pt)
\end{itemize}

\vfill

\begin{flushright}
Mars 2024 \\
Prof : M. TRAORÉ
\end{flushright}

\end{document}
