\documentclass[12pt,a4paper]{article}
\usepackage{amsmath,amssymb,mathrsfs,tikz,times,pifont}
\usepackage[T1]{fontenc}
\usepackage{enumitem}
\newcommand\circitem[1]{%
\tikz[baseline=(char.base)]{
\node[circle,draw=gray, fill=red!55,
minimum size=1.2em,inner sep=0] (char) {#1};}}
\newcommand\boxitem[1]{%
\tikz[baseline=(char.base)]{
\node[fill=cyan,
minimum size=1.2em,inner sep=0] (char) {#1};}}
\setlist[enumerate,1]{label=\protect\circitem{\arabic*}}
\setlist[enumerate,2]{label=\protect\boxitem{\alph*}}
%%%::::::by chnini ameur :::::::%%%
\everymath{\displaystyle}
\usepackage[left=1cm,right=1cm,top=1cm,bottom=1.7cm]{geometry}
\usepackage[colorlinks=true, linkcolor=blue, urlcolor=blue, citecolor=blue]{hyperref}
\usepackage{array,multirow}
\usepackage[most]{tcolorbox}
\usepackage{varwidth}
\usepackage{float} %pour utiliser l'option [H] qui force l'image à apparaître exactement à l'endroit où elle est placée dans le code.
\tcbuselibrary{skins,hooks}
\usetikzlibrary{patterns}
%%%::::::by chnini ameur :::::::%%%
\newtcolorbox{exa}[2][]{enhanced,breakable,before skip=2mm,after skip=5mm,
colback=yellow!20!white,colframe=black!20!blue,boxrule=0.5mm,
attach boxed title to top left ={xshift=0.6cm,yshift*=1mm-\tcboxedtitleheight},
fonttitle=\bfseries,
title={#2},#1,
% varwidth boxed title*=-3cm,
boxed title style={frame code={
\path[fill=tcbcolback!30!black]
([yshift=-1mm,xshift=-1mm]frame.north west)
arc[start angle=0,end angle=180,radius=1mm]
([yshift=-1mm,xshift=1mm]frame.north east)
arc[start angle=180,end angle=0,radius=1mm];
\path[left color=tcbcolback!60!black,right color = tcbcolback!60!black,
middle color = tcbcolback!80!black]
([xshift=-2mm]frame.north west) -- ([xshift=2mm]frame.north east)
[rounded corners=1mm]-- ([xshift=1mm,yshift=-1mm]frame.north east)
-- (frame.south east) -- (frame.south west)
-- ([xshift=-1mm,yshift=-1mm]frame.north west)
[sharp corners]-- cycle;
},interior engine=empty,
},interior style={top color=yellow!5}}
%%%%%%%%%%%%%%%%%%%%%%%

\usepackage{fancyhdr}
\usepackage{eso-pic}         % Pour ajouter des éléments en arrière-plan
% Commande pour ajouter du texte en arrière-plan
\usepackage{tkz-tab}
\AddToShipoutPicture{
    \AtTextCenter{%
        \makebox[0pt]{\rotatebox{80}{\textcolor[gray]{0.7}{\fontsize{5cm}{5cm}\selectfont PGB}}}
    }
}
\usepackage{lastpage}
\fancyhf{}
\pagestyle{fancy}
\renewcommand{\footrulewidth}{1pt}
\renewcommand{\headrulewidth}{0pt}
\renewcommand{\footruleskip}{10pt}
\fancyfoot[R]{
\color{blue}\ding{45}\ \textbf{2025}
}
\fancyfoot[L]{
\color{blue}\ding{45}\ \textbf{Prof:M. BA}
}
\cfoot{\bf
\thepage /
\pageref{LastPage}}
\usetikzlibrary{trees} % Bibliothèque pour les arbres
\begin{document}
\renewcommand{\arraystretch}{1.5}
\renewcommand{\arrayrulewidth}{1.2pt}
\begin{tikzpicture}[overlay,remember picture]
\node[draw=blue,line width=1.2pt,fill=purple,text=blue,inner sep=3mm,rounded corners,pattern=dots]at ([yshift=-2.5cm]current page.north) {\begingroup\setlength{\fboxsep}{0pt}\colorbox{white}{\begin{tabular}{|*1{>{\centering \arraybackslash}p{0.28\textwidth}} |*2{>{\centering \arraybackslash}p{0.2\textwidth}|} *1{>{\centering \arraybackslash}p{0.19\textwidth}|} }
\hline
\multicolumn{3}{|c|} {\parbox[c]{10cm}{\begin{center}
\textbf{{\Large\sffamily BAC 2025 }}
\end{center}}} \\ \hline
\textbf{Matière: Mathématiques}& \textbf{Niveau : T}\textbf{L}  \\ \hline
\multicolumn{4}{|c|}{\parbox[c]{10cm}{\begin{center}
\textbf{{\Large\sffamily Correction }}
\end{center}}} \\ \hline
\end{tabular}}\endgroup};
\end{tikzpicture}
% Définition de la commande C_n^k
\newcommand{\Cnk}[2]{C_{#1}^{#2}}
\vspace{3cm}
\section*{\underline{Exercice 1 :($06$ pts)}}
\begin{enumerate}
\item Déterminons $a$ et $b$
\[
\bar{x} = \frac{1 + 2 + 3 + 4 + 4 + a}{6} = 3{,}25 
\Rightarrow \frac{14 + a}{6} = 3{,}25 
\Rightarrow 14 + a = 19{,}5 
\Rightarrow a = 5{,}5
\] \textbf{0,5 pt }

\[
\bar{y} = \frac{7 + 5 + 5 + 4 + 3 + b}{6} = 4{,}45 
\Rightarrow \frac{24 + b}{6} = 4{,}45 
\Rightarrow 24 + b = 26{,}7 
\Rightarrow b = 2{,}7
\]\textbf{0,5 pt }

\[
\begin{array}{|c|c|c|c|c|c|c|c|}
\hline
&  &  &  &  &  &  & \textbf{Totale} \\
\hline
x_i & 1 & 2 & 3 & 4 & 4 & 5{,}5 & \textbf{19{,}5} \\
\hline
y_i & 7 & 5 & 5 & 4 & 3 & 2{,}7 & \textbf{26{,}7} \\
\hline
x_i - \bar{x} & -2{,}25 & -1{,}25 & -0{,}25 & 0{,}75 & 0{,}75 & 2{,}25 & \textbf{0} \\
\hline
y_i - \bar{y} & 2{,}55 & 0{,}55 & 0{,}55 & -0{,}45 & -1{,}45 & -1{,}75 & \textbf{0} \\
\hline
(x_i - \bar{x})(y_i - \bar{y}) & -5{,}7375 & -0{,}6875 & -0{,}1375 & -0{,}3375 & -1{,}0875 & -3{,}9375 & \textbf{-11{,}925} \\
\hline
(x_i - \bar{x})^2 & 5{,}0625 & 1{,}5625 & 0{,}0625 & 0{,}5625 & 0{,}5625 & 5{,}0625 & \textbf{12{,}875} \\
\hline
(y_i - \bar{y})^2 & 6{,}5025 & 0{,}3025 & 0{,}3025 & 0{,}2025 & 2{,}1025 & 3{,}0625 & \textbf{12{,}475} \\
\hline
\end{array}
\]
\begin{enumerate}
    \item Calcul du coefficient \( r \)
\item Comme \( |r| \approx 0{,}92 > 0{,}8 \), il y a une bonne corrélation négative.
\end{enumerate}

\item Déterminons l'équation de la droite de régression de \( Y \) en \( X \) :

\[
\text{Cov}(X,Y) = \frac{-11{,}925}{6} \approx -1{,}9875
\quad \text{et} \quad 
\text{Var}(X) = \frac{12{,}875}{6} \approx 2{,}1458
\]\textbf{0,5 + 0,5 pt }

\[
a = \frac{\text{Cov}(X,Y)}{\text{Var}(X)} = \frac{-1{,}9875}{2{,}1458} \approx -0{,}926
\quad ; \quad 
b = \bar{y} - a \bar{x} = 4{,}45 - (-0{,}926 \times 3{,}25) \approx 7{,}46
\]

\[
\boxed{y = -0{,}926x + 7{,}46}
\]
\end{enumerate}

\section*{\underline{Exercice 2 :($08$ pts)}}

Moussa commence à travailler en janvier 2011 avec un salaire mensuel de \( 450\,000 \) FCFA.  
Chaque 1\textsuperscript{er} janvier, son salaire augmente de 2\%.  
Il commence à épargner quand son salaire atteint \( 600\,000 \) FCFA.

\subsection*{1. Modélisation}

Le salaire de Moussa suit une suite géométrique :

\[
S_n = 450\,000 \cdot (1{,}02)^n
\]

où \( n \) représente le nombre d'années après 2011.

\subsection*{2. Vérification en 2021}

En 2021, on a \( n = 10 \) (car 2021 - 2011 = 10)

\[
S_{10} = 450\,000 \cdot (1{,}02)^{10} \approx 450\,000 \cdot 1{,}219 = 548\,550
\]

\[
548\,550 < 600\,000 \Rightarrow \text{Moussa ne peut pas commencer à épargner en 2021.}
\]

\subsection*{3. Détermination de l'année de début d'épargne}

On cherche \( n \) tel que :

\[
450\,000 \cdot (1{,}02)^n \geq 600\,000
\Rightarrow (1{,}02)^n \geq \frac{600\,000}{450\,000} = \frac{4}{3}
\]

\[
n \geq \frac{\ln \left(\frac{4}{3}\right)}{\ln(1{,}02)} \approx \frac{0{,}2877}{0{,}0198} \approx 14{,}53
\]

Donc la première année entière où il atteint ce seuil est :

\[
2011 + 15 = \boxed{2026}
\]

\section*{\underline{Exercice 3 :($06$ pts)}}

Une urne contient 12 boules : 3 rouges, 4 vertes et 5 blanches.  
On tire 3 boules simultanément.

Nombre total de tirages possibles :
\[
\Cnk{12}{3} = 220
\]

\subsection*{A. Les 3 boules sont blanches}

\[
\Cnk{5}{3} = 10 \Rightarrow P(A) = \frac{10}{220} = \boxed{\frac{1}{22}}
\]

\subsection*{B. Les 3 boules ont la même couleur}

\[
\Cnk{5}{3} + \Cnk{4}{3} + \Cnk{3}{3} = 10 + 4 + 1 = 15
\Rightarrow P(B) = \frac{15}{220} = \boxed{\frac{3}{44}}
\]

\subsection*{C. 2 vertes et 1 rouge}

\[
\Cnk{4}{2} \cdot \Cnk{3}{1} = 6 \cdot 3 = 18
\Rightarrow P(C) = \frac{18}{220} = \boxed{\frac{9}{110}}
\]

\subsection*{D. Aucune boule blanche}

On tire 3 boules parmi 7 (rouges et vertes) :
\[
\Cnk{7}{3} = 35 \Rightarrow P(D) = \frac{35}{220} = \boxed{\frac{7}{44}}
\]

\subsection*{E. Au moins une boule blanche}

\[
P(E) = 1 - P(D) = 1 - \frac{7}{44} = \boxed{\frac{37}{44}}
\]

\end{document}
