\documentclass{article}
\usepackage[utf8]{inputenc}
\usepackage[french]{babel}
\usepackage{amsmath}

% Définition de la commande C_n^k
\newcommand{\Cnk}[2]{C_{#1}^{#2}}

\begin{document}

\section*{Exercice 3 (6 points)}

Une urne contient 12 boules : 3 rouges, 4 vertes et 5 blanches.  
On tire 3 boules simultanément.

Nombre total de tirages possibles :
\[
\Cnk{12}{3} = 220
\]

\subsection*{A. Les 3 boules sont blanches}

\[
\Cnk{5}{3} = 10 \Rightarrow P(A) = \frac{10}{220} = \boxed{\frac{1}{22}}
\]

\subsection*{B. Les 3 boules ont la même couleur}

\[
\Cnk{5}{3} + \Cnk{4}{3} + \Cnk{3}{3} = 10 + 4 + 1 = 15
\Rightarrow P(B) = \frac{15}{220} = \boxed{\frac{3}{44}}
\]

\subsection*{C. 2 vertes et 1 rouge}

\[
\Cnk{4}{2} \cdot \Cnk{3}{1} = 6 \cdot 3 = 18
\Rightarrow P(C) = \frac{18}{220} = \boxed{\frac{9}{110}}
\]

\subsection*{D. Aucune boule blanche}

On tire 3 boules parmi 7 (rouges et vertes) :
\[
\Cnk{7}{3} = 35 \Rightarrow P(D) = \frac{35}{220} = \boxed{\frac{7}{44}}
\]

\subsection*{E. Au moins une boule blanche}

\[
P(E) = 1 - P(D) = 1 - \frac{7}{44} = \boxed{\frac{37}{44}}
\]

\end{document}
