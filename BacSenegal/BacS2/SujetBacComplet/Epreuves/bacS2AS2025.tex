\documentclass[11pt]{article}
\usepackage[utf8]{inputenc}
\usepackage[T1]{fontenc}
\usepackage[french]{babel}
\usepackage{amsmath, amssymb}
\usepackage{graphicx}
\usepackage{multicol}
\usepackage{fancyhdr}
\usepackage{enumitem}
\usepackage[top=2cm, bottom=2cm, left=2.2cm, right=2.2cm]{geometry}

\pagestyle{fancy}
\fancyhf{}
\rhead{2025GS26NA0132}
\lhead{UNIVERSITÉ CHEIKH ANTA DIOP DE DAKAR}
\rfoot{\thepage/2}

\begin{document}

\begin{center}
    \textbf{\Large OFFICE DU BACCALAURÉAT} \\
    \texttt{Email : office@ucad.sn} \\
    \texttt{Site web : officedubac.sn} \\
    \vspace{0.3cm}
    \textbf{\LARGE MATHÉMATIQUES} \\
    \vspace{0.2cm}
    Séries : S2–S2A–S4–S5 \hfill Coef. 5 \\
    \textbf{Épreuve du 1\textsuperscript{er} groupe} \\
    Durée : 4 heures
\end{center}

\vspace{0.2cm}
\noindent
\textbf{Les calculatrices électroniques non imprimantes avec entrée unique par clavier sont autorisées. Les calculatrices permettant d’afficher des formulaires ou des tracés de courbe sont interdites. Leur utilisation sera considérée comme une fraude (Cf. Circulaire n° 5990/OB/Dir. du 12 08 1998).}

\vspace{0.5cm}
\noindent
\textbf{EXERCICE 1 :}  \textbf{(05,5 points)}

\begin{enumerate}[label=\Alph*.]
\item Soit deux nombres complexes \( z_1, z_2 \) avec \( z_2 \ne 0 \). Rappeler les propriétés algébriques suivantes : 
\(
\left| \frac{z_1}{z_2} \right|, \quad \arg\left(\frac{z_1}{z_2}\right), \quad |z_1 \times z_2| \text{ et } \arg(z_1 \times z_2).
\)
\hfill \textbf{(01 point)}

\item On considère le polynôme \( P \) défini par :
\(
P(z) = z^3 + (4 - 5i)z^2 - (5 + 12i)z - 8 + i \quad \text{où } z \in \mathbb{C}.
\)
\begin{enumerate}
    \item Montrer que \( P \) admet une racine imaginaire pure \( \beta \) que l’on déterminera. \hfill (0,5 point)
    \item Déterminer le polynôme \( g \) tel que \( P(z) = (z - \beta)g(z) \). \hfill (0,5 point)
    \item Factoriser \( g(z) \) puis en déduire une factorisation de \( P(z) \). \hfill (1 point)
\end{enumerate}

\item Dans le plan complexe muni d’un repère orthonormé \((O; \vec{u}, \vec{v})\), unité 1 cm, on considère les points \( A, B \) et \( C \) d’affixes respectives :
\(
z_A = 1 + 2i,\quad z_B = 2 + i,\quad z_C = -1 + i.
\)
\begin{enumerate}
    \item Déterminer la nature du triangle ABC. \hfill (0,5 point)
    \item Déterminer le point \( D \) tel que \( \overrightarrow{AB} + \overrightarrow{AD} = \overrightarrow{AC} \). \hfill (0,5 point)
    \item Soit \( f \) la translation qui transforme \( A \) en \( B \). En déduire l’image du point \( C \) par \( f \). \hfill (0,5 point)
\end{enumerate}

\item Soit \( S \) la similitude plane directe qui laisse invariant \( A \) et transforme \( B \) en \( C \).
\begin{enumerate}
    \item Donner l’écriture complexe et l’écriture analytique de \( S \). \hfill (0,5 point)
    \item Déterminer les éléments caractéristiques de \( S \). \hfill (0,75 point)
    \item Soit le cercle \( (C) \) d’équation \( (x + 3)^2 + (y - 2)^2 = 4 \). \\
    Déterminer l’équation de \( (C') \), image de \( (C) \) par \( S \). \hfill (0,5 point)
    \item Tracer \( (C) \) et \( (C') \) dans le repère. \hfill (0,5 point)
\end{enumerate}
\end{enumerate}

\vspace{0.5cm}
\noindent
\textbf{EXERCICE 2 :} \textbf{(06 points)}

\noindent
Une usine fabrique des téléphones portables que l’on ne peut pas discerner en les touchant. Les téléphones produits seront vendus 45 000 FCFA l’unité. Le chef de l’usine remarque que 2 téléphones sur 100 sont défectueux. Il décide de mettre tous les téléphones fabriqués depuis 10 jours dans une caisse et de vérifier leur état. Il procède comme suit : il prend un téléphone de la caisse, vérifie son état et le remet dans la caisse. Il recommence ainsi de suite le processus.

\begin{enumerate}
    \item Aide le chef de l’usine à déterminer le nombre \( n \) de téléphones à produire pour que la chance d’obtenir au moins un téléphone défectueux soit supérieure ou égale à 0,999. \hfill (03 points)
    
    \item Sachant que la production de ces téléphones coûte en moyenne 11 970 000 FCFA à l’usine, on tirera-t-elle profit après la vente de tous les téléphones non défectueux ? \hfill (03 points)
\end{enumerate}

\vspace{0.4cm}
\noindent
\textit{Les résultats seront donnés à \( 10^{-3} \) près avec 3 chiffres significatifs.}

\vspace{0.2cm}
\textbf{PROBLÈME} \textbf{(08,5 points)}

\vspace{0.4cm}
\noindent
\textbf{PARTIE A}

\noindent
Soit \( g \) la fonction définie sur \( [1; +\infty[ \) par : \( g(x) = 1 - x \ln x \).

\begin{enumerate}
    \item Montrer que \( \forall x \in [1; +\infty[ \), \( g'(x) < 0 \). \hfill (0,25 point)
    \item Dresser le tableau de variations de \( g \). \hfill (0,5 point)
    \item Montrer que \( g \) est une bijection de \( [1; +\infty[ \) sur un intervalle à préciser. \hfill (0,5 point)
    \item Montrer que l’équation \( g(x) = 0 \) admet une unique solution \( \alpha \) dans \( [1; +\infty[ \). \hfill (0,25 point)
    \item Montrer que \( 1{,}7 < \alpha < 1{,}8 \). \hfill (0,25 point)
    \item Préciser le signe de \( g \) sur \( [1; +\infty[ \). \hfill (0,25 point)
\end{enumerate}

\vspace{0.4cm}
\noindent
\textbf{PARTIE B}

\noindent
Soit \( f \) la fonction définie par :
\(
f(x) = 
\begin{cases}
(-x + 1)\ln(-x + 1) & \text{si } x < 1 \\
e^{-x+1} \ln x & \text{si } x \geq 1
\end{cases}
\)

\noindent
\( (C_f) \) sa courbe représentative dans le plan muni d’un repère orthonormé \( (O; I, J) \) d’unité 2 cm.
\begin{enumerate}
\item
\begin{enumerate}
    \item[a)] Montrer que \( f \) est définie sur \( \mathbb{R} \). \hfill (0,5 point)

    \item[b)] Rappeler les limites suivantes :
    \(
    \lim\limits_{x \to 0^+} x \ln x, \quad \lim\limits_{x \to 0^+} \frac{\ln x}{x}, \quad \lim\limits_{x \to +\infty} \frac{\ln x}{x^a} \text{ avec } a > 0.
    \)
    \hfill (0,5 point)

    \item[c)] Déterminer \( \lim\limits_{x \to -\infty} f(x), \quad \lim\limits_{x \to +\infty} f(x) \) et interpréter graphiquement, si possible, les résultats obtenus. \hfill (0,75 point)

    \item[d)] Étudier la branche infinie de la courbe \( (C_f) \) de \( f \) en \( -\infty \). \hfill (0,25 point)
\end{enumerate}

\vspace{0.3cm}
\noindent
\item
\begin{enumerate}
    \item[a)] Montrer que \( f \) est continue en 1. \hfill (0,5 point)

    \item[b)] Étudier la dérivabilité de \( f \) en 1 et interpréter géométriquement, si possible, les résultats obtenus. \hfill (0,5 point)
\end{enumerate}

\vspace{0.2cm}
\noindent
\item
\begin{enumerate}
    \item[a)] Calculer \( f'(x) \) sur \( ] -\infty ; 1[ \) et étudier son signe. \hfill (0,5 point)
    
    \item[b)] Montrer que \( \forall x \in ]1 ; +\infty[ \),
    \(
    f'(x) = \frac{g(x) e^{x+1}}{x}
    \)
    En déduire le signe de \( f'(x) \) sur \( ]1 ; +\infty[ \). \hfill (0,25 point)
\end{enumerate}

\vspace{0.2cm}
\noindent
\item Dresser le tableau de variations de \( f \). \hfill (0,5 point)

\noindent
\item Tracer la courbe représentative \( (C_f) \) de \( f \) dans le plan muni du repère \( (O; \vec{i}, \vec{j}) \). \hfill (0,75 point)

\end{enumerate}

\textbf{PARTIE C}
\begin{enumerate}

\item Soit \( h \) la restriction de \( f \) à l’intervalle \( [\alpha ; +\infty[ \)

\begin{enumerate}
    \item[a)] Montrer que \( h \) admet une fonction réciproque \( h^{-1} \). \hfill (0,25 point)
    
    \item[b)] Préciser l’ensemble de définition de \( h^{-1} \) et tracer sa courbe dans le plan muni du repère \( (O; \vec{i}, \vec{j}) \). \hfill (0,5 point)
\end{enumerate}

\vspace{0.2cm}
\noindent
\item Calculer en \( \text{cm}^2 \) l’aire \( \mathcal{A}(g) \) de la partie \( \mathcal{D} \) du plan comprise entre l’axe des abscisses, la droite d’équation \( x =-1 \), l’axe des ordonnées et la courbe \( (C_f) \) de \( f \). \hfill (1 point)
\end{enumerate}

\end{document}
