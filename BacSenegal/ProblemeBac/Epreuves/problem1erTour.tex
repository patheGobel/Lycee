\documentclass[12pt,a4paper]{article}
\usepackage{amsmath,amssymb,mathrsfs,tikz,times,pifont}
\usepackage{enumitem}
\usepackage{multicol}
\usepackage{lmodern}
\newcommand\circitem[1]{%
\tikz[baseline=(char.base)]{
\node[circle,draw=gray, fill=red!55,
minimum size=1.2em,inner sep=0] (char) {#1};}}
\newcommand\boxitem[1]{%
\tikz[baseline=(char.base)]{
\node[fill=cyan,
minimum size=1.2em,inner sep=0] (char) {#1};}}
\setlist[enumerate,1]{label=\protect\circitem{\arabic*}}
\setlist[enumerate,2]{label=\protect\boxitem{\alph*}}
\everymath{\displaystyle}
\usepackage[left=1cm,right=1cm,top=1cm,bottom=1.7cm]{geometry}
\usepackage[colorlinks=true, linkcolor=blue, urlcolor=blue, citecolor=blue]{hyperref}
\usepackage{array,multirow}
\usepackage[most]{tcolorbox}
\usepackage{varwidth}
\usepackage{float}
\tcbuselibrary{skins,hooks}
\usetikzlibrary{patterns}

\newtcolorbox{exa}[2][]{enhanced,breakable,before skip=2mm,after skip=5mm,
colback=yellow!20!white,colframe=black!20!blue,boxrule=0.5mm,
attach boxed title to top left ={xshift=0.6cm,yshift*=1mm-\tcboxedtitleheight},
fonttitle=\bfseries,
title={#2},#1,
boxed title style={frame code={
\path[fill=tcbcolback!30!black]
([yshift=-1mm,xshift=-1mm]frame.north west)
arc[start angle=0,end angle=180,radius=1mm]
([yshift=-1mm,xshift=1mm]frame.north east)
arc[start angle=180,end angle=0,radius=1mm];
\path[left color=tcbcolback!60!black,right color = tcbcolback!60!black,
middle color = tcbcolback!80!black]
([xshift=-2mm]frame.north west) -- ([xshift=2mm]frame.north east)
[rounded corners=1mm]-- ([xshift=1mm,yshift=-1mm]frame.north east)
-- (frame.south east) -- (frame.south west)
-- ([xshift=-1mm,yshift=-1mm]frame.north west)
[sharp corners]-- cycle;
},interior engine=empty,
},interior style={top color=yellow!5}}

\usepackage{fancyhdr}
\usepackage{eso-pic}
\usepackage{tkz-tab}
\AddToShipoutPicture{
    \AtTextCenter{%
        \makebox[0pt]{\rotatebox{80}{\textcolor[gray]{0.7}{\fontsize{5cm}{5cm}\selectfont PGB}}}
    }
}
\usepackage{lastpage}
\fancyhf{}
\pagestyle{fancy}
\renewcommand{\footrulewidth}{1pt}
\renewcommand{\headrulewidth}{0pt}
\renewcommand{\footruleskip}{10pt}
\fancyfoot[R]{\color{blue}\ding{45}\ \textbf{2025}}
\fancyfoot[L]{\color{blue}\ding{45}\ \textbf{Prof : M. BA}}
\cfoot{\bf \thepage / \pageref{LastPage}}

\newcommand{\exo}[1]{%
        \textbf{\underline{Problème #1}}
}

\begin{document}
\renewcommand{\arraystretch}{1.5}
\renewcommand{\arrayrulewidth}{1.2pt}
\begin{tikzpicture}[overlay,remember picture]
    \node[draw=blue,line width=1.2pt,fill=purple,text=blue,inner sep=3mm,rounded corners,pattern=dots]at ([yshift=-2.5cm]current page.north) {\begingroup\setlength{\fboxsep}{0pt}\colorbox{white}{\begin{tabular}{|*1{>{\centering \arraybackslash}p{0.28\textwidth}} |*2{>{\centering \arraybackslash}p{0.2\textwidth}|} *1{>{\centering \arraybackslash}p{0.19\textwidth}|} }
                \hline
                \multicolumn{3}{|c|}{$\diamond$$\diamond$$\diamond$\ \textbf{Lycée de Dindéfélo}\ $\diamond$$\diamond$$\diamond$ } & \textbf{A.S. : 2024/2025} \\ \hline
                \textbf{Matière : Mathématiques} & \textbf{Niveau : T S2} & \textbf{Date : 09/06/2025} & \textbf{} \\ \hline
                \multicolumn{4}{|c|}{\parbox[c]{10cm}{\begin{center}
                  \textbf{{\Large\sffamily Problèmes proposés au BAC S2 Sénégal de 1999 à 2022}}
                \end{center}}} \\ \hline
            \end{tabular}}\endgroup};
\end{tikzpicture}
\vspace{3cm}

\begin{multicols}{2}
\small
\setlength{\columnseprule}{0.1mm}

\exo{1} \textbf{Extrait BAC 1999 $1^{er}$ groupe}
On considère la fonction \( f \) définie par :

\( 
f(x) =
\begin{cases}
x + \ln\left|\dfrac{x - 1}{x + 1}\right| & \text{si } x \in ]-\infty, -1[ \cup ]-1, 0[ \\
x^2 e^{-x} & \text{si } x \in [0, +\infty[
\end{cases}
\)
et \( (C_f) \) sa courbe représentative dans un repère orthonormé \( (O; \vec{i}, \vec{j}) \), d’unité 2 cm.

\subsection*{Partie A}
\begin{enumerate}
  \item Déterminer l’ensemble de définition \( D_f \) de \( f \). Calculer \( f(-2) \) et \( f(3) \).
  \item Calculer les limites aux bornes de \( D_f \).
  \item Étudier la continuité de \( f \) en 0.
  \item 
  \begin{enumerate}
    \item Établir que la dérivée de \( f \) est donnée par :
    
\(
    f'(x) =
    \begin{cases}
    \dfrac{x^2 + 1}{x^2 - 1}  \text{si } x \in ]-\infty, -1[ \cup ]-1, 0[ \\
    x e^{-x}(2 - x) \text{si } x \in [0, +\infty[
    \end{cases}
\)
    \item La fonction \( f \) est-elle dérivable en 0 ? Justifier votre réponse.
    \item Dresser le tableau de variations de \( f \).
  \end{enumerate}
  \item Démontrer que l’équation \( f(x) = 0 \) admet une solution unique \( \alpha \) comprise entre \( -1{,}6 \) et \( -1{,}5 \).
  \item 
  \begin{enumerate}
    \item Justifier que la droite \( (D) \) d’équation 
    
    \( y = x \) est une asymptote à la courbe \( (C_f) \) en \( -\infty \).
    \item Étudier la position relative de \( (C_f) \) par rapport à la droite \( (D) \) pour
    
     \( x \in ]-\infty, -1[ \cup ]-1, 0[ \).
    \item Tracer \( (C_f) \).
  \end{enumerate}
\end{enumerate}

\subsection*{Partie B : } Soit \( g \) la restriction de \( f \) à \( I = [0;2] \).
\begin{enumerate}
    \item Montrer que \( g \) définit une bijection de \( I \) vers un intervalle \( J \) à préciser.
    \item On note \( g^{-1} \) la bijection réciproque de \( g \).
    \begin{enumerate}
        \item Résoudre l’équation \( g^{-1}(x) = 1 \).
        \item Montrer que \( \left(g^{-1}\right)'\left(\dfrac{1}{e}\right) = e \).
        \item Construire \( (C_{g^{-1}}) \), la courbe de \( g^{-1} \).
    \end{enumerate}
\end{enumerate}


\subsection*{Partie C : } \( \beta \) étant un réel strictement positif, on pose :

\(I(\beta) = \int_0^{\beta} f(x)\,dx \)
\begin{enumerate}
    \item 
    \begin{enumerate}
        \item Interpréter graphiquement \( I(\beta) \).
        \item En procédant par une intégration par parties, calculer \( I(\beta) \).
    \end{enumerate}
    
    \item Calculer \( \lim_{\beta \to -\infty} I(\beta) \).
    
    \item On pose \( \beta = 2 \).
    \begin{enumerate}
        \item Calculer \( I(2) \).
        \item En déduire la valeur en \( \text{cm}^2 \) de l’aire du domaine du plan délimité par la courbe \( (C_f) \), l’axe des abscisses et les droites d’équations \( x = 0 \) et \( x = \dfrac{4}{e^2} \).
    \end{enumerate}
\end{enumerate}

\exo{2} \textbf{BAC 1999 Remplacement}

Soit \( f \) la fonction définie sur \( \mathbb{R} \) par :\\
\(
f(x) =
\begin{cases}
e^{-\frac{1}{x^2}} & \text{si } x \in ]-\infty ; 0[ \\
\ln\left|\dfrac{x - 1}{x + 1}\right| & \text{si } x \in [0 ; 1[ \cup ]1 ; +\infty[
\end{cases}
\)\\
et \( (C_f) \) sa courbe représentative dans un repère orthonormé \( (O ; \vec{i} ; \vec{j}) \) d’unité \( 2\ cm \).

\subsection*{Partie A : }
\begin{enumerate}
    \item Étudier la continuité de \( f \) en 0.
    \item 
    \begin{enumerate}
        \item Montrer que \( \forall x \in ]0 ; 1[ \),\\
        \( \dfrac{f(x)}{x} = \dfrac{\ln(1 - x)}{x} - \dfrac{\ln(1 + x)}{x} \)
        \item Étudier la dérivabilité de \( f \) en 0.
        \item En déduire que \( (C_f) \) admet au point d’abscisse 0 deux demi-tangentes dont on donnera les équations.
    \end{enumerate}
        \item Étudier les variations de \( f \).
				\item Tracer \( (C_f) \).
\end{enumerate}				
    \subsection*{Partie B : } Soit \( g \) la restriction de \( f \) à \( ]1 ; +\infty[ \).
\begin{enumerate}

    \item Montrer que \( g \) est une bijection de \( ]1 ; +\infty[ \) vers un intervalle \( J \) à préciser.\\
    On notera \( g^{-1} \) la bijection réciproque de \( g \).

    \item Montrer que l’équation \( g(x) = -e \) admet une unique solution \( \alpha \) sur l’intervalle \( ]1 ; +\infty[ \).\\
    \textbf{(On ne demande pas de calculer \( \alpha \)).}

    \item Montrer que \( \forall x \in J \), \( g^{-1}(x) = 1 - \dfrac{e^x}{e^x - 1} \).

    \item Construire \( (C_{g^{-1}}) \).\\
    \textbf{(On indiquera la nature et l’équation de chacune des asymptotes à \( (C_g) \) et \( (C_{g^{-1}}) \)).}
\item Calculer en \( \text{cm}^2 \) l’aire \( A \) de l’ensemble des points \( M(x ; y) \) défini par :

\(
\left\{
\begin{array}{l}
-\ln 7 \leq x \leq -1 \\
0 \leq y \leq g^{-1}(x)
\end{array}
\right.
\)

\end{enumerate}

\exo{3} \textbf{BAC 2000 1\textsuperscript{er} groupe}
Soit \( f \) la fonction définie de \( \mathbb{R} \) dans \( \mathbb{R} \) par :\\
\(
f(x) =
\begin{cases}
x \ln(x+1) & \text{si } x \geq 0 \\
x e^{\frac{1}{x}} & \text{si } x < 0
\end{cases}
\)

Le plan est muni d’un repère orthonormé \( (O ; \vec{i} ; \vec{j}) \) \textbf{(unité graphique 2 cm)}.\\
On désigne par \( (C) \) la courbe représentative de \( f \) et \( (\Delta) \) la droite d’équation \( y = x \).

\textbf{Partie A :}
\begin{enumerate}
    \item 
    \begin{enumerate}
        \item Montrer que \( f \) est continue en 0.
        \item Étudier la dérivabilité de \( f \) en 0.
        \item Interpréter les résultats précédents.
    \end{enumerate}

    \item 
    \begin{enumerate}
        \item Montrer que \( \forall x < 0 \), \( f'(x) > 0 \).
        \item Étudier les variations de \( f \) sur \( [0 ; +\infty[ \).\\
        En déduire que \( \forall x > 0 \), \( f'(x) > 0 \).
        \item Donner le tableau de variations de \( f \).
    \end{enumerate}
    \item
\begin{enumerate}
    \item Déterminer \( \lim\limits_{x \to -\infty} x \left( e^{\frac{1}{x}} - 1 \right) \), \textbf{(on pourra poser \( u = \dfrac{1}{x} \))}.
    \item Montrer que la droite \( (D) : y = x + 1 \) est asymptote à \( (C) \) au voisinage de \( -\infty \).\\
    On admettra que \( (C) \) est en dessous de \( (D) \).
    \item Déterminer la nature de la branche infinie de \( (C) \) en \( +\infty \).
\end{enumerate}

\item Construire \( (C) \), on précisera les coordonnées de \( I \), intersection de \( (C) \) et \( (\Delta) \) pour \( x > 0 \).

\end{enumerate}
\textbf{Partie B :}
\begin{enumerate}
    \item Déterminer les réels \( a \), \( b \) et \( c \) tels que :
    
    \(
    \forall x \in \mathbb{R}_+, \quad \dfrac{x^2}{x + 1} = ax + b + \dfrac{c}{x + 1}.
    \)

    \item En déduire au moyen d’une intégration par parties que la fonction \( F \) telle que :

    \(
    F(x) = \dfrac{(x^2 - 1)\ln(x + 1)}{2} - \dfrac{1}{4}(x^2 - 2x)
    \)

    est une primitive de \( f \) sur \( \mathbb{R}_+ \).

    \item En déduire en \( \text{cm}^2 \) l’aire \( A \) de la partie du plan délimitée par \( (\Delta) \), \( (C) \) et les droites d’équations \( x = 0 \) et \( x = e - 1 \).
\end{enumerate}

\textbf{Partie C :}
\begin{enumerate}
\item 
    \begin{enumerate}
        \item Montrer que \( f \) admet une bijection réciproque notée \( f^{-1} \).
        \item \( f^{-1} \) est-elle dérivable en 0 ? Préciser la nature de la tangente en 0 à la courbe \( f^{-1} \).
    \end{enumerate}
    \item Construire \( (C_0) \), la courbe de \( f^{-1} \) dans le repère \( (O ; \vec{i} ; \vec{j}) \).

    \item Déduire du \textbf{B.3)} l’aire du domaine \( D \) définie par : \( M(x ; y) \) tels que :
    
    \(
    \left\{
    \begin{array}{l}
    0 \leq x \leq e - 1 \\
    f(x) \leq y \leq f^{-1}(x)
    \end{array}
    \right.
    \)
\end{enumerate}

\exo{4} \textbf{BAC 2000 Remplacement}

\textbf{Partie A :} Soit \( g \) la fonction définie par 

\( g(x) = 1 - x e^{-x} \).

\begin{enumerate}
    \item Étudier les variations de \( g \).
    \item En déduire le signe de \( g(x) \) suivant les valeurs de \( x \).
\end{enumerate}

\textbf{Partie B :} Soit \( f \) la fonction définie par :\\
\(
f(x) =
\begin{cases}
\ln(-x) & \text{si } x < -1 \\
(x + 1)(1 + e^{-x}) & \text{si } x \leq -1
\end{cases}
\)

On désigne par \( (C) \) sa courbe représentative dans un repère orthonormé \( (0 ; \vec{i} ; \vec{j}) \) \textbf{(unité graphique 2 cm)}.

\begin{enumerate}
    \item Étudier la continuité et la dérivabilité de \( f \) sur \( \mathbb{R} \).

    \item Étudier les variations de \( f \), puis dresser le tableau de variations de \( f \).

    \item
    \begin{enumerate}
        \item Montrer que la droite \( (D) : y = x + 1 \) est une asymptote à \( (C_f) \) en \( +\infty \).
        \item Étudier la position relative de \( (C_f) \) par rapport à \( (D) \) sur \( [-1 ; +\infty[ \).
    \end{enumerate}

    \item Montrer qu’il existe un unique point de la courbe \( (C_f) \) où la tangente \( (T) \) est parallèle à la droite \( (D) \).

    \item Tracer \( (C_f) \), l’asymptote \( (D) \) et la tangente \( (T) \),\\
    on précisera la tangente ou les demi-tangentes à \( (C_f) \) au point d’abscisse \( -1 \).

    \item
    \begin{enumerate}
        \item Montrer que \( f \) est une bijection de \( [-1 ; +\infty[ \) sur un ensemble \( J \) que l’on précisera.
        \item Construire \( (C_0) \), la courbe de \( f^{-1} \) sur le même graphique que la courbe \( (C_f) \).
    \end{enumerate}
\end{enumerate}

\textbf{Partie C :} Pour \( \beta \geq -1 \), on note \( A(\beta) \) l’aire en \( \text{cm}^2 \) de la partie du plan définie par :

\(
\left\{
\begin{array}{l}
-1 \leq x \leq \beta \\
x + 1 \leq y \leq f(x)
\end{array}
\right.
\)

\begin{enumerate}
    \item Calculer \( A(\beta) \) à l’aide d’une intégration par parties.
    \item Montrer que \( A(\beta) \) admet une limite finie lorsque \( \beta \to +\infty \).\\
    Interpréter graphiquement cette limite.
\end{enumerate}

\exo{5} \textbf{BAC 2001 1\textsuperscript{er} groupe}

On considère la fonction \( g \) définie par :\\
\(
f(x) =
\begin{cases}
x(1 - \ln x)^2 & \text{si } x > 0 \\
0 & \text{si } x = 0
\end{cases}
\)

On appelle \( (C) \) sa courbe représentative dans un repère orthonormé \( (O ; \vec{i} ; \vec{j}) \).

\begin{enumerate}
    \item Étudier la continuité et la dérivabilité de \( g \) sur son ensemble de définition.
    \item Étudier les variations de \( g \) puis dresser son tableau de variations.
    \item Tracer \( (C) \).
    \item Soit \( \alpha \) un réel appartenant à l’intervalle \( ]0 ; e[ \).
    \begin{enumerate}
        \item Calculer à l’aide de deux intégrales par parties, l’aire \( A(\alpha) \) du domaine plan délimitée par l’axe des abscisses, la courbe \( (C) \) et les droites d’équations respectives \( x = \alpha \) et \( x = e \).
        \item Calculer
    \end{enumerate}
    \item
    \begin{enumerate}
        \item Déterminer les coordonnées des points d’intersection de la courbe \( (C) \) et la droite \( (\Delta) \) d’équation \( y = x \).
        \item Pour quelles valeurs de \( m \) la droite \( (\Delta_m) \) d’équation \( y = mx \) recoupe-t-elle la courbe \( (C) \) en deux points \( M_1 \) et \( M_2 \) autres que \( O \) ?
        \item La droite \( (\Delta_m) \) coupe la droite \( (D) \) d’équation \( x = e \) en \( P \). 
        
        Montrer que \( OM_1 \times OM_2 = OP^2 \).
    \end{enumerate}

    \item
    \begin{enumerate}
        \item Montrer que la restriction \( h \) de la fonction \( g \) à l’intervalle \( [e ; +\infty[ \) admet une réciproque \( h^{-1} \) dont on précisera l’ensemble de définition.
        \item Sur quel ensemble \( h^{-1} \) est-elle dérivable ?\\
        Calculer \( h(e^2) \) ; en déduire \( (h^{-1})'(e^2) \).
        \item Construire la courbe de \( h^{-1} \).
    \end{enumerate}

\end{enumerate}


\exo{6} \textbf{BAC 2002 1\textsuperscript{er} groupe}

\textbf{Partie A :} On considère la fonction \( g \) définie sur \( \mathbb{R}^+ \setminus \{1\} \) par :

\(
g(x) =
\begin{cases}
\dfrac{1}{(\ln x)^2} - \dfrac{1}{\ln x} & \text{si } x > 0 \text{ et } x \neq 1 \\
0 & \text{si } x = 0
\end{cases}
\)

\begin{enumerate}
    \item Montrer que \( g \) est continue en 0.
    \item Étudier les limites de \( g \) aux bornes de son ensemble de définition.
    \item Dresser le tableau de variations de \( g \).
    \item En déduire le signe de \( g(x) \) en fonction de \( x \).
    \item Calculer en \( \text{cm}^2 \) l’aire de la partie plane comprise entre la courbe de \( g \), l’axe des abscisses et les droites d’équations respectives : \( x = e \) et \( x = e^2 \).
\end{enumerate}

\textbf{Partie B :} On considère la fonction \( f \) définie sur \( \mathbb{R}^+ \setminus \{1\} \) par :
\(
f(x) =
\begin{cases}
-\dfrac{x}{\ln x} & \text{si } x > 0 \text{ et } x \neq 1 \\
0 & \text{si } x = 0
\end{cases}
\)

\begin{enumerate}
    \item Montrer que \( f \) est continue à droite et dérivable à droite au point 0.\\
    En déduire l’existence d’une demi-tangente à la courbe représentative \( (C) \) de \( f \) au point d’abscisse 0.

    \item Étudier les limites aux bornes de son ensemble de définition.

    \item Comparer \( f'(x) \) et \( g(x) \).\\
    En déduire les variations de \( f \) et son tableau de variations.

    \item Déterminer l’équation de la tangente \( (D) \) à la courbe \( (C) \) au point d’abscisse \( e^2 \).
    \item Soit \( M \) le point de \( (C) \) d’abscisse \( x \) et \( N \) le point de \( (D) \) de même abscisse \( x \).\\
    On pose \( \varphi(x) = \overline{MN} \).
    \begin{enumerate}
        \item Montrer que \( \varphi(x) = f(x) + \dfrac{x + e^2}{4} \).
        \item Déduire de la \textbf{partie A} le tableau de variations de \( f'(x) \) puis le signe de \( f'(x) \) sur \( ]1 ; +\infty[ \).
        \item En déduire le signe de \( \varphi(x) \) sur \( ]1 ; +\infty[ \) et la position de \( (C) \) par rapport à \( (D) \) pour les points d’abscisse \( x > 1 \).
    \end{enumerate}

    \item Représenter dans le plan rapporté à un repère orthonormé la courbe \( (C) \) et la droite \( (D) \) \textbf{unité 2cm}.

\end{enumerate}

\exo{7} \textbf{BAC 2003 1\textsuperscript{er} GROUPE}

\textbf{Partie A :} On considère la fonction \( u \) définie sur \( [0 ; +\infty[ \) par :
\[
u(x) = \ln\left| \frac{x+1}{x-1} \right| - \frac{2x}{x^2 - 1}
\]

\begin{enumerate}
    \item 
    \begin{enumerate}
        \item Déterminer l’ensemble de définition de \( u \).
        \item Calculer \( u(0) \) et \( \lim\limits_{x \to +\infty} u(x) \).
    \end{enumerate}

    \item Étudier les variations de \( u \).\\
    \textbf{(il n’est pas nécessaire de calculer la limite de \( u \) en 1).}

    \item Déduire des résultats précédents que :
    \begin{enumerate}
        \item \( \forall x \in [0 ; 1[, \ u(x) \geq 0 \).
        \item \( \forall x \in ]1 ; +\infty[, \ u(x) < 0 \).
    \end{enumerate}
\end{enumerate}

\textbf{Partie B :} Soit \( g \) la fonction définie sur \( [0 ; +\infty[ \) par :
\[
g(x) = x \ln\left| \frac{x+1}{x-1} \right| - 1
\]

\begin{enumerate}
    \item Déterminer \( D_g \) (le domaine de définition de \( g \)) ;\\
    puis étudier la limite de \( g \) en 1.
\end{enumerate}
\begin{enumerate}
    \item
    \begin{enumerate}
        \item Vérifier que \( \dfrac{x+1}{x-1} = 1 + \dfrac{2}{x-1} \).
        \item En déduire que 
        \[
        \lim_{x \to +\infty} \dfrac{x - 1}{2} \ln\left(1 + \dfrac{2}{x - 1} \right) = 1.
        \]
        \item En déduire que \( \lim\limits_{x \to +\infty} g(x) = 1 \). Interpréter géométriquement ce résultat.
        \item Dresser le tableau de variations de \( g \).
        \item Montrer qu’il existe un réel \( \alpha \) unique appartenant à \( ]0 ; 1[ \) tel que \( g(\alpha) = 0 \).\\
        Donner un encadrement d’ordre 1 de \( \alpha \).
    \end{enumerate}

    \item Tracer la courbe \( (C_g) \) de \( g \) dans le plan rapporté à un repère orthonormé \textbf{(unité 2 cm)}.
\end{enumerate}

\textbf{Partie C :} Soit \( h \) la fonction définie sur \( [0 ; 1[ \) et
\[
f(x) = (x^2 - 1) \ln \sqrt{ \dfrac{x+1}{x-1} }.
\]

\begin{enumerate}
    \item Montrer que \( f \) est dérivable sur \( [0 ; 1[ \) et que :\\
    \( f'(x) = g(x) \), \( \forall x \in [0 ; 1[ \).
    
    \item Déterminer l’aire du domaine plan limité par la courbe \( (C_g) \), l’axe des abscisses et la droite d’équation \( x = \alpha \).
\end{enumerate}

\exo{8} \textbf{BAC 2004 1\textsuperscript{er} GROUPE}

Soit \( f \) la fonction définie par :
\[
f(x) = \frac{(2x - 1)e^x - 2x + 2}{e^x - 1}.
\]

On note \( (C) \) la représentation graphique de la fonction \( f \) dans un repère orthonormé \( (O ; \vec{i} ; \vec{j}) \), dont l’unité est 2 cm.

\begin{enumerate}
    \item Déterminer l’ensemble de définition \( D_f \) de la fonction \( f \), et trouver les réels \( a, b \) et \( c \) tels que pour tout \( x \in D_f \), on ait :
    \[
    f(x) = ax + b + \frac{c}{e^x - 1}.
    \]

    \item Déterminer les limites de \( f \) aux bornes de \( D_f \).

    \item 
    \begin{enumerate}
        \item Déterminer la fonction dérivée de \( f \).
        \item Résoudre dans \( \mathbb{R} \), l’équation :
        \[
        2e^{2x} - 5e^x + 2 = 0.
        \]
        \item En déduire le sens de variation de \( f \) et dresser le tableau de variations de \( f \).
    \end{enumerate}

    \item Démontrer que les droites d’équations respectives \( y = 2x - 1 \) et \( y = 2x - 2 \) sont des asymptotes de \( (C) \) respectivement en \( +\infty \) et en \( -\infty \).\\
    Préciser l’autre asymptote.

    \item Soit \( x \) un réel de \( D_f \). On considère les deux points \( M \) et \( M' \) de \( (C) \) d’abscisses respectives \( x \) et \( -x \).\\
    Déterminer les coordonnées du milieu \( \Omega \) du segment \( [MM'] \).\\
    Que peut-on en déduire pour la courbe \( (C) \) ?
    \item Tracer la courbe \( (C) \).
    \item
    \begin{enumerate}
        \item Trouver les réels \( \alpha \) et \( \beta \) tels que, pour tout réel \( x \) de l’ensemble \( D_f \), on ait :
        \[
        f(x) = 2x + \alpha + \frac{\beta e^x}{e^x - 1}.
        \]

        \item Soit \( k \) un réel supérieur ou égal à 2.\\
        Déterminer l’aire \( A(k) \) en \( \text{cm}^2 \) de l’ensemble des points du plan dont les coordonnées \( (x ; y) \) vérifient :
        \[
        \ln 2 \leq x \leq \ln k \quad \text{et} \quad 2x - 1 \leq y \leq f(x).
        \]
\end{enumerate}

\end{enumerate}

\exo{9} \textbf{BAC 2004 Remplacement}

\textbf{Partie A :} Soit l’équation différentielle
\[
(E) : -\tfrac{1}{2}y'' + \tfrac{3}{2}y' - y = 0
\]
Déterminer la solution \( g \) de \( (E) \) dont la courbe représentative \( (C) \) passe par le point \( A(0 ; -1) \) et dont la tangente en ce point est parallèle à l’axe des abscisses.

\textbf{Partie B :} Soit la fonction définie sur \( \mathbb{R} \) par :
\[
f(x) = e^{2x} - e^x.
\]
On note \( (\Gamma) \) sa courbe représentative dans un repère orthonormé \( (O ; \vec{i} ; \vec{j}) \) \textbf{(unité 2 cm)}.

\begin{enumerate}
    \item Étudier les variations de \( f \).
    \item Déterminer l’équation de la tangente à \( (\Gamma) \) au point d’abscisse \( \ln 2 \).
    \item Calculer \( \lim\limits_{x \to +\infty} \dfrac{f(x)}{x} \). Interpréter graphiquement le résultat.
\end{enumerate}

\textbf{Partie C :} Soit \( h \) la restriction de \( f \) à l’intervalle \( [0 ; +\infty[ \).

\begin{enumerate}
    \item Démontrer que \( h \) réalise une bijection de \( [0 ; +\infty[ \) sur un intervalle \( J \) à préciser.
    \item Démontrer que \( h^{-1} \) est dérivable en \( 3 \), puis calculer \( (h^{-1})'(3) \).
    \item Déterminer \( h^{-1}(x) \) pour tout \( x \in J \).
    \item Tracer \( (C_0) \), la courbe représentative de \( h^{-1} \) dans le repère \( (O ; \vec{i} ; \vec{j}) \).
\end{enumerate}

\exo{10} \textbf{BAC 2005 1\textsuperscript{er} GROUPE}

\textbf{Partie A :} Soit la fonction \( f \) définie sur \( \mathbb{R} \) par :
\[
f(x) = \frac{e^x}{e^x + 1} - \ln(e^x + 1).
\]

On note \( (C) \) la représentation graphique de la fonction \( f \) dans un plan muni d’un repère orthonormal \( (O ; \vec{i} ; \vec{j}) \) \textbf{(unité 2 cm)}.

\begin{enumerate}
    \item Étudier les variations de \( f \).
    \item Montrer que \( \lim\limits_{x \to +\infty} \left[ f(x) - 1 + x \right] = 0 \).\\
    Que peut-on en déduire pour \( (C_f) \) ?
    \item Construire \( (C_f) \).
    \item Montrer que \( f \) réalise une bijection de \( ]-\infty ; +\infty[ \) sur \( ]-\infty ; 0[ \).
\end{enumerate}
\textbf{Partie B :} Soit \( g \) la fonction définie par :
\[
f(x) = e^x \ln(1 + e^x)
\]
On note \( (C_g) \) sa courbe représentative.

\begin{enumerate}
    \item Montrer que \( g \) est dérivable sur \( \mathbb{R} \).

    \item Montrer que, pour tout réel \( x \),\\
    \( f'(x) = e^{-x} f(x) \).

    \item Montrer que \( \lim\limits_{x \to +\infty} g(x) = 0 \) et \( \lim\limits_{x \to -\infty} g(x) = 1 \).

    \item En déduire la nature des branches infinies.

    \item Dresser le tableau de variations de \( g \).

    \item Construire \( (C_g) \) dans le repère précédent.
    
    \setcounter{enumi}{6}
    \item 
    \begin{enumerate}
        \item Montrer que \( \dfrac{1}{e^x + 1} = \dfrac{e^{-x}}{e^{-x} + 1} \).
        \item À tout réel \( \beta \), on associe le réel
        \[
        I(\beta) = \int_0^{\beta} g(x)\,dx.
        \]
        Justifier l’existence de \( I(\beta) \).
        \item Calculer \( I(\beta) \) à l’aide d’une intégration par parties.
        \item Calculer \( \lim\limits_{\beta \to +\infty} I(\beta) \).
    \end{enumerate}

\end{enumerate}
\textbf{Partie C :} On considère l’équation différentielle :
\[
(E) : y' + y = \dfrac{e^{-x}}{e^{-x} + 1}.
\]

\begin{enumerate}
    \item Vérifier que la fonction \( g \) étudiée dans la \textbf{partie B} est solution de \( (E) \).

    \item Montrer qu’une fonction \( \phi \) est solution de \( (E) \) si et seulement si \( \phi - g \) est solution de l’équation différentielle \( (E_0) : y' + y = 0 \).

    \item Résoudre \( (E_0) \) et en déduire les solutions de \( (E) \).

    \item Déterminer la solution de \( (E) \) qui s’annule en \( \ln 2 \).
\end{enumerate}

\exo{11} \textbf{BAC 2006 1\textsuperscript{er} GROUPE}

\textbf{Partie A :} Soit \( h \) la fonction définie sur \( \mathbb{R} \) par :
\[
h(x) = 1 + (1 - x)e^{2 - x}
\]

\begin{enumerate}
    \item Étudier les variations de \( h \) \textbf{(on ne demande pas de calculer les limites aux bornes de \( D_h \))}.
    \item En déduire le signe de \( h(x) \) sur \( \mathbb{R} \).
\end{enumerate}

\textbf{Partie B :} Soit \( f \) la fonction définie sur \( \mathbb{R} \) par :
\[
f(x) = x(1 + e^{2 - x})
\]
et \( (C_f) \) sa courbe représentative dans un repère orthonormé \( (O ; \vec{i} ; \vec{j}) \) \textbf{(unité 2 cm)}.

\begin{enumerate}
    \item 
    \begin{enumerate}
        \item Étudier les limites de \( f \) en \( +\infty \) et en \( -\infty \).
        \item Préciser la nature de la branche infinie en \( -\infty \).
                \item Calculer \( \lim\limits_{x \to +\infty} \left[ f(x) - x \right] \), puis interpréter le résultat obtenu.

        \item Préciser la position de \( (C_f) \) par rapport à la droite \( (\Delta) : y = x \).
    \end{enumerate}
    \item 
    \begin{enumerate} 
            \item Dresser le tableau de variations de \( f \).

            \item Montrer que \( f \) admet une bijection réciproque notée \( f^{-1} \), définie sur \( \mathbb{R} \).

            \item \( f^{-1} \) est-elle dérivable en 4 ?

            \item Étudier la position de \( (C_f) \) par rapport à sa tangente au point d’abscisse 2.

            \item Construire \( (C_f) \) \textbf{(on tracera la tangente au point d’abscisse 2)}.

            \item Construire \( (C_{f^{-1}}) \), la courbe de \( f^{-1} \) dans le repère précédent.
    \end{enumerate}
\end{enumerate}
\textbf{Partie C :} Soit \( \beta \) un réel strictement positif sur \( \mathbb{R} \). La région du plan est délimitée par les droites d’équations \( x = 0 \) et \( x = \beta \), et les courbes d’équations respectives \( y = f(x) \) et \( y = x \). Soit \( A(\beta) \) l’aire de cette région \( R_\beta \), en \( \text{cm}^2 \).

\begin{enumerate}
    \item Calculer \( A(\beta) \) en fonction de \( \beta \).
    \item Déterminer \( \alpha = \lim\limits_{\beta \to +\infty} A(\beta) \).\\
    Interpréter graphiquement le résultat obtenu.
\end{enumerate}

\exo{12} \textbf{BAC 2007 1\textsuperscript{er} GROUPE}

\textbf{Partie A :} Soit \( g \) la fonction définie sur \( ]0 ; +\infty[ \) par :
\[
g(x) = 1 + x + \ln x.
\]

\begin{enumerate}
    \item Dresser le tableau de variations de \( g \).
    \item Montrer qu’il existe un unique \( \alpha \) solution de l’équation \( g(x) = 0 \).\\
    Vérifier que \( 0{,}2 < \alpha < 0{,}3 \).
    \item En déduire le signe de \( g \) sur \( ]0 ; +\infty[ \).
    \item Établir la relation : \( \ln \alpha = -1 - \alpha \).
\end{enumerate}

\textbf{Partie B :} On considère la fonction \( f \) définie par :
\[
f(x) = 
\begin{cases}
\dfrac{x \ln x}{1 + x} & \text{si } x > 0 \\
0 & \text{si } x = 0
\end{cases}
\]
et \( (C_f) \) sa courbe représentative dans un repère orthonormé \( (O ; \vec{i} ; \vec{j}) \) \textbf{(unité 5 cm)}.

\begin{enumerate}
    \item Montrer que la fonction \( f \) est continue en 0 puis sur \( [0 ; +\infty[ \).
    \item Étudier la dérivabilité de \( f \) en 0.\\
    Interpréter graphiquement ce résultat.
    \item Déterminer la limite de \( f \) en \( +\infty \).
    \item Montrer que \( \forall x \in ]0 ; +\infty[ \),
    \[
    f'(x) = \dfrac{g(x)}{(1 + x)^2}.
    \]
    En déduire le signe de \( f'(x) \) sur \( ]0 ; +\infty[ \).
    \item Montrer que \( f(\alpha) = -\alpha \).
    \item Dresser le tableau de variations de \( f \).
    \item Construire \( (C_f) \). \textbf{(on prendra \( \alpha = 0{,}3 \)).}
\end{enumerate}
\textbf{Partie C :} Soit \( h \) la restriction de \( f \) à l’intervalle \( I = [1 ; +\infty[ \).

\begin{enumerate}
    \item Montrer que \( h \) réalise une bijection de \( I \) vers un intervalle \( J \) à préciser.
    \item Soit \( h^{-1} \) la bijection réciproque de \( h \).\\
    Étudier la dérivabilité de \( h^{-1} \) sur \( J \).
    \item Calculer \( h(2) \) et \( \left(h^{-1}\right)'\left( \dfrac{2 \ln 2}{3} \right) \).
    \item Construire \( (C_{h^{-1}}) \), la courbe de \( h^{-1} \) dans le repère \( (O ; \vec{i} ; \vec{j}) \).
\end{enumerate}

\vspace{1em}
\textbf{Partie D :}

\begin{enumerate}
    \item À l’aide d’une intégration par parties, calculer l’intégrale 
    \[
    I = \int_1^e x \ln x\,dx.
    \]

    \item Montrer que pour tout \( x \in [1 ; e] \),
    \[
    \dfrac{x \ln x}{e + 1} \leq f(x) \leq \dfrac{x \ln x}{2}.
    \]

    \item En déduire que :
    \[
    \dfrac{e^2 + 1}{4(e + 1)} \leq \int_1^e f(x)\,dx \leq \dfrac{e^2 + 1}{8}.
    \]
\end{enumerate}

\exo{13} \textbf{BAC 2008 1\textsuperscript{er} GROUPE}

\textbf{Partie A :} Soit \( f \) la fonction numérique définie par :
\[
f(x) = 
\begin{cases}
x + 2 + \ln\left( \dfrac{x - 1}{x + 1} \right) & \text{si } x < 0 \\
(2 + x)e^{-x} & \text{si } x \geq 0
\end{cases}
\]
et \( (C_f) \) sa courbe représentative dans un repère orthonormé \( (O ; \vec{i} ; \vec{j}) \) \textbf{(unité 1 cm)}.

\begin{enumerate}
    \item Montrer que \( f \) est définie sur \( \mathbb{R} \setminus \{-1\} \).
    
    \item
    \begin{enumerate}
        \item Calculer les limites aux bornes du domaine de définition de \( f \).\\
        Préciser les asymptotes parallèles aux axes.

        \item Calculer \( \lim\limits_{x \to +\infty} \left[ f(x) - (x - 2) \right] \).\\
        Interpréter graphiquement le résultat.
    \end{enumerate}
    
    \item
    \begin{enumerate}
        \item Étudier la continuité de \( f \) en 0.

        \item Démontrer que : 
        \[
        \lim\limits_{x \to 0} \dfrac{e^{-x} - 1}{x} = -1
        \quad \text{et} \quad
        \lim\limits_{x \to 0} \dfrac{\ln(1 - x)}{x} = -1.
        \]

        \item En déduire que \( f \) est dérivable à gauche et à droite en 0.\\
        \( f \) est-elle dérivable en 0 ?
    \end{enumerate}

    \item Calculer \( f'(x) \) pour :
    \begin{enumerate}
        \item \( x \in ]0 ; +\infty[ \).
        \item \( x \in ]-\infty ; -1[ \cup ]-1 ; 0[ \).
    \end{enumerate}

    \item Étudier le signe de \( f'(x) \) pour :
    \begin{enumerate}
        \item \( x \in ]0 ; +\infty[ \).
        \item \( x \in ]-\infty ; -1[ \cup ]-1 ; 0[ \).
    \end{enumerate}

    \item Dresser le tableau de variations de \( f \).
    \item Montrer que l’équation \( f(x) = 0 \) admet une unique solution \( \alpha \) appartenant à \( ]-3 ; -2[ \).
    
    \item Tracer \( (C_f) \). On mettra en évidence l’allure de \( (C_f) \) au point d’abscisse 0 et les droites asymptotes.
\end{enumerate}

\textbf{Partie B :} Soit \( g \) la restriction de \( f \) à l’intervalle \( ]-\infty ; -1[ \).

\begin{enumerate}
    \item Montrer que \( g \) définit une bijection de \( ]-\infty ; -1[ \) sur un intervalle \( J \) à préciser.
    
    \item On note \( g^{-1} \) sa bijection réciproque.
    \begin{enumerate}
        \item Calculer \( g(-2) \). Montrer que \( g^{-1} \) est dérivable en \( \ln 3 \).
        \item Calculer \( (g^{-1})'(\ln 3) \).
        \item Représenter la courbe de \( g^{-1} \) dans le repère précédent.
    \end{enumerate}
\end{enumerate}

\textbf{Partie C :} Soit \( A \) l’aire de la région du plan délimitée par les droites d’équations respectives \( x = -2 \), \( x = -3 \) et \( y = x + 2 \) et la courbe de \( f \).

\exo{14} \textbf{BAC 2009 1\textsuperscript{er} GROUPE}

\begin{enumerate}
    \item 
    \begin{enumerate}
        \item Étudier les variations de la fonction \( f \) définie sur \( ]-1 ; +\infty[ \) par : \( f(x) = 2 \ln(x + 1) \).\\
        Tracer sa courbe représentative \( (C) \) et \( (T) \) dans le repère orthonormal \( (O ; \vec{i} ; \vec{j}) \), unité 2 cm.

        \item Démontrer que sur \( [2 ; +\infty[ \), la fonction \( l \), définie par \( l(x) = f(x) - x \), est bijective et que l’équation \( l(x) = 0 \) admet une solution unique \( \lambda \).
    \end{enumerate}

    \item On considère la suite \( (U_n)_{n \in \mathbb{N}} \) définie par :
    \[
    \begin{cases}
    U_0 = 5 \\
    U_{n+1} = 2 \ln(1 + U_n)
    \end{cases}
    \]

    \begin{enumerate}
        \item Sans faire de calcul, représenter les quatre premiers termes de la suite sur le graphique.

        \item Démontrer par récurrence que pour tout \( n \), \( U_n \geq 2 \).

        \item Montrer que pour tout \( x \in [2 ; +\infty[ \), on a :
        \[
        |f'(x)| \leq \dfrac{2}{3}.
        \]

        \item En déduire que pour tout \( n \), on a :
        \[
        |U_{n+1} - \lambda| \leq \dfrac{2}{3} |U_n - \lambda|,
        \]
        \[
        |U_n - \lambda| \leq 2 \left( \dfrac{2}{3} \right)^n,
        \]
        et que \( (U_n) \) converge vers \( \lambda \).

        \item Déterminer le plus petit entier naturel \( p \) tel que :
        \[
        |U_p - \lambda| \leq 10^{-2}.
        \]
        Que représente \( U_p \) pour \( \lambda \) ?
    \end{enumerate}
\end{enumerate}

\end{multicols}

\end{document}
