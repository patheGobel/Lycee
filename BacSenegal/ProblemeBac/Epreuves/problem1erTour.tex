\documentclass[12pt,a4paper]{article}
\usepackage{amsmath,amssymb,mathrsfs,tikz,times,pifont}
\usepackage{enumitem}
\usepackage{multicol}
\usepackage{lmodern}
\newcommand\circitem[1]{%
\tikz[baseline=(char.base)]{
\node[circle,draw=gray, fill=red!55,
minimum size=1.2em,inner sep=0] (char) {#1};}}
\newcommand\boxitem[1]{%
\tikz[baseline=(char.base)]{
\node[fill=cyan,
minimum size=1.2em,inner sep=0] (char) {#1};}}
\setlist[enumerate,1]{label=\protect\circitem{\arabic*}}
\setlist[enumerate,2]{label=\protect\boxitem{\alph*}}
\everymath{\displaystyle}
\usepackage[left=1cm,right=1cm,top=1cm,bottom=1.7cm]{geometry}
\usepackage[colorlinks=true, linkcolor=blue, urlcolor=blue, citecolor=blue]{hyperref}
\usepackage{array,multirow}
\usepackage[most]{tcolorbox}
\usepackage{varwidth}
\usepackage{float}
\tcbuselibrary{skins,hooks}
\usetikzlibrary{patterns}

\newtcolorbox{exa}[2][]{enhanced,breakable,before skip=2mm,after skip=5mm,
colback=yellow!20!white,colframe=black!20!blue,boxrule=0.5mm,
attach boxed title to top left ={xshift=0.6cm,yshift*=1mm-\tcboxedtitleheight},
fonttitle=\bfseries,
title={#2},#1,
boxed title style={frame code={
\path[fill=tcbcolback!30!black]
([yshift=-1mm,xshift=-1mm]frame.north west)
arc[start angle=0,end angle=180,radius=1mm]
([yshift=-1mm,xshift=1mm]frame.north east)
arc[start angle=180,end angle=0,radius=1mm];
\path[left color=tcbcolback!60!black,right color = tcbcolback!60!black,
middle color = tcbcolback!80!black]
([xshift=-2mm]frame.north west) -- ([xshift=2mm]frame.north east)
[rounded corners=1mm]-- ([xshift=1mm,yshift=-1mm]frame.north east)
-- (frame.south east) -- (frame.south west)
-- ([xshift=-1mm,yshift=-1mm]frame.north west)
[sharp corners]-- cycle;
},interior engine=empty,
},interior style={top color=yellow!5}}

\usepackage{fancyhdr}
\usepackage{eso-pic}
\usepackage{tkz-tab}
\AddToShipoutPicture{
    \AtTextCenter{%
        \makebox[0pt]{\rotatebox{80}{\textcolor[gray]{0.7}{\fontsize{5cm}{5cm}\selectfont PGB}}}
    }
}
\usepackage{lastpage}
\fancyhf{}
\pagestyle{fancy}
\renewcommand{\footrulewidth}{1pt}
\renewcommand{\headrulewidth}{0pt}
\renewcommand{\footruleskip}{10pt}
\fancyfoot[R]{\color{blue}\ding{45}\ \textbf{2025}}
\fancyfoot[L]{\color{blue}\ding{45}\ \textbf{Prof : M. BA}}
\cfoot{\bf \thepage / \pageref{LastPage}}

\newcommand{\exo}[1]{%
        \textbf{\underline{Problème #1}}
}

\begin{document}
\renewcommand{\arraystretch}{1.5}
\renewcommand{\arrayrulewidth}{1.2pt}
\begin{tikzpicture}[overlay,remember picture]
    \node[draw=blue,line width=1.2pt,fill=purple,text=blue,inner sep=3mm,rounded corners,pattern=dots]at ([yshift=-2.5cm]current page.north) {\begingroup\setlength{\fboxsep}{0pt}\colorbox{white}{\begin{tabular}{|*1{>{\centering \arraybackslash}p{0.28\textwidth}} |*2{>{\centering \arraybackslash}p{0.2\textwidth}|} *1{>{\centering \arraybackslash}p{0.19\textwidth}|} }
                \hline
                \multicolumn{3}{|c|}{$\diamond$$\diamond$$\diamond$\ \textbf{Lycée de Dindéfélo}\ $\diamond$$\diamond$$\diamond$ } & \textbf{A.S. : 2024/2025} \\ \hline
                \textbf{Matière : Mathématiques} & \textbf{Niveau : T S2} & \textbf{Date : 09/06/2025} & \textbf{} \\ \hline
                \multicolumn{4}{|c|}{\parbox[c]{10cm}{\begin{center}
                  \textbf{{\Large\sffamily Problèmes proposés au BAC S2 Sénégal de 1999 à 2022}}
                \end{center}}} \\ \hline
            \end{tabular}}\endgroup};
\end{tikzpicture}
\vspace{3cm}

\begin{multicols}{2}
\small
\setlength{\columnseprule}{0.1mm}

\exo{1} \textbf{Extrait BAC 1999 $1^{er}$ groupe}
On considère la fonction \( f \) définie par :

\( 
f(x) =
\begin{cases}
x + \ln\left|\dfrac{x - 1}{x + 1}\right| & \text{si } x \in ]-\infty, -1[ \cup ]-1, 0[ \\
x^2 e^{-x} & \text{si } x \in [0, +\infty[
\end{cases}
\)
et \( (C_f) \) sa courbe représentative dans un repère orthonormé \( (O; \vec{i}, \vec{j}) \), d’unité 2 cm.

\subsection*{Partie A}
\begin{enumerate}
  \item Déterminer l’ensemble de définition \( D_f \) de \( f \). Calculer \( f(-2) \) et \( f(3) \).
  \item Calculer les limites aux bornes de \( D_f \).
  \item Étudier la continuité de \( f \) en 0.
  \item 
  \begin{enumerate}
    \item Établir que la dérivée de \( f \) est donnée par :
    
\(
    f'(x) =
    \begin{cases}
    \dfrac{x^2 + 1}{x^2 - 1}  \text{si } x \in ]-\infty, -1[ \cup ]-1, 0[ \\
    x e^{-x}(2 - x) \text{si } x \in [0, +\infty[
    \end{cases}
\)
    \item La fonction \( f \) est-elle dérivable en 0 ? Justifier votre réponse.
    \item Dresser le tableau de variations de \( f \).
  \end{enumerate}
  \item Démontrer que l’équation \( f(x) = 0 \) admet une solution unique \( \alpha \) comprise entre \( -1{,}6 \) et \( -1{,}5 \).
  \item 
  \begin{enumerate}
    \item Justifier que la droite \( (D) \) d’équation 
    
    \( y = x \) est une asymptote à la courbe \( (C_f) \) en \( -\infty \).
    \item Étudier la position relative de \( (C_f) \) par rapport à la droite \( (D) \) pour
    
     \( x \in ]-\infty, -1[ \cup ]-1, 0[ \).
    \item Tracer \( (C_f) \).
  \end{enumerate}
\end{enumerate}

\subsection*{Partie B : } Soit \( g \) la restriction de \( f \) à \( I = [0;2] \).
\begin{enumerate}
    \item Montrer que \( g \) définit une bijection de \( I \) vers un intervalle \( J \) à préciser.
    \item On note \( g^{-1} \) la bijection réciproque de \( g \).
    \begin{enumerate}
        \item Résoudre l’équation \( g^{-1}(x) = 1 \).
        \item Montrer que \( \left(g^{-1}\right)'\left(\dfrac{1}{e}\right) = e \).
        \item Construire \( (C_{g^{-1}}) \), la courbe de \( g^{-1} \).
    \end{enumerate}
\end{enumerate}


\subsection*{Partie C : } \( \beta \) étant un réel strictement positif, on pose :

\(I(\beta) = \int_0^{\beta} f(x)\,dx \)
\begin{enumerate}
    \item 
    \begin{enumerate}
        \item Interpréter graphiquement \( I(\beta) \).
        \item En procédant par une intégration par parties, calculer \( I(\beta) \).
    \end{enumerate}
    
    \item Calculer \( \lim_{\beta \to -\infty} I(\beta) \).
    
    \item On pose \( \beta = 2 \).
    \begin{enumerate}
        \item Calculer \( I(2) \).
        \item En déduire la valeur en \( \text{cm}^2 \) de l’aire du domaine du plan délimité par la courbe \( (C_f) \), l’axe des abscisses et les droites d’équations \( x = 0 \) et \( x = \dfrac{4}{e^2} \).
    \end{enumerate}
\end{enumerate}

\exo{2} \textbf{BAC 1999 Remplacement}

Soit \( f \) la fonction définie sur \( \mathbb{R} \) par :\\
\(
f(x) =
\begin{cases}
e^{-\frac{1}{x^2}} & \text{si } x \in ]-\infty ; 0[ \\
\ln\left|\dfrac{x - 1}{x + 1}\right| & \text{si } x \in [0 ; 1[ \cup ]1 ; +\infty[
\end{cases}
\)\\
et \( (C_f) \) sa courbe représentative dans un repère orthonormé \( (O ; \vec{i} ; \vec{j}) \) d’unité \( 2\ cm \).

\subsection*{Partie A : }
\begin{enumerate}
    \item Étudier la continuité de \( f \) en 0.
    \item 
    \begin{enumerate}
        \item Montrer que \( \forall x \in ]0 ; 1[ \),\\
        \( \dfrac{f(x)}{x} = \dfrac{\ln(1 - x)}{x} - \dfrac{\ln(1 + x)}{x} \)
        \item Étudier la dérivabilité de \( f \) en 0.
        \item En déduire que \( (C_f) \) admet au point d’abscisse 0 deux demi-tangentes dont on donnera les équations.
    \end{enumerate}
        \item Étudier les variations de \( f \).
				\item Tracer \( (C_f) \).
\end{enumerate}				
    \subsection*{Partie B : } Soit \( g \) la restriction de \( f \) à \( ]1 ; +\infty[ \).
\begin{enumerate}

    \item Montrer que \( g \) est une bijection de \( ]1 ; +\infty[ \) vers un intervalle \( J \) à préciser.\\
    On notera \( g^{-1} \) la bijection réciproque de \( g \).

    \item Montrer que l’équation \( g(x) = -e \) admet une unique solution \( \alpha \) sur l’intervalle \( ]1 ; +\infty[ \).\\
    \textbf{(On ne demande pas de calculer \( \alpha \)).}

    \item Montrer que \( \forall x \in J \), \( g^{-1}(x) = 1 - \dfrac{e^x}{e^x - 1} \).

    \item Construire \( (C_{g^{-1}}) \).\\
    \textbf{(On indiquera la nature et l’équation de chacune des asymptotes à \( (C_g) \) et \( (C_{g^{-1}}) \)).}
\item Calculer en \( \text{cm}^2 \) l’aire \( A \) de l’ensemble des points \( M(x ; y) \) défini par :

\(
\left\{
\begin{array}{l}
-\ln 7 \leq x \leq -1 \\
0 \leq y \leq g^{-1}(x)
\end{array}
\right.
\)

\end{enumerate}


\exo{3} 
\begin{enumerate}
    \item 
\end{enumerate}

\vspace{1em}

\end{multicols}

\end{document}
