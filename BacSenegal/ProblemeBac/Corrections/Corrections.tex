\documentclass[12pt]{article}
\usepackage[utf8]{inputenc}
\usepackage[french]{babel}
\usepackage{amsmath}
\usepackage{amssymb}
\usepackage{geometry}
\geometry{margin=2.5cm}

\begin{document}

\section*{Problème 9 --- BAC 2004 Remplacement}

\textbf{Partie A :} Soit l’équation différentielle :  
\[
(E) : -\frac{1}{2} y'' + \frac{3}{2} y' - y = 0
\]

Déterminer la solution \( g \) de \( (E) \) dont la courbe représentative \( (C) \) passe par le point \( A(0 ; -1) \) et dont la tangente en ce point est parallèle à l’axe des abscisses.

\subsection*{1. Mise sous forme canonique}
On commence par multiplier l'équation par \( -2 \) :
\[
y'' - 3 y' + 2 y = 0
\]

\subsection*{2. Résolution de l'équation homogène}
C'est une équation linéaire homogène à coefficients constants. On cherche une solution sous la forme \( y = e^{rx} \).

On résout l'équation caractéristique :
\[
r^2 - 3r + 2 = 0 \Rightarrow (r - 1)(r - 2) = 0
\Rightarrow r_1 = 1 \quad \text{et} \quad r_2 = 2
\]

Donc la solution générale est :
\[
y(x) = \lambda e^{x} + \mu e^{2x}
\]

\subsection*{3. Détermination des constantes}
On cherche \( \lambda \) et \( \mu \) tels que :
\[
g(0) = -1 \quad \text{et} \quad g'(0) = 0
\]

On calcule :
\[
g(0) = \lambda e^0 + \mu e^0 = \lambda + \mu = -1 \tag{1}
\]
\[
g'(x) = \lambda e^{x} + 2\mu e^{2x} \Rightarrow g'(0) = \lambda + 2\mu = 0 \tag{2}
\]

Résolvons le système :
\[
\begin{cases}
\lambda + \mu = -1 \\
\lambda + 2\mu = 0
\end{cases}
\Rightarrow \text{Soustraction : } (\lambda + 2\mu) - (\lambda + \mu) = \mu = 1
\Rightarrow \lambda = -2
\]

\subsection*{4. Solution particulière}

\[
\boxed{g(x) = -2 e^x + e^{2x}}
\]

\end{document}
