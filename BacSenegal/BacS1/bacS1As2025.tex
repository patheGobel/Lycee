\documentclass[11pt]{article}
\usepackage[utf8]{inputenc}
\usepackage[T1]{fontenc}
\usepackage[french]{babel}
\usepackage{amsmath, amssymb}
\usepackage{enumitem}
\usepackage[top=2cm, bottom=2cm, left=2.2cm, right=2.2cm]{geometry}

\begin{document}

\begin{center}
    \textbf{UNIVERSITÉ CHEIKH ANTA DIOP DE DAKAR} \\
    \textbf{OFFICE DU BACCALAURÉAT} \\
    Email : \texttt{office@ucad.edu.sn} \\
    Site web : \texttt{officedubac.sn} \\
    \vspace{0.3cm}
    \textbf{\LARGE MATHÉMATIQUES} \\
    Séries : S1–S1A–S3 \hfill Coef. 8 \\
    \textbf{Épreuve du 1\textsuperscript{er} groupe} \\
    Durée : 4 heures
\end{center}

\vspace{0.2cm}
\noindent
\textbf{Les calculatrices électroniques non imprimantes avec entrée unique par clavier sont autorisées. Les calculatrices permettant d’afficher des formulaires ou des tracés de courbe sont interdites. Leur utilisation sera considérée comme une fraude (Cf. Circulaire n° 5990/OB/DIR. du 12 08 1998).}

\vspace{0.5cm}
\noindent
\textbf{EXERCICE 1 :} \hfill (04,75 points)

\vspace{0.2cm}
\noindent
Pour préparer les tests de sélection aux Jeux Olympiques de la Jeunesse de Dakar 2026, les athlètes disposent de deux stades \( A \) et \( B \) pour les entraînements.

\vspace{0.3cm}
\noindent
\textbf{PARTIE I} \hfill (02,5 points)

\vspace{0.2cm}
\noindent
Un athlète doit s’entraîner deux jours consécutifs.

\begin{itemize}
    \item Le premier jour, la probabilité qu’il choisisse le stade \( A \) est égale à \( \alpha \).
    \item Le second jour, on admet que la probabilité qu’il choisisse un stade différent de celui fréquenté la veille est \( 0{,}8 \).
\end{itemize}

\noindent
Pour \( j \in \{1,2\} \), on note les événements suivants :
\[
A_j : \text{« l’athlète choisit le stade } A \text{ le } j^\text{ième} \text{ jour »}, \quad
B_j : \text{« l’athlète choisit le stade } B \text{ le } j^\text{ième} \text{ jour »}.
\]

\begin{enumerate}
    \item Déterminer la valeur de \( \alpha \) pour que les événements \( A_1 \) et \( A_2 \) aient la même probabilité. \hfill (01 point)

    \noindent \textit{Dans toute la suite de l’exercice, on prendra \( \alpha = 0{,}5 \).}

    \item Calculer la probabilité qu’un athlète se rende au même stade pendant les deux jours. \hfill (0,75 point)

    \item Au deuxième jour, on aperçoit un athlète sortant du stade \( B \). Quelle est la probabilité qu’il se soit entraîné au même stade la veille ? \hfill (0,75 point)
\end{enumerate}

\vspace{0.4cm}
\noindent
\textbf{PARTIE II} \hfill (02,25 points)

\vspace{0.2cm}
\noindent
Au premier jour, on a \( n \) athlètes (\( n \geq 3 \)) qui doivent s’entraîner. Chacun d’entre eux choisit, au hasard et indépendamment des choix des autres, l’un des deux stades où il doit s’entraîner.

\noindent
On suppose que les deux stades ne contiennent aucun athlète au départ.

\noindent
On dit qu’un athlète est heureux s’il se trouve seul dans un stade où s’entraîner.

\begin{enumerate}
    \item Quelle est la probabilité qu’il y ait deux athlètes heureux ? \hfill (0,5 point)

    \item Soit \( p_n \) la probabilité qu’il y ait un athlète heureux parmi ces \( n \) athlètes.
    \begin{enumerate}
        \item[a)] Montrer, par un raisonnement naturel (\( n \geq 3 \)), que : \( p_n = \frac{2n}{2^{n-1}} \). \hfill (0,75 point)
        \item[b)] Étudier la variation de la suite \( (p_n)_{n \geq 3} \). \hfill (0,5 point)
        \item[c)] Déterminer la plus petite valeur de \( n \) pour laquelle la probabilité d’avoir un athlète heureux est inférieure à \( 0{,}01 \). \hfill (0,5 point)
    \end{enumerate}
\end{enumerate}

\vfill
\begin{flushright}
    \textit{.../2}
\end{flushright}

\end{document}
