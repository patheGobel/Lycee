\documentclass[12pt,a4paper]{article}
\usepackage{amsmath,amssymb,mathrsfs,tikz,times,pifont}
\usepackage[T1]{fontenc}
\usepackage{enumitem}
\newcommand\circitem[1]{%
\tikz[baseline=(char.base)]{
\node[circle,draw=gray, fill=red!55,
minimum size=1.2em,inner sep=0] (char) {#1};}}
\newcommand\boxitem[1]{%
\tikz[baseline=(char.base)]{
\node[fill=cyan,
minimum size=1.2em,inner sep=0] (char) {#1};}}
\setlist[enumerate,1]{label=\protect\circitem{\arabic*}}
\setlist[enumerate,2]{label=\protect\boxitem{\alph*}}
%%%::::::by chnini ameur :::::::%%%
\everymath{\displaystyle}
\usepackage[left=1cm,right=1cm,top=1cm,bottom=1.7cm]{geometry}
\usepackage[colorlinks=true, linkcolor=blue, urlcolor=blue, citecolor=blue]{hyperref}
\usepackage{array,multirow}
\usepackage[most]{tcolorbox}
\usepackage{varwidth}
\usepackage{float} %pour utiliser l'option [H] qui force l'image à apparaître exactement à l'endroit où elle est placée dans le code.
\tcbuselibrary{skins,hooks}
\usetikzlibrary{patterns}
%%%::::::by chnini ameur :::::::%%%
\newtcolorbox{exa}[2][]{enhanced,breakable,before skip=2mm,after skip=5mm,
colback=yellow!20!white,colframe=black!20!blue,boxrule=0.5mm,
attach boxed title to top left ={xshift=0.6cm,yshift*=1mm-\tcboxedtitleheight},
fonttitle=\bfseries,
title={#2},#1,
% varwidth boxed title*=-3cm,
boxed title style={frame code={
\path[fill=tcbcolback!30!black]
([yshift=-1mm,xshift=-1mm]frame.north west)
arc[start angle=0,end angle=180,radius=1mm]
([yshift=-1mm,xshift=1mm]frame.north east)
arc[start angle=180,end angle=0,radius=1mm];
\path[left color=tcbcolback!60!black,right color = tcbcolback!60!black,
middle color = tcbcolback!80!black]
([xshift=-2mm]frame.north west) -- ([xshift=2mm]frame.north east)
[rounded corners=1mm]-- ([xshift=1mm,yshift=-1mm]frame.north east)
-- (frame.south east) -- (frame.south west)
-- ([xshift=-1mm,yshift=-1mm]frame.north west)
[sharp corners]-- cycle;
},interior engine=empty,
},interior style={top color=yellow!5}}
%%%%%%%%%%%%%%%%%%%%%%%

\usepackage{fancyhdr}
\usepackage{eso-pic}         % Pour ajouter des éléments en arrière-plan
% Commande pour ajouter du texte en arrière-plan
\usepackage{tkz-tab}
\AddToShipoutPicture{
    \AtTextCenter{%
        \makebox[0pt]{\rotatebox{80}{\textcolor[gray]{0.7}{\fontsize{5cm}{5cm}\selectfont PGB}}}
    }
}
\usepackage{lastpage}
\fancyhf{}
\pagestyle{fancy}
\renewcommand{\footrulewidth}{1pt}
\renewcommand{\headrulewidth}{0pt}
\renewcommand{\footruleskip}{10pt}
\fancyfoot[R]{
\color{blue}\ding{45}\ \textbf{2025}
}
\fancyfoot[L]{
\color{blue}\ding{45}\ \textbf{Prof:M. BA}
}
\cfoot{\bf
\thepage /
\pageref{LastPage}}
\usetikzlibrary{trees} % Bibliothèque pour les arbres
\begin{document}
\renewcommand{\arraystretch}{1.5}
\renewcommand{\arrayrulewidth}{1.2pt}
\begin{tikzpicture}[overlay,remember picture]
\node[draw=blue,line width=1.2pt,fill=purple,text=blue,inner sep=3mm,rounded corners,pattern=dots]at ([yshift=-2.5cm]current page.north) {\begingroup\setlength{\fboxsep}{0pt}\colorbox{white}{\begin{tabular}{|*1{>{\centering \arraybackslash}p{0.28\textwidth}} |*2{>{\centering \arraybackslash}p{0.2\textwidth}|} *1{>{\centering \arraybackslash}p{0.19\textwidth}|} }
\hline
\multicolumn{3}{|c|}{$\diamond$$\diamond$$\diamond$\ \textbf{Lycée de Dindéfélo}\ $\diamond$$\diamond$$\diamond$ }& \textbf{A.S. : 2024/2025} \\ \hline
\textbf{Matière: Mathématiques}& \textbf{Niveau : T}\textbf{S2} &\textbf{Date: 22/05/2025} & \textbf{Durée : 4 heures} \\ \hline
\multicolumn{4}{|c|}{\parbox[c]{10cm}{\begin{center}
\textbf{{\Large\sffamily Correction Composition Du 2$ ^\text{\bf nd} $ Semestre}}
\end{center}}} \\ \hline
\end{tabular}}\endgroup};
\end{tikzpicture}
\vspace{3cm}
\section*{\underline{Exercice 1 :($04.75$ pts)}}
\subsection*{Partie I : (02,5 points)}
Construisons un arbre pondère correspondant à cette épreuve.
% Styles pour les différents niveaux de l'arbre
\tikzstyle{level 1}=[level distance=2cm, sibling distance=3.5cm] % Niveau 1 : distance verticale et entre branches
\tikzstyle{level 2}=[level distance=3.5cm, sibling distance=3cm] % Niveau 2 : plus resserré
\tikzstyle{bag} = [text centered] % Style pour les nœuds intermédiaires
\tikzstyle{end} = [text centered] % Style pour les feuilles
\definecolor{darkgreen}{RGB}{0,100,0} % Couleur vert foncé (non utilisée ici)
\definecolor{darkred}{RGB}{139,0,0} % Couleur rouge foncé

\begin{tikzpicture}[grow=right, sloped] % Arbre qui se développe vers la droite, étiquettes inclinées

% Nœud racine (départ de l'univers)
\node[bag] {\textcolor{darkred}{}} 
    child {
        % Première branche : événement B
        node[bag] {\textcolor{darkred}{$B_1$}} 
            child {
                % Sous-branche de B : issue V
                node[end] {\textcolor{darkred}{$A_2$}} 
                edge from parent
                node[above] {\textcolor{black}{$0,8$}} % Probabilité conditionnelle : P(V|B)
            }
            child {
                % Sous-branche de B : issue R
                node[end] {\textcolor{darkred}{$B_2$}} 
                edge from parent
                node[above] {\textcolor{black}{$0,2$}} % Probabilité à compléter
            }
        edge from parent
        node[above] {\textcolor{black}{$1-\alpha$}} % Probabilité de B : P(B)
    }
    child {
        % Deuxième branche : événement A
        node[bag] {\textcolor{darkred}{$A_1$}} 
            child {
                % Sous-branche de A : issue V
                node[end] {\textcolor{darkred}{$B_2$}} 
                edge from parent
                node[above] {\textcolor{black}{$0,8$}} % Probabilité à compléter
            }
            child {
                % Sous-branche de A : issue R
                node[end] {\textcolor{darkred}{$A_2$}} 
                edge from parent
                node[above] {\textcolor{black}{$0,2$}} % Probabilité conditionnelle : P(R|A)
            }
        edge from parent
        node[above] {\textcolor{black}{$\alpha$}} % Probabilité de A : P(A), à compléter
    };
\end{tikzpicture}

\begin{enumerate}
\item Déterminons la valeur de $\alpha$

$
\begin{aligned}
P(A_2) &= P(A_1) \times P_{A_1}(A_2) + P(B_1) \times P_{B_1}(A_2)\\
			 &= \alpha \times 0,2 + (1-\alpha) \times 0,8\\
			 &= 0,2\alpha + 0,8-0,8\alpha\\
			 &= -0,6\alpha + 0,8\\
\end{aligned}
$

$
\begin{aligned}
\textbf{Si } P(A_1)=P(A_2) &\implies \alpha = -0,6\alpha + 0,8\\
													 &\implies 1,6\alpha = 0,8\\
													 &\alpha = \frac{0,8}{1,6}\\
													 &\alpha = 0,5
\end{aligned}
$
\item Calculons la probabilité qu’un athlète se rende au même stade pendant les deux jours.

$A_1$ :  «\text{ l'athlète choisit le stade A le $1^{er}$jour} »

$B_1$ :  «\text{ l'athlète choisit le stade B le $1^{er}$jour} »

$A_2$ :  «\text{ l'athlète choisit le stade A le $2^{er}$jour} »

$B_2$ :  «\text{ l'athlète choisit le stade B le $2^{er}$jour} »

Un athlète se rende au même stade pendant les deux jours se traduit par: $A_1\cap A_2$ ou $B_1\cap B_2$

$
\begin{aligned}
P((A_1\cap A_2) \cup (B_1\cap B_2)) &= P(A_1\cap A_2) + P(B_1\cap B_2)\\
																	  &=P(A_1) \times P_{A_1}(A_2) + P(B_1) \times P_{B_1}(A_2)\\
																	  &=0,5 \times 0,2 + 0,5 \times 0,8\\
																	  &=0,1 + 0,4\\
																	  &=0,5\\
\end{aligned}
$

\end{enumerate}

\end{document}
