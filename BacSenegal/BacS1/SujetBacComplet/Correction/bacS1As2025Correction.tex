\documentclass[12pt,a4paper]{article}
\usepackage{amsmath,amssymb,mathrsfs,tikz,times,pifont}
\usepackage[T1]{fontenc}
\usepackage{enumitem}
\newcommand\circitem[1]{%
\tikz[baseline=(char.base)]{
\node[circle,draw=gray, fill=red!55,
minimum size=1.2em,inner sep=0] (char) {#1};}}
\newcommand\boxitem[1]{%
\tikz[baseline=(char.base)]{
\node[fill=cyan,
minimum size=1.2em,inner sep=0] (char) {#1};}}
\setlist[enumerate,1]{label=\protect\circitem{\arabic*}}
\setlist[enumerate,2]{label=\protect\boxitem{\alph*}}
%%%::::::by chnini ameur :::::::%%%
\everymath{\displaystyle}
\usepackage[left=1cm,right=1cm,top=1cm,bottom=1.7cm]{geometry}
\usepackage[colorlinks=true, linkcolor=blue, urlcolor=blue, citecolor=blue]{hyperref}
\usepackage{array,multirow}
\usepackage[most]{tcolorbox}
\usepackage{varwidth}
\usepackage{float} %pour utiliser l'option [H] qui force l'image à apparaître exactement à l'endroit où elle est placée dans le code.
\tcbuselibrary{skins,hooks}
\usetikzlibrary{patterns}
%%%::::::by chnini ameur :::::::%%%
\newtcolorbox{exa}[2][]{enhanced,breakable,before skip=2mm,after skip=5mm,
colback=yellow!20!white,colframe=black!20!blue,boxrule=0.5mm,
attach boxed title to top left ={xshift=0.6cm,yshift*=1mm-\tcboxedtitleheight},
fonttitle=\bfseries,
title={#2},#1,
% varwidth boxed title*=-3cm,
boxed title style={frame code={
\path[fill=tcbcolback!30!black]
([yshift=-1mm,xshift=-1mm]frame.north west)
arc[start angle=0,end angle=180,radius=1mm]
([yshift=-1mm,xshift=1mm]frame.north east)
arc[start angle=180,end angle=0,radius=1mm];
\path[left color=tcbcolback!60!black,right color = tcbcolback!60!black,
middle color = tcbcolback!80!black]
([xshift=-2mm]frame.north west) -- ([xshift=2mm]frame.north east)
[rounded corners=1mm]-- ([xshift=1mm,yshift=-1mm]frame.north east)
-- (frame.south east) -- (frame.south west)
-- ([xshift=-1mm,yshift=-1mm]frame.north west)
[sharp corners]-- cycle;
},interior engine=empty,
},interior style={top color=yellow!5}}
%%%%%%%%%%%%%%%%%%%%%%%

\usepackage{fancyhdr}
\usepackage{eso-pic}         % Pour ajouter des éléments en arrière-plan
% Commande pour ajouter du texte en arrière-plan
\usepackage{tkz-tab}
\AddToShipoutPicture{
    \AtTextCenter{%
        \makebox[0pt]{\rotatebox{80}{\textcolor[gray]{0.7}{\fontsize{5cm}{5cm}\selectfont PGB}}}
    }
}
\usepackage{lastpage}
\fancyhf{}
\pagestyle{fancy}
\renewcommand{\footrulewidth}{1pt}
\renewcommand{\headrulewidth}{0pt}
\renewcommand{\footruleskip}{10pt}
\fancyfoot[R]{
\color{blue}\ding{45}\ \textbf{2025}
}
\fancyfoot[L]{
\color{blue}\ding{45}\ \textbf{Prof:M. BA}
}
\cfoot{\bf
\thepage /
\pageref{LastPage}}
\usetikzlibrary{trees} % Bibliothèque pour les arbres
\begin{document}
\renewcommand{\arraystretch}{1.5}
\renewcommand{\arrayrulewidth}{1.2pt}
\begin{tikzpicture}[overlay,remember picture]
\node[draw=blue,line width=1.2pt,fill=purple,text=blue,inner sep=3mm,rounded corners,pattern=dots]at ([yshift=-2.5cm]current page.north) {\begingroup\setlength{\fboxsep}{0pt}\colorbox{white}{\begin{tabular}{|*1{>{\centering \arraybackslash}p{0.28\textwidth}} |*2{>{\centering \arraybackslash}p{0.2\textwidth}|} *1{>{\centering \arraybackslash}p{0.19\textwidth}|} }
\hline
\multicolumn{3}{|c|} {\parbox[c]{10cm}{\begin{center}
\textbf{{\Large\sffamily BAC 2025 }}
\end{center}}} \\ \hline
\textbf{Matière: Mathématiques}& \textbf{Niveau : T}\textbf{S1}  \\ \hline
\multicolumn{4}{|c|}{\parbox[c]{10cm}{\begin{center}
\textbf{{\Large\sffamily Correction }}
\end{center}}} \\ \hline
\end{tabular}}\endgroup};
\end{tikzpicture}
\vspace{3cm}
\section*{\underline{Exercice 1 :($04.75$ pts)}}
\subsection*{Partie I : (02,5 points)}
Construisons un arbre pondère correspondant à cette épreuve.
% Styles pour les différents niveaux de l'arbre
\tikzstyle{level 1}=[level distance=2cm, sibling distance=3.5cm] % Niveau 1 : distance verticale et entre branches
\tikzstyle{level 2}=[level distance=3.5cm, sibling distance=3cm] % Niveau 2 : plus resserré
\tikzstyle{bag} = [text centered] % Style pour les nœuds intermédiaires
\tikzstyle{end} = [text centered] % Style pour les feuilles
\definecolor{darkgreen}{RGB}{0,100,0} % Couleur vert foncé (non utilisée ici)
\definecolor{darkred}{RGB}{139,0,0} % Couleur rouge foncé

\begin{tikzpicture}[grow=right, sloped] % Arbre qui se développe vers la droite, étiquettes inclinées

% Nœud racine (départ de l'univers)
\node[bag] {\textcolor{darkred}{}} 
    child {
        % Première branche : événement B
        node[bag] {\textcolor{darkred}{$B_1$}} 
            child {
                % Sous-branche de B : issue V
                node[end] {\textcolor{darkred}{$A_2$}} 
                edge from parent
                node[above] {\textcolor{black}{$0,8$}} % Probabilité conditionnelle : P(V|B)
            }
            child {
                % Sous-branche de B : issue R
                node[end] {\textcolor{darkred}{$B_2$}} 
                edge from parent
                node[above] {\textcolor{black}{$0,2$}} % Probabilité à compléter
            }
        edge from parent
        node[above] {\textcolor{black}{$1-\alpha$}} % Probabilité de B : P(B)
    }
    child {
        % Deuxième branche : événement A
        node[bag] {\textcolor{darkred}{$A_1$}} 
            child {
                % Sous-branche de A : issue V
                node[end] {\textcolor{darkred}{$B_2$}} 
                edge from parent
                node[above] {\textcolor{black}{$0,8$}} % Probabilité à compléter
            }
            child {
                % Sous-branche de A : issue R
                node[end] {\textcolor{darkred}{$A_2$}} 
                edge from parent
                node[above] {\textcolor{black}{$0,2$}} % Probabilité conditionnelle : P(R|A)
            }
        edge from parent
        node[above] {\textcolor{black}{$\alpha$}} % Probabilité de A : P(A), à compléter
    };
\end{tikzpicture}

\begin{enumerate}
\item Déterminons la valeur de $\alpha$

$
\begin{aligned}
P(A_2) &= P(A_1) \times P_{A_1}(A_2) + P(B_1) \times P_{B_1}(A_2)\\
			 &= \alpha \times 0,2 + (1-\alpha) \times 0,8\\
			 &= 0,2\alpha + 0,8-0,8\alpha\\
			 &= -0,6\alpha + 0,8\\
\end{aligned}
$

$
\begin{aligned}
\textbf{Si } P(A_1)=P(A_2) &\implies \alpha = -0,6\alpha + 0,8\\
													 &\implies 1,6\alpha = 0,8\\
													 &\alpha = \frac{0,8}{1,6}\\
													 &\alpha = 0,5
\end{aligned}
$
\hfill \textbf{(01 point)}
\item Calculons la probabilité qu’un athlète se rende au même stade pendant les deux jours.

$A_1$ :  «\text{ l'athlète choisit le stade A le $1^{er}$jour} »

$B_1$ :  «\text{ l'athlète choisit le stade B le $1^{er}$jour} »

$A_2$ :  «\text{ l'athlète choisit le stade A le $2^{er}$jour} »

$B_2$ :  «\text{ l'athlète choisit le stade B le $2^{er}$jour} »

Un athlète se rende au même stade pendant les deux jours se traduit par: $A_1\cap A_2$ ou $B_1\cap B_2$

$
\begin{aligned}
P((A_1\cap A_2) \cup (B_1\cap B_2)) &= P(A_1\cap A_2) + P(B_1\cap B_2)\\
																	  &=P(A_1) \times P_{A_1}(A_2) + P(B_1) \times P_{B_1}(B_2)\\
																	  &=0,5 \times 0,2 + 0,5 \times 0,2\\
																	  &=0,1 + 0,1\\
																	  &=0,2\\
\end{aligned}
$


\begin{center}
$\boxed{P((A_1\cap A_2) \cup (B_1\cap B_2))=0,2}$\hfill \textbf{(0,75 point)}
\end{center}

		\item Au deuxième jour, on aperçoit un athlète sortant du stade \( B \). La probabilité qu’il se soit entraîné au même stade la veille

Se traduit par : entrer dans \( B \) deux jours successifs, c’est-à-dire : \( B_1 \textit{ sachant } B_2 \)

\(
\begin{aligned}
P_{B_2}(B_1) &= \dfrac{P(B_1 \cap B_2)}{P(B_2)} \\
            &= \dfrac{P(B_1) \times P_{B_1}(B_2)}{P(A_1) \times P_{A_1}(B_2) + P(B_1) \times P_{B_1}(B_2)} \\
            &= \dfrac{0,5 \times 0,2}{0,5 \times 0,8 + 0,5 \times 0,2} \\
            &= \dfrac{0,1}{0,4 + 0,1} \\
            &= \dfrac{0,1}{0,5} \\
            &= 0,2
\end{aligned}
\)

\begin{center}
\(\boxed{P_{B_2}(B_1) = 0,2}\) \hfill \textbf{(0,75 point)}
\end{center}



\end{enumerate}

\subsection*{Partie II :  (02,25 points)}

\begin{enumerate}
    \item La probabilité qu’il y ait deux athlètes heureux

		Il ne peut pas y avoir deux athlètes heureux car il ya au moins 3 athlètes  donc l'un des stades sera occupé par au moins deux athlètes 
		
		\begin{center}
 				\fbox{\textcolor{red}{La probabilité qu'il y ait deux athlètes heureux est nulle}} \hfill \textbf{(0,5 point)}
		\end{center}
     
     \item Un athlète est heureux s’il est seul dans un stade. On note la probabilité que cela arrive parmi \( n \) athlètes :

\noindent
Pour un athlète donné, l’épreuve qui consiste à choisir un stade est une \textbf{épreuve de Bernoulli} dont la probabilité du succès (le stade \( A \)) est \( 0,5 \).\\

\begin{enumerate}
\item Cette épreuve étant effectuée \( n \) fois de suite (par les \( n \) athlètes) et de manière indépendante, on a un \textbf{schéma de Bernoulli}.

\vspace{0.2cm}
\noindent
On a deux cas :
\begin{itemize}
    \item[\textbullet] Un athlète se présente dans le stade \( A \) et \( n - 1 \) athlètes dans le stade \( B \) : 1 succès ;
    \item[\textbullet] Un athlète se présente dans le stade \( B \) et \( n - 1 \) athlètes dans le stade \( A \) : \( n - 1 \) succès.
\end{itemize}

\vspace{0.2cm}
\noindent
La probabilité qu’il y ait un athlète heureux parmi ces \( n \) athlètes est :

\[
p_n = C_n^1 (0{,}5)^1 (1 - 0{,}5)^{n - 1} + C_n^1 (0{,}5)^{n - 1}(1 - 0{,}5)^1 = 2n \cdot (0{,}5)^n = \frac{n}{2^{n - 1}}
\]

\vspace{0.2cm}
\[
\boxed{p_n = \dfrac{n}{2^{n - 1}}} \quad \textbf{(0,75 pt)}
\]


\textcolor{red}{\textbf{Autre Approche}}

    \vspace{0.4cm}
\noindent

\textbf{Remarque : si on voulait le voir comme une variable aléatoire}

\vspace{0.2cm}
On peut modéliser le choix de chaque athlète comme une variable aléatoire de Bernoulli :
\begin{itemize}
    \item On note \( X \) le nombre d’athlètes qui choisissent le stade \( A \),
    \item Chaque athlète a une probabilité \( \frac{1}{2} \) de choisir \( A \) ou \( B \),
    \item Donc \( X \sim \mathcal{B}(n, \frac{1}{2}) \), c’est-à-dire une loi binomiale.
\end{itemize}

Un athlète est \textbf{heureux} s’il est \textbf{seul} dans un stade, ce qui correspond à :
\[
\text{un seul athlète dans } A \quad \text{ou} \quad \text{un seul athlète dans } B
\]
Autrement dit :
\[
p_n = \mathbb{P}(X = 1) + \mathbb{P}(X = n - 1)
\]
Avec la loi binomiale :
\[
\mathbb{P}(X = k) = \binom{n}{k} \left( \frac{1}{2} \right)^k \left( \frac{1}{2} \right)^{n - k} = \binom{n}{k} \left( \frac{1}{2} \right)^n
\]
Donc :
\[
p_n = \left[ \binom{n}{1} + \binom{n}{n - 1} \right] \cdot \left( \frac{1}{2} \right)^n = 2n \cdot \left( \frac{1}{2} \right)^n = \frac{2n}{2^n}
\]

Et comme :
\[
2^n = 2 \cdot 2^{n - 1} \Rightarrow \frac{2n}{2^n} = \frac{n}{2^{n - 1}}
\]
    
\[
\boxed{ p_n = \dfrac{n}{2^{n-1}} } \hfill \textbf{(0,75 point)}
\]

\item Étudions la variation de la suite \( (p_n)_{n \geq 3} \). \hfill \textbf{(0,5 point)}

		\( p_n = \dfrac{n}{2^{n-1}} \)
		
		On a 
		
		\(
		\begin{aligned}
		p_{n+1} - p_n &= \dfrac{(n+1)}{2^{n}} - \dfrac{n}{2^{n-1}}\\
									&= \dfrac{n+1}{2^{n}} - \dfrac{n}{2^{n-1}}\\
									&= \dfrac{n+1}{2^{n}} - \dfrac{2n}{2^{n}}\\
									&= \dfrac{-n+1}{2^{n}}\\
		\end{aligned}
		\)

Donc $p_{n+1} - p_n = \dfrac{-n+1}{2^{n}}$

Comme $n \geq 3$ alors $-n+1 < 0$ donc $\dfrac{-n+1}{2^{n} } < 0$ d'où $\forall n \geq 3$, $p_n <0$

Ainsi la suite $(p_n)$ est décroissante

La convergence de la suite 

\(
\begin{aligned} 
p_n = \dfrac{n}{2^{n-1}} &\implies p_n = \dfrac{2n}{2^{n}}\\
												 &\implies p_n = \dfrac{2n}{e^{\ln(2^{n})}}\\
												 &\implies p_n = \dfrac{2n}{e^{n\ln(2)}}\\
												 &\implies p_n = \dfrac{2}{\ln(2)}\times\dfrac{n\ln(2)}{e^{n\ln(2)}}\\
\end{aligned}
\)

Donc \( p_n = \dfrac{2}{\ln(2)}\times\dfrac{n\ln(2)}{e^{n\ln(2)}} \)

 \(
 \begin{aligned} 
  \lim\limits_{n \to +\infty} p_n &= \lim\limits_{n \to +\infty} \dfrac{2}{\ln(2)}\times\dfrac{n\ln(2)}{e^{n\ln(2)}} \\
  																&=\dfrac{2}{\ln(2)}\times  \lim\limits_{n \to +\infty} \dfrac{n\ln(2)}{e^{n\ln(2)}}\\
  																&=\dfrac{2}{\ln(2)}\times  0\\
  																&=0
  \end{aligned}
 \)
 
 D'où \( \lim\limits_{n \to +\infty} p_n =0 \) donc la suite \( (p_n)_{n \geq 3} \) converge vers \( 0 \)

        \item Calculons \( p_{10} \)
        
        \( p_n = \dfrac{n}{2^{n-1}} \)
        
        \(
        \begin{aligned} 
        	p_{10} &= \dfrac{10}{2^{10-1}}\\
        				 &= \dfrac{10}{2^{9}}\\
        				 &= \dfrac{5}{2^{8}}\\
        \end{aligned}
        \)

				\[ \boxed{p_{10} = \dfrac{5}{2^{8}} \approx 0,019}  \] \hfill \textbf{ (0,25 point) }    
        
        Déterminons la plus grande valeur de \( n \) pour laquelle la probabilité d’avoir un athlète heureux est supérieur à \( 0,005 \). \hfill \textbf{ (0,25 point) }

La suite $(p_{n})$ étant strictement décroissante, on va chercher la plus grande valeur de $n$ supérieur à 10
telles que $p_{n}$ reste supérieur à $0,005$      

    \vspace{0.2cm}
\noindent

\( p_n = \dfrac{n}{2^{n-1}} \)

Calculs :

\(
\begin{aligned}
p_{11} &= \dfrac{11}{2^{10}} \approx 0,01 > 0,005\\
p_{12} &= \dfrac{12}{2^{11}} \approx 0,0059 > 0,005\\
p_{13} &= \dfrac{13}{2^{12}} \approx 0,0031 < 0,005\\
\end{aligned}
\)

\vspace{0.2cm}
\noindent
\textbf{Conclusion :} La plus grande valeur de \( n \) pour laquelle la probabilité d’avoir un athlète heureux soit supérieur à $0,005$
est $12$.

\[
\boxed{n = 12}
\]    
\end{enumerate}
 
\end{enumerate}

\section*{\underline{Exercice 2 :(04,25 pts)}}

Soient \( (\Delta_1) \) et \( (\Delta_2) \) deux droites distinctes de l’espace.

\noindent
On note \( R_1 \) et \( R_2 \) les demi-tours d’axes respectifs \( (\Delta_1) \) et \( (\Delta_2) \).

\medskip
Le but de cet exercice est de déterminer une condition nécessaire et suffisante portant sur \( (\Delta_1) \) et \( (\Delta_2) \) pour que : \( R_1 \circ R_2 = R_2 \circ R_1. \)

\begin{enumerate}
\item \textbf{On suppose que \( (\Delta_1) \) et \( (\Delta_2) \) sont perpendiculaires en un point noté \( O \).}

On adoptera les notations suivantes :
\begin{itemize}
    \item Le plan contenant \( (\Delta_1) \) et \( (\Delta_2) \) est noté \( (P) \).
    \item La droite perpendiculaire en \( O \) au plan \( (P) \) est notée \( (\Delta) \).
    \item Le plan contenant \( (\Delta) \) et \( (\Delta_1) \) est noté \( (P_1) \).
    \item Le plan contenant \( (\Delta) \) et \( (\Delta_2) \) est noté \( (P_2) \).
    \item Les réflexions par rapport aux plans \( (P), (P_1), (P_2) \) sont respectivement notées \( S_P, S_{P_1}, S_{P_2} \).
\end{itemize}

\begin{enumerate}
    \item Faisons une figure en faisant apparaître clairement le point \( O \), les plans \( (P), (P_1), (P_2) \) ainsi que les droites \( (\Delta), (\Delta_1), (\Delta_2) \). \hfill \textbf{(0,75 point)}
    
    La figure, voir ce qui suit
    
    \item Déterminons \( S_{p} \circ S_{p_{1}} \) et \( S_{p_{2}} \circ S_{p} \)
    
    La transformation \( S_{p} \circ S_{p_{1}} \) est la rotation d’axe  $(P) \cap (P_{1}) = \Delta_{1}$ et  d’angle 2 $ \times $ [ l’angle formé par $(P) $ et $ (P_{1})$ ] $= 2 \times \frac{\pi}{2} = \pi $

C’est donc le demi-tour d’axe  $ \Delta_1 $ c’est à dire $ R_1 $ \hfill \textbf{(0,5 point)}

		La transformation \( S_{p_{2}} \circ S_{p}  \) est la rotation d’axe  $(P) \cap (P_{2}) = \Delta_{2}$ et  d’angle 2 $ \times $ [ l’angle formé par $(P) $ et $ (P_{2})$ ] $= 2 \times \frac{\pi}{2} = \pi $

C’est donc le demi-tour d’axe  $ \Delta_2 $ c’est à dire $ R_2 $ \hfill \textbf{(0,5 point)}

		\item En déduire que \( R_{2} \circ R_{1} \) est un demi-tour dont on précisera l’axe.
		
		 	\( R_{2} \circ R_{1} = ( S_{p_{2}} \circ S_{p}) \circ (S_{p} \circ S_{p_{1}}) = S_{p_{2}} \circ S_{p_{1}}\)
		\item Prouver alors que $R_1 \circ R_2 = R_2 \circ R_1$.
		
		On peut aussi écrire : $R_1 = S_{P_1} \circ S_P$. On a alors :
		
		\[ R_1 \circ R_2 = (S_{P_1} \circ S_P) \circ (S_P \circ S_{P_2}) = S_{P_1} \circ S_{P_2}. \]
		
		Or, $S_{P_1} \circ S_{P_2}$ est aussi la rotation d’axe $(P_1) \cap (P_2) = \Delta$ et d’angle $2 \times \frac{\pi}{2} = \pi$. 
		
		C’est donc le \textbf{demi-tour d’axe} $\Delta$.
		Finalement, $R_1 \circ R_2 = R_2 \circ R_1$.
\end{enumerate}
\item Réciproquement, on suppose que $R_1 \circ R_2 = R_2 \circ R_1$.
		
		Soit $A$ un point de $(\Delta_1)$ qui n’appartient pas à $(\Delta_2)$ et $B$ l’image de $A$ par $R_2$.
		\begin{enumerate}
			\item Montrer que la droite $(AB)$ et la droite $(\Delta_2)$ sont perpendiculaires.
			
			Soit $(Q)$ le plan passant par $A$ et orthogonal à $(\Delta_2)$ et soit $I$ le point d’intersection de $(Q)$ et $(\Delta_2)$. Dire que le 				point $B$ est l’image du point $A$ par $R_2$ signifie que $B$ est l’image du point $A$ par la restriction de $R_2$ à $(Q)$ qui est la symétrie centrale de centre $I$. On a $(AB) \subset (Q)$, donc $(\Delta_2)$ est orthogonal à $(AB)$. De plus, $A$, $I$ et $B$ sont alignés. Donc $I \in (\Delta_2) \cap (AB)$. Ainsi, la droite $(AB)$ et la droite $\Delta_2$ sont perpendiculaires en $I$.
			
			\item En utilisant la relation $R_1 \circ R_2 = R_2 \circ R_1$, prouver que $B = R_1(B)$.
			
					\[ R_1(B) = R_1(R_2(A)) = R_2(R_1(A)) = R_2(A) = B. \]
					
			\item En déduire que $(\Delta_1)$ et $(\Delta_2)$ sont perpendiculaires.
			
			On a :
			
			$R_1(B) = B$ ; donc $B \in (\Delta_1)$.
			
			$A \in \Delta_1$.
			
			$A \neq B$ car si $A = B$, on aurait $R_2(A) = A$ ; ce qui signifierait que $A \in (\Delta_2)$.
			
			On en déduit que $(AB) = (\Delta_1)$. D’après 2.a), la droite $(AB)$ et la droite $(\Delta_2)$ sont perpendiculaires. On en déduit que $(\Delta_1)$ et $(\Delta_2)$ sont perpendiculaires.
		\end{enumerate}
\item En utilisant ce qui précède, énoncer une condition nécessaire et suffisante portant sur $(\Delta_1)$ et $(\Delta_2)$ pour que $R_1 \circ R_2 = R_2 \circ R_1$.

Soient $(\Delta_1)$ et $(\Delta_2)$ deux droites distinctes de l’espace, $R_1$ et $R_2$ les demi-tours d’axes respectifs $(\Delta_1)$ et $(\Delta_2)$.

\[ R_1 \circ R_2 = R_2 \circ R_1 \iff (\Delta_1) \perp (\Delta_2). \]
\end{enumerate}

\section*{\underline{Problème :(11 pts)}}

On considère le plan complexe $\mathbb{P}$ rapporté à un repère orthonormé direct $(O ; \vec{u}, \vec{v})$.

	\section*{\underline{PARTIE A (2 points)}}

\begin{enumerate}
\item Soit $(a, b) \in \mathbb{R}^* \times \mathbb{R}$ et $F_{a,b}$ l’application de $\mathbb{P}$ dans $\mathbb{P}$ qui au point $M$ d’affixe $z$ fait correspondre le point $M'$ d’affixe $z'$ telle que : $z' = a \overline{z} + ib$ où $\overline{z}$ est le conjugué de $z$.
\begin{enumerate}
\item Exprimons les coordonnées $x'$ et $y'$ de $M'$ en fonction des coordonnées $x$ et $y$ de $M$.
\[
\begin{cases}
x' = a x \\
y' = b - a y
\end{cases}
\]
\item Déterminer, suivant les valeurs de $a$ et $b$, l’ensemble des points invariants par $F_{a,b}$.
\[ F_{a,b}(M) = M \Leftrightarrow 
\begin{cases} 
x = a x \\ 
y = b - a y 
\end{cases}
\Leftrightarrow 
\begin{cases} 
x(a - 1) = 0 \\ 
y(1 + a) = b 
\end{cases} \]

     Si $a = 1$ alors $y = \frac{1}{2} b$ et $x$ peut prendre n’importe quelle valeur réelle. L’ensemble des points invariants est la droite d’équation $y = \frac{1}{2} b$.
     
     Si $a \neq 1$ alors $x = 0$.
    
         Si $a = -1$ et $b \neq 0$, l’ensemble des points invariants est vide.
         
         Si $a = -1$ et $b = 0$, $y$ peut prendre n’importe quelle valeur réelle. L’ensemble des points invariants est la droite $(D)$ d’équation $x = 0$.
         
         Si $a \neq -1$, alors $y = \frac{b}{1 + a}$. L’ensemble des points invariants est $\Omega \left( \frac{ib}{1 + a} \right)$.
\end{enumerate}
\item On suppose $|a| \neq 1$. Montrer que $F_{a,b} = S_\Delta \circ h$, où $S_\Delta$ est la symétrie orthogonale d’axe la droite $(\Delta)$ d’équation $y = \frac{b}{a + 1}$ et $h$ l’homothétie de centre le point $\Omega$ d’affixe $\frac{ib}{1 + a}$ et de rapport $a$.

Déterminons les expressions analytiques de $S_\Delta$ et de $h$.

Soit $M(x, y)$ et $M'(x', y')$.

Expression analytique de $S_\Delta$ :
\[ S_\Delta(M) = M' \Leftrightarrow
\begin{cases} 
MM' \text{ est orthogonal à } \vec{u} \\ 
\text{Le milieu } I \text{ de } [MM'] \text{ appartient à } \Delta 
\end{cases} \]
\[
\begin{cases}
x' = x \\ 
y' = -y + \frac{2b}{1 + a}
\end{cases} \]
Expression analytique de $h$:
\[ h(M) = M' \Leftrightarrow \overline{\Omega M'} = a \overline{\Omega M} \]
\[
\begin{cases} 
x' = a x \\ 
y' = a y + \frac{b}{1 + a}(1 - a)
\end{cases} \]
La composée $S_\Delta \circ h$ a pour expression analytique :
\[
\begin{cases} 
x' = a x \\ 
y' = -\left(a y + \frac{b}{1 + a}(1 - a)\right) + \frac{2b}{1 + a}
\end{cases} \]
\[ S_\Delta \circ h : 
\begin{cases} 
x' = a x \\ 
y' = -a y + b
\end{cases} \]
Finalement, $S_\Delta \circ h$ a pour écriture complexe $z' = x' + i y' = a x - a i y + i b = a(x - i y) + i b = a \overline{z} + i b$. Donc $S_\Delta \circ h = F_{a,b}$.
$F_{a,b}$ est la composée de la symétrie orthogonale par rapport à la droite $\Delta$ d’équation $y = \frac{b}{a + 1}$ et de l’homothétie $h$ de centre $\Omega$ d’affixe $Z_\Omega = \frac{i b}{1 + a}$ et de rapport $a$.
\item Soit $(c, d) \in \mathbb{R}^* \times \mathbb{R}$ et $G_{c,d}$ l’application de $\mathbb{P}$ dans $\mathbb{P}$ qui, au point $N$ d’affixe $z$, fait correspondre le point $N'$ d’affixe $z'$ tel que : $z' = c z + i d$.

Déterminer, suivant les valeurs de $c$ et $d$, la nature et les éléments géométriques caractéristiques de $G_{c,d}$.

     Si $c = 1$, $G_{c,d}$ est la translation de vecteur $\vec{v}$ d’affixe $i d$.
     
     Si $c \neq 1$, $G_{c,d}$ est l’homothétie de centre $\Pi$ d’affixe $\frac{i d}{1 - c}$ et de rapport $c$.

\end{enumerate}
		\section*{\underline{PARTIE B (2 points)}}
		\begin{enumerate}
		\item Dans cette , on suppose que $|a| \neq 1$.
		
			On définit la suite de points $(M_n)_{n \geq 1}$ par :
			
			\[
				\begin{cases} 
					M_1 \text{ est le point d’affixe } u_1 = a + i b \\
					\forall n \geq 1, M_{n+1} = F_{a,b}(M_n)
			\end{cases} \]
			\begin{enumerate}
			\item Déterminons l’affixe $u_2$ du point $M_2$.
			
				\[ M_2 = F_{a,b}(M_1) \Leftrightarrow u_2 = a \overline{u_1} + i b \]
				\[ \Leftrightarrow u_2 = a^2 + i b(1 - a) \]
				\item Montrer que pour tout entier naturel $n$ non nul, $M_n$ a pour affixe :
				\[ u_n = a^n + i b \left( \frac{1 - (-a)^n}{1 + a} \right). \]
				On va faire une démonstration par récurrence.
				
				La formule est vraie pour $n = 1$. En effet, $u_1 = a + i b = a^1 + i b \left( \frac{1 - (-a)^1}{1 + a} \right)$.
				
     Supposons que la formule est vraie à l’ordre $n$, c’est-à-dire que $M_n$ a pour affixe :
     
      $u_n = a^n + i b \left( \frac{1 - (-a)^n}{1 + a} \right)$ et montrons que $M_{n+1}$ a pour affixe : $u_{n+1} = a^{n+1} + i b \left( \frac{1 - (-a)^{n+1}}{1 + a} \right)$.
    \[ M_{n+1} = F_{a,b}(M_n) \Leftrightarrow u_{n+1} = a \overline{u_n} + i b \]
    \[ \Leftrightarrow u_{n+1} = a \left( a^n - i b \left( \frac{1 - (-a)^n}{1 + a} \right) \right) + i b \]
    \[ \Leftrightarrow u_{n+1} = a^{n+1} + i b \left( \frac{1 - (-a)^{n+1}}{1 + a} \right) \]

Conclusion : $M_n$ a pour affixe $u_n = a^n + i b \left( \frac{1 - (-a)^n}{1 + a} \right)$.
			\end{enumerate}
		\end{enumerate}
		
\end{document}
