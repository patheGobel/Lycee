\documentclass[12pt,a4paper]{article}
\usepackage{amsmath,amssymb,mathrsfs,tikz,times,pifont}
\usepackage{enumitem}
\newcommand\circitem[1]{%
\tikz[baseline=(char.base)]{
\node[circle,draw=gray, fill=red!55,
minimum size=1.2em,inner sep=0] (char) {#1};}}
\newcommand\boxitem[1]{%
\tikz[baseline=(char.base)]{
\node[fill=cyan,
minimum size=1.2em,inner sep=0] (char) {#1};}}
\setlist[enumerate,1]{label=\protect\circitem{\arabic*}}
\setlist[enumerate,2]{label=\protect\boxitem{\alph*}}
%%%::::::by chnini ameur :::::::%%%
\everymath{\displaystyle}
\usepackage[left=1cm,right=1cm,top=1cm,bottom=1.7cm]{geometry}
\usepackage{array,multirow}
\usepackage[most]{tcolorbox}
\usepackage{varwidth}
\tcbuselibrary{skins,hooks}
\usetikzlibrary{patterns}
%%%::::::by chnini ameur :::::::%%%
\newtcolorbox{exa}[2][]{enhanced,breakable,before skip=2mm,after skip=5mm,
colback=yellow!20!white,colframe=black!20!blue,boxrule=0.5mm,
attach boxed title to top left ={xshift=0.6cm,yshift*=1mm-\tcboxedtitleheight},
fonttitle=\bfseries,
title={#2},#1,
% varwidth boxed title*=-3cm,
boxed title style={frame code={
\path[fill=tcbcolback!30!black]
([yshift=-1mm,xshift=-1mm]frame.north west)
arc[start angle=0,end angle=180,radius=1mm]
([yshift=-1mm,xshift=1mm]frame.north east)
arc[start angle=180,end angle=0,radius=1mm];
\path[left color=tcbcolback!60!black,right color = tcbcolback!60!black,
middle color = tcbcolback!80!black]
([xshift=-2mm]frame.north west) -- ([xshift=2mm]frame.north east)
[rounded corners=1mm]-- ([xshift=1mm,yshift=-1mm]frame.north east)
-- (frame.south east) -- (frame.south west)
-- ([xshift=-1mm,yshift=-1mm]frame.north west)
[sharp corners]-- cycle;
},interior engine=empty,
},interior style={top color=yellow!5}}
%%%%%%%%%%%%%%%%%%%%%%%

\usepackage{fancyhdr}
\usepackage{eso-pic}         % Pour ajouter des éléments en arrière-plan
% Commande pour ajouter du texte en arrière-plan
\AddToShipoutPicture{
    \AtTextCenter{%
        \makebox[0pt]{\rotatebox{80}{\textcolor[gray]{0.7}{\fontsize{5cm}{5cm}\selectfont PGB}}}
    }
}
\usepackage{lastpage}
\fancyhf{}
\pagestyle{fancy}
\renewcommand{\footrulewidth}{1pt}
\renewcommand{\headrulewidth}{0pt}
\renewcommand{\footruleskip}{10pt}
\fancyfoot[R]{
\color{blue}\ding{45}\ \textbf{2025}
}
\fancyfoot[L]{
\color{blue}\ding{45}\ \textbf{Prof:M. BA}
}
\cfoot{\bf
\thepage /
\pageref{LastPage}}
\begin{document}
\renewcommand{\arraystretch}{1.5}
\renewcommand{\arrayrulewidth}{1.2pt}
\begin{tikzpicture}[overlay,remember picture]
\node[draw=blue,line width=1.2pt,fill=purple,text=blue,inner sep=3mm,rounded corners,pattern=dots]at ([yshift=-2.5cm]current page.north) {\begingroup\setlength{\fboxsep}{0pt}\colorbox{white}{\begin{tabular}{|*1{>{\centering \arraybackslash}p{0.28\textwidth}} |*2{>{\centering \arraybackslash}p{0.2\textwidth}|} *1{>{\centering \arraybackslash}p{0.19\textwidth}|} }
\hline
\multicolumn{3}{|c|}{$\diamond$$\diamond$$\diamond$\ \textbf{Lycée de Dindéfélo}\ $\diamond$$\diamond$$\diamond$ }& \textbf{A.S. : 2024/2025} \\ \hline
\textbf{Matière: Mathématiques}& \textbf{Niveau : 2nd}\textbf{L} &\textbf{Date: 10/05/2025} & \textbf{Durée : 3 heures} \\ \hline
\multicolumn{4}{|c|}{\parbox[c]{10cm}{\begin{center}
\textbf{{\Large\sffamily Devoir n$ ^{\circ} $ 2 Du 2$ ^\text{\bf nd} $ Semestre}}
\end{center}}} \\ \hline
\end{tabular}}\endgroup};
\end{tikzpicture}
\vspace{3cm}

\section*{\underline{Exercice 1 :} 6 points ( Application Affine et Factorisation  )}
\begin{enumerate}
\item  Soit la fonction affine \( f(x) = 2x + 1 \)

\begin{enumerate}
    \item Calculer \( f(1) \) puis conclure 
    \item Résoudre \( f(x) = 5 \) puis conclure 
\end{enumerate}
\item Déterminer le taux d’accroissement de la fonction affine telle que : \( f(3)=3 \) et \( f(2)=4\)
\item Factoriser au mieux :

\( \textbf{A}(x) = x^3 - 8 + 2(x - 2)^2  \)
\item Factoriser, si possible, les trinômes suivants :
\[
\textbf{B}(x) = -2x^2 + 4x + 6\quad\quad \textbf{C}(x) = x^2 - 8x + 17 \quad\quad \textbf{D}(x) = -9x^2 + 6x - 1 
\]
\end{enumerate}


\section*{\underline{Exercice 2 :} 4 points ( Résolution de systèmes par la méthode de Cramer )}
Résolvons chacun des systèmes suivants en utilisant la méthode de Cramer :

\(\textbf{a)}\quad
\begin{cases}
3x - y + 2 = 0 \\
3x - y - 1 = 0
\end{cases}\implies
\begin{cases}
3x - y = -2 \\
3x - y = 1
\end{cases}
\)


\(
\text{Calcul du déterminant de } : \\
\Delta = \left| \begin{matrix} 3 & -1 \\ 3 & -1 \end{matrix} \right| = (3)(-1) - (3)(-1) = -3 + 3 = 0.
\)

\(
\Delta_{x} = \left| \begin{matrix} -2 & -1 \\ 1 & -1 \end{matrix} \right| = (-2)(-1) - (1)(-1) = 2 + 1 = 3.
\)

\(
\Delta_{y} = \left| \begin{matrix} 3 & -2 \\ 3 & 1 \end{matrix} \right| = (3)(1) - (3)(-2) = 3 + 6 = 9.
\)

\(
\Delta=0, \Delta_{x}\neq 0 \textbf{ et } \Delta_{y}\neq 0
\)

\(
\textcolor{red}{\boxed{\text{Le système n'admet pas de solution.}}}\\
\)

\(\textbf{b)} \quad
\begin{aligned}
\begin{cases}
2y + x = 5 \\
-y + 7 = 4
\end{cases}&\implies
\begin{cases}
x+2y = 5 \\
0-y  = -3
\end{cases}
\end{aligned}
\)


\(
\text{Calcul du déterminant de } : \\
\Delta = \left| \begin{matrix} 1 & 2 \\ 0 & -1 \end{matrix} \right| = (1)(-1) - (0)(2) = -1.
\)

\(
\text{Le déterminant est non nul, nous pouvons utiliser la règle de Cramer.}
\)

\(
\text{Calcul de } x : \quad
x = \frac{\Delta_{x}}{\Delta} = \frac{\left| \begin{matrix} 5 & 2 \\ 3 & -1 \end{matrix} \right|}{-1}
= \frac{(5)(-1) - (3)(2)}{-1} = \frac{-5 - 6}{-1} = \frac{-11}{-1} = 11.
\)

\(
\text{Calcul de } y : \quad
y = \frac{\Delta_{y}}{\Delta} = \frac{\left| \begin{matrix} 1 & 5 \\ 0 & 3 \end{matrix} \right|}{-1}
= \frac{(1)(3) - (0)(5)}{-1} = \frac{3}{-1} = -3.
\)

\(
\text{Solution du système :} \quad
\textcolor{red}{\boxed{S=\left\lbrace  \left( 11, \, -3 \right)  \right\rbrace }}.
\)

\(\textbf{c)} \quad
\begin{cases}
3x - y = 2 \\
2x - y = 1
\end{cases}
\)


\(
\text{Calcul du déterminant de } : \\
\Delta = \left| \begin{matrix} 3 & -1 \\ 2 & -1 \end{matrix} \right| = (3)(-1) - (2)(-1) = -3 + 2 = -1.
\)

\(
\text{Le déterminant est non nul, nous pouvons utiliser la règle de Cramer.}
\)

\(
\text{Calcul de } x : \quad
x = \frac{\Delta_{x}}{\Delta} = \frac{\left| \begin{matrix} 2 & -1 \\ 1 & -1 \end{matrix} \right|}{-1}
= \frac{(2)(-1) - (1)(-1)}{-1} = \frac{-2 + 1}{-1} = \frac{-1}{-1} = 1.
\)

\(
\text{Calcul de } y : \quad
y = \frac{\Delta_{y}}{\Delta} = \frac{\left| \begin{matrix} 3 & 2 \\ 2 & 1 \end{matrix} \right|}{-1}
= \frac{(3)(1) - (2)(2)}{-1} = \frac{3 - 4}{-1} = \frac{-1}{-1} = 1.
\)

\(
\text{Solution du système :} \quad
\textcolor{red}{\boxed{S=\left\lbrace  \left( 1, \, 1 \right)  \right\rbrace }}.
\)

\(\textbf{d)} \quad
\begin{aligned}
\begin{cases}
x + y = 3 \\
-y + 4 = x - 2
\end{cases}&\implies
\begin{cases}
x + y = 3 \\
-x-y  = - 6
\end{cases}\implies
\begin{cases}
x + y = 3 \\
x+y  =  6
\end{cases}
\end{aligned}
\)\\

\(
\text{Calcul du déterminant de }: \\
\Delta = \left| \begin{matrix} 1 & 1 \\ -1 & -1 \end{matrix} \right| = (1)(1) - (1)(1) = -1 + 1 = 0.
\)

\(
\Delta_{x} = \left| \begin{matrix} 3 & 1 \\ 6 & 1 \end{matrix} \right| = (3)(1) - (6)(1) = -3 + 6 = 3.
\)

\(
\Delta_{y} = \left| \begin{matrix} 1 & 3 \\ 1 & 6 \end{matrix} \right| = (1)(6) - (1)(3) = -6 + 3 = -3.
\)

\(
\Delta=0, \Delta_{x}\neq 0 \textbf{ et } \Delta_{y}\neq 0
\)

\(
\textcolor{red}{\boxed{\text{Le système n'admet pas de solution.}}}
\)
\section*{\underline{Exercice 3 :} 6 points ( Équations et inéquations du second degré )}

Résoudre dans \(\mathbb{R}\) les inéquations suivantes :

\[
\textbf{a)} \, -x^2 - x + 2 = 0 \quad\quad \quad\quad\textbf{b)} \, \, -9x^2 + 12x - 3 = 0
\]

\[
\textbf{c)} \, -3x^2 + 4x - 2 \leq 0 \quad\quad\quad\quad \textbf{d)} \, 8x^2 + 34x + 21 < 0
\]

\[
\textbf{e)} \, 2x^2 - x - 3 > 0 \quad\quad\quad\quad \textbf{f)} \, 4 - 9x^2 \geq 0
\]
\section*{\underline{Exercice 4 :} 4 points ( Union et Intersection d'Intervalles )}  

\begin{enumerate}  
\item On considère \( I = [-1, 5] \) et \( J = [0, 7] \). Déterminer \( I \cup J \) et \( I \cap J \).  

\item On considère \( K = [-7, -3] \) et \( L = [1, 3] \). Déterminer \( K \cup L \) et \( K \cap L \).  
\end{enumerate}

\end{document}
