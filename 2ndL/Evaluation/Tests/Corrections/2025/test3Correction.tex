\documentclass[a4paper,12pt]{article}
\usepackage{graphicx}
\usepackage[a4paper, top=0cm, bottom=2cm, left=2cm, right=2cm]{geometry} % Ajuste les marges
\usepackage{xcolor} % Pour ajouter des couleurs
\usepackage{hyperref} % Pour avoir des r\'ef\'erences color\'ees si n\'ecessaire
\usepackage{eso-pic} % Pour ajouter des \'el\'ements en arri\`ere-plan
\usepackage[french]{babel}
\usepackage[T1]{fontenc}
\usepackage{mathrsfs}
\usepackage{amsmath}
\usepackage{amsfonts}
\usepackage{amssymb}
\usepackage{tkz-tab}
\usepackage{tcolorbox} % Pour encadrer avec fond color\'e
\usepackage{tikz}
\usetikzlibrary{arrows, shapes.geometric, fit}

% D\'efinition de l'encadr\'e adaptatif avec fond jaune
\newtcolorbox{resultbox}{
    colback=red!30, % Fond rouge clair
    colframe=black, % Bordure noire fine
    sharp corners, % Coins nets
    boxrule=0.5pt, % Contour l\'eger
    boxsep=2pt, % Espacement interne
    left=5pt, right=5pt, top=2pt, bottom=2pt, % Marges internes
}
% Commande pour ajouter du texte en arrière-plan
\AddToShipoutPicture{
    \AtTextCenter{%
        \makebox[0pt]{\rotatebox{80}{\textcolor[gray]{0.4}{\fontsize{8cm}{13cm}\selectfont PGB}}}
    }
}
\begin{document}

\hrule % Barre horizontale
% En-t\^ete
\begin{center}
    {\Large \textbf{Correction du Test 3}}
\end{center}
\hrule

Résolvons dans \(\mathbb{R}\): \( x^2 - x - 6 < 0\)

Posons \( x^2 - x - 6 = 0\)

\textbf{Calcul du discriminant} 
 
Le discriminant \(\Delta\) est donné par :

\[
\begin{aligned}
 \Delta &= b^2 - 4ac \\
        &= (-1)^2 - 4 \times 1 \times (-6) \\
        &= 1 + 24 \\
        &= 25
\end{aligned}
\]

Ainsi, le discriminant est : 

\begin{resultbox}
    \[
    \mathbf{\Delta = 25}
    \]
\end{resultbox}

mme \(\Delta > 0\), le trinôme admet deux racines réelles distinctes \( x_1 \) et \( x_2 \) :

Les racines sont donc données par :

\[
\begin{aligned}
        x_1 &= \frac{-b - \sqrt{\Delta}}{2a} \\
            &= \frac{-(-1) - \sqrt{25}}{2 \times 1} \\
            &= \frac{1 - 5}{2} \\
            &= \frac{-4}{2} \\
            &= -2
\end{aligned}
\]

\begin{resultbox}
    \[
    \mathbf{x_1 = -2}
    \]
\end{resultbox}

\[
\begin{aligned}
        x_1 &= \frac{-b - \sqrt{\Delta}}{2a} \\
            &= \frac{-(-1) + \sqrt{25}}{2 \times 1} \\
            &= \frac{1 + 5}{2} \\
            &= \frac{6}{2} \\
            &= 3
\end{aligned}
\]

\begin{resultbox}
    \[
    \mathbf{x_2 = 3}
    \]
\end{resultbox}

                    \begin{center}
                        \begin{tikzpicture}
                            \tkzTabInit[lgt=3,espcl=1.5]
                            {$x$ / 1 , $x^2 - x - 6$ / 1}
                            {$-\infty$, $-2$, $3$, $-\infty$}
                            \tkzTabLine
                            {, +, z ,  -  , z , + ,}
                        \end{tikzpicture}
                    \end{center}
\begin{resultbox}
    \[
    \mathbf{S = ]-2;3[}
    \]
\end{resultbox}
\end{document}
