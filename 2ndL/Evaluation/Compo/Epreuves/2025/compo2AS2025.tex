\documentclass[12pt,a4paper]{article}
\usepackage{amsmath,amssymb,mathrsfs,tikz,times,pifont}
\usepackage{enumitem}
\usepackage{float}
\newcommand\circitem[1]{%
\tikz[baseline=(char.base)]{
\node[circle,draw=gray, fill=red!55,
minimum size=1.2em,inner sep=0] (char) {#1};}}
\newcommand\boxitem[1]{%
\tikz[baseline=(char.base)]{
\node[fill=cyan,
minimum size=1.2em,inner sep=0] (char) {#1};}}
\setlist[enumerate,1]{label=\protect\circitem{\arabic*}}
\setlist[enumerate,2]{label=\protect\boxitem{\alph*}}
%%%::::::by chnini ameur :::::::%%%
\everymath{\displaystyle}
\usepackage[left=1cm,right=1cm,top=1cm,bottom=1.7cm]{geometry}
\usepackage{array,multirow}
\usepackage[most]{tcolorbox}
\usepackage{varwidth}
\tcbuselibrary{skins,hooks}
\usetikzlibrary{patterns}
%%%::::::by chnini ameur :::::::%%%
\newtcolorbox{exa}[2][]{enhanced,breakable,before skip=2mm,after skip=5mm,
colback=yellow!20!white,colframe=black!20!blue,boxrule=0.5mm,
attach boxed title to top left ={xshift=0.6cm,yshift*=1mm-\tcboxedtitleheight},
fonttitle=\bfseries,
title={#2},#1,
% varwidth boxed title*=-3cm,
boxed title style={frame code={
\path[fill=tcbcolback!30!black]
([yshift=-1mm,xshift=-1mm]frame.north west)
arc[start angle=0,end angle=180,radius=1mm]
([yshift=-1mm,xshift=1mm]frame.north east)
arc[start angle=180,end angle=0,radius=1mm];
\path[left color=tcbcolback!60!black,right color = tcbcolback!60!black,
middle color = tcbcolback!80!black]
([xshift=-2mm]frame.north west) -- ([xshift=2mm]frame.north east)
[rounded corners=1mm]-- ([xshift=1mm,yshift=-1mm]frame.north east)
-- (frame.south east) -- (frame.south west)
-- ([xshift=-1mm,yshift=-1mm]frame.north west)
[sharp corners]-- cycle;
},interior engine=empty,
},interior style={top color=yellow!5}}
%%%%%%%%%%%%%%%%%%%%%%%

\usepackage{fancyhdr}
\usepackage{eso-pic}         % Pour ajouter des éléments en arrière-plan
% Commande pour ajouter du texte en arrière-plan
\AddToShipoutPicture{
    \AtTextCenter{%
        \makebox[0pt]{\rotatebox{80}{\textcolor[gray]{0.7}{\fontsize{5cm}{5cm}\selectfont PGB}}}
    }
}
\usepackage{lastpage}
\fancyhf{}
\pagestyle{fancy}
\renewcommand{\footrulewidth}{1pt}
\renewcommand{\headrulewidth}{0pt}
\renewcommand{\footruleskip}{10pt}
\fancyfoot[R]{
\color{blue}\ding{45}\ \textbf{2025}
}
\fancyfoot[L]{
\color{blue}\ding{45}\ \textbf{Prof:M. BA}
}
\cfoot{\bf
\thepage /
\pageref{LastPage}}
\begin{document}
\renewcommand{\arraystretch}{1.5}
\renewcommand{\arrayrulewidth}{1.2pt}
\begin{tikzpicture}[overlay,remember picture]
    \node[draw=blue,line width=1.2pt,fill=purple,text=blue,inner sep=3mm,rounded corners,pattern=dots]at ([yshift=-2.5cm]current page.north) {\begingroup\setlength{\fboxsep}{0pt}\colorbox{white}{\begin{tabular}{|*1{>{\centering \arraybackslash}p{0.28\textwidth}} |*2{>{\centering \arraybackslash}p{0.2\textwidth}|} *1{>{\centering \arraybackslash}p{0.19\textwidth}|} }
                \hline
                \multicolumn{3}{|c|}{$\diamond$$\diamond$$\diamond$\ \textbf{Lycée de Dindéfélo}\ $\diamond$$\diamond$$\diamond$ } & \textbf{A.S. : 2024/2025}                                                                     \\ \hline
                \textbf{Matière: Mathématiques}                                                                                    & \textbf{Niveau : 2}\textbf{$^{nd}$L} & \textbf{Date: 17/06/2025} & \textbf{Durée : 3 heures} \\ \hline
                \multicolumn{4}{|c|}{\parbox[c]{10cm}{\begin{center}
                                                                  \textbf{{\Large\sffamily Composition n$ ^{\circ} $ 2 Du 2$ ^\text{\bf nd} $ Semestre}}
                                                              \end{center}}}                                                                                                                               \\ \hline
            \end{tabular}}\endgroup};
\end{tikzpicture}
\vspace{3cm}

\section*{\underline{Exercice 1 :} 4 pts }

\begin{enumerate}
    \item Résous le système suivant par la méthode de ton choix (Cramer, substitution, addition,\ldots) \hfill \textbf{2 pts}
    \[
    \left\{
    \begin{aligned}
        2x + y &= 3 \\
        3x + 2y &= 6
    \end{aligned}
    \right.
    \]
    
    \item Résous graphiquement le système : \hfill \textbf{2 pts}
    \[
    \left\{
    \begin{aligned}
        3x + y - 4 &\leq 0 \\
        2x + y - 1 &\geq 0
    \end{aligned}
    \right.
    \]
\end{enumerate}

\vspace{0.5cm}

\section*{\underline{Exercice 2 :} 8 pts }
\begin{enumerate}
    \item On considère le trinôme suivant : \( f(x) = -2x^2 + 7x - 5 \).\\
    Montre que le discriminant de \( f(x) \) est \( \Delta = 9 \). Déduis-en sa forme canonique et sa forme factorisée. \hfill \textbf{3 pts}
    
    \item Résous dans \( \mathbb{R} \) l’équation \( x^2 - 3x - 10 = 0 \). \hfill \textbf{1 pt}
    
    \item Résous dans \( \mathbb{R} \) l’inéquation \( x^2 - 3x - 10 \leq 0 \). \hfill \textbf{2 pts}
    
    \item Résous dans \( \mathbb{R}^2 \) le système ci-dessous : \hfill \textbf{2 pts}
    \[
    \left\{
    \begin{aligned}
        x + y &= -5 \\
        x - y &= 6
    \end{aligned}
    \right.
    \]
\end{enumerate}

\section*{\underline{ Exercice 3 :} 8 pts}

\subsection*{\underline{PARTIE A :} 5 pts}

On considère les fonctions suivantes :
\[
f(x) = -\frac{1}{2}x + 3 \quad ; \quad g(x) = 5x + 10 \quad ; \quad h(x) = 3
\]
\begin{enumerate}
    \item Donne le sens de variation des fonctions \( f \), \( g \) et \( h \). \hfill \textbf{1,5 pt}
    
    \item Soit \( k(x) = -3x + 2 \)
    \begin{enumerate}
        \item Calcule l’image de \(-1\) et de \(0\) par la fonction \(k\). \hfill \textbf{1 pt}
        \item Détermine les antécédents de \(7\) et \(\dfrac{1}{2}\) par la fonction \(k\). \hfill \textbf{1 pt}
        \item Trace la représentation graphique de \(k\). \hfill \textbf{1,5 pt}
    \end{enumerate}
\end{enumerate}

\vspace{0.5cm}

\subsection*{\underline{PARTIE B :} 3 pts}

\begin{enumerate}
    \item Dans le plan, muni d’un repère orthonormé \((O; \vec{i}; \vec{j})\), on donne les droites \\
    \((D) : y = -7x + 8\) et \((D') : y = -7x + 2\).\\
    Les droites \((D)\) et \((D')\) sont-elles parallèles ? Justifie ta réponse. \hfill \textbf{1,5 pt}
    
    \item Soit la droite \((D_1)\) : \(y = mx + 2\) avec \(m\) un réel. Détermine la valeur de \(m\) pour que la droite \((D_1)\) soit parallèle à la droite \((D_2)\) : \(4x - 2y + 3 = 0\). \hfill \textbf{1,5 pt}
\end{enumerate}

\end{document}
