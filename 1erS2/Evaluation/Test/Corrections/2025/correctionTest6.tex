\documentclass[a4paper,12pt]{article}
\usepackage{graphicx}
\usepackage[a4paper, top=0cm, bottom=2cm, left=2cm, right=2cm]{geometry} % Ajuste les marges
\usepackage{xcolor} % Pour ajouter des couleurs
\usepackage{hyperref} % Pour avoir des références colorées si nécessaire
\usepackage{eso-pic} % Pour ajouter des éléments en arrière-plan

\usepackage[french]{babel}
\usepackage[T1]{fontenc}
\usepackage{mathrsfs}
\usepackage{amsmath}
\usepackage{amsfonts}
\usepackage{amssymb}
\usepackage{tkz-tab}
\usepackage{tcolorbox} % Pour encadrer avec fond coloré

\usepackage{tikz}
\usetikzlibrary{arrows, shapes.geometric, fit}
\newcounter{correction} % Compteur pour les questions

% Définir la commande pour afficher une question numérotée
\newcommand{\question}{%
  \refstepcounter{correction}%
  \textbf{\textcolor{black}{Question \thecorrection : 2,5 pts}} \ignorespaces
}

% Commande pour ajouter du texte en arrière-plan
\AddToShipoutPicture{
    \AtTextCenter{%
        \makebox[0pt]{\rotatebox{80}{\textcolor[gray]{0.6}{\fontsize{10cm}{10cm}\selectfont PGB}}}
    }
}

\begin{document}
\hrule % Barre horizontale
% En-tête
\begin{center}
    \begin{tabular}{@{} p{5cm} p{5cm} p{5cm} @{}} % 3 colonnes avec largeurs fixées
        Lycée Dindéfélo & \quad\quad\quad Test n°6 & 15 mars 2025 \\
    \end{tabular}
    \\[-0.01cm] % Ajuster l'espace vertical entre le tableau et la barre
    \hrule % Barre horizontale
\end{center}

\begin{center}
    \textbf{\Large Produit scalaire et fonctions} \\[0.2cm]
    \textbf{\large Professeur : M. BA} \\[0.2cm]
    \textbf{Classe : $1^\text{ère}$ S2} \\[0.2cm]
    \textbf{\small Durée : 10 minutes} \\[0.2cm]
    \textbf{\small Note :\quad\quad /5}
\end{center}

% Nom de l'élève
\textbf{\small Nom de l’élève :} \underline{\hspace{8cm}} \\[0.5cm]

\question
\begin{center}
    \begin{tikzpicture}
        % Définition des points
        \coordinate (B) at (0,0);
        \coordinate (C) at (6,0);
        \coordinate (A) at (2,3);

        % Dessin du triangle
        \draw (B) -- (A) -- (C) -- cycle;

        % Étiquetage des sommets
        \node[above left] at (A) {\textbf{A}};
        \node[below left] at (B) {\textbf{B}};
        \node[below right] at (C) {\textbf{C}};

        % Étiquetage des côtés
        \node[below] at (3,0) {\textit{a}};
        \node[above right] at (3,2.1) {\textit{b}};
        \node[left] at (1,1.5) {\textit{c}};
    \end{tikzpicture}
\end{center}

\begin{enumerate}
    \item Donnons une formule d’Al-Kashi appliquée à ce triangle.\textbf{ 0,5 pt}\\[0.2cm]
        \begin{tcolorbox}[colback=green!20, colframe=green!80, sharp corners]
        \[
        \mathbf{a^2 = b^2 + c^2 - 2bc\cos(A)}
        \]
    \end{tcolorbox}
    \item Donnons la loi des sinus.\textbf{ 0,5 pt}\\[0.2cm]
        \begin{tcolorbox}[colback=green!20, colframe=green!80, sharp corners]
        \[
        \mathbf{\frac{a}{\sin(A)} = \frac{b}{\sin(B)} = \frac{c}{\sin(C)}}
        \]
    \end{tcolorbox}
    \item Donnons l'équation du cercle de centre I $\begin{pmatrix} a \\ b \end{pmatrix}$ et de rayon r .\textbf{ 0,5 pt}\\[0.2cm]
        \begin{tcolorbox}[colback=green!20, colframe=green!80, sharp corners]
        \[
        \mathbf{(x - a)^2 + (y - b)^2 = r^2}
        \]
    \end{tcolorbox}
    \item Soient \( A \) et \( B \) deux points du plan et \( I \) milieu du segment \([AB]\). Pour tout point \( M \)  du plan, donner la formule du Théorème de la médiane.\textbf{ 0,75 pt}\\[0.2cm]
    $$MA^{2}+MB^{2} = 2MI^{2}+\frac{AB^{2}}{2}$$
\end{enumerate}

\question\\
Dans chaque cas, donner le domaine de définition\textbf{ 1 pt $+$ 1,75 pt}.

$f(x) = \sqrt{x^{2}+16}$\\[0.2cm]

$f \quad \exists $ ssi  $ x^{2}+16 \geq 0 $

$\begin{aligned}
    x^{2}+16 \geq 0 &\iff x^{2} \geq -16\textbf{ Tjv}\\
                &\iff x \in \mathbb{R}
\end{aligned}$

$Df=\mathbb{R}$

    \begin{tcolorbox}[colback=green!20, colframe=green!80, sharp corners]
        \[
        \mathbf{Df=\mathbb{R}}
        \]
    \end{tcolorbox}
    
$g(x) = \frac{\sqrt{x}}{\sqrt{x^{2}+9}}$\\[0.2cm]
$f \quad \exists $ ssi  $ x^{2}+9 > 0 $ et $x \geq 0$

$\begin{aligned}
    x^{2}+9 > 0 &\iff x^{2} > -9 \text{ et } x\geq 0\\
                &\iff \textbf{ Tjv} \text{ et } x\in [0; +\infty[ \\
                &\iff x \in \mathbb{R} \text{ et } x\in [0; +\infty[ \\
                &\iff x \in \mathbb{R} \cap [0; +\infty[ \\
\end{aligned}$

$ Df=[0; +\infty[ $

    \begin{tcolorbox}[colback=green!20, colframe=green!80, sharp corners]
        \[
        \mathbf{Df=[0; +\infty[}
        \]
    \end{tcolorbox}
\end{document}