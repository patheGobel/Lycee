\documentclass[12pt,a4paper]{article}
\usepackage{amsmath,amssymb,mathrsfs,tikz,times,pifont}
\usepackage{enumitem}
\usepackage{multicol}
\usepackage{lmodern}
\newcommand\circitem[1]{%
\tikz[baseline=(char.base)]{
\node[circle,draw=gray, fill=red!55,
minimum size=1.2em,inner sep=0] (char) {#1};}}
\newcommand\boxitem[1]{%
\tikz[baseline=(char.base)]{
\node[fill=cyan,
minimum size=1.2em,inner sep=0] (char) {#1};}}
\setlist[enumerate,1]{label=\protect\circitem{\arabic*}}
\setlist[enumerate,2]{label=\protect\boxitem{\alph*}}
%%%::::::by chnini ameur :::::::%%%
\everymath{\displaystyle}
\usepackage[left=1cm,right=1cm,top=1cm,bottom=1.7cm]{geometry}
\usepackage[colorlinks=true, linkcolor=blue, urlcolor=blue, citecolor=blue]{hyperref}
\usepackage{array,multirow}
\usepackage[most]{tcolorbox}
\usepackage{varwidth}
\usepackage{float} %pour utiliser l'option [H] qui force l'image à apparaître exactement à l'endroit où elle est placée dans le code.
\tcbuselibrary{skins,hooks}
\usetikzlibrary{patterns}
%%%::::::by chnini ameur :::::::%%%
\newtcolorbox{exa}[2][]{enhanced,breakable,before skip=2mm,after skip=5mm,
colback=yellow!20!white,colframe=black!20!blue,boxrule=0.5mm,
attach boxed title to top left ={xshift=0.6cm,yshift*=1mm-\tcboxedtitleheight},
fonttitle=\bfseries,
title={#2},#1,
% varwidth boxed title*=-3cm,
boxed title style={frame code={
\path[fill=tcbcolback!30!black]
([yshift=-1mm,xshift=-1mm]frame.north west)
arc[start angle=0,end angle=180,radius=1mm]
([yshift=-1mm,xshift=1mm]frame.north east)
arc[start angle=180,end angle=0,radius=1mm];
\path[left color=tcbcolback!60!black,right color = tcbcolback!60!black,
middle color = tcbcolback!80!black]
([xshift=-2mm]frame.north west) -- ([xshift=2mm]frame.north east)
[rounded corners=1mm]-- ([xshift=1mm,yshift=-1mm]frame.north east)
-- (frame.south east) -- (frame.south west)
-- ([xshift=-1mm,yshift=-1mm]frame.north west)
[sharp corners]-- cycle;
},interior engine=empty,
},interior style={top color=yellow!5}}
%%%%%%%%%%%%%%%%%%%%%%%

\usepackage{fancyhdr}
\usepackage{eso-pic}         % Pour ajouter des éléments en arrière-plan
% Commande pour ajouter du texte en arrière-plan
\usepackage{tkz-tab}
\AddToShipoutPicture{
    \AtTextCenter{%
        \makebox[0pt]{\rotatebox{80}{\textcolor[gray]{0.7}{\fontsize{5cm}{5cm}\selectfont PGB}}}
    }
}
\usepackage{lastpage}
\fancyhf{}
\pagestyle{fancy}
\renewcommand{\footrulewidth}{1pt}
\renewcommand{\headrulewidth}{0pt}
\renewcommand{\footruleskip}{10pt}
\fancyfoot[R]{
\color{blue}\ding{45}\ \textbf{2025}
}
\fancyfoot[L]{
\color{blue}\ding{45}\ \textbf{Prof:M. BA}
}
\cfoot{\bf
\thepage /
\pageref{LastPage}}
\begin{document}
\renewcommand{\arraystretch}{1.5}
\renewcommand{\arrayrulewidth}{1.2pt}
\begin{tikzpicture}[overlay,remember picture]
    \node[draw=blue,line width=1.2pt,fill=purple,text=blue,inner sep=3mm,rounded corners,pattern=dots]at ([yshift=-2.5cm]current page.north) {\begingroup\setlength{\fboxsep}{0pt}\colorbox{white}{\begin{tabular}{|*1{>{\centering \arraybackslash}p{0.28\textwidth}} |*2{>{\centering \arraybackslash}p{0.2\textwidth}|} *1{>{\centering \arraybackslash}p{0.19\textwidth}|} }
                \hline
                \multicolumn{3}{|c|}{$\diamond$$\diamond$$\diamond$\ \textbf{Lycée de Dindéfélo}\ $\diamond$$\diamond$$\diamond$ } & \textbf{A.S. : 2024/2025}                                              \\ \hline
                \textbf{Matière: Mathématiques}                                                                                    & \textbf{Niveau : 1}\textbf{S2} & \textbf{Date: 29/04/2025} & \textbf{} \\ \hline
                \multicolumn{4}{|c|}{\parbox[c]{10cm}{\begin{center}
                                                                  \textbf{{\Large\sffamily Td Dérivée et Application}}
                                                              \end{center}}}                                                                                                        \\ \hline
            \end{tabular}}\endgroup};
\end{tikzpicture}
\vspace{3cm}

\begin{multicols}{2}
\setlength{\columnseprule}{0.1mm} % La largeur de la ligne verticale entre les colonnes
\textbf{\underline{Exercice 1}}

Étudier la dérivabilité de \( f \) en \( x_0 \)
\begin{enumerate}[align=left]
    \item \( \displaystyle f(x) = 3x^2 - 4x + 1, \, x_0 = -2 \)
    \item \( \displaystyle f(x) = (1 - x)\sqrt{x - 1}, \, x_0 = 1 \)
    \item \( \displaystyle f(x) = x^2 + |x| + 2, \, x_0 = -2 \)
    \item \( \displaystyle f(x) = \frac{3x - 1}{x + 1}, \, x_0 = 0 \)
    \item \( \displaystyle  f(x) = 
    \begin{cases}
        \frac{x + 2}{x + 1} & \text{si } x < -1 \\
        x^2 - 1 & \text{si } x \geq -1
    \end{cases}
    \quad x_0 = -1 \)
    
    \item \( \displaystyle f(x) = 
    \begin{cases}
        x\sqrt{3 - x} & \text{si } x \leq 3 \\
        (x - 3)\sqrt{x - 3} & \text{si } x > 3
    \end{cases}
    \quad x_0 = 3 \)
\end{enumerate}
\textbf{\underline{Exercice 2}} Soit la fonction \( f \) définie par :\\
\(f(x) = 
\begin{cases}
    \sqrt{-x - 1} & \text{si } x \in ]-\infty, -1] \\
    x^2 - 1 & \text{si } x \in ]-1, 1] \\
    2\sin(x - 1) & \text{si } x \in ]1, +\infty[
\end{cases}\)

\begin{enumerate}
    \item Étudier la continuité de \( f \) en \( -1 \) et \( 1 \).
    \item Étudier la dérivabilité de \( f \) en \( -1 \) et \( 1 \).
    \item Déterminer les intervalles sur lesquels \( f \) est dérivable et calculer la fonction dérivée de \( f \).
\end{enumerate}
\textbf{\underline{Exercice 3}} Soit la fonction \( f \) définie par :

\( f(x) = 
\begin{cases}
    \frac{\sqrt{x^2(x - 1)^2}}{(x - 1)(|x| + 1)} & \text{si } x \neq 1 \\
    -\frac{1}{2} & \text{si } x = 1
\end{cases} \)

Étudier la continuité et la dérivabilité de \( f \) en \( x = 0 \) et \( x = 1 \).\\
\textbf{\underline{Exercice 4} Fonction dérivée}\\
Calculer la dérivée de f dans chacun des cas suivants
\begin{enumerate}
    \item \( f(x) = 3x^2 - 3x + 5 \)
    \item \( f(x) = -5x^3 + 2x + 1 \)
    \item \( f(x) = (2x^2 + 3x - 5)(-6x + 7) \)
    \item \( f(x) = x + \frac{1}{x} \)
    \item \( f(x) = \left( -3x^2 + 2x \right)^4 \)
    \item \( f(x) = \frac{2x + 3}{x - 1} \)
    \item \( f(x) = -\frac{3}{x^2 - 4} \)
    \item \( f(x) = \frac{x^2 + x - 2}{2x^2 + 3x - 5} \)
    \item \( f(x) = \frac{1}{(4 - 5x)^3} \)
    \item \( f(x) = \sqrt{3x - 4} \)
    \item \( f(x) = \sqrt{3x^2 + x + 1} \)
    \item \( f(x) = \frac{x + 1}{x - 1} \)
    \item \( f(x) = \left( \frac{3x - 1}{2x + 4} \right)^4 \)
    \item \( f(x) = (2x - 1)^2(4 - 3x)^3 \)
    \item \( f(x) = \cos(3x + 5) \)
    \item \( f(x) = \sqrt{\cos x} \)
    \item \( f(x) = \frac{1 + \cos x}{\cos x - 3} \)
    \item \( f(x) = \tan^2 x + \cos^2 x \)
    \item \( f(x) = (x^2-x)\sqrt{-x^2 + 9} \)
    \item \( f(x) = \sqrt{\frac{3x + 7}{x^2 + 2x + 1}} \)
\end{enumerate}
\textbf{\underline{Exercice 5}}
\begin{enumerate}
    \item Soit la fonction \( f(x) = \frac{3x^2 + ax + b}{x^2 + 1} \). Déterminer les réels \( a \) et \( b \) tels que la courbe \( (C_f) \) passe par \( A(0;3) \) et admet en ce point une tangente d'équation : \( y = 4x + 3 \).
    
    \item Soit la fonction \( g(x) = ax + b + \frac{1}{x} \). Déterminer les réels \( a \) et \( b \) pour que la courbe \( (C_g) \) passe par le point \( A\left(\frac{\sqrt{2}}{2}, 2\sqrt{2} - 1 \right) \) et admet en ce point une tangente parallèle à l'axe des abscisses.
    
    \item Déterminer les réels \( a \), \( b \), et \( c \) pour que l'hyperbole \( (H) \) d'équation \( y = \frac{ax + b}{x + c} \) ait le point \( \Omega(-1; 2) \) comme centre de symétrie et admette au point d'abscisse 1, une tangente parallèle à la droite \( (D) \) d'équation \( y = -x \).
\end{enumerate}
\textbf{\underline{Exercice 6} TVI}

Soit la fonction \( f \) définie sur \( \mathbb{R} \) par :\\ \( f(x) = x^3 - 3x + 1 \).

\begin{enumerate}
    \item Étudier les variations de \( f \).
    \item En déduire le tableau de variations de \( f \).
    \item Démontrer que l'équation \( f(x) = 0 \) admet trois solutions.
    \item Donner un encadrement d'amplitude \( 10^{-2} \) près de chacune de ces solutions.
\end{enumerate}
\textbf{\underline{Exercice 7}}
\begin{enumerate}
    \item Soit la fonction \( g \) définie par :\\ 
\( g(x) = x\sqrt{x^2 + 1} - 1. \)
    \begin{enumerate}
        \item Étudier les variations de \( g \).
        \item Montrer qu'il existe un unique réel\\ \( \alpha \in [0, 7] \) tel que \( g(\alpha) = 0 \).
    \end{enumerate}
    \item Soit la fonction \( f \) définie par :\\ 
\( f(x) = \frac{x^3}{3} - \sqrt{1 + x^2} \quad \text{et} \quad (C_f) \) sa courbe représentative.
    \begin{enumerate}
        \item Étudier les limites de \( f \) aux bornes de \( D_f \), puis étudier les branches infinies de \( (C_f) \).
        \item Montrer que : \( f'(x) = \frac{xg(x)}{\sqrt{1 + x^2}} \).\\ 
        En déduire les variations de \( f \).
    \end{enumerate}
\end{enumerate}
\textbf{\underline{Exercice 8}}
Soit la fonction \( f \) définie par :
\[
f(x) = 
\begin{cases}
\frac{(x - 1)^2}{\sqrt{x^2 - 1}} & \text{si } |x| > 1 \\
x^2 - 3x + 2 & \text{si } |x| < 1
\end{cases}
\]

\begin{enumerate}
    \item Déterminer \( D_f \).
    \item Peut-on prolonger \( f \) par continuité en 1 ? En -1 ? Si oui, préciser le prolongement par continuité.
    \item Montrer que la fonction \( g \), restriction de \( f \) à l'intervalle \( ]-1;1[ \), définit une bijection de \( ]-1;1[ \) vers un intervalle \( J \) à préciser.
    \item \( g^{-1} \) est-elle dérivable en \( \frac{3}{4} \) ? Justifier.
    \item Déterminer \( g^{-1}(x) \).
\end{enumerate}
\textbf{\underline{Exercice 9}}
Dresser le tableau de variation des fonctions \( f \) suivantes :

\begin{enumerate}
    \item \( f(x) = x^2 + 4x - 1 \)
    \item \( f(x) = ax^2 + bx + c \, (a \neq 0) \)
    \item \( f(x) = \frac{2x - 5}{x + 1} \)
    \item \( f(x) = \frac{ax + b}{cx + d} \)
    \item \( f(x) = x^3 - 4x^2 + 5 \)
    \item \( f(x) = -4x^3 + 3x \)
    \item \( f(x) = x^4 + 2x^2 - 10 \)
    \item \( f(x) = -x^4 + 8x^2 - 5 \)
    \item \( f(x) = \frac{x^2 - x}{x^2 + x} \)
    \item \( f(x) = \frac{x^2 + x}{x^2 - 4} \)
    \item \( f(x) = \frac{5x^2 + 2x - 11}{x^2 - x - 2} \)
    \item \( f(x) = \frac{x^2 + x - 2}{x^2} \)
    \item \( f(x) = \frac{x}{2} - \frac{3}{2x} \)
    \item \( f(x) = x^2 + 2x - 3 \)
    \item \( f(x) = \sqrt{x + 3} \)
    \item \( f(x) = \frac{2x^2 + |x - 3|}{x^2 - |x - 1|} \)
\end{enumerate}
\end{multicols}

\end{document}
