\documentclass[a4paper,12pt]{article}
\usepackage[french]{babel}
\usepackage{amsmath, amssymb}
\usepackage{tikz}
\usetikzlibrary{positioning}

\begin{document}

\textbf{Chapitre 02 : Fonction Numériques}

\section*{I.Définition de fonction et application}
\subsection*{1.Définition et présentation d'une fonction}

Soient $E$ et $F$ deux ensembles, on appelle \textbf{fonction} toute relation qui associe à tout élément de $E$ \underline{au plus} un élément de $F$.

    \begin{tikzpicture}[scale=2]
        % set A
        \draw[line width=1pt] (0,0) ellipse (1cm and 2cm);
        \node[above] at (0,2.2) (A) {$A$};
        \fill (0,1.4) circle(1pt) node[left] (a) {$a$}; 
        \fill (-0.5,0.7) circle(1pt) node[left] (b) {$b$};
        \fill (0,0) circle(1pt) node[left] (c) {$c$};
        \fill (-0.5,-0.7) circle(1pt) node[left] (d) {$d$};
        \fill (0,-1.4) circle(1pt) node[left] (e) {$e$};
        
        %set B
        \draw[line width=1pt] (5,0) ellipse (1cm and 2cm);
        \node[above] at (5,2.2) (B) {$B$};
        \fill (5,1.4) circle(1pt) node[right] (p) {$p$}; 
        \fill (4.5,0.6) circle(1pt) node[right] (q) {$q$};
        \fill (5,-0.2) circle(1pt) node[right] (r) {$r$};
        \fill (4.5,-1.0) circle(1pt) node[right] (s) {$s$};
        
        % relations
        \draw[-latex,cyan] (a) -- (r);
        \draw[-latex,cyan] (b) -- (p);
        \draw[-latex,cyan] (c) -- (s);
        \draw[-latex,cyan] (d) -- (s);
        \draw[-latex,cyan] (e) -- (q);
    \end{tikzpicture}

$h$ \textbf{est une fonction} car élément de A est associé à au maximum 1 élément de B.

    \begin{tikzpicture}[scale=2]
        % set A
        \draw[line width=1pt] (0,0) ellipse (1cm and 2cm);
        \node[above] at (0,2.2) (A) {$A$};
        \fill (0,1.4) circle(1pt) node[left] (a) {$a$}; 
        \fill (-0.5,0.7) circle(1pt) node[left] (b) {$b$};
        \fill (0,0) circle(1pt) node[left] (c) {$c$};
        \fill (-0.5,-0.7) circle(1pt) node[left] (d) {$d$};
        \fill (0,-1.4) circle(1pt) node[left] (e) {$e$};
        
        %set B
        \draw[line width=1pt] (5,0) ellipse (1cm and 2cm);
        \node[above] at (5,2.2) (B) {$B$};
        \fill (5,1.4) circle(1pt) node[right] (p) {$p$}; 
        \fill (4.5,0.6) circle(1pt) node[right] (q) {$q$};
        \fill (5,-0.2) circle(1pt) node[right] (r) {$r$};
        \fill (4.5,-1.0) circle(1pt) node[right] (s) {$s$};
        
        % relations
        \draw[-latex,cyan] (a) -- (r);
        \draw[-latex,cyan] (d) -- (s);
        \draw[-latex,cyan] (e) -- (q);
    \end{tikzpicture}

$h$ \textbf{est une fonction} car  est associé à deux valeurs distinctes.


    \begin{tikzpicture}[scale=2]
        % set A
        \draw[line width=1pt] (0,0) ellipse (1cm and 2cm);
        \node[above] at (0,2.2) (A) {$A$};
        \fill (0,1.4) circle(1pt) node[left] (a) {$a$}; 
        \fill (-0.5,0.7) circle(1pt) node[left] (b) {$b$};
        \fill (0,0) circle(1pt) node[left] (c) {$c$};
        \fill (-0.5,-0.7) circle(1pt) node[left] (d) {$d$};
        \fill (0,-1.4) circle(1pt) node[left] (e) {$e$};
        
        %set B
        \draw[line width=1pt] (5,0) ellipse (1cm and 2cm);
        \node[above] at (5,2.2) (B) {$B$};
        \fill (5,1.4) circle(1pt) node[right] (p) {$p$}; 
        \fill (4.5,0.6) circle(1pt) node[right] (q) {$q$};
        \fill (5,-0.2) circle(1pt) node[right] (r) {$r$};
        \fill (4.5,-1.0) circle(1pt) node[right] (s) {$s$};
        
        % relations
        \draw[-latex,cyan] (a) -- (p);
        \draw[-latex,cyan] (c) -- (p);
        \draw[-latex,cyan] (c) -- (s);
        \draw[-latex,cyan] (d) -- (s);
        \draw[-latex,cyan] (e) -- (q);
    \end{tikzpicture}

$h$ \textbf{n'est pas une fonction} car $d$ est associé à deux valeurs distinctes.

\textbf{NB :}
\begin{itemize}
    \item \( E \) est appelé \textbf{l'ensemble de départ} et \( F \) est appelé \textbf{l'ensemble d'arrivée}.
    \item Si \( f \) est une fonction, on dit que \( f \) est une fonction de \( E \) dans \( F \) ou \( E \) dans \( F \) on note \( E \to F \).
    \item On appelle \textbf{fonction réelle} toute relation qui à chaque élément de \( \mathbb{R} \) associe un élément de \( \mathbb{R} \).
\end{itemize}

\subsection*{Notation et Vocabulaire}

La fonction réelle \( f \) est notée :
\[
    f : \mathbb{R} \to \mathbb{R}
\]
\[
    x \mapsto f(x)
\]

\begin{itemize}
    \item \( x \) est appelé \textbf{antécédent}.
    \item Le réel \( y = f(x) \) est appelé \textbf{image} de \( x \) par \( f \).
    \item On dit que \( f \) est une fonction dont la variable \( x \) est réelle.
\end{itemize}

\subsection*{Exemples de fonctions réelles}

\begin{enumerate}

     \item[a)] Soit la fonction \( f \) définie par :
    \[
    \begin{aligned}
        f : \mathbb{R} &\to \mathbb{R} \\
        x &\mapsto x + 1
    \end{aligned}
    \]
    \item[b] Soit la fonction \( g \) définie par :
    \[
    \begin{aligned}
        g : \mathbb{R} &\to \mathbb{R} \\
        x &\mapsto x^2 + 8x + 1
    \end{aligned}
    \]
    
    \item[c)] Soit la fonction \( h \) définie par :
    \[
    \begin{aligned}
        h : \mathbb{R} &\to \mathbb{R} \\
        x &\mapsto \sqrt{8x + 1}
    \end{aligned}
    \]
    
    \item[d)] Soit la fonction \( k \) définie par :
    \[
    \begin{aligned}
        k : \mathbb{R} &\to \mathbb{R} \\
        x &\mapsto \frac{2x}{x+1}
    \end{aligned}
    \]
\end{enumerate}

\textbf{Remarque :}
\begin{itemize}
    \item \( f \) et \( g \) sont appelées \textbf{fonctions polynomiales}.
    \item \( h \) est une fonction \textbf{irrationnelle}.
    \item \( k \) est une fonction \textbf{rationnelle}.
\end{itemize}

\subsection*{Exemple : Exercice d'application}

\begin{center}
\begin{tikzpicture}
    % Ensemble A
    \draw[line width=1pt] (-2,2) ellipse (1 and 3.4);
    \node[above] at (-2,5.5) {\textbf{A}};

    % Éléments de A
    \node (a1) at (-2,4.3) {1};
    \node (a2) at (-2,3.2) {2};
    \node (a3) at (-2,2) {3};
    \node (a4) at (-2,0.9) {4};
    \node (a5) at (-2,-0.2) {5};
    \node (a6) at (-2,-1) {6};

    % Ensemble B
    \draw[line width=1pt] (2,2) ellipse (1 and 3.4);
    \node[above] at (2,5.5) {\textbf{B}};

    % Éléments de B
    \node (b1) at (2,4.3) {12};
    \node (b2) at (2,3.5) {6};
    \node (b3) at (2,3) {4};
    \node (b4) at (2,2) {17};
    \node (b5) at (2,1) {0};
    \node (b6) at (2,0.2) {11};
    \node (b6) at (2,-0.5) {15};

    % Flèches représentant la fonction
    \draw[->,red,thick] (a1) -- (b4);
    \draw[->,red,thick] (a2) -- (b3);
    \draw[->,red,thick] (a3) -- (b5);
    \draw[->,red,thick] (a4) -- (b4);
    \draw[->,red,thick] (a5) -- (b1);
    
\end{tikzpicture}
\end{center}

\textbf{Questions :}
\begin{enumerate}
    \item Donner l'image de 1, de 4 et de 6.
    \item Donner l'antécédent de 12, de 11, de 4 et de 6.
\end{enumerate}

\textbf{\textcolor{red}{Correction}}

\subsection*{2)Définition et présentation d'une application}

\subsubsection*{a.Activité}
\textbf{En considérant les trois diagrammes suivants :}
\begin{enumerate}
    \item Pour chaque fonction supprimmer les antécédents qui n'ont pas d'image.
    \item Dans chaque cas, attribue une nouvelle lettre à l'ensemble de départ.
\end{enumerate}

\begin{center}
\begin{tikzpicture}[scale=1.5]

    % Premier diagramme
    % Ensemble A
    \draw[line width=1pt] (-4,2) ellipse (1.2 and 2.5);
    \node[above] at (-4,4) {\textbf{A}};

    % Éléments de A
    \node (a1) at (-4,3.5) {1};
    \node (a2) at (-4,2.5) {8.5};
    \node (a3) at (-4,1.5) {14};
    \node (a4) at (-4,0.5) {7};
    \node (a5) at (-4,-0.5) {4.5};
    \node (a6) at (-4,-1.5) {9};

    % Ensemble B
    \draw[line width=1pt] (-1,2) ellipse (1.2 and 2.5);
    \node[above] at (-1,4) {\textbf{B}};

    % Éléments de B
    \node (b1) at (-1,3.5) {1,2};
    \node (b2) at (-1,2.5) {0};
    \node (b3) at (-1,1.5) {0.7};
    \node (b4) at (-1,0.5) {7};
    \node (b5) at (-1,-0.5) {13};
    \node (b6) at (-1,-1.5) {10.5};

    % Flèches
    \draw[->,red,thick] (a1) -- (b1);
    \draw[->,red,thick] (a2) -- (b3);
    \draw[->,red,thick] (a3) -- (b4);
    \draw[->,red,thick] (a4) -- (b5);
    \draw[->,red,thick] (a5) -- (b6);

    % Deuxième diagramme
    % Ensemble A
    \draw[line width=1pt] (3,2) ellipse (1.2 and 2.5);
    \node[above] at (3,4) {\textbf{A}};

    % Éléments de A
    \node (c1) at (3,3.5) {1,2};
    \node (c2) at (3,2.5) {4,5};
    \node (c3) at (3,1.5) {5,7};
    \node (c4) at (3,0.5) {7,8};
    \node (c5) at (3,-0.5) {4,5};
    \node (c6) at (3,-1.5) {5,7};

    % Ensemble B
    \draw[line width=1pt] (6,2) ellipse (1.2 and 2.5);
    \node[above] at (6,4) {\textbf{B}};

    % Éléments de B
    \node (d1) at (6,3.5) {12};
    \node (d2) at (6,2.5) {6,4};
    \node (d3) at (6,1.5) {17};
    \node (d4) at (6,0.5) {0};
    \node (d5) at (6,-0.5) {9};

    % Flèches
    \draw[->,red,thick] (c1) -- (d1);
    \draw[->,red,thick] (c2) -- (d2);
    \draw[->,red,thick] (c3) -- (d3);
    \draw[->,red,thick] (c4) -- (d4);
    \draw[->,red,thick] (c5) -- (d5);

\end{tikzpicture}
\end{center}
\subsubsection*{b.Domaine de définition ou ensemble de définition}
Soit $f$ une fonction numérique de variable réelle.

\section*{\color{red}IV. Image directe - Image réciproque par une application}

\subsection*{\color{red}1. Image directe}

\subsubsection*{\color{red}a. Définition}
Soit $f$ une fonction définie de \( E \to F \) et \( A \) une partie de \( E \). On appelle image directe (image de \( A \) par \( f \)), notée \( f(A) \), l'ensemble des images par \( f \) de tous les éléments de \( A \cap Df \).

\[
f(A) = \{ y \in F \mid \exists x \in A \cap Df, y = f(x) \}
\]
\subsection*{\color{red}b. Exemple} % Seule la section est en rouge
\text{Soit }
    \[
    \begin{aligned}
 f : \mathbb{R} &\to \mathbb{R} \\
        x &\mapsto \frac{1}{x + 1}
    \end{aligned}
    \]

Trouver les images par \( f \) de chacun des sous-intervalles de \( \mathbb{R} \) suivants :
\[
A = ]-1, 4] \quad ; \quad B = [-5, 3]
\]

$f$ existe ssi $x\neq -1$ 

Donc $Df=\mathbb{R}\setminus\{-1\}$

$\star \forall x \in A$, $x\in Df$, calculons $f(A)$ 

$
\begin{aligned}
x \in A &\implies -1<x\leq 4\\
&\implies 0<x+1\leq 5\\
&\implies \frac{1}{5}\leq\frac{1}{x+1}\\
&\implies \frac{1}{5}\leq f(x)\\
&\implies f(x)\in \left]\frac{1}{5};+\infty\right[\\
\end{aligned}
$

$
f(A) = \left[ \frac{1}{5}, +\infty \right]
$

\[
\forall x \in B \cap \mathcal{D}_f \Rightarrow x \in [-5; -1[ \cup ]-1; 3]
\]

\[
x \in [-5; -1[ \quad \text{ou} \quad x \in ]-1; 3]
\]

\[
-5 \leq x < -1 \quad \text{ou} \quad -1 < x \leq 3
\]

\[
-4 \leq x+1 < 0 \quad \text{ou} \quad 0 < x+1 \leq 4
\]

\[
-\frac{1}{4} > \frac{1}{x+1} \quad \text{ou} \quad \frac{1}{4} \leq \frac{1}{x+1}
\]

\[
f(x) \leq -\frac{1}{4} \quad \text{ou} \quad f(x) \geq \frac{1}{4}
\]

\[
f(B) = ]-\infty; -\frac{1}{4}] \cup [\frac{1}{4}; +\infty[
\]
\section*{\textcolor{red}{2) Image réciproque}}

\subsection*{\textcolor{red}{a) Définition :}}
Soit \( f \) une fonction définie de \( E \) vers \( F \), d'ensemble de définition \( \mathcal{D}_f \), et \( B \) une partie de \( F \). On appelle \textcolor{red}{image réciproque} de \( B \) par \( f \), la \textcolor{red}{partie de \( \mathcal{D}_f \)} notée : \( f^{-1}(B) \), constituée des antécédents par \( f \) de tous les éléments de \( B \).

\subsection*{\textcolor{red}{b) Remarque : Comment trouver l'image réciproque ?}}
Pour trouver l'image réciproque d'un intervalle \( B \) dans \( \mathbb{R} \) par une fonction \( f \), d'ensemble de \noindent \textbf{\textcolor{blue}{définition}} \textcolor{red}{\( \mathcal{D}_f \)}\textbf{\textcolor{blue}{, on résout l’un des systèmes d’inconnues suivants :}}

\begin{itemize}
    \item \textbf{\textcolor{red}{Si \( B = [a ; b] \), alors on résout :}}
    \[
    \begin{cases}
        x \in \mathcal{D}_f \\
        a \leq f(x) \leq b
    \end{cases}
    \]

    \item \textbf{\textcolor{red}{Si \( B = ]a ; b] \), alors on résout :}}
    \[
    \begin{cases}
        x \in \mathcal{D}_f \\
        a < f(x) \leq b
    \end{cases}
    \]

    \item \textbf{\textcolor{red}{Si \( B = [a ; b[ \), alors on résout :}}
    \[
    \begin{cases}
        x \in \mathcal{D}_f \\
        a \leq f(x) < b
    \end{cases}
    \]

    \item \textbf{\textcolor{red}{Si \( B = ]-\infty ; b] \), alors on résout :}}
    \[
    \begin{cases}
        x \in \mathcal{D}_f \\
        f(x) \leq b
    \end{cases}
    \]

    \item \textbf{\textcolor{red}{Si \( B = [a ; +\infty[ \), alors on résout :}}
    \[
    \begin{cases}
        x \in \mathcal{D}_f \\
        f(x) \geq a
    \end{cases}
    \]

    \item \textbf{\textcolor{red}{Si \( B = \{ b \} \), alors on résout :}}
    \[
    \begin{cases}
        x \in \mathcal{D}_f \\
        f(x) = b
    \end{cases}
    \]
\end{itemize}

\subsection*{\textcolor{red}{c) Exemple :}}
Soit
    \[
    \begin{aligned}
 f : \mathbb{R} &\to \mathbb{R} \\
        x &\mapsto 2x - 1
    \end{aligned}
    \]

    \[
    \begin{aligned}
 g : \mathbb{R} &\to \mathbb{R} \\
        x &\mapsto x^2 - x - 2
    \end{aligned}
    \]

\begin{enumerate}
    \item Trouver l’image réciproque par \( f \) de \( B = [1,3] \).
    \item Trouver l’image réciproque par \( g \) de \( B = \{0\} \).
\end{enumerate}

\textbf{\underline{Résolution}}

\begin{enumerate}
    \item[1)] Pour \( B = [1,3] \), déterminons l’image réciproque par \( f \).

    \begin{itemize}
        \item \( f(x) \) :
        
        \[
        \left\{
        \begin{array}{l}
            x \in \mathcal{D}_f  \\
            1 \leq f(x) \leq 3
        \end{array}
        \right.
        \]

        \[
        \left\{
        \begin{array}{l}
            x \in \mathbb{R} \\
            1 \leq 2x - 1 \leq 3
        \end{array}
        \right.
        \]

        \[
        \left\{
        \begin{array}{l}
            x \in \mathbb{R} \\
            1 + 1 \leq 2x - 1 + 1 \leq 3 + 1
        \end{array}
        \right.
        \]

        \[
        \left\{
        \begin{array}{l}
            x \in \mathbb{R} \\
            2 \leq 2x \leq 4
        \end{array}
        \right.
        \]

        \[
        \left\{
        \begin{array}{l}
            x \in \mathbb{R} \\
            \frac{2}{2} \leq \frac{2x}{2} \leq \frac{4}{2}
        \end{array}
        \right.
        \]

        \[
        \left\{
        \begin{array}{l}
            x \in \mathbb{R} \\
            1 \leq x \leq 2
        \end{array}
        \right.
        \]

    \end{itemize}

    \[
    \boxed{f^{-1}(B) = [1,2]}
    \]

    \item[2)] Pour \( g(x) \) :
\end{enumerate}
 \[
    D_g = \mathbb{R}
    \]

    \[
    \left\{
    \begin{array}{l}
        x \in \mathbb{R} \\
        x^2 - x - 2 = 0
    \end{array}
    \right.
    \]

    \[
    \left\{
    \begin{array}{l}
        x \in \mathbb{R} \\
        (x - 1)(x + 2) = 0
    \end{array}
    \right.
    \]

    \[
    \left\{
    \begin{array}{l}
        x \in \mathbb{R} \\
        x = -1 \quad \text{ou} \quad x = 2
    \end{array}
    \right.
    \]

    \[
    \boxed{g^{-1}(B) = \{-1,2\}}
    \]

\vspace{0.5cm}
\textcolor{red}{\textbf{\underline{REMARQUE !!!}}}  

Toute application à la fois \underline{injective} et \underline{surjective} est forcément \textcolor{red}{\underline{bijective}}.
\end{document}
