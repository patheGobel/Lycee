\documentclass{article} % Classe de document standard
\usepackage{tikz} % Package pour les dessins vectoriels
\usetikzlibrary{trees} % Bibliothèque pour les arbres
\usepackage{xcolor} % Pour la gestion des couleurs

\begin{document}

% Styles pour les différents niveaux de l'arbre
\tikzstyle{level 1}=[level distance=2.5cm, sibling distance=3.5cm] % Niveau 1 : distance verticale et entre branches
\tikzstyle{level 2}=[level distance=2.5cm, sibling distance=1.5cm] % Niveau 2 : plus resserré
\tikzstyle{bag} = [text centered] % Style pour les nœuds intermédiaires
\tikzstyle{end} = [text centered] % Style pour les feuilles
\definecolor{darkgreen}{RGB}{0,100,0} % Couleur vert foncé (non utilisée ici)
\definecolor{darkred}{RGB}{139,0,0} % Couleur rouge foncé

\begin{tikzpicture}[grow=right, sloped] % Arbre qui se développe vers la droite, étiquettes inclinées

% Nœud racine (départ de l'univers)
\node[bag] {\textcolor{darkred}{$\Omega$}} 
    child {
        % Première branche : événement B
        node[bag] {\textcolor{darkred}{$B$}} 
            child {
                % Sous-branche de B : issue V
                node[end] {\textcolor{darkred}{$V$}} 
                edge from parent
                node[above] {\textcolor{black}{$\frac{1}{16}$}} % Probabilité conditionnelle : P(V|B)
            }
            child {
                % Sous-branche de B : issue M
                node[end] {\textcolor{darkred}{$M$}} 
                edge from parent
                node[above] {\textcolor{black}{$\cdots$}} % Probabilité à compléter
            }
            child {
                % Sous-branche de B : issue R
                node[end] {\textcolor{darkred}{$R$}} 
                edge from parent
                node[above] {\textcolor{black}{$\cdots$}} % Probabilité à compléter
            }
        edge from parent
        node[above] {\textcolor{black}{$\frac{1}{2}$}} % Probabilité de B : P(B)
    }
    child {
        % Deuxième branche : événement A
        node[bag] {\textcolor{darkred}{$A$}} 
            child {
                % Sous-branche de A : issue V
                node[end] {\textcolor{darkred}{$V$}} 
                edge from parent
                node[above] {\textcolor{black}{$\cdots$}} % Probabilité à compléter
            }
            child {
                % Sous-branche de A : issue M
                node[end] {\textcolor{darkred}{$M$}} 
                edge from parent
                node[above] {\textcolor{black}{$\cdots$}} % Probabilité à compléter
            }
            child {
                % Sous-branche de A : issue R
                node[end] {\textcolor{darkred}{$R$}} 
                edge from parent
                node[above] {\textcolor{black}{$\frac{3}{10}$}} % Probabilité conditionnelle : P(R|A)
            }
        edge from parent
        node[above] {\textcolor{black}{$\cdots$}} % Probabilité de A : P(A), à compléter
    };

\end{tikzpicture}

\end{document}
